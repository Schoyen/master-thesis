\chapter{Coupled cluster theory}
    % TODO: Why is coupled cluster so awesome?
    % Discuss size-consistency and size-extensivity.
    % See Crawford & Schaefer for a good intro to these topics.


    \section{Time-independent coupled cluster theory}
        This section follows closely the derivation done in
        \citetitle{crawford2000introduction} by
        \citeauthor{crawford2000introduction} \cite{crawford2000introduction}.
        Starting with the time-independent Schrödinger equation, the coupled
        cluster method seeks to find a many-body wave function $\ketcc$ which
        solves
        \begin{align}
            \hamil\ketcc = \energycc\ketcc.
        \end{align}
        Start from a single reference Slater determinant $\ketslat$ coupled
        cluster makes the ansatz that the true many-body wave function is given
        by
        \begin{align}
            \ketcc \equiv e^{\clust}\ketslat
            = \sum_{n = 0}^{\infty}\frac{1}{n!}\clust^n\ketslat,
            \label{eq:cc_wave_function}
        \end{align}
        where the \emph{cluster operator} $\clust$ is given by a sum of excitation
        operators $\clust_{\mu}$.
        \begin{align}
            \clust &= \sum_{\mu = 1}^{\infty} \clust_{\mu}
            = \clustamp^a_i\ccr{a}\can{i}
            + \para{\frac{1}{2!}}^2\clustamp^{ab}_{ij}\ccr{a}\ccr{b}\can{i}\can{j}
            + \para{\frac{1}{3!}}^2\clustamp^{abc}_{ijk}
            \ccr{a}\ccr{b}\ccr{c}\can{i}\can{j}\can{k}
            + \dots,
            \label{eq:cluster-operator}
        \end{align}
        where the cluster amplitudes $\clustamp^{abc\dots}_{ijk\dots}$ are the
        primary unknowns.
        In this formulation, the cluster amplitudes satisify the anti-symmetric
        properties
        \begin{align}
            \clustamp^{abc\dots}_{ijk\dots}
            = -\clustamp^{bac\dots}_{ijk\dots}
            = -\clustamp^{abc\dots}_{jik\dots}
            = \clustamp^{bac\dots}_{jik\dots},
        \end{align}
        where interchange of a pair of indices gives rise to a sign change.
        Note that interchange of indices from upper to lower, or vice-versa, are
        not permitted.
        By inserting \autoref{eq:cc_wave_function} into the time-independent
        Schrödinger equation we get
        \begin{align}
            \hamil\ketcc
            = e^{\clust}\ketslat
            = \energycc e^{\clust}\ketslat
            = \energycc \ketcc.
        \end{align}
        Projecting onto the reference Slater determinant we are left with
        \begin{align}
            \braslat\hamil\ketcc
            &= \braslat\energycc e^{\clust}\ketslat
            = \energycc \braslat e^{\clust}\ketslat
            = \energycc,
        \end{align}
        where for orthonormal Slater determinants, the coupled cluster wave
        function assumes \emph{intermediate normalization}, viz.
        \begin{align}
            \braket{\slat}{\cc}
            &= \bra{\slat}\para{
                \1 + \clust + \frac{1}{2!}\clust^2 + \dots
            }\ket{\slat}
            = 1.
        \end{align}


        \subsection{Truncated coupled cluster theory}

    \section{Time-dependent coupled cluster theory}
        Stuff!


        \subsection{Measuring the quality of the reference state in time}
            When evolving a system in time, we can -- and will -- get in a
            situation where the our initial state has little to no overlap with
            the time-evolved state.
            That is, we can get in a situation where
            \begin{align}
                \braket{\psi(t)}{\psi(0)} \approx 0,
            \end{align}
            where $\ket{\psi(0)}$ is some initial state and $\ket{\psi(t)}$ is
            the time-evolved state at a later time $t$.
            In time-dependent configuration interaction theory\footnote{
                Without intermediate normalization.
                % TODO: Check if this applies in general
            } this is handled automatically by the
            zero-order amplitude.
            \begin{gather}
                \ket{\ci(t)} = C_0(t)\ketslat + C^{a}_{i}(t)\ket{\slat^{a}_{i}} + \dots
                \implies \braket{\slat}{\ci(t)} = C_0(t),
            \end{gather}
            where $\ketslat$ is the reference determinant and we have
            orthonormal determinants.
            In time-dependent coupled cluster theory where we assume
            intermediate normalization, there is however a different story to be
            told.
            For the time-dependent coupled cluster wavefunction we have that
            \begin{gather}
                \ket{\cc(t)} = e^{\clust(t)}\ketslat
                = \para{
                    \1
                    + \clust(t)
                    + \dots
                }\ketslat
                \implies
                \braket{\slat}{\cc(t)} = 1.
            \end{gather}
            This might not look like much, but for strong fields, or long
            time-evolution, we quickly find states which have little to no
            overlap with the original reference state.
            When this is the case the cluster amplitudes are left with the job
            of both ``removing'' the reference state from the cluster wave
            function \emph{and} providing contribution from higher excited
            determinants \cite{pedersen2018symplectic}.
            Frustratingly enough the formulation of time-dependent coupled
            cluster with static spin-orbitals is not able to solve this problem.
            When $\braket{\slat}{\cc(t)} \to 0$, the magnitude of the amplitudes
            sky-rocket, the energy is no longer conserved (when the
            time-dependence in the Hamiltonian is turned off), and the data
            produced can no longer be trusted.
            % TODO: This can be shown theoretically. See article by Thomas and
            % Simen for the steps.
            Luckily, there is a solution\footnote{
                As far as the author can tell.
            } where we allow the spin-orbitals change in time along with the
            amplitudes.
            We will in the coming sections explore two versions of this scheme
            and demonstrate that this fixes the problems described by
            \citeauthor{pedersen2018symplectic} \cite{pedersen2018symplectic}.

            But, before we try to solve the problem, we can formulate a
            ``zero-order'' amplitude for the time-dependent coupled cluster wave
            function which does not solve the problem, but at least lets us
            measure how good or bad the reference state is in time.
            This is done by introducing a phase factor $\clustamp_0(t) \in
            \mathbb{C}$ which we insert in the exponential ansatz.
            \begin{gather}
                \ket{\cc(t)}
                = e^{\clustamp_0(t) + \clust(t)}\ketslat
                = e^{\clustamp_0(t)} e^{\clust(t)}\ketslat,
                \\
                \bra{\bicc(t)}
                = \braslat\para{\1 + \clustl(t)}
                e^{-\clustamp_0(t) - \clust(t)}
                = \braslat\para{\1 + \clustl(t)}
                e^{-\clust(t)} e^{-\clustamp_0(t)},
            \end{gather}
            where we've used that $\clustamp_0(t)$ is a number to separate it
            from the cluster amplitudes in the exponential.
            Regardless if $\clustamp_0(t)$ is real or not, the inclusion of the
            phase does not change any of the equations used in coupled cluster.
            However, we can now see that
            \begin{align}
                \braket{\slat}{\cc(t)} = e^{\clustamp_0(t)},
            \end{align}
            which tells us how good the reference state $\ketslat$ is at a later
            time $t$.
            Note that the inclusion of $\clustamp_0(t)$ is in no way able to be
            used as a amplitude to lower the presence of the reference as in the
            configuration interaction method.
            We also see that for $\clustamp_0(t) = 0$ we recover intermediate
            normalization and the known time-dependent coupled cluster method.
            The question remains how we can evaluate $\clustamp_0(t)$ from the
            coupled cluster equations.
            In general we find the time derivative of the cluster amplitudes
            from
            \begin{align}
                i\hslash \dpd[]{}{t}\clustamp_{\mu}
                &=
                \bra{\slat_{\mu}} e^{-\clust(t)} \hamil(t) e^{\clust(t)}
                \ketslat,
            \end{align}
            where $\mu$ denotes the excitation level of the reference slater
            determinant and the order of the cluster amplitudes as in
            \autoref{eq:cluster-operator}.
            By choosing $\mu = 0$ we are left with
            \begin{align}
                i\hslash \dpd[]{}{t}\clustamp_{0}
                &=
                \bra{\slat} e^{-\clust(t)} \hamil(t) e^{\clust(t)}
                \ketslat
                =
                \bra{\slat} \hamil(t) e^{\clust(t)}
                \ketslat,
            \end{align}
            which should look familiar as the equation for the ground state
            energy in the time-independent coupled cluster method if the
            explicit time-dependence is removed from the Hamiltonian and the
            cluster amplitudes.
            % TODO: Elaborate on why lambda_0(t) = 1.
