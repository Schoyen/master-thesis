\chapter{Coupled cluster theory}
    \label{chap:cc}
    In the previous chapter on the configuration interaction method we saw how
    the direct diagonalization of the full Slater determinant space led to an
    exponential scaling in the number of determinants.
    In \autoref{chap:hf} we demonstrated an approximation to the full many-body
    wave function using a single Slater determinant.
    The Hartree-Fock and the full configuration interaction methods therefore
    serve as endpoints to the approximations that can be done for the exact
    many-body wave function.
    The former is too simple, whereas the latter proves too challenging.
    The latter does open up for further approximations by truncating the Slater
    determinant space, but this leads to a size-inconsistent model which scales
    badly once we increase the number of particles.

    In lieu of these considerations, the coupled-cluster method provides an
    approximation to the exact many-body wave function by an exponential ansatz.
    This method provides higher order correlations beyond the single determinant
    approximation given in the Hartree-Fock method, and maintains a polynomial
    scaling for the Slater determinant basis if the cluster operators are
    truncated.
    In the untruncated limit the coupled-cluster method recovers the full
    configuration interaction method.
    We shall also see how the truncated coupled-cluster method is
    size-consistent.


    \section{Time-independent coupled-cluster theory}
        The coupled-cluster method seeks to find a many-body wave function
        $\ket*{\Psi}$ which solves the time-independent Schrödinger equation.
        Starting from a single reference Slater determinant $\ket*{\slat}$ built
        from an orthonormal basis of single-particle states\footnote{%
            We often use orthonormal molecular orbitals from the Hartree-Fock
            minimization procedure as our basis.
            This is the reason why coupled-cluster is often called a \emph{post
            Hartree-Fock} method.
        } coupled-cluster makes the ansatz that the true many-body wave function
        can be approximated by the exponential ansatz
        \begin{align}
            \ket*{\Psi} \equiv \exponential(\clust)\ket*{\slat}
            = \sum_{n = 0}^{\infty}\frac{1}{n!}\clust^n\ket*{\slat},
            \label{eq:cc_wave_function}
        \end{align}
        where the \emph{cluster operator} $\clust$ is given by a sum of excitation
        operators $\hat{X}{\mu}$ and cluster amplitudes $\clustamp_{\mu}$
        \begin{align}
            \clust &= \sum_{\mu = 1}^{\infty} \hat{X}_{\mu}\clustamp_{\mu}
            = \clustamp^a_i\ccr{a}\can{i}
            + \para{\frac{1}{2!}}^2\clustamp^{ab}_{ij}\ccr{a}\ccr{b}\can{i}\can{j}
            + \dots.
            \label{eq:cluster-operator}
        \end{align}
        Here the cluster amplitudes are the primary unknowns.
        Due to the antisymmetric properties of the excitation operators, the
        cluster amplitudes will also be antisymmetric with respect to the
        exchange of two virtual-virtual or occupied-occupied indices as seen in
        \autoref{eq:antisymmetric-excitation}.
        Furthermore, the cluster amplitudes commute with one another as the
        creation and annihilation operators run over non-overlapping sets of
        indices.
        Note that interchange of indices from upper to lower, or vice-versa, are
        not permitted.
        By inserting \autoref{eq:cc_wave_function} into the time-independent
        Schrödinger equation we get
        \begin{align}
            \hamil\ket*{\Psi}
            = \exponential(\clust)\ket*{\slat}
            = \energy \exponential(\clust)\ket*{\slat}
            = \energy \ket*{\Psi}.
        \end{align}
        Projecting onto the reference Slater determinant we are left with
        \begin{align}
            \mel*{\slat}{\hamil}{\Psi}
            &= \energy\mel*{\slat}{\exponential(\clust)}{\slat}
            = \energy,
            \label{eq:cc_energy}
        \end{align}
        where the coupled-cluster wave function assumes \emph{intermediate
        normalization}, viz.
        \begin{align}
            \braket*{\slat}{\Psi}
            &= \mel*{\slat}{
                \para{
                    \1 + \clust + \frac{1}{2!}\clust^2 + \dots
                }
            }{\slat}
            = 1.
        \end{align}
        Using the same projection technique as for the energy equation, we can
        obtain expressions for the cluster amplitudes.
        All we have to do is project onto an excited reference state, where the
        order of the excitation decides which order of the amplitudes we are
        solving for.
        That is,
        \begin{align}
            \mel*{\slat_{\mu}}{
                \hamil \exponential(\clust)
            }{\slat}
            &= \energy \mel*{\slat_{\mu}}{
                \exponential(\clust)
            }{\slat}.
            \label{eq:cc_amp}
        \end{align}
        Here the excitation order $\mu$ decides which amplitudes we are solving
        for, e.g., for the doubles cluster amplitudes $\clustamp^{ab}_{ij}$ we
        project onto the doubly excited reference state
        $\ket*{\slat^{ab}_{ij}}$.
        These equations are non-linear due to the presence of the exponential
        function creating cross terms with other excited amplitudes.
        Furthermore, if we don't truncate the cluster operator $\clust$ we
        recover the exact wave function as in the full configuration
        interaction method \cite{coester1958421, monkhorst1977421}.
        %The amplitude equations in \autoref{eq:cc_amp} must be solved
        %iteratively in order to find a minimum of the energy.

        By expanding the exponential cluster operator in a power series, as in
        \autoref{eq:cc_wave_function}, and inserting it into
        \autoref{eq:cc_energy} we get
        \begin{align}
            \mel*{\slat}{\hamil}{\Psi}
            &=
            \mel*{\slat}{
                \hamil
                \sum_{n = 0}^{\infty} \frac{1}{n!}\clust^n
            }{\slat}
            = \energy.
        \end{align}
        Depending on the form of the Hamiltonian, we will get a natural
        truncation of the cluster operators.
        A single cluster operator is at least a singles excitation operator and
        a cluster operator raised to the power of $n$ will at least excite $n$
        particles from the reference state.
        We restrict ourselves to at most two-body interactions in the
        Hamiltonian and we are therefore at most able to relax two particles
        from an excited state.
        As we are projecting onto the reference state the Hamiltonian is forced
        to relax the excitation done by the cluster operators in order to get a
        non-zero contribution.
        Thus, for a Hamiltonian with at most two-body operators the exponential
        power series truncates at $n = 2$ for the cluster operators.
        That is
        \begin{align}
            \mel*{\slat}{\hamil}{\Psi}
            &=
            \mel*{\slat}{\hamil}{\slat}
            + \mel*{\slat}{\hamil \clust}{\slat}
            + \frac{1}{2!}
            \mel*{\slat}{\hamil \clust^2}{\slat}
            = \energy.
            \label{eq:cc-energy-initial}
        \end{align}
        Note that this result is general irrespective of the truncation level
        chosen for the cluster operators themselves.
        This means that all cluster operators higher than doubles excitations
        will only contribute indirectly to the energy calculation via the
        singles and doubles amplitudes.

        \subsection{Rewriting the coupled-cluster equations}
            In order to get amiable equations for the coupled-cluster energy and
            amplitudes, we left multiply the Hamiltonian by the inverse of the
            exponential cluster operator.
            We then get the equations
            \begin{gather}
                \mel*{\slat}{
                    \exponential(-\clust)
                    \hamil
                    \exponential(\clust)
                }{\slat}
                = \energy,
                \label{eq:cc_energy_sim}
                \\
                \mel*{\slat_{\mu}}{
                    \exponential(-\clust)
                    \hamil
                    \exponential(\clust)
                }{\slat}
                = 0.
                \label{eq:cc_amp_sim}
            \end{gather}
            Even though \autoref{eq:cc_energy} and \autoref{eq:cc_amp} are
            formally correct, they are not tractable for numerical
            implementation, unlike \autoref{eq:cc_energy_sim} and
            \autoref{eq:cc_amp_sim} \cite{crawford2000introduction}.
            Furthermore, the amplitude equations are now decoupled from the
            energy equation, which will ensure size-extensivity.
            By applying the Baker-Campbell-Hausdorff formula we get
            \begin{align}
                \simhamil
                &=
                \exponential(-\clust)
                \hamil
                \exponential(\clust)
                =
                \hamil
                + \com{\hamil}{\clust}
                + \frac{1}{2!}\com{
                    \com{
                        \hamil
                    }{
                        \clust
                    }
                }{
                    \clust
                }
                + \dots.
            \end{align}
            We refer to $\simhamil$ as the \emph{similarity transformed
            Hamiltonian}.
            We now introduce the normal-ordered Hamiltonian $\hamil_N$ as defined in
            \autoref{subsec:normal-ordered-hamiltonian} with the reference
            energy from \autoref{subsec:reference-energy}.
            As the reference energy is just a number, the similarity transformed
            normal-ordered Hamiltonian takes on the form
            \begin{align}
                \simhamil_N
                &=
                \exponential(-\clust)
                \hamil_N
                \exponential(\clust)
                =
                \exponential(-\clust)
                \hamil
                \exponential(\clust)
                - \mel*{\slat}{\hamil}{\slat}
                = \simhamil - \energyref.
                \label{eq:sim-normal-hamil}
            \end{align}
            If the reference state is the Hartree-Fock state%
            \footnote{
                The reference state is more or less always the Hartree-Fock
                state in quantum chemistry computations.
            }
            all energy contributions beyond the reference energy is known as
            \emph{correlation energy}.
            Next, we apply Baker-Campbell-Hausdorff formula to
            \autoref{eq:sim-normal-hamil} and use generalized Wick's theorem to
            contract the Hamiltonian and the cluster operators.
            The only nonzero terms in this expansion will be the terms
            containing at least one contraction between the normal-ordered
            Hamiltonian and every cluster operator on its right.
            This is called the ``connected cluster'' theorem and yields a
            natural truncation of the Baker-Campbell-Hausdorff expansion
            \cite{crawford2000introduction, shavitt2009many}.
            For a Hamiltonian with at most two-body interactions the expansion
            contains at most four cluster operators to the right of the
            Hamiltonian,
            \begin{align}
                \simhamil_N
                &=
                \brak{
                    \hamil_N
                    + \hamil_N \clust
                    + \frac{1}{2!} \hamil_N \clust^2
                    + \frac{1}{3!} \hamil_N \clust^3
                    + \frac{1}{4!} \hamil_N \clust^4
                }_c.
                \label{eq:connected-hamil}
            \end{align}
            Here the subscript $c$ signifies that we only include the terms that
            are connected, i.e., that the Hamiltonian has at least one
            contraction with every cluster operator to its right.
            Note that the connected cluster theorem is a statement about the
            similarity transformed normal-ordered Hamiltonian.
            It is therefore applicable to the amplitude equations as well.

        \subsection{Energy equations}
            \label{subsec:cc-energy-equations}
            We are now ready to generate equations for the energy of the
            many-body system using the coupled-cluster method.
            We do this by computing
            \begin{align}
                \energy
                =
                \mel*{\slat}{
                    \exponential(-\clust)
                    \hamil
                    \exponential(\clust)
                }{\slat}
                = \mel*{\slat}{\simhamil_N}{\slat}
                + \mel*{\slat}{\hamil}{\slat}.
                \label{eq:cc-energy-normal}
            \end{align}
            Having already found an expression for the reference energy from
            \autoref{subsec:reference-energy}, our job is now to compute the
            expectation value of the similarity transformed normal-ordered
            Hamiltonian, or more concisely, the coupled-cluster correlation
            energy.
            We now insert the expression for the Hamiltonian from
            \autoref{eq:connected-hamil} into \autoref{eq:cc-energy-normal}.
            The first term is the expectation value of the normal-ordered
            Hamiltonian which is zero by construction, viz.
            \begin{align}
                \mel*{\slat}{\hamil_N}{\slat} = 0.
            \end{align}
            From \autoref{eq:cc-energy-initial} we know that all higher powers of
            the cluster operator than $2$ will annihilate the overlap.
            If we assume that the cluster operators contain at least the
            singles and doubles cluster operators, that is,
            \begin{align}
                \clust = \clust_1 + \clust_2 + \dots,
            \end{align}
            we get the energy contributions
            \begin{align}
                \mel*{\slat}{
                    \brak{
                        \hamil_N \clust
                    }_c
                }{\slat}
                =
                \fockten^{i}_{a}\clustamp^{a}_{i}
                +
                \frac{1}{4}\twoten^{ij}_{ab} \clustamp^{ab}_{ij},
                \label{eq:cc-energy-linear-contrib}
            \end{align}
            for the lone cluster operator, whereas the squared operator yields
            \begin{align}
                \frac{1}{2!}\mel*{\slat}{
                    \brak{
                        \hamil_N \clust^2
                    }_c
                }{\slat}
                &=
                \half \twoten^{ij}_{ab}\clustamp^{a}_{i}\clustamp^{b}_{j}.
                \label{eq:cc-energy-squared-contrib}
            \end{align}
            For a complete derivation of these expressions see
            \autoref{app:cc-energy-equations}.
            Including the reference energy, the total energy equation in the
            coupled-cluster method is given by
            \begin{align}
                \energy
                &= \mel*{\slat}{\simhamil}{\slat}
                =
                \fockten^{i}_{i}
                - \half \twoten^{ij}_{ij}
                + \fockten^{i}_{a}\clustamp^{a}_{i}
                + \half\twoten^{ij}_{ab}\brak{
                    \half\clustamp^{ab}_{ij}
                    + \clustamp^{a}_{i}\clustamp^{b}_{j}
                }.
                \label{eq:cc-energy-equation}
            \end{align}
            Depending on the truncation level of the cluster operators, we can
            adjust the energy expression in \autoref{eq:cc-energy-equation}.
            For example, in the coupled-cluster doubles (CCD) approximation, we
            get the energy equation
            \begin{align}
                \energy
                &= \mel*{\slat}{
                    \exponential(-\clust_2)
                    \hamil
                    \exponential(\clust_2)
                }{\slat}
                =
                \fockten^{i}_{i}
                - \half \twoten^{ij}_{ij}
                + \frac{1}{4}\twoten^{ij}_{ab} \clustamp^{ab}_{ij}.
                \label{eq:ccd-energy-equation}
            \end{align}
            As already mentioned, only the reference, singly and doubly excited
            states contribute to the final energy expression.
            All higher excitations couple to the energy, but only indirectly
            through the singles and doubles amplitudes.


        \subsection{Amplitude equations}
            Using \autoref{eq:cc_amp_sim} and Wick's theorem we find a set of
            non-linear equations which we solve iteratively to construct the
            coupled-cluster amplitudes.
            Unlike the energy equations in \autoref{eq:cc_energy_sim}, we are no
            longer projecting onto the reference state.
            This means that in addition to the similarity transformed
            Hamiltonian, we get an extra set of normal-ordered creation and
            annihilation operators from the excited state we are projecting
            onto.
            This seemingly benign inclusion of pairs of operators leads to a
            significant increase in the number of terms arising from Wick's
            theorem and can quickly lead to long and winded computations prone
            to much error.
            There are several ways to attack this problem, either by direct
            computation using Wick's theorem \cite{crawford2000introduction} or
            the more elegant solution of using diagrams
            \cite{crawford2000introduction, shavitt2009many}.
            The task at hand is to evaluate the equations
            \begin{align}
                \mel*{\slat_{\mu}}{\simhamil}{\slat}
                = \mel*{\slat_{\mu}}{\simhamil_N}{\slat}
                = \mel*{\slat}{
                    \hat{X}^{\dagger}_{\mu}\simhamil_N
                }{\slat}
                = 0,
            \end{align}
            where we we've simplified by similarity transformed Hamiltonian
            by inserting the normal ordered similiarty transformed Hamiltonian
            as the overlap between the excited state and the reference state is
            zero.
            Furthermore, we've expressed the excited determinant via the
            relaxation operator from \autoref{def:excitation-operator}.
            As the operator $\hat{X}^{\dagger}_{\mu}$ relaxes the incoming state
            from the right by an order $\mu$ we need only keep terms from the
            normal-ordered similarity transformed Hamiltonian that leaves the
            reference state excited at $\mu$.
            We will therefore look for terms in \autoref{eq:connected-hamil}
            that yields
            \begin{align}
                \simhamil_N\ket*{\slat}
                &\propto \ket*{\slat_{\mu}},
            \end{align}
            as all other terms will vanish.
            For each set of amplitudes we get a corresponding set of amplitude
            equations.
            As an example, for the coupled-cluster singles and doubles (CCSD)
            method we get the pair of amplitude equations
            \begin{gather}
                \mel*{\slat^{a}_{i}}{
                    \simhamil_N
                }{\slat}
                =
                \mel*{\slat}{
                    (\hat{X}^{a}_{i})^{\dagger}
                    \simhamil_N
                }{\slat}
                = 0,
                \\
                \mel*{\slat^{ab}_{ij}}{
                    \simhamil_N
                }{\slat}
                =
                \mel*{\slat}{
                    (\hat{X}^{ab}_{ij})^{\dagger}
                    \simhamil_N
                }{\slat}
                = 0.
            \end{gather}
            We'll describe both the process of finding amplitude equations and
            how we can solve these equations in \autoref{sec:cc-solver}.


        \subsection{Size-consistency}
            Having discussed the lack of size-consistency in the truncated
            configuration interaction method in
            \autoref{subsec:size-consistency}, we now turn our attention to the
            exponential ansatz used in the coupled-cluster method.
            We assume that the Hamiltonian is additively separable as in
            \autoref{eq:additively-separable}, and that the second quantized
            operators for the two subsystems $A$ and $B$ anticommute with one
            another.
            We now need to demonstrate that the compound coupled-cluster wave
            function $\ket*{\Psi^{AB}}$ is multiplicately separable as in
            \autoref{eq:multiplicately-separable} in order for the method to be
            size-consistent \cite{helgaker-molecular}.
            Now, as the cluster operators in the exponential ansatz consist of
            excitation operators -- which we from
            \autoref{subsec:excited-determinants} know to anticommute -- we have
            that
            \begin{align}
                \com{\clust_{A}}{\clust_{B}} = 0.
            \end{align}
            Furthermore, this extends to the cluster operator in subsystem $A$
            commuting with any other second quantized operator in subsystem $B$
            due to there being an even number of anticommutation.
            The compound wave function can now be found to be
            \begin{align}
                \ket*{\Psi^{AB}}
                &= \exponential(\clust^{AB})\ket*{\slat^{AB}}
                =
                \exponential(\clust^{A})\exp(\clust^{B})
                \ket*{\slat^{A}}\ket*{\slat^{B}}
                \\
                &=
                \exponential(\clust^{A})\ket*{\slat^{A}}
                \exponential(\clust^{B})\ket*{\slat^{B}}
                = \ket*{\Psi^{A}}\ket*{\Psi^{B}}.
            \end{align}
            The size-consistent property is retained for all exponential wave
            function ansatzes due to the separability of the exponential
            operator \cite{helgaker-molecular, size-extensivity}.

        \subsection{Non-variational coupled-cluster}
            \label{subsec:non-variational-coupled-cluster}
            From the postulates of quantum mechanics we know that every physical
            observable $q$ is described by a Hermitian operator $\hat{Q}$ acting
            on the Hilbert space of state vectors.
            Having defined the coupled-cluster wave function $\ket*{\Psi}$ by
            \begin{align}
                \ket*{\Psi} = \exponential(\clust)\ket*{\slat},
            \end{align}
            we would expect to be able to construct a solution to the amplitudes
            by virtue of the variational principle.
            That is, by requiring that the amplitudes minimize the functional
            \begin{align}
                \energy[\Psi, \Psi*]
                = \frac{
                    \mel*{\Psi}{\hamil}{\Psi}
                }{
                    \braket*{\Psi}{\Psi}
                }
                = \frac{
                    \mel*{\slat}{
                        (\exponential(\clust))^{\dagger}
                        \hamil
                        \exponential(\clust)
                    }{\slat}
                }{
                    \mel*{\slat}{
                       (\exponential(\clust))^{\dagger}
                       \exponential(\clust)
                   }{\slat}
                }.
                \label{eq:variational-cc}
            \end{align}
            However, there are some serious limitations in
            \autoref{eq:variational-cc}.
            By computing the series expansion of the exponential operators we
            find
            \begin{align}
                \mel*{\slat}{
                   (\exponential(\clust))^{\dagger}
                   \hamil
                   \exponential(\clust)
               }{\slat}
                = \mel*{\slat}{
                    \para{
                        \1 + \clust^{\dagger} + \dots
                    }
                    \hamil
                    \para{
                        \1 + \clust + \dots
                    }
                }{\slat},
                \label{eq:expansion-variational-cc}
            \end{align}
            where the relaxation operators on the left will act as excitation
            operators on the reference state we are projecting onto.
            This means that we no longer get a natural truncation with respect
            to the Hamiltonian \cite{crawford2000introduction}.

            %As the variational formulation of coupled-cluster yields intractable
            %equations and we note that the similarity transformed Hamiltonian is
            %non-Hermitian thus violating the postulates of quantum mechanics, we
            %are seemingly in a lot of trouble.
            %Luckily, by not truncating the cluster operator\footnote{%
            %    Or, for $N$ particles truncating at $\clust_N$.
            %} we will recover the formally exact solution within the space
            %spanned by our basis of spin-orbitals \cite{coester1958421,
            %monkhorst1977421}.

        \subsection{The coupled-cluster Lagrangian}
            Even though we conclude that the non-variational formulation of
            the coupled-cluster method yields an exact solution in the
            untruncated limit, and is accurate enough in the truncated limit
            that the lack of a bounded energy is not a large problem,
            we will get problems when looking at other properties than the
            energy as the conditions for the Hellmann-Feynman theorem are not
            met \cite{helgaker-molecular}.
            There have been attempts at computing observable quantities by using
            the variational expectation value for the operator in question
            \cite{exp-value-cizek, fink1974163}
            \begin{align}
                \expval{O}
                &= \frac{\mel*{\Psi}{\hat{O}}{\Psi}}{\braket*{\Psi}{\Psi}}
                = \frac{
                    \mel*{\slat}{
                        (\exponential(\clust))^{\dagger}
                        \hat{O}
                        \exponential(\clust)
                    }{\slat}
                }{
                    \mel*{\slat}{
                        (\exponential(\clust))^{\dagger}
                        \exponential(\clust)
                    }{\slat}
                },
            \end{align}
            but as $\ket*{\Psi}$ is not found from optimization of the energy
            using the variational principle in \autoref{eq:variational-cc}, we
            do not necessarily have that the energy found from the variational
            principle is the same as the energy found from the projected energy
            equation \cite{kvaal2013variational}.
            Replacing the Hamiltonian in \autoref{eq:cc_energy_sim} with the
            operator of the observable in question, we do get an expression for
            the observable.
            However, we return to the problem of computing first order
            properties being a cumbersome problem as the Hellmann-Feynman
            theorem is still not satisfied \cite{helgaker-molecular,
            kvaal2013variational}.

            In order for us to conform to the Hellmann-Feynman theorem, we can
            use a technique where we construct a variational Lagrangian
            \cite{helgaker1989, helgaker-molecular}.
            To motivate the need for a variational Lagrangian, we consider the
            coupled-cluster energy expression subject to a small perturbation
            $g$ \cite{monkhorst1977421}
            \begin{align}
                E(g, \vfg{\clustamp})
                &= \mel*{\slat}{
                    \exponential(-\clust(g))
                    \hamil(g)
                    \exponential(\clust(g))
                }{\slat},
                \label{eq:cc-perturbed-energy}
            \end{align}
            where $\mu$ denotes an excitation level and $\vfg{\clustamp}$ is the
            set of all cluster amplitudes.
            We also note that as the perturbation is represented on a grid, we
            will in the second quantized formulation absorb the pertubation in
            the matrix elements of the wave function instead of the wave
            function itself \cite{helgaker-molecular}.
            The expansion of the operators will then become
            \cite{monkhorst1977421}
            \begin{align}
                \hamil(g) &= \hamil + g\hamil^{(1)} + g^2\hamil^{(2)} + \dots, \\
                \clust(g) &= \clust + g\clust^{(1)} + g^2\clust^{(2)} + \dots.
            \end{align}
            We find the optimized energy of \autoref{eq:cc-perturbed-energy} by
            \begin{align}
                E(g) &= E(g, \vfg{\clustamp}^{*})
            \end{align}
            where $\vfg{\clustamp}^{*}$ for a given $g$ represents the solution
            to the set of amplitude equations
            \begin{align}
                f_{\mu}(g, \vfg{\clustamp}_{*})
                &=
                \left.
                \mel*{\Phi_{\mu}}{
                    \exponential(-\clust(g))
                    \hamil(g)
                    \exponential(\clust(g))
                }{\slat}
                \right\rvert_{\vfg{\clustamp} = \vfg{\clustamp}^{*}}
                = 0,
                \label{eq:cc-amplitude-set}
            \end{align}
            for all excitations $\mu$.
            Now, if the coupled-cluster wave function was indeed variational,
            this condition corresponds to finding the stationary points of the
            coupled-cluster energy by
            \begin{align}
                \left.
                \dpd[]{E(g, \vfg{\clustamp})}{\clustamp_{\mu}}
                \right\rvert_{\vfg{\clustamp} = \vfg{\clustamp}^{*}}
                = 0,
                \label{eq:cc-variational-condition}
            \end{align}
            for every $g$.
            %As the wave function in fact isn't variational, this means that
            %\begin{align}
            %    \dod[]{E(g)}{g}
            %    &=
            %    \left.
            %    \dod[]{E[g, \clustamp_{\mu}]}{g}
            %    \right\rvert_{\clustamp_{\mu} = \clustamp^{*}_{\mu}}
            %    =
            %    \left.
            %    \dpd[]{E[g, \clustamp_{\mu}]}{g}
            %    \right\rvert_{\clustamp_{\mu} = \clustamp^{*}_{\mu}}
            %    +
            %    \left.
            %    \dpd[]{E[g, \clustamp_{\mu}]}{\clustamp_{\nu}}
            %    \right\rvert_{\clustamp_{\mu} = \clustamp^{*}_{\mu}}
            %    \left.
            %    \dpd[]{\clustamp_{\nu}}{g}
            %    \right\rvert_{\clustamp_{\mu} = \clustamp^{*}_{\mu}}
            %    \\
            %    &\neq
            %    \left.
            %    \dpd[]{E[g, \clustamp_{\mu}]}{g}
            %    \right\rvert_{\clustamp_{\mu} = \clustamp^{*}_{\mu}},
            %\end{align}
            %as \autoref{eq:cc-variational-condition} is not satisified.
            In order to satisty \autoref{eq:cc-variational-condition} and the
            Hellmann-Feynman theorem we use Lagrange's method of undetermined
            multipliers to introduce a new set of parameters $\vfg{\clustlamp}$,
            one for every cluster amplitude \cite{helgaker-molecular}.
            We incorporate the original energy function in
            \autoref{eq:cc-perturbed-energy} into a Lagrangian with the set of
            amplitude equations in \autoref{eq:cc-amplitude-set} multiplied with
            the multipliers.
            That is,
            \begin{align}
                L(g, \vfg{\clustamp}, \vfg{\clustlamp})
                &=
                \mel*{\slat}{
                    (\1 + \clustl)
                    \exponential(-\clust(g))
                    \hamil(g)
                    \exponential(\clust(g))
                }{\slat},
                \label{eq:cc-perturbed-lagrangian}
            \end{align}
            where we've introduced the relaxation operator $\clustl$ from the
            Lagrange multipliers along with the string of relaxation operators
            from the excited state $\ket{\slat_{\mu}}$ we project onto in the
            amplitude equations.
            That is,
            \begin{align}
                \clustl
                &\equiv
                \clustlamp_{\mu}\hat{X}^{\dagger}_{\mu}
                = \clustlamp^{i}_{a}\ccr{i}\can{a}
                + \frac{1}{4}\clustlamp^{ij}_{ab}\ccr{i}\can{a}\ccr{j}\can{b}
                + \dots
                \equiv
                \clustl_1 + \clustl_2 + \dots,
            \end{align}
            with the truncation level $\mu$ matching the truncation of the
            cluster operators $\clust$.
            Optimizing \autoref{eq:cc-perturbed-lagrangian} with respect to the
            parameters $\vfg{\clustamp}$ and $\vfg{\clustlamp}$ we get equations
            which lets us find the stationary points $\vfg{\clustamp}^{*}$ and
            $\vfg{\clustlamp}^{*}$,
            \begin{gather}
                \dpd[]{L(g, \vfg{\clustamp}, \vfg{\clustlamp})}
                {\clustlamp_{\mu}}
                = \mel*{\slat_{\mu}}{
                    \exponential(-\clust(g))
                    \hamil(g)
                    \exponential(\clust(g))
                }{\slat}
                = 0,
                \\
                \dpd[]{L(g, \vfg{\clustamp}, \vfg{\clustlamp})}
                {\clustamp_{\mu}}
                = \mel*{\slat}{
                    (\1 + \clustl)
                    \exponential(-\clust(g))
                    \com{\hamil(g)}{\hat{X}_{\mu}}
                    \exponential(\clust(g))
                }{\slat}
                = 0,
            \end{gather}
            where we note that the first equation in the absence of the
            perturbation corresponds to the projected amplitude equations in
            \autoref{eq:cc_amp_sim}.
            The second equation yields the solution to the Lagrange multipliers.
            This equation is slightly more involved as the extra set of
            amplitudes generates an increased amount of contractions to be
            solved.
            We can use Wick's generalized theorem in order to find equations for
            the Lagrange multipliers.
            In the abscence of the perturbation we write the coupled-cluster
            Lagrangian as
            \begin{align}
                L(\vfg{\clustamp}, \vfg{\clustlamp})
                &= \mel*{\slat}{
                    (\1 + \clustl)
                    \exponential(-\clust)
                    \hamil
                    \exponential(\clust)
                }{\slat}
                \equiv \mel*{\tilde{\Psi}}{\hamil}{\Psi}
                = E(\vfg{\clustamp}, \vfg{\clustlamp}),
                \label{eq:cc-energy-functional}
            \end{align}
            which we interpret as the variational coupled-cluster energy
            function.
            Here we've denoted the right- and left-hand coupled-cluster wave
            functions by
            \begin{gather}
                \ket*{\Psi} = \exponential(\clust)\ket{\slat},
                \label{eq:cc-ket}
                \\
                \bra*{\tilde{\Psi}} = \bra*{\slat}(\1 + \clustl)
                \exponential(-\clust).
                \label{eq:cc-bra}
            \end{gather}
            The stationary points of the cluster amplitudes and the Lagrange
            multipliers are then
            \begin{gather}
                \dpd[]{L(\vfg{\clustamp}, \vfg{\clustlamp})}
                {\clustlamp_{\mu}}
                = \mel*{\slat_{\mu}}{
                    \exponential(-\clust)
                    \hamil
                    \exponential(\clust)
                }{\slat}
                = 0,
                \label{eq:cc-lagrangian-tau}
                \\
                \dpd[]{L(\vfg{\clustamp}, \vfg{\clustlamp})}
                {\clustamp_{\mu}}
                = \mel*{\slat}{
                    (\1 + \clustl)
                    \exponential(-\clust)
                    \com{\hamil}{\hat{X}_{\mu}}
                    \exponential(\clust)
                }{\slat}
                \label{eq:cc-lagrangian-lambda}
                = 0.
            \end{gather}
            The introduction of the coupled-cluster wave functions in
            \autoref{eq:cc-ket} and \autoref{eq:cc-bra} motivates the
            coupled-cluster density operator \cite{kvaal2013variational}
            \begin{align}
                \density
                &= \dyad*{\Psi}{\tilde{\Psi}}
                = e^{\clust}\dyad{\slat}{\slat}\para{\1 + \clustl}e^{\clust}.
                \label{eq:cc-density-operator}
            \end{align}
            Using this definition for the coupled-cluster density operator we
            have that
            \begin{align}
                \tr(\density)
                &= \tr{\dyad*{\Psi}{\tilde{\Psi}}}
                = \bra*{\tilde{\Psi}}\ket{\Psi}
                \bra*{\tilde{\Psi}}\ket{\Psi}
                = 1,
            \end{align}
            which shows that the probabilities sum up to unity.
            To see this, observe that the exponentiated cluster operators cancel
            each other and the $\clustl$-operator destroys the reference.
            We are thus left left with the overlap between the normalized
            reference states.
            Furthermore, we see that
            \begin{align}
                \density^2
                = \density\density
                = \ket{\Psi}\braket*{\tilde{\Psi}}{\Psi}
                \bra*{\tilde{\Psi}}
                = \dyad*{\Psi}{\tilde{\Psi}}
                = \density,
            \end{align}
            making the density operator idempotent.
            This also means that the coupled-cluster wave function is a pure
            state wave function.
            However, not surprisingly, we have that
            \begin{align}
                \density \neq \density^{\dagger}.
            \end{align}
            The observation that the coupled-cluster ansatz with the Lagrange
            multipliers describe a pure, non-Hermitian, state leads to the
            interpretation of the coupled-cluster wave function using the
            \emph{bi-variational principle} \cite{arponen1983311, lowdin-bi,
            kvaal2012ab, kvaal2013variational}
            We will expand on this notion when we discuss the orbital-adaptive
            time-dependent coupled-cluster method later.

        \subsection{Density matrices}
            \label{subsec:cc-density-matrices}
            The variational energy function in \autoref{eq:cc-energy-functional}
            can be written in terms of the one- and two-body density matrices,
            \begin{align}
                E
                &= \mel*{\tilde{\Psi}}{\hamil}{\Psi}
                = \oneten^{p}_{q}
                \mel*{\tilde{\Psi}}{\ccr{p}\can{q}}{\Psi}
                + \frac{1}{4}
                \twoten^{pq}_{rs}
                \mel*{\tilde{\Psi}}{\ccr{p}\ccr{q}\can{s}\can{r}}{\Psi}
                \\
                &=
                \oneten^{p}_{q} \densityten^{q}_{p}
                + \frac{1}{4}
                \twoten^{pq}_{rs}
                \densityten^{rs}_{pq},
            \end{align}
            where the density matrices are treated as first order properties of
            the energy function.
            Defining the coupled-cluster one- and two-body density matrices by
            \begin{align}
                \densityten^{q}_{p}
                &\equiv \mel*{\tilde{\Psi}}{\ccr{p}\can{q}}{\Psi},
                \label{eq:cc-one-body-density}
                \\
                \densityten^{rs}_{pq}
                &\equiv
                \mel*{\tilde{\Psi}}{
                    \ccr{p}\ccr{q}\can{s}\can{r}
                }{\Psi},
                \label{eq:cc-two-body-density}
            \end{align}
            we recognize the ability to compute expectation values of one- and
            two-body operators from \autoref{sec:mb-density-matrices}.
            As the cluster amplitudes are now found using a variational
            technique, these expectation values will conform to the
            Hellmann-Feynman theorem.

        \subsection{The non-orthogonal coupled-cluster method}
            \label{subsec:nocc}
            Including explicit orbital rotations in the coupled-cluster
            Lagrangian is a technique to make the coupled-cluster method gauge
            invariant \cite{gauge-invariant-thomas, gauge-invariant-thomas-2}
            and have it satisfy Ehrenfest's theorem \cite{rolf-nocc}.
            There exists several methods that include orbital rotations in the
            coupled-cluster formalism, such as, the orbital optimized coupled
            cluster method, and the Brueckner coupled-cluster method.
            % TODO: Add citations from Rolf to these?
            These methods do suffer from certain defects, where the former does
            not converge to the full configuration interaction method in the
            untruncated limit, and the latter yields unphysical results in the
            second order poles in the response function \cite{rolf-nocc}.
            However, the method we will discuss is the non-orthogonal coupled
            cluster method \cite{gauge-invariant-thomas-2} in which we relax the
            requirement that the orbital transformations should be unitary
            \cite{rolf-nocc}.
            This method has been demonstrated to converge to the full
            configuration interaction method in the untruncated limit
            \cite{rolf-nocc}.
            This yields biorthogonal orbitals leading to a bi-variational
            Lagrangian \cite{kvaal2013variational} in a formulation resembling
            that of \citeauthor{arponen1983311} \cite{arponen1983311}.

            In the non-orthogonal method we introduce a singles operators
            $\nocc$ given by
            \begin{align}
                \nocc = \noccten_{pq}\biccr{p}\bican{q},
            \end{align}
            where $\biccr{p}$ and $\bican{p}$ are creation and annihilation
            operators two biorthogonal Hilbert spaces.
            They are defined in terms of the original static creation and
            annihilation operators via the transformation
            \cite{helgaker-molecular, rolf-nocc}
            \begin{gather}
                \biccr{p}
                = \exp(-\nocc)\ccr{p}\exp(\nocc),
                \label{eq:biccr}
                \\
                \bican{p}
                = \exp(-\nocc)\can{p}\exp(\nocc).
                \label{eq:bican}
            \end{gather}
            The approach taken in the orthogonal coupled-cluster method is to
            require that $\nocc$ is antihermitian as this yields unitary
            transformations for the new creation and annihilation operators, and
            we have that the operators are the adjoint of one another.
            However, in the non-orthogonal coupled-cluster method, $\nocc$ is no
            longer antihermitian and we have in general
            \begin{align}
                \bican{p}^{\dagger} \neq \biccr{p},
            \end{align}
            as can be seen by taking the Hermitian conjugate of
            \autoref{eq:biccr} and see that this is not the same as
            \autoref{eq:bican} due to $\nocc$ not being antihermitian.
            Luckily even though the operators not being Hermitian conjugates of
            one another, they still retain their anticommutation relation
            \cite{balian1969, lowdin-bi}
            \begin{align}
                \acom{\bican{p}}{\biccr{q}} = \delta_{pq},
            \end{align}
            and we can still utilize Wick's theorem \cite{rolf-nocc,
            kvaal2012ab}, as long as the original atomic orbital basis is
            orthonormal.
            From the orthogonal coupled-cluster method we have that
            $\noccten_{aa} = \noccten_{ii} = 0$ which lets us separate the
            orbital rotation operator into two blocks.
            This property transfers to the non-orthogonal method and we have
            \begin{align}
                \nocc
                = \noccten_{ai} \biccr{a}\bican{i}
                + \noccten_{ia} \biccr{i}\bican{a}
                \equiv
                \nocc^{u} + \nocc^{d},
            \end{align}
            where we've introduced the labels $u$ and $d$ for up and down
            respectively \cite{rolf-nocc}.
            Constructing the bi-variational non-orthogonal coupled-cluster
            Lagrangian
            \begin{align}
                L(\vfg{\clustamp}, \vfg{\clustlamp}, \noccten^{u},
                \noccten^{d})
                = \mel*{\tilde{\slat}}{
                    (\1 + \clustl)
                    \exponential(-\clust)
                    \exponential(-\nocc)
                    \hamil
                    \exponential(\nocc)
                    \exponential(\clust)
                }{\slat},
                \label{eq:nocc-lagrangian}
            \end{align}
            where we've introduced the left reference state
            \begin{align}
                \bra*{\tilde{\slat}}
                = \bra*{\tilde{\phi}_1, \dots, \tilde{\phi}_N}
                = \bra{\vac}\bican{1}\dots\bican{N},
            \end{align}
            as the biorthogonal dual state of the right reference state defined
            in the normal way.
            It is interesting to compare the exponential operator with $\nocc$
            to Thouless theorem from \autoref{theorem:thouless}.
            Thouless theorem can be compared to the singles amplitudes in
            regular coupled-cluster theory as the orbital rotations are
            formulated in a completely analogous way with a singles excitation
            operator.
            In the non-orthogonal formulation $\exp[\nocc]$ contains both an
            excitation and an annihilation operator.
            This makes the method include the singles ampitudes, and open up for
            more degrees of freedom for the relaxation operator.
            By including the relaxation operator the method opens up for two
            sets of spin-orbitals unlike Thouless theorem which only rotates a
            single set of spin-orbitals \cite{helgaker-molecular}.
            % TODO: Verify this.
            In fact, by incorporating the singles excitations as an orbital
            transformation of the Hamiltonian, we can ignore the $\clust_1$
            contribution to the coupled-cluster amplitudes.
            This also applies for the non-orthogonal formulation and we have the
            doubles excitations with $\clust_2$ and $\clustl_2$ will be equally
            good as the singles-and-doubles excitations.

            Returning to the process of finding the stationary conditions of the
            non-orthogonal coupled-cluster Lagrangian, we require that
            $\bra*{\tilde{\Psi}}$ and $\ket{\Psi}$ must satisfy the standard
            coupled-cluster equations \cite{rolf-nocc}.
            As $\nocc$ does not commute with $\clust$ and $\clustl$ we express
            the amplitude equations in the optimized basis where $\nocc = 0$
            \cite{ugur-occ, rolf-nocc}.
            The stationary conditions for $\clustamp_{\mu}$ and
            $\clustlamp_{\mu}$ are thus the same as in
            \autoref{eq:cc-lagrangian-tau} and
            \autoref{eq:cc-lagrangian-lambda}.
            For the orbital rotations we compute the right-hand sides and set
            $\nocc = 0$ afterwards \cite{ugur-occ}.
            This yields the stationary conditions
            \begin{gather}
                \left.
                \dpd{
                    L(\vfg{\clustamp}, \vfg{\clustlamp},
                    \noccten^{u}, \noccten^{d})
                }{
                    {\noccten^{u}}
                }
                \right\rvert_{\noccten = 0}
                = \mel*{\tilde{\slat}}{
                    (\1 + \clustl)
                    \exponential(-\clust)
                    \com{\hamil}{\biccr{a}\bican{i}}
                    \exponential(\clust)
                }{\slat},
                \label{eq:nocc-kappa-up-rhs}
                \\
                \left.
                \dpd{
                    L(\vfg{\clustamp}, \vfg{\clustlamp},
                    \noccten^{u}, \noccten^{d})
                }{
                    {\noccten^{d}}
                }
                \right\rvert_{\noccten = 0}
                = \mel*{\tilde{\slat}}{
                    (\1 + \clustl)
                    \exponential(-\clust)
                    \com{\hamil}{\biccr{i}\bican{a}}
                    \exponential(\clust)
                }{\slat},
                \label{eq:nocc-kappa-down-rhs}
            \end{gather}
            where it is important to note that all creation and annihilation
            operators in the cluster amplitudes and the Hamiltonian are the
            biorthogonal operators.

    \section{Time-dependent coupled-cluster theory}
        Having explored the time-independent coupled-cluster theory, the
        question now arises how we can move to the time-dependent situation.
        The formulation of the time-dependent coupled-cluster theory initially
        started with a non-variational formulation prior to the introduction of
        the Lagrange multipliers.
        In 1978 \citeauthor{tdcc-hoodbhoy} \cite{tdcc-hoodbhoy, tdcc-hoodbhoy-2}
        proposed the time-dependent formulation of coupled-cluster theory by
        inserting the coupled-cluster wave function into the time-dependent
        Schrödinger equation,
        \begin{align}
            i\hslash
            \exponential(-\clust(t))
            \partial_t
            \exponential(\clust(t))
            \ket{\slat}
            = \exponential(-\clust(t))
            \hamil(t)
            \exponential(\clust(t))
            \ket{\slat},
        \end{align}
        which in many ways is the ``natural'' approach to take as this is the
        analog to the projected time-independent Schrödinger equation for the
        coupled-cluster wave function.
        The truncation level of this approach can be done in a similar manner as
        in the time-independent case, i.e., by projecting onto an excited state
        $\ket*{\slat_{\mu}}$, and the right-hand sides lead to the familiar
        projected amplitude equations in \autoref{eq:cc_amp_sim} and energy
        equation \autoref{eq:cc_energy_sim}.
        The time-dependence is kept in the wave function by the amplitudes and
        in the Hamiltonian in accordance with the Schrödinger picture.
        As the cluster amplitudes commute, we have that the time-derivative of
        coupled-cluster wave function is given by
        \begin{align}
            i\hslash\partial_t
            \exponential(\clust(t))
            \ket{\slat}
            = i\hslash \para{
                \partial_t\clustamp_{\mu}(t)
            }
            \hat{X}_{\mu}
            \exponential(\clust(t))
            \ket{\slat}.
        \end{align}
        Left-projecting with an excited state we find that
        \begin{align}
            i\hslash
            \mel*{\slat_{\mu}}{
                \exponential(-\clust(t))
                \partial_t
                \exponential(\clust(t))
            }{\slat}
            =
            i\hslash
            \partial_t
            \clustamp_{\mu}(t).
            \label{eq:td-clustamp}
        \end{align}
        This means that the time-evolution of the amplitudes is given by the
        amplitude equations in time as their right-hand sides.
        The time-dependent formulation described above suffers in that it is not
        variational \cite{tdcc-hoodbhoy, tdcc-hoodbhoy-2, tdcc-huber} and the
        process of computing observables in time suffers from the same problems
        as in the non-variational time-independent case.

        To fix this we reuse the variational coupled-cluster Lagrangian in the
        time-dependent variational principle yielding the time-dependent
        coupled-cluster action functional \cite{kvaal2013variational}
        \begin{align}
            S[\Psi, \tilde{\Psi}]
            = \int\dd t
            \mel*{\tilde{\Psi}(t)}{
                \para{
                    i\hslash\partial_t
                    - \hamil(t)
                }
            }{\Psi(t)}
            = \int\dd t
            L[\Psi, \tilde{\Psi}],
            \label{eq:cc-action-functional}
        \end{align}
        where the coupled-cluster Lagrangian now includes the time-derivative.
        We can write the Lagrangian as
        \begin{align}
            L[\Psi, \tilde{\Psi}]
            = i\hslash \clustlamp_{\mu}\partial_t\clustamp_{\mu}
            - E[\Psi, \tilde{\Psi}],
        \end{align}
        where $E[\Psi, \tilde{\Psi}]$ is the energy functional formulation of
        the coupled-cluster energy function from
        \autoref{eq:cc-energy-functional}.
        Finding the stationary conditions for the action functional now consists
        of solving the two sets of equations
        \begin{align}
            \dpd{L[\Psi, \tilde{\Psi}]}{\clustlamp_{\mu}} = 0,
            \qquad
            \dpd{L[\Psi, \tilde{\Psi}]}{\clustamp_{\mu}} = 0.
        \end{align}
        The former of these two equations yield the time-evolution of the
        $\clustamp_{\mu}$-amplitudes whereas the latter yield the
        time-evolution for the $\clustlamp_{\mu}$-amplitudes.
        For the former equation we find
        \begin{gather}
            \dpd{L[\Psi, \tilde{\Psi}]}{\clustlamp_{\mu}}
            =
            i\hslash\partial_t\clustamp_{\mu}
            - \dpd{E[\Psi, \tilde{\Psi}]}{\clustlamp_{\mu}}
            = 0
            \\
            \implies
            i\hslash\partial_t\clustamp_{\mu}
            =
            \dpd{E[\Psi, \tilde{\Psi}]}{\clustlamp_{\mu}}
            =
            \mel*{\slat_{\mu}}{
                \exponential(-\clust(t))
                \hamil(t)
                \exponential(\clust(t))
            }{\slat},
            \label{eq:time-evolution-tau}
        \end{gather}
        where the excited determinant comes from the derivative of the
        $\clustl(t)$ with respect to $\clustlamp_{\mu}$ yielding a relaxation
        operator $\hat{X}^{\dagger}_{\mu}$.
        We see that \autoref{eq:time-evolution-tau} is the familiar projected
        amplitude equations in the non-variational formulation of the
        coupled-cluster wave function similar to the method introduced by
        \citeauthor{tdcc-hoodbhoy} \cite{tdcc-hoodbhoy, tdcc-hoodbhoy-2}.
        The second stationary condition gives
        \begin{gather}
            \dpd{L[\Psi, \tilde{\Psi}]}{\clustamp_{\mu}}
            = i\hslash\dpd{}{\clustamp_{\mu}}\brak{
                \partial_t\para{
                    \clustlamp_{\nu}
                    \clustamp_{\nu}
                }
                - \para{\partial_t\clustlamp_{\nu}}
                \clustamp_{\nu}
            }
            - \dpd{E[\Psi, \tilde{\Psi}]}{\clustamp_{\mu}}
            = 0 \\
            \implies
            -i\hslash\partial_t\clustlamp_{\mu}
            =
            \dpd{E[\Psi, \tilde{\Psi}]}{\clustamp_{\mu}}
            =
            \mel*{\tilde{\Psi}(t)}{
                \com{\hamil(t)}{\hat{X}_{\mu}}
            }{\Psi(t)},
            \label{eq:time-evolution-lambda}
        \end{gather}
        where we in the product rule for the derivatives of the amplitudes used
        that the boundary term must disappear in accordance with the
        time-dependent variational principle.
        The commutator comes from the left and right exponential cluster
        operators.
        Now, \autoref{eq:time-evolution-tau} and
        \autoref{eq:time-evolution-lambda} yields a variational formulation of
        the time-dependent coupled-cluster methods where $\mu$ decides the
        truncation level.
        For \autoref{eq:time-evolution-tau} we can reuse the regular
        $\clust$-amplitudes as the right-hand side, and for
        \autoref{eq:time-evolution-lambda} we use the right-hand sides for the
        Lagrange multipliers from the time-independent situation.

    \section{Orbital-adaptive time-dependent coupled-cluster}
        \label{sec:oatdcc}
        In the article titled \citetitle{kvaal2012ab}
        \citeauthor{kvaal2012ab} \cite{kvaal2012ab} employs the
        bi-variational principle as discussed by \citeauthor{arponen1983311}
        \cite{arponen1983311} to create a new family of coupled-cluster
        methods dubbed ``orbital-adaptive time-dependent
        coupled-cluster''.
        They provide an approximation to the multi-configurational
        time-dependent Hartree-Fock methods, which themselves are
        approximations to the time-dependent full configuration interaction
        method.
        Similar to the non-orthogonal coupled-cluster method discussed in
        \autoref{subsec:nocc} the orbital-adaptive method opens up for
        orbital rotations along with the coupled-cluster amplitudes
        $\clust(t)$ and $\clustl(t)$, with the orbitals subject to the
        constraint that they are biorthogonal in time, that is,
        \begin{align}
            \braket*{\tilde{\phi}_p(t)}{\phi_q(t)} = \delta_{pq}
            \implies
            \acom{\bican{p}}{\biccr{q}}
            = \delta_{pq},
        \end{align}
        where we denote the bra states by tilde to distinguish it from the
        adjoint state.
        In general we have $\ket{\phi_p(t)}^{\dagger} \neq
        \bra*{\tilde{\phi}_p(t)}$ as the two orbitals are defined in two
        separate Hilbert spaces \cite{kvaal2012ab}.
        This extends through to the Slater determinants being biorthogonal
        in time, that is
        \begin{align}
            \braket*{\tilde{\slat}_{\mu}(t)}{\slat_{\nu}(t)}
            = \delta_{\mu\nu}.
        \end{align}

        In order to avoid two sets of second quantized operators, we
        transform from the atomic orbital basis in a similar fashion as done
        in the non-orthogonal coupled-cluster method.
        However, the formulation of the orbital-adaptive method does not
        contain an explicit single-particle operator as in the
        non-orthogonal method.
        Therefore, instead of transforming via \autoref{eq:biccr} and
        \autoref{eq:bican}, we use the projection operator of the
        biorthogonal Slater determinant basis \cite{kvaal2012ab}
        \begin{align}
            \Pi(t)
            = \dyad*{\slat_{\mu}(t)}{\tilde{\slat}_{\mu}(t)}.
        \end{align}
        This lets us transform the Hamiltonian to the biorthogonal basis by
        \begin{align}
            \mel*{\tilde{\Psi}(t)}{\hamil(t)}{\Psi(t)}
            =
            \mel*{\tilde{\Psi}(t)}{\Pi(t)\hamil(t)\Pi(t)}{\Psi(t)},
        \end{align}
        where we stress that $\Pi(t) \neq \Pi^{\dagger}(t)$.  This
        corresponds to replacing all static second quantized operators by
        the biorthogonal operators.
        Instead of going through a description of the time-dependent
        bivariational principle \cite{kvaal2012ab, arponen1983311}, we
        define the bivariational form of the coupled-cluster wave functions
        to be
        \begin{gather}
            \ket*{\Psi(t)} = \exponential(\clust(t))\ket*{\slat(t)}, \\
            \bra*{\tilde{\Psi}(t)}
            = \bra*{\tilde{\slat}(t)}(\1 + \clustl(t))
            \exponential(-\clust(t)),
        \end{gather}
        where we impose the normalization that
        \begin{align}
            \braket*{\tilde{\Psi}(t)}{\Psi(t)} = 1.
        \end{align}
        From here on out we remove the explicit time-dependence on the
        states and operators to avoid too much clutter.
        We now define the action-like functional\footnote{%
            This must not be mistaken for an actual action functional
            \cite{arponen1983311}.
        } by \cite{kvaal2012ab, arponen1983311}
        \begin{align}
            S[\vfg{\clustamp}, \vfg{\clustlamp}, \slat, \tilde{\slat}]
            =
            \int \dd t
            \mel*{\tilde{\Psi}}{
                \para{
                    i\hslash\partial_t - \hamil
                }
            }{\Psi}
            =
            \int \dd t
            L[\vfg{\clustamp}, \vfg{\clustlamp}, \slat, \tilde{\slat}],
            \label{eq:oacc-action-functional}
        \end{align}
        which closely resembles the coupled-cluster action functional from
        \autoref{eq:cc-action-functional} except that
        \autoref{eq:oacc-action-functional} opens up for independent
        variations over the orbitals.
        To go from here we evaluate the Lagrangian functional.
        The time-derivative yields
        \begin{align}
            \mel*{\tilde{\Psi}}{\partial_t}{\Psi}
            =
            \mel*{\tilde{\Psi}}{
                \para{
                    \partial_t \clustamp_{\mu}\hat{X}_{\mu}
                    +
                    \hat{\eta}
                }
            }{\Psi}
            =
            \clustlamp_{\mu}\partial_t\clustamp_{\mu}
            + \mel*{\tilde{\Psi}}{\hat{\eta}}{\Psi},
        \end{align}
        where the first term comes from the time-derivative of the cluster
        amplitudes as seen in \autoref{eq:td-clustamp}, and the second term
        comes from the time-derivative of the orbitals.
        This operator is given by
        \begin{align}
            \hat{\eta}
            = \mel*{\tilde{\phi}_p}{\partial_t}{\phi_q}
            \biccr{p}\bican{q}
            = \eta^{p}_{q}\biccr{p}\bican{q}
        \end{align}
        which is a one-body operator similar to the projected one-body
        Hamiltonian.
        We define the time-dependent energy functional by
        \begin{align}
            \mathcal{E}[
                \vfg{\clustamp}, \vfg{\clustlamp}, \slat, \tilde{\slat}
            ]
            \equiv \mel*{\tilde{\Psi}}{
                \para{
                    \hamil - i\hslash\hat{\eta}
                }
            }{\Psi},
        \end{align}
        which should be compared with \autoref{eq:cc-energy-functional}.
        We can now write the Lagrangian functional as
        \begin{align}
            L[\vfg{\clustamp}, \vfg{\clustlamp}, \slat, \tilde{\slat}]
            =
            i\hslash\clustlamp_{\mu}\partial_t\clustamp_{\mu}
            -
            \mathcal{E}[
                \vfg{\clustamp}, \vfg{\clustlamp}, \slat, \tilde{\slat}
            ].
            \label{eq:oatdcc-lagrangian}
        \end{align}
        We are now interested in finding stationary conditions for the
        action-like functional by performing variations over all dependent
        variables in the Lagrangian functional.
        The stationary conditions for the $\clustl$-amplitudes are given by
        \begin{align}
            i\hslash\partial_t \clustamp_{\mu}
            = \dpd{\mathcal{E}[
                \clustamp_{\nu}, \clustlamp_{\nu}
                \slat, \tilde{\slat}
            ]}{\clustlamp_{\mu}}
            =
            \mel*{\tilde{\slat}_{\mu}}{
                \exponential(-\clust)\brak{
                    \hamil
                    - i\hslash\hat{\eta}
                }\exponential(\clust)
            }{\slat}.
        \end{align}
        Comparing this to \autoref{eq:time-evolution-tau} we see that the
        only difference is the inclusion of the one-body operator
        $i\hslash\hat{\eta}$.
        The stationary conditions for the $\clust$-amplitudes then yields
        \begin{align}
            -i\hslash\partial_t\clustlamp_{\mu}
            &=
            \dpd{
                \mathcal{E}[
                    \vfg{\clustamp}, \vfg{\clustlamp},
                    \slat, \tilde{\slat}
                ]
            }{\clustamp_{\mu}}
            =
            \mel*{\tilde{\Psi}}{
                \com{
                    \hamil
                    - i\hslash\hat{\eta}
                }{
                    \hat{X}_{\mu}
                }
            }{\Psi},
        \end{align}
        which is also comparable to the time-dependent coupled-cluster
        method with stationary orbitals in
        \autoref{eq:time-evolution-lambda}.

        \subsection{Orbital rotations}
            For the variation of the orbitals we start by noting that the four
            sets of parameters $\clust$, $\clustl$, $\slat$, and $\tilde{\slat}$
            are overdetermined \cite{kvaal2012ab}.
            This means that we can choose a gauge condition for the orbital
            rotations in order to remove parametric redundancies.
            Similarly to the non-orthogonal coupled-cluster method we choose
            \begin{align}
                \hat{\eta}
                = \eta^{i}_{a}\biccr{i}\bican{a}
                + \eta^{a}_{i}\biccr{a}\bican{i},
                \label{eq:eta-gauge}
            \end{align}
            that is, the occupied-occupied and virtual-virtual rotations are set
            to zero.
            Furthermore, the orbital rotations makes the singles amplitudes
            $\clustamp^{a}_{i}$ redundant, and to avoid having more parameters
            in $\bra*{\tilde{\Psi}}$ than in $\ket*{\Psi}$ we set
            $\clustlamp^{i}_{a}$ to zero as well \cite{kvaal2012ab}.
            We now perform variations over the orbitals
            \begin{align}
                \ket*{\phi_p'} = \ket*{\phi_p} + \ket*{\delta\phi_p},
            \end{align}
            and similarly for the dual single-particle state
            $\bra*{\tilde{\phi}_p}$.
            We restrict our attention to the variation $\ket*{\delta\phi_p}$ and
            ignore the full state $\ket*{\phi_p'}$.
            Now, we note that
            \begin{align}
                \ket*{\delta\phi_p}
                = \hat{P}\ket*{\delta\phi_p}
                + \hat{Q}\ket*{\delta\phi_p},
            \end{align}
            where the single-particle projection operator $\hat{P}$ is given by
            \begin{align}
                \hat{P}
                = \dyad*{\tilde{\phi}_p}{\phi_p}
                \implies
                \hat{Q} = \1 - \hat{P}.
            \end{align}
            Due to the gauge condition imposed on $\eta^{p}_{q}$ we know that we
            only need to concern ourselves with variations over occupied-virtual
            and virtual-occupied pairs.
            We can thus construct the two sets of orbital variations
            \begin{gather}
                \ket{\delta\phi_i}
                = \epsilon^{a}_{i}\ket{\phi_a}
                + \hat{Q}\ket{\delta\phi_i},
                \\
                \ket{\delta\phi_a}
                = \epsilon^{a}_{i}\ket{\phi_i}
                + \hat{Q}\ket{\delta\phi_a},
            \end{gather}
            where $\epsilon^{a}_{i}$ is an arbitrary function of time.
            Furthermore, the functions $\hat{Q}\ket{\delta\phi_p}$ are also
            arbitrary and independent of $\epsilon^{a}_{i}$.
            We denote the equations from the variations over the first and
            second term as $P$-space equations and $Q$-space equations
            respectively.
            We can then perform the variations over the $P$-space equations
            independently of the variations over the $Q$-space equations.

        \subsection{$P$-space equations}
            We start by defining the variations over the orbitals to be
            \begin{align}
                \ket{\delta\phi_i} = \epsilon\ket{\phi_a},
                \qquad
                \ket{\delta\phi_a} = \epsilon\ket{\phi_i},
            \end{align}
            where $\epsilon(t)$ is now an arbitrary function of time that is
            independent of the labels $i$ and $a$.
            Due to the biorthogonality of the orbitals, we can find the
            corresponding variations over the dual states
            \begin{gather}
                0 = \delta\para{\braket*{\tilde{\phi}_a}{\phi_i}}
                = \braket*{\delta\tilde{\phi}_a}{\phi_i}
                + \braket*{\tilde{\phi}_a}{\delta\phi_i}
                \\
                \implies
                \braket*{\delta\tilde{\phi}_a}{\phi_i}
                = -\braket*{\tilde{\phi}_a}{\delta\phi_i}
                = -\epsilon,
            \end{gather}
            which yields
            \begin{gather}
                \bra*{\delta\tilde{\phi}_i}
                = -\epsilon\bra*{\tilde{\phi}_a},
                \qquad
                \bra*{\delta\tilde{\phi}_a}
                = -\epsilon\bra*{\tilde{\phi}_i}.
            \end{gather}
            By inspecting the Lagrangian functional from
            \autoref{eq:oatdcc-lagrangian} we note that only the energy
            functional depends explicitly on the orbitals.
            In terms of the biorthogonal second quantized operators we can
            describe the variation over an orbital in an arbitrary state by the
            action
            \begin{gather}
                \ket*{\delta\Psi}
                = \epsilon\brak{
                    \biccr{a}\bican{i}
                    + \biccr{i}\bican{a}
                }\ket*{\Psi},
                \\
                \bra*{\delta\tilde{\Psi}}
                = -\epsilon\bra*{\tilde{\Psi}}\brak{
                    \biccr{a}\bican{i}
                    + \biccr{i}\bican{a}
                },
            \end{gather}
            as the variation is performed by replacing a virtual orbital by an
            occupied orbital and vice versa.
            Now, the Hamiltonian and the cluster amplitudes do not explictly
            depend on the orbitals and will therefore be unaffected by the
            orbital variation.
            However, $\hat{\eta}$ will be varied.
            We see the effect of this variation by considering the matrix
            elements $\eta^{p}_{q}$ prior to imposing the gauge condition for
            the orbital rotations.
            This yields
            \begin{align}
                \delta\eta^{p}_{q}
                &= \mel*{\delta\tilde{\phi}_p}{
                    \partial_t
                }{\phi_q}
                + \mel*{\tilde{\phi}_p}{
                    \partial_t
                }{\delta\phi_q}
                \\
                &=
                -\epsilon\brak{
                    \delta_{pa}\bra*{\tilde{\phi}_i}
                    + \delta_{pi}\bra*{\tilde{\phi}_a}
                }\partial_t\ket*{\phi_q}
                \nonumber \\
                &\qquad
                + \bra*{\tilde{\phi}_p}\partial_t\brak{
                    \epsilon \delta_{qa}\ket*{\phi_i}
                    + \epsilon \delta_{qi}\ket*{\phi_a}
                },
            \end{align}
            where the Kronecker-Deltas run over free indices and should be
            treated more as running over occupied or virtual indices.
            Now, due to the gauge conditions, we get
            \begin{align}
                \delta\eta^{p}_{q}
                &=
                - \epsilon\brak{
                    \delta_{pa}\delta_{qb}\eta^{i}_{b}
                    + \delta_{pi}\delta_{qj}\eta^{a}_{j}
                }
                + \partial_t \epsilon \brak{
                    \delta_{qa}\delta_{pi}
                    + \delta_{qi}\delta_{pa}
                }
                \nonumber \\
                &\qquad
                + \epsilon\brak{
                    \delta_{qa}\delta_{pb}\eta^{b}_{i}
                    + \delta_{qi}\delta_{pj}\eta^{a}_{j}
                }
                \\
                &=
                \partial_t \epsilon \brak{
                    \delta_{qa}\delta_{pi}
                    + \delta_{qi}\delta_{pa}
                },
            \end{align}
            where the two other terms cancel as can be seen by shuffling the
            free indices.
            The variation over the full orbital operator is thus given by
            \begin{align}
                \delta\hat{\eta}
                = \delta\eta^{p}_{q}\biccr{p}\bican{q}
                = \partial_t\epsilon\brak{
                    \biccr{a}\bican{i}
                    + \biccr{i}\bican{a}
                }.
            \end{align}
            We find the expectation value over the variation of $\hat{\eta}$ to
            be
            \begin{align}
                \mel*{\tilde{\Psi}}{\delta\hat{\eta}}{\Psi}
                = \para{\partial_t\epsilon}
                \brak{
                    \densityten^{i}_{a}
                    + \densityten^{a}_{i}
                }
                =
                -\epsilon\brak{
                    \partial_t\densityten^{i}_{a}
                    + \partial_t\densityten^{a}_{i}
                },
            \end{align}
            where the boundary term disappears by definition of the variation
            \cite{kvaal2012ab}.
            Now, as we have set $\clustlamp^{i}_{a} = 0$, we have that
            \begin{align}
                \densityten^{i}_{a}
                = \mel*{\tilde{\Psi}}{\biccr{a}\bican{i}}{\Psi}
                = \clustlamp^{i}_{a}
                = 0.
            \end{align}
            This is however not the case for $\densityten^{a}_{i}$ as the
            operator string acts as a relaxtion operator.

            The variation over the occupied orbitals and the virtual orbitals
            can be done independently \cite{kvaal2012ab}.
            This yields two sets of equations for the $P$-space variations.
            We start by looking at the variation over the occupied orbitals in
            the energy functional.
            \begin{align}
                \delta\mathcal{E}
                &=
                \mel*{\delta\tilde{\Psi}}{\hamil - i\hslash\hat{\eta}}{\Psi}
                +
                \mel*{\tilde{\Psi}}{\hamil - i\hslash\hat{\eta}}{\delta\Psi}
                -i\hslash
                \mel*{\tilde{\Psi}}{\delta\hat{\eta}}{\Psi}
                \\
                &=
                \epsilon
                \mel*{\tilde{\Psi}}{
                    \com{
                        \hamil - i\hslash\hat{\eta}
                    }{
                        \biccr{a}\bican{i}
                    }
                }{\Psi}
                + \epsilon i\hslash\partial_t\densityten^{i}_{a}
                =
                \epsilon
                \mel*{\tilde{\Psi}}{
                    \com{
                        \hamil - i\hslash\hat{\eta}
                    }{
                        \biccr{a}\bican{i}
                    }
                }{\Psi}.
                \label{eq:occupied-variation}
            \end{align}
            The variation over the virtual orbitals gives
            \begin{align}
                \delta\mathcal{E}
                &=
                \epsilon
                \mel*{\tilde{\Psi}}{
                    \com{
                        \hamil - i\hslash\hat{\eta}
                    }{
                        \biccr{i}\bican{a}
                    }
                }{\Psi}
                + \epsilon i\hslash\partial_t\densityten^{a}_{i},
                \label{eq:virtual-variation}
            \end{align}
            where the time-derivative of the one-body density matrix is
            retained.
            Recalling that the amplitudes do not depend on the orbital
            variations directly we can compute the variation over the orbitals
            in the action-like functional by
            \begin{align}
                \delta S
                = \int \dd t \delta L
                = \int \dd t \delta\mathcal{E}
                = 0.
            \end{align}
            The requirement that $\delta S$ is zero for all $\epsilon$ implies
            that we can remove the integrand and $\epsilon$ as it factors out.
            This means that we can write \autoref{eq:occupied-variation} and
            \autoref{eq:virtual-variation} on the form
            \begin{gather}
                i\hslash\mel*{\tilde{\Psi}}{
                    \com{
                        \hat{\eta}
                    }{
                        \biccr{a}\bican{i}
                    }
                }{\Psi}
                =
                \mel*{\tilde{\Psi}}{
                    \com{
                        \hamil
                    }{
                        \biccr{a}\bican{i}
                    }
                }{\Psi},
                \\
                i\hslash\mel*{\tilde{\Psi}}{
                    \com{
                        \hat{\eta}
                    }{
                        \biccr{i}\bican{a}
                    }
                }{\Psi}
                =
                \mel*{\tilde{\Psi}}{
                    \com{
                        \hamil
                    }{
                        \biccr{i}\bican{a}
                    }
                }{\Psi}
                + \epsilon i\hslash\partial_t\densityten^{a}_{i}.
            \end{gather}
            Looking at the gauge condition for $\hat{\eta}$ given in
            \autoref{eq:eta-gauge} we see that the only non-zero terms in the
            commutators on the left-hand side becomes
            \begin{gather}
                i\hslash\eta^{j}_{b}\mel*{\tilde{\Psi}}{
                    \com{
                        \biccr{j}\bican{b}
                    }{
                        \biccr{a}\bican{i}
                    }
                }{\Psi}
                = i\hslash\eta^{j}_{b}\brak{
                    \delta_{ba}\densityten^{i}_{j}
                    - \delta_{ij}\densityten^{b}_{a}
                }
                \equiv
                i\hslash\eta^{j}_{b}A^{ib}_{aj},
                \\
                i\hslash\eta^{b}_{j}\mel*{\tilde{\Psi}}{
                    \com{
                        \biccr{b}\bican{j}
                    }{
                        \biccr{i}\bican{a}
                    }
                }{\Psi}
                =
                i\hslash\eta^{b}_{j}\brak{
                    \delta_{ji}\densityten^{a}_{b}
                    - \delta_{ab}\densityten^{j}_{i}
                }
                \equiv
                -i\hslash\eta^{b}_{j}
                A^{ja}_{bi},
            \end{gather}
            where we've defined the coefficient matrix $\vfg{A}$.
            What remains is to evaluate the commutators with the Hamiltonian
            matrix on the right-hand side.
            For brevity we restrict ourselves to only looking at the operator
            strings.
            We have for the one-body Hamiltonian
            \begin{gather}
                \com{\biccr{p}\bican{q}}{\biccr{a}\bican{i}}
                = \delta_{qa}\biccr{p}\bican{i}
                - \delta_{ip}\biccr{a}\bican{q}, \\
                \com{\biccr{p}\bican{q}}{\biccr{i}\bican{a}}
                = \delta_{qi}\biccr{p}\bican{a}
                - \delta_{ap}\biccr{i}\bican{q}.
            \end{gather}
            The expectation value of these operator strings will be made into
            one-body density matrices.
            That is, for the one-body Hamiltonian we get
            \begin{gather}
                \mel*{\tilde{\Psi}}{
                    \com{
                        \onehamil
                    }{
                        \biccr{a}
                        \bican{i}
                    }
                }{\Psi}
                = \oneten^{p}_{a}\densityten^{i}_{p}
                - \oneten^{i}_{q}\densityten^{q}_{a},
                \\
                \mel*{\tilde{\Psi}}{
                    \com{
                        \onehamil
                    }{
                        \biccr{i}
                        \bican{a}
                    }
                }{\Psi}
                =
                \oneten^{p}_{i}\densityten^{a}_{p}
                -
                \oneten^{a}_{q}\densityten^{q}_{i}.
            \end{gather}
            For the two-body operator we get
            \begin{align}
                \com{\biccr{p}\biccr{q}\bican{s}\bican{r}}{\biccr{a}\bican{i}}
                &=
                \delta_{ra}\biccr{p}\biccr{q}\bican{s}\bican{i}
                - \delta_{sa}\biccr{p}\biccr{q}\bican{r}\bican{i}
                \nonumber \\
                &\qquad
                - \delta_{ip}\biccr{a}\biccr{q}\bican{s}\bican{r}
                + \delta_{iq}\biccr{a}\biccr{p}\bican{s}\bican{r},
                \\
                \com{\biccr{p}\biccr{q}\bican{s}\bican{r}}{\biccr{i}\bican{a}}
                &=
                \delta_{ri}\biccr{p}\biccr{q}\bican{s}\bican{a}
                - \delta_{si}\biccr{p}\biccr{q}\bican{r}\bican{a}
                \nonumber \\
                &\qquad
                - \delta_{ap}\biccr{i}\biccr{q}\bican{s}\bican{r}
                + \delta_{aq}\biccr{i}\biccr{p}\bican{s}\bican{r}.
            \end{align}
            Similarly to the one-body Hamiltonain, the expectation value will
            yield two-body density matrices for the operator strings.
            Using the antisymmetry of the two-body matrix elements, we get
            \begin{gather}
                \mel*{\tilde{\Psi}}{
                    \com{
                        \twohamil
                    }{
                        \biccr{a}
                        \bican{i}
                    }
                }{\Psi}
                =
                \half\twoten^{pq}_{is}\densityten^{as}_{pq}
                - \half\twoten^{aq}_{rs}\densityten^{sr}_{iq},
                \\
                \mel*{\tilde{\Psi}}{
                    \com{
                        \twohamil
                    }{
                        \biccr{i}
                        \bican{a}
                    }
                }{\Psi}
                =
                \half
                \twoten^{pq}_{is}\densityten^{sa}_{pq}
                -
                \half
                \twoten^{aq}_{rs}\densityten^{sr}_{iq}.
            \end{gather}
            Putting everything together we get equations for the
            occupied-virtual part of $\hat{\eta}$ and the virtual-occupied
            block.
            The former is given by
            \begin{align}
                i\hslash A^{ib}_{aj}\eta^{j}_{b}
                = \oneten^{p}_{a}\densityten^{i}_{p}
                - \oneten^{i}_{q}\densityten^{q}_{a}
                + \half\twoten^{pq}_{is}\densityten^{as}_{pq}
                - \half\twoten^{aq}_{rs}\densityten^{sr}_{iq}.
                \label{eq:eta-jb}
            \end{align}
            This is a linear equation system that can be solved for
            $\eta^{j}_{b}$.
            The virtual-occupied block equation is
            \begin{align}
                -i\hslash\eta^{b}_{j}
                A^{ja}_{bi}
                =
                \oneten^{p}_{i}\densityten^{a}_{p}
                -
                \oneten^{a}_{q}\densityten^{q}_{i}
                +
                \half
                \twoten^{pq}_{is}\densityten^{sa}_{pq}
                -
                \half
                \twoten^{aq}_{rs}\densityten^{sr}_{iq}
                + i\hslash\partial_t\densityten^{a}_{i},
                \label{eq:eta-bj}
            \end{align}
            which resembles a transposed linear equation.
            Note that the time-derivative of the one-body density matrix does
            not disappear in the latter equation.
            Solving both \autoref{eq:eta-jb} and \autoref{eq:eta-bj} we are able
            to construct the $\vfg{\eta}$ matrix.
            We shall in the following derivation of the $Q$-space equations see
            how this gives an expression for the time-evolution of the orbitals.

        \subsection{$Q$-space equations}
            For the $Q$-space equations we define the variation over the
            orbitals to be
            \begin{gather}
                \ket*{\delta\phi_p}
                \equiv \ket*{\theta}
                = \hat{Q}\ket*{\theta},
                \\
                \bra*{\delta\tilde{\phi}_p}
                \equiv \bra*{\theta}
                = \bra*{\theta}\hat{Q},
            \end{gather}
            where the requirement that $\ket*{\theta}$ can be defined in terms
            of itself acted upon by $\hat{Q}$ yields
            \begin{align}
                \braket*{\theta}{\phi_p}
                = \braket*{\tilde{\phi}_p}{\theta}
                = 0,
            \end{align}
            for all $p$.
            This helps explain why we were able to split up the orbital
            variations over $\hat{P}$ and $\hat{Q}$ as it demonstrates that the
            $Q$-space equations are independent of the $P$-space equations.
            The variation over $Q$-space will only apply to the energy
            functional $\mathcal{E}$ as the amplitude term in the Lagrangian
            does not explicitly depend on the orbitals.
            We write the energy functional in terms of one- and two-body
            matrices
            \begin{align}
                \mathcal{E}
                &= \mel*{\tilde{\Psi}}{
                    \para{
                        \hamil - i\hslash\hat{\eta}
                    }
                }{\Psi}
                =
                \densityten^{q}_{p}\para{
                    \oneten^{p}_{q}
                    - i\hslash\eta^{p}_{q}
                }
                + \frac{1}{4}\densityten^{rs}_{pq}
                \twoten^{pq}_{rs}
                \\
                &= \densityten^{q}_{p}
                \mel*{\tilde{\phi}_p}{
                    \para{
                        \onehamil
                        - i\hslash\partial_t
                    }
                }{\phi_{q}}
                +
                \frac{1}{4}\densityten^{rs}_{pq}
                \mel*{\tilde{\phi}_p\tilde{\phi}_q}{
                    \twohamil
                }{\phi_r\phi_s}_{AS},
            \end{align}
            where we've written out the explicit orbital dependencies in the
            one- and two-body Hamiltonian and the time-derivative.
            Requiring that the first order variations over the orbitals should
            make the action functional disappear, we have
            \begin{align}
                \delta S
                &=
                \int\dd t \brak{
                    \rho^{q}_{p}
                    \mel*{\theta}{
                        \para{
                            \onehamil
                            - i\hslash\partial_t
                        }
                    }{\phi_q}
                    +
                    \densityten^{rs}_{pq}
                    \mel*{\theta\tilde{\phi}_q}{
                        \twohamil
                    }{\phi_r\phi_s}
                }
                = 0,
            \end{align}
            where we stress that $p$ is now a free index, and where the two-body
            elements no longer are antisymmetric.
            The factor $1/4$ disappeared from the two-body integrals due to the
            antisymmetric properties of the two-body density matrix and from the
            symmetric properties of the two-body elements.\footnote{%
                To see this expand the antisymmetric two-body Hamiltonian in its
                constituent parts, and permute the orbitals using the symmetric
                properties of the two-body elements.
            }
            As the variation must be valid for all $\bra*{\theta}\hat{Q}$ we are
            left with
            \begin{align}
                i\hslash\densityten^{q}_{p}\hat{Q}\partial_t\ket{\phi_q}
                = \densityten^{q}_{p}\hat{Q}\onehamil\ket{\phi_q}
                + \densityten^{rs}_{pq}\hat{Q}\hat{W}^{q}_{s}\ket{\phi_{r}},
                \label{eq:ket-q}
            \end{align}
            where we've defined the mean-field potential $W^{q}_{s}$ to be
            \cite{kvaal2012ab}
            \begin{align}
                \mel*{\tilde{\phi}_p}{
                    \hat{W}^{q}_{s}
                }{\phi_r}
                = \int\dd x_1 \dd x_2
                \tilde{\phi}_{p}(x_1)
                \tilde{\phi}_{q}(x_2)
                \twohamil(x_1, x_2)
                \phi_s(x_2)
                \phi_r(x_1),
            \end{align}
            similarly to the Coulomb operator in the Hartree-Fock theory.
            Performing the variations over the ket-side of the energy functional
            we have
            \begin{align}
                \delta S
                = \int \dd t
                \brak{
                    \densityten^{q}_{p}
                    \mel*{\tilde{\phi}_p}{
                        \para{
                            \onehamil
                            - i\hslash \partial_t
                        }
                    }{\theta}
                    + \densityten^{rs}_{pq}
                    \mel*{\tilde{\phi}_p\tilde{\phi}_q}{
                        \twohamil
                    }{\theta\phi_s}
                }
                = 0,
            \end{align}
            where now $q$ in the first term and $r$ in the second term are free
            indices.
            This variation must be valid for all $\hat{Q}\ket{\theta}$ and we
            get the equation
            \begin{align}
                -i\hslash\densityten^{q}_{p}\para{
                    \partial_t \bra*{\tilde{\phi}_p}
                }\hat{Q}
                =
                \densityten^{q}_{p}\bra*{\tilde{\phi}_p}
                \onehamil\hat{Q}
                + \densityten^{rs}_{pq}
                \bra*{\tilde{\phi}_p}
                \hat{W}^{q}_{s}\hat{Q},
                \label{eq:bra-q}
            \end{align}
            where we get a change of sign on the left-hand side due to
            integration by parts and removal of the boundary term.
            Now \autoref{eq:ket-q} and \autoref{eq:bra-q} along with the
            $P$-space equations provide the equations of motion for the orbital
            rotations.

            When \citeauthor{kvaal2012ab} formulated the orbital-adaptive
            coupled-cluster method the intention was to provide an approximation
            to the multi-configuration Hartree-Fock method.
            The latter method approximates the full configuration interaction
            method by truncating the single-particle basis beyond the originally
            truncated basis set.
            The orbital-adaptive coupled-cluster method approximates the
            multi-configuration Hartree-Fock method by truncating the Slater
            determinant space by using the exponential ansatz from
            coupled-cluster.
            Thus, we are in a position to do two truncations from the
            full-configuration method; the Slater basis and the single-particle
            basis.
            Now, the former truncation comes from the choice of the cluster
            operator truncation and the latter from the $Q$-space equations.
            We shall discuss this in more detail when we look at the
            implementation of the orbital-adaptive coupled-cluster method in
            \autoref{subsec:oatdcc-implementation}.



        \subsection{Normalization of the coupled-cluster wave functions}
            We compute the normalization of the coupled-cluster wave function by
            \begin{align}
                \braket*{\tilde{\Psi}(t)}{\Psi(t)} = N(t).
            \end{align}
            When using static orbitals this reduces to
            \begin{align}
                \braket*{\tilde{\Psi}(t)}{\Psi(t)}
                = \mel*{\slat}{
                    (\1 + \clustl(t))
                    \exponential(-\clust(t))
                    \exponential(\clust(t))
                }{\slat}
                = \braket*{\slat}{\slat}
                = 1,
            \end{align}
            where we assume that the reference determinant is normalized to
            unity.
            We can then see that the static formulation of the coupled-cluster
            methods are, by construction, normalized to unity.
            However, for the orbital-adaptive methods where the orbitals are
            allowed to vary in time, the picture changes.
            We then have
            \begin{align}
                \braket*{\tilde{\Psi}(t)}{\Psi(t)}
                &=
                \mel*{\tilde{\slat}(t)}{
                    (\1 + \clustl(t))
                    \exponential(-\clust(t))
                    \exponential(\clust(t))
                }{\slat(t)}
                \\
                &= \braket*{\tilde{\slat}(t)}{\slat(t)}
                = \det[\vfg{T}(t)]
                = N(t),
                \label{eq:norm-oatdcc}
            \end{align}
            where we have introduced the matrix $\vfg{T}(t)$ as the product of
            the coefficient matrices for occupied indices.
            That is,
            \begin{align}
                T_{ij}(t) = \tilde{C}_{i\alpha}(t) C_{\alpha j}(t)
            \end{align}
            where the coefficients are limited to run over the occupied orbitals
            in the reference Slater determinants, but $\alpha$ runs over all the
            atomic orbitals\footnote{%
                There is no problem in computing the determinant of the full
                product between the coefficient matrices as this should yield
                the identity, but formally only the occupied orbitals are
                included in the reference states.
            }.
            Here we observe that the normalization constant can change in time.
            Mathematically, we construct the product between the coefficients to
            be the identity, i.e.,
            \begin{align}
                \tilde{\vfg{C}}(t)\vfg{C}(t)
                = \1,
                \label{eq:oatdcc-coeff-identity}
            \end{align}
            where this product runs over all indices $p$ and $q$.
            We therefore expect that $N(t) = 1$ for all times $t$, but due to
            integration errors in the numerical differential equation solvers we
            will observe a time-dependent drift.
            By measuring the time-dependent norm from \autoref{eq:norm-oatdcc}
            and the preservation of the identity for the coefficients from
            \autoref{eq:oatdcc-coeff-identity} in the orbital-adaptive regime,
            we can make an educated guess as to when the integration breaks.
            This motivates the need for good differential equation integrators
            which preserve the norm.
            % TODO: If this should be left in the text, we must include some
            % results demonstrating this.

        \subsection{Measuring the quality of the reference state in time}
            \label{subsec:cc-phase}
            When evolving a system in time, we can -- and will -- get in a
            situation where our initial state has little overlap with the
            time-evolved state \cite{pedersen2018symplectic}.\footnote{%
                Note that even though there might be a nonzero overlap with our
                initial state, this overlap can be so small that within
                computational precision it is zero.
            }
            That is, we can get in a situation where
            \begin{align}
                \braket*{\psi(t)}{\psi(0)} \to 0,
            \end{align}
            where $\ket{\psi(0)}$ is some initial state and $\ket{\psi(t)}$ is
            the time-evolved state at a later time $t$.
            In time-dependent configuration interaction theory this is handled
            automatically by the zero-order amplitude, viz.
            \begin{gather}
                \ket*{\Psi(t)}
                = C_0(t)\ket*{\slat}
                + C^{a}_{i}(t)\ket*{\slat^{a}_{i}}
                + \dots
                \implies
                \braket*{\slat}{\Psi(t)} = C_0(t),
            \end{gather}
            where $\ketslat$ is the reference determinant and we assume an
            orthonormal set of single-particle states as our basis.
            In time-dependent coupled-cluster theory where we assume
            intermediate normalization, there is however a different story to be
            told.
            For the time-dependent coupled-cluster wavefunction we have that
            \begin{gather}
                \ket*{\Psi(t)}
                = \exponential(\clust(t))\ket*{\slat}
                = \para{
                    \1
                    + \clust(t)
                    + \dots
                }\ket*{\slat}
                \implies
                \braket*{\slat}{\Psi(t)} = 1.
            \end{gather}
            This might not look like much, but for strong fields, or long
            time-evolution, we can find states which have very little overlap
            with the reference state.
            In numerical simulations the lack of infinite precision leads to
            numerial problems when the overlap goes to zero.
            When this is the case the cluster amplitudes are left with the job
            of both ``removing'' the reference state from the cluster wave
            function \emph{and} providing contribution from higher excited
            determinants \cite{pedersen2018symplectic}.
            Frustratingly enough the formulation of the time-dependent
            coupled-cluster method with static spin-orbitals is seemingly not
            able to solve this problem.
            % TODO: Add snide remark on lowering the integration time-step?
            When
            \begin{align}
                \braket*{\slat}{\Psi(t)} \to 0,
            \end{align}
            the magnitude of the amplitudes sky-rocket, the energy is no longer
            conserved (when the time-dependence in the Hamiltonian is turned
            off), and the data produced can no longer be trusted
            \cite{pedersen2018symplectic}.
            Luckily, the orbital-adaptive time-dependent coupled-cluster method
            seems to remove much of these problems.
            We will in the coming sections demonstrate that the orbital-adaptive
            formulation of coupled-cluster fixes some of the problems described
            by \citeauthor{pedersen2018symplectic}
            \cite{pedersen2018symplectic}.
            However, the orbital-adaptive formulation does not seem to solve all
            stability problems, and an ongoing study into this is being
            conducted by \citeauthor{oa-stability} \cite{oa-stability}.
            % TODO: Should we include any study at all of the stability?

            But, before we try to solve the problem, we can formulate a
            ``zero-order'' amplitude for the time-dependent coupled-cluster wave
            function which does not solve the problem, but at least lets us
            measure how good or bad the reference state is in time.
            This is done by introducing a phase factor $\clustamp_0(t) \in
            \mathbb{C}$ which we insert in the exponential ansatz, viz.
            \begin{gather}
                \ket*{\Psi(t)}
                = \exponential(\clustamp_0(t) + \clust(t))\ket*{\slat},
                \\
                \bra*{\tilde{\Psi}(t)}
                = \bra*{\slat}(\1 + \clustl(t))
                \exponential(-\clustamp_0(t) - \clust(t)),
            \end{gather}
            where $\clustamp_0(t)$ is a number and therefore commutes with the
            cluster operators.
            Regardless if $\clustamp_0(t)$ is real or not, the inclusion of the
            phase does not change any of the equations used in coupled-cluster
            methods as it oes not couple amplitude equations.
            However, we can now see that
            \begin{gather}
                \braket*{\slat}{\Psi(t)} = \exponential(\clustamp_0(t)),
                \\
                \braket*{\tilde{\Psi}(t)}{\slat}
                = \exponential(-\clustamp_0(t)) \brak{
                    1
                    - \mel*{\slat}{
                        \clustl(t)
                        \clust(t)
                    }{\slat}
                },
            \end{gather}
            which tells us how much overlap the reference state $\ket*{\slat}$ has
            with the time-evolved state at a time $t$.
            Stated differently, it measures how well the reference state
            is at approximating the time-evolved state.
            Note that the inclusion of $\clustamp_0(t)$ is in no way able to be
            used as an amplitude to lower the presence of the reference as in
            the configuration interaction method.
            We also see that for $\clustamp_0(t) = 0$ we recover intermediate
            normalization and the known time-dependent coupled-cluster method.
            The question remains how we can evaluate $\clustamp_0(t)$ from the
            coupled-cluster equations.
            In general we find the time derivative of the cluster amplitudes
            from
            \begin{align}
                i\hslash \partial_t\clustamp_{\mu}
                &=
                \mel*{\slat_{\mu}}{
                    \exponential(-\clust(t))
                    \hamil(t)
                    \exponential(\clust(t))
                }{\slat},
            \end{align}
            where $\mu$ denotes the excitation level of the reference slater
            determinant and the order of the cluster amplitudes as in
            \autoref{eq:cluster-operator}.
            By choosing $\mu = 0$ we are left with
            \begin{align}
                i\hslash \partial_t\clustamp_{0}
                &=
                \mel*{\slat}{
                    \exponential(-\clust(t))
                    \hamil(t)
                    \exponential(\clust(t))
                }{\slat},
            \end{align}
            which looks familiar as the equation for the ground state
            energy in the time-independent coupled-cluster method if the
            explicit time-dependence is removed from the Hamiltonian and the
            cluster amplitudes.
