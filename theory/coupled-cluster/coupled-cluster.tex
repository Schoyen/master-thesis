\chapter{Coupled cluster theory}
    % TODO: Why is coupled cluster so awesome?
    % Discuss size-consistency and size-extensivity.
    % See Crawford & Schaefer for a good intro to these topics.


    \section{Time-independent coupled cluster theory}
        This section follows closely the derivation done in
        \citetitle{crawford2000introduction} by
        \citeauthor{crawford2000introduction} \cite{crawford2000introduction}.
        Starting with the time-independent Schrödinger equation, the coupled
        cluster method seeks to find a many-body wave function $\ketcc$ which
        solves
        \begin{align}
            \hamil\ketcc = \energycc\ketcc.
        \end{align}
        Start from a single reference Slater determinant $\ketslat$ coupled
        cluster makes the ansatz that the true many-body wave function is given
        by
        \begin{align}
            \ketcc \equiv e^{\clust}\ketslat
            = \sum_{n = 0}^{\infty}\frac{1}{n!}\clust^n\ketslat,
            \label{eq:cc_wave_function}
        \end{align}
        where the \emph{cluster operator} $\clust$ is given by a sum of excitation
        operators $\clust_{p}$.
        \begin{align}
            \clust &= \sum_{p = 1}^{n} \clust_p
            = \clustamp^a_i\ccr{a}\can{i}
            + \para{\frac{1}{2!}}^2\clustamp^{ab}_{ij}\ccr{a}\ccr{b}\can{i}\can{j}
            + \para{\frac{1}{3!}}^2\clustamp^{abc}_{ijk}
            \ccr{a}\ccr{b}\ccr{c}\can{i}\can{j}\can{k}
            + \dots,
        \end{align}
        where the cluster amplitudes $\clustamp^{abc...}_{ijk...}$ are the
        primary unknowns.
        By inserting \autoref{eq:cc_wave_function} into the time-independent
        Schrödinger equation we get
        \begin{align}
            \hamil\ketcc
            = e^{\clust}\ketslat
            = \energycc e^{\clust}\ketslat
            = \energycc \ketcc.
        \end{align}
        Projecting onto the reference Slater determinant we are left with
        \begin{align}
            \braslat\hamil\ketcc
            &= \braslat\energycc e^{\clust}\ketslat
            = \energycc \braslat e^{\clust}\ketslat
            = \energycc,
        \end{align}
        where for orthonormal Slater determinants, the coupled cluster wave
        function assumes \emph{intermediate normalization}, viz.
        \begin{align}
            \braket{\slat}{\cc}
            &= \bra{\slat}\para{
                \1 + \clust + \frac{1}{2!}\clust^2 + \dots
            }\ket{\slat}
            = 1.
        \end{align}
