\chapter{Quantum Mechanics}
    \epigraph{The underlying physical laws necessary for the mathematical
    theory of a large part of physics and the whole of chemistry are thus
    completely known, and the difficulty is only that the exact application of
    these laws leads to equations much too complicated to be soluble.}
    {--- P. A. M. Dirac}

    \section{Pauli exclusion principle}
        % TODO: Include the spin-statistics theorem.

    \section{Spin-orbitals}
        In this thesis we will be looking at many-particle systems. Wave
        functions of many-particle systems depend on the coordinates of all the
        particles contained in the system. We will typically build the wave
        functions of the full system from \emph{single particle functions}. In
        this work we'll limit our attention to particles with spin one half ,
        i.e., fermions, which means that the total wave function must satisfy
        the \emph{Pauli exclusion principle} due to the particles being
        indistinguishable. The single particle functions will as a consequence
        depend on the particle's spatial orientation, i.e., the \emph{orbital}
        part, and the spin. We call these wave functions \emph{spin-orbitals}.
        \begin{align}
            \psi(x) \equiv \psi(\vf{r}, \sigma),
        \end{align}
        where $x = (\vf{r}, \sigma)$ is a generalized coordinate of both
        position, $\vf{r}$, and spin, $\sigma$. For fermions we have only two
        allowed spin states
        \begin{align}
            s_z = \pm \half\hslash,
        \end{align}
        where $s_z$ is the spin along $z$-direction. As there are only two
        allowed states we have that $\sigma \in \brac{\upspin, \downspin}$,
        where a positive value for $s_z$ corresponds to $\sigma = \upspin$,
        i.e., spin up, and a negative value for $s_z$ to $\sigma = \downspin$,
        i.e., spin down.  We denote
        \begin{align}
            \psi_1(\vf{r}) \equiv \psi\para{\vf{r}, \upspin},
            \qquad
            \psi_2(\vf{r}) \equiv \psi\para{\vf{r}, \downspin},
        \end{align}
        for the two different spin-directions. We can thus represent the
        generalized spin-orbital $\psi(x)$ as a two-dimensional vector
        \begin{align}
            \psi(x) = \begin{pmatrix}
                \psi_1(\vf{r}) \\
                \psi_2(\vf{r})
            \end{pmatrix}.
        \end{align}
        We separate the spin dependence from the spatial part of the
        spin-orbitals by introducing separate ``spin functions'' for spin-up and
        spin-down, viz.
        \begin{align}
            \alpha \equiv \alpha(\sigma) = \begin{pmatrix}
                1 \\
                0
            \end{pmatrix},
            \qquad
            \beta \equiv \beta(\sigma) = \begin{pmatrix}
                0 \\
                1
            \end{pmatrix},
            \label{eq:spin-basis}
        \end{align}
        where we use the same convention as in much of the many-body quantum
        mechanics litterature in labelling $\alpha$ as spin up and $\beta$ as
        spin down. Evaluating the spin functions thus yields
        \begin{gather}
            \alpha\para{\upspin} = 1, \qquad \alpha\para{\downspin} = 0, \\
            \beta\para{\upspin} = 0, \qquad \beta\para{\downspin} = 1.
        \end{gather}
        Using \autoref{eq:spin-basis} we see that we write the
        generalized spin-orbital as a linear combination of the spin basis
        functions by
        \begin{align}
            \psi(x)
            = \psi_1(\vf{r})\alpha(\sigma)
            + \psi_2(\vf{r})\beta(\sigma).
        \end{align}
        % TODO: Consider adding theory on spin as in Mayer.



        %A general spin-orbital
        %can thus be written as a linear combination of both spin directions
        %\begin{align}
        %    \psi(x)
        %    &= \psi\para{\vf{r}, \half} + \psi\para{\vf{r}, -\half}
        %    \equiv \psi_1(\vf{r}) + \psi_2(\vf{r})
        %    = \begin{pmatrix}
        %        \psi_1(\vf{r}) \\
        %        \psi_2(\vf{r})
        %    \end{pmatrix},
        %\end{align}
        %where we've in the last equality consider the two different
        %spin-directions as unit vectors in a two-dimensional space.

        %A general spin-orbital is a two-level system where the coefficients of
        %the two spin-directions serve as the spatial part of the wave function,
        %viz.
        %\begin{align}
        %    \ket{\psi}
        %    = \ket{\psi_1}\otimes\ket{\upspin}
        %    + \ket{\psi_2}\otimes\ket{\downspin}
        %    =
        %    \begin{pmatrix}
        %        \ket{\psi_1} \\
        %        \ket{\psi_2}
        %    \end{pmatrix},
        %    \label{eq:general-spin-orbital}
        %\end{align}
        %where $\psi_i$ can be function of position, $\vf{r}$, or momentum,
        %$\vf{p}$, or any other desirable coordinate.  In the last equation we've
        %expressed the two spin directions as unit vectors in a two-dimensional
        %vector space\footnote{The choice which spin direction will represent
        %which unit vector is arbitrary. Feel free to create any orthonormal unit
        %vectors in two dimensions.}, viz.
        %\begin{align}
        %    \ket{\upspin}
        %    =
        %    \begin{pmatrix}
        %        1 \\
        %        0
        %    \end{pmatrix},
        %    \qquad
        %    \ket{\downspin}
        %    =
        %    \begin{pmatrix}
        %        0 \\
        %        1
        %    \end{pmatrix}.
        %\end{align}
        %However, we will usually work with spin-orbitals that are either
        %definitely in the spin-up or the spin-down state, that is, we refrain
        %from using general spin-orbitals such as in
        %\autoref{eq:general-spin-orbital} when we construct our systems.
        %% TODO: Reconsider this statement.
        %% TODO: Consider including longer discussion of spin as in Mayer.


    \section{Slater determinants}
        As we are working with fermions the total many-body wave function must
        be \emph{antisymmetric} with respect to interchange of
        particles\footnote{We will restrict ourselves to the interchange of a
        full spin-orbital.}. Mathematically the full wave function of $n$
        fermions should satisfy
        \begin{align}
            \Psi(1, \dots, i, \dots, j, \dots, n)
            =
            -\Psi(1, \dots, j, \dots, i, \dots, n),
        \end{align}
        where we for brevity have introduced the notation $i \equiv x_i =
        (\vf{r}_i, \sigma_i)$. A fully antisymmetric, normalized, $n$-body wave
        function is given by the \emph{Slater determinant} built from $n$
        individual spin-orbitals with $n$ generalized coordinates.
        \begin{align}
            \Psi(1, \dots, n)
            &= \frac{1}{\sqrt{n!}}
            \begin{vmatrix}
                \psi_1(1) & \dots & \psi_n(1) \\
                \vdots & \ddots & \vdots \\
                \psi_1(n) & \dots & \psi_n(n)
            \end{vmatrix},
        \end{align}
        where $\psi_i(j) \equiv \psi_i(\vf{r}_j, \sigma_j)$ for the
        spin-orbitals\footnote{Note that the index on the spin-orbitals has
        nothing to do with which spin component is used as in the previous
        section on spin-orbitals.}.

    \section{Second quantization}

    \section{Density matrices}
        When working with many-body quantum mechanics, computing expectation
        values can at times prove easier when done using density matrices. A
        general density matrix of a \emph{pure state} is on the form
        \begin{align}
            \densitymatrix = \ket{\psi}\bra{\psi},
        \end{align}
        that is, a pure state is a quantum state $\ket{\psi}$ containing the
        maximum amount of information about a given system. For a \emph{mixed
        state}, i.e., a linear combination of pure states $\ket{\psi_k}$ with a
        classical probability $p_k$ associated with the state, we get a density
        matrix on the form
        \begin{align}
            \densitymatrix = \sum_{k} p_k \ket{\psi_k}\bra{\psi_k}.
        \end{align}
        Any density operator must satisfy the following
        properties \cite{modern-qm}:
        \begin{enumerate}
            \item Hermiticity, that is
                \begin{align}
                    p_k = p_k^{*} \implies \densitymatrix = \densitymatrix^{\dagger}.
                \end{align}
                This translates to the probabilities being real, $p_k \in
                \mathbb{R}$.
            \item Positivity,
                \begin{align}
                    p_k \geq 0 \implies \bra{\chi}\densitymatrix\ket{\chi} \geq 0.
                \end{align}
                In other words, density matrices are \emph{positive
                semidefinite}.
            \item Normalization of the probabilities,
                \begin{align}
                    \sum_{k} p_k = 1 \implies \tr(\densitymatrix),
                \end{align}
                that is, the probabilities must sum up to one.
        \end{enumerate}
        Furthermore, by squaring the density matrix and taking the trace we can
        infer if the system we are perusing is in a mixed state or a pure state
        \cite{modern-qm}.
        \begin{align}
            \tr(\densitymatrix^2) = \sum_{k} p_k^2 \leq 1,
        \end{align}
        with equality if, and only if, the system is in a pure state, viz.
        \begin{align}
            \densitymatrix = \ket{\psi}\bra{\psi}
            \implies \densitymatrix^2 = \densitymatrix
            \implies \tr(\densitymatrix^2) = 1.
        \end{align}
        Using density matrices, we can compute the expectation value of any
        operator $\hat{O}$, by \cite{modern-qm}
        \begin{align}
            \expv{\hat{O}} = \tr(\hat{O}\densitymatrix).
        \end{align}
        % TODO: Work this out further

        \subsection{Many-body density matrices}
            In a seminal paper by Löwdin \cite{lowdin-density-matrices}, the
            concept of a many-body density matrix in terms of the orbitals of a
            Slater determinant is discussed. These are dubbed $N$-body density
            matrices, where $N$ depends on the $N$-body interaction, that is,
            the number of particles included in the interaction. Of the most
            useful for our work, we have the one- and two-body density matrices.
            As we will work almost exclusively in second quantization, we will
            follow the derivation of the one- and two-body density matrices done
            by \citeauthor{helgaker-molecular}. Löwdin's paper
            \cite{lowdin-density-matrices} did not employ second quantization,
            and all matrices are expressed in the coordinate representation. We
            will list these as they arrive.
