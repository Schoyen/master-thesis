\chapter{Quantum Mechanics}
    \epigraph{The underlying physical laws necessary for the mathematical
    theory of a large part of physics and the whole of chemistry are thus
    completely known, and the difficulty is only that the exact application of
    these laws leads to equations much too complicated to be soluble.}
    {--- P. A. M. Dirac}

    \section{Spin-orbitals}
        In this thesis we will be looking at many-particle systems. Wave
        functions of many-particle systems depend on the coordinates of all the
        particles contained in the system. We will typically build the wave
        functions of the full system from \emph{single particle functions}. In
        this work we'll limit our attention to particles with spin one half ,
        i.e., fermions, which means that the total wave function must satisfy
        the \emph{Pauli exclusion principle}. The single particle functions
        will as a consequence depend on the particle's spatial
        orientation\footnote{Note that this covers both position and momentum
        space as well as an arbitrary dimensionality.} and spin. We call
        these wave functions \emph{spin-orbitals}. The spatial part of a
        spin-orbtial will be denoted \emph{orbital} whereas the spin-part is
        simply called spin. A general spin-orbital is a two-level system where
        the coefficients of the two spin-directions serve as the spatial part of
        the wave function, viz.
        \begin{align}
            \ket{\psi} = \psi_1\ket{\upspin} + \psi_2\ket{\downspin}
            =
            \begin{pmatrix}
                \psi_1 \\
                \psi_2
            \end{pmatrix},
        \end{align}
        where $\psi_i$ can be function of position, $\vf{r}$, or momentum,
        $\vf{p}$, or any other desirable coordinate.  In the last equation we've
        expressed the two spin directions as unit vectors in a two-dimensional
        vector space\footnote{The choice which spin direction will represent
        which unit vector is arbitrary. Feel free to create any orthonormal unit
        vectors in two dimensions.}, viz.
        \begin{align}
            \ket{\upspin}
            =
            \begin{pmatrix}
                1 \\
                0
            \end{pmatrix},
            \qquad
            \ket{\downspin}
            =
            \begin{pmatrix}
                0 \\
                1
            \end{pmatrix}.
        \end{align}


        %In a functional basis we
        %denote the spin-orbitals by
        %\begin{align}
        %    \psi(x) \equiv \psi(\vf{r}, \sigma) = \psi(\vf{r})\ket{\sigma},
        %\end{align}
        %where $\vf{r}$ is a spatial position and $\sigma \in \brac{\upspin,
        %\downspin}$, is either spin up, $\upspin$, or spin down, $\downspin$.
        %For brevity we introduce $x = (\vf{r}, \sigma)$ as a generalized
        %coordinate. A general spin-orbital can be written as a linear
        %combination of two spatial orbitals\footnote{We will refer to the
        %spatial part of a spin-orbital simply as an orbital.} where each orbital
        %is combined with a spin direction. %TODO: Re-phrase this sentence.  This
        %can be represented by
        %\begin{align}
        %    \psi(x)
        %    &= \psi_1(\vf{r})\ket{\upspin} + \psi_2(\vf{r})\ket{\downspin}
        %    =
        %    \begin{pmatrix}
        %        \psi_1(\vf{r}) \\
        %        \psi_2(\vf{r})
        %    \end{pmatrix},
        %\end{align}
        %where we've in the last equation expressed the two spin directions as
        %unit vectors in two-dimensional vector space\footnote{The choice which
        %spin direction will be which unit vector is arbitrary. Feel free to
        %create any orthonormal unit vectors in two dimensions.}, viz.
        %\begin{align}
        %    \ket{\upspin}
        %    =
        %    \begin{pmatrix}
        %        1 \\
        %        0
        %    \end{pmatrix},
        %    \qquad
        %    \ket{\downspin}
        %    =
        %    \begin{pmatrix}
        %        0 \\
        %        1
        %    \end{pmatrix}.
        %\end{align}

    \section{Slater determinants}

    \section{Second quantization}

    \section{Density matrices}
        When working with many-body quantum mechanics, computing expectation
        values can at times prove easier when done using density matrices. A
        general density matrix of a \emph{pure state} is on the form
        \begin{align}
            \densitymatrix = \ket{\psi}\bra{\psi},
        \end{align}
        that is, a pure state is a quantum state $\ket{\psi}$ containing the
        maximum amount of information about a given system. For a \emph{mixed
        state}, i.e., a linear combination of pure states $\ket{\psi_k}$ with a
        classical probability $p_k$ associated with the state, we get a density
        matrix on the form
        \begin{align}
            \densitymatrix = \sum_{k} p_k \ket{\psi_k}\bra{\psi_k}.
        \end{align}
        Any density operator must satisfy the following
        properties \cite{modern-qm}:
        \begin{enumerate}
            \item Hermiticity, that is
                \begin{align}
                    p_k = p_k^{*} \implies \densitymatrix = \densitymatrix^{\dagger}.
                \end{align}
                This translates to the probabilities being real, $p_k \in
                \mathbb{R}$.
            \item Positivity,
                \begin{align}
                    p_k \geq 0 \implies \bra{\chi}\densitymatrix\ket{\chi} \geq 0.
                \end{align}
                In other words, density matrices are \emph{positive
                semidefinite}.
            \item Normalization of the probabilities,
                \begin{align}
                    \sum_{k} p_k = 1 \implies \tr(\densitymatrix),
                \end{align}
                that is, the probabilities must sum up to one.
        \end{enumerate}
        Furthermore, by squaring the density matrix and taking the trace we can
        infer if the system we are perusing is in a mixed state or a pure state
        \cite{modern-qm}.
        \begin{align}
            \tr(\densitymatrix^2) = \sum_{k} p_k^2 \leq 1,
        \end{align}
        with equality if, and only if, the system is in a pure state, viz.
        \begin{align}
            \densitymatrix = \ket{\psi}\bra{\psi}
            \implies \densitymatrix^2 = \densitymatrix
            \implies \tr(\densitymatrix^2) = 1.
        \end{align}
        Using density matrices, we can compute the expectation value of any
        operator $\hat{O}$, by \cite{modern-qm}
        \begin{align}
            \expv{\hat{O}} = \tr(\hat{O}\densitymatrix).
        \end{align}
        % TODO: Work this out further

        \subsection{Many-body density matrices}
            In a seminal paper by Löwdin \cite{lowdin-density-matrices}, the
            concept of a many-body density matrix in terms of the orbitals of a
            Slater determinant is discussed. These are dubbed $N$-body density
            matrices, where $N$ depends on the $N$-body interaction, that is,
            the number of particles included in the interaction. Of the most
            useful for our work, we have the one- and two-body density matrices.
            As we will work almost exclusively in second quantization, we will
            follow the derivation of the one- and two-body density matrices done
            by \citeauthor{helgaker-molecular}. Löwdin's paper
            \cite{lowdin-density-matrices} did not employ second quantization,
            and all matrices are expressed in the coordinate representation. We
            will list these as they arrive.
