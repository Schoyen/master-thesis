\chapter{Quantum Mechanics}
    \epigraph{YOLO}{P. A. M. Dirac}

    \section{Density matrices}
        When working with many-body quantum mechanics, computing expectation
        values can at times prove easier when done using density matrices. A
        general density matrix of a \emph{pure state} is on the form
        \begin{align}
            \densitymatrix = \ket{\psi}\bra{\psi},
        \end{align}
        that is, a pure state is a quantum state $\ket{\psi}$ containing the
        maximum amount of information about a given system. For a \emph{mixed
        state}, i.e., a linear combination of pure states $\ket{\psi_k}$ with a
        classical probability $p_k$ associated with the state, we get a density
        matrix on the form
        \begin{align}
            \densitymatrix = \sum_{k} p_k \ket{\psi_k}\bra{\psi_k}.
        \end{align}
        Any density operator must satisfy the following
        properties \cite{modern-qm}:
        \begin{enumerate}
            \item Hermiticity, that is
                \begin{align}
                    p_k = p_k^{*} \implies \densitymatrix = \densitymatrix^{\dagger}.
                \end{align}
                This translates to the probabilities being real, $p_k \in
                \mathbb{R}$.
            \item Positivity,
                \begin{align}
                    p_k \geq 0 \implies \bra{\chi}\densitymatrix\ket{\chi} \geq 0.
                \end{align}
                In other words, density matrices are \emph{positive
                semidefinite}.
            \item Normalization of the probabilities,
                \begin{align}
                    \sum_{k} p_k = 1 \implies \tr(\densitymatrix),
                \end{align}
                that is, the probabilities must sum up to one.
        \end{enumerate}
        Furthermore, by squaring the density matrix and taking the trace we can
        infer if the system we are perusing is in a mixed state or a pure state
        \cite{modern-qm}.
        \begin{align}
            \tr(\densitymatrix^2) = \sum_{k} p_k^2 \leq 1,
        \end{align}
        with equality if, and only if, the system is in a pure state, viz.
        \begin{align}
            \densitymatrix = \ket{\psi}\bra{\psi}
            \implies \densitymatrix^2 = \densitymatrix
            \implies \tr(\densitymatrix^2) = 1.
        \end{align}
        Using density matrices, we can compute the expectation value of any
        operator $\hat{O}$, by \cite{modern-qm}
        \begin{align}
            \expv{\hat{O}} = \tr(\hat{O}\densitymatrix).
        \end{align}
        % TODO: Work this out further

        \subsection{Many-body density matrices}
            In a seminal paper by Löwdin \cite{lowdin-density-matrices}, the
            concept of a many-body density matrix in terms of the orbitals of a
            Slater determinant is discussed. These are dubbed $N$-body density
            matrices, where $N$ depends on the $N$-body interaction, that is,
            the number of particles included in the interaction. Of the most
            useful for our work, we have the one- and two-body density matrices.
            As we will work almost exclusively in second quantization, we will
            follow the derivation of the one- and two-body density matrices done
            by \citeauthor{helgaker-molecular}. Löwdin's paper
            \cite{lowdin-density-matrices} did not employ second quantization,
            and all matrices are expressed in the coordinate representation. We
            will list these as they arrive.
