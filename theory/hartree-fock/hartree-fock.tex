\chapter{Hartree-Fock theory}
    One can not tackle the subject of many-body theory without meeting the
    Hartree-Fock approximation. It serves as an excellent initial approximation,
    and in many cases the \emph{only} approximation, to the many-body
    wavefunction for a given system. It is a rather inexpensive approximation,
    in terms of computational intensity, and explains much of the underlying
    physics of a given system of many particles.

    \section{Deriving the Hartree-Fock equations}
        The main goal of the Hartree-Fock approximation is to solve the
        time-independent Schrödinger equation,
        \begin{align}
            \hamil\ket{\Psi} = E\ket{\Psi},
        \end{align}
        for a given Hamiltonian, $\hamil$. The goal of Hartree-Fock is to find
        the ``best'' wavefunction, $\ket{\Psi}$, under the constraints that it
        consists of a single Slater determinant,
        \begin{align}
            \ket{\Psi} = \ketslat
            = \ket{\mo{1}\dots\mo{N}}
        \end{align}
        where $\brac{\mo{i}}_{i = 1}^{L}$ is the basis of unknown
        \emph{molecular orbitals}, and $N$ is the number of occupied orbitals.
