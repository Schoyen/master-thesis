\chapter{Hartree-Fock theory}
    One can not tackle the subject of many-body theory without a discussion of
    the Hartree-Fock method. It serves as an excellent initial approximation,
    and in many cases the \emph{only} approximation, to the many-body
    wavefunction for a given system. It is a rather cheap method, in terms of
    computational intensity, and explains much of the underlying physics of a
    given system of many particles.

    \section{Assumptions used in the Hartree-Fock method}
        In the Hartree-Fock method we make five assumptions in order to make the
        many-body problem tractable.
        \begin{enumerate}
            \item We assume that the \emph{Born-Oppenheimer approximation} is a
                good approximation.
                % TODO: Add theory on the Born-Oppenheimer approximation
            \item We assume that the motion of the electrons can be described
                non-relativistically.
            \item We assume that the solution to the variational problem can be
                represented as a linear combination of a finite number of basis
                functions.
            \item The energy eigenfunctions of the time-independent Schrödinger
                equation can be described by a single Slater determinant.
            \item We assume that correlation between particles can be described
                in the \emph{mean-field approximation}.
        \end{enumerate}
        These assumptions provide the basis for the Hartree-Fock method. We
        shall see later that we quickly reach a limit where these assumptions
        break apart thus motivating the use of \emph{post Hartree-Fock methods}
        such as the coupled cluster method.

    \section{Deriving the time-independent Hartree-Fock equations}
        Much of the theory shown in this section draws from the excellent
        lecture notes by \citeauthor{kvaal2017notes} \cite{kvaal2017notes} and
        the book \citetitle{szabo1996modern} by \citeauthor{szabo1996modern}
        \cite{szabo1996modern}.
        Starting from the Schrödinger equation
        \begin{align}
            \hamil\ket{\Psi} = \energy\ket{\Psi},
        \end{align}
        where $\hamil$ is the electronic Hamiltonian with at most two-body
        interactions on the form
        \begin{align}
            \hamil = \onehamil + \twohamil,
        \end{align}
        where $\onehamil$ is the one-body part of the Hamiltonian and
        $\twohamil$ the higher order correlations.
        We know that the ground state of $\onehamil$ will be a single Slater
        determinant.
        If the two-body interactions are ``small'' we can assume that there will
        exist a Slater determinant which will capture most of the true ground
        state of the full Hamiltonian\footnote{We will see that it does not take
        much before the two-body interaction becomes a little more than just a
        small perturbation.}.
        This motivates the approximation that our many-body wave function
        $\ket{\Psi}$ can be approximated by a single Slater determinant.
        \begin{align}
            \ket{\Psi} = \ketslat = \ket{\phi_1, \phi_2, \dots, \phi_N},
        \end{align}
        where the \emph{molecular orbitals} $\brac{\phi_i}_{i = 1}^{N}$ are the
        primary unknowns.
        They are subject to the constraint that they are orthonormal.
        That is,
        \begin{align}
            \braket{\phi_i}{\phi_j} = \delta_{ij}
            \implies \braket{\slat}{\slat} = 1.
            % TODO: Check if this is an actual implication
        \end{align}
        Defining the energy functional
        \begin{align}
            \energyfunc{\slat}
            \equiv \braslat\hamil\ketslat
            =
            \bra{\phi_i}\onehamil\ket{\phi_i}
            + \half\bra{\phi_i\phi_j}\twohamil\ket{\phi_i\phi_j}_{AS},
            \label{eq:energy_func_hf}
        \end{align}
        where the Einstein summation convention is understood.
        The variational principle\footnote{%
            Note that we treat the Slater determinants as orthonormal thus
            removing the need to include the normalization factor in the energy
            functional.
        }
        tells us that the true ground state energy, $\energygs$, will be a lower
        bound to the energy found from the energy functional in
        \autoref{eq:energy_func_hf} for any normalized trial wavefunction
        $\ketslat$.
        That is,
        \begin{align}
            \energygs
            \leq \energyfunc{\slat} = \braslat\hamil\ketslat.
        \end{align}
        Our task is now to find the molecular orbitals $\brac{\phi_i}_{i =
        1}^{N}$ that minimizes the energy functional%
        \footnote{%
            This is done by finding a stationary state for $\energyfunc{\slat}$,
            which does not guarantee that we have found a minimum, but often the
            stationary state will be a minimum.
        }.
        By performing a variation in the Slater determinant,
        \begin{align}
            \slat \to \slat + \delta\slat,
        \end{align}
        we find the that the energy functional is changed by
        \begin{align}
            \energyfunc{\slat + \delta\slat}
            &=
            \bra{\slat + \delta\slat}\hamil\ket{\slat + \delta\slat}
            \\
            &= \energyfunc{\slat}
            + \bra{\delta\slat}\hamil\ketslat + \braslat\hamil\ket{\delta\slat}
            + \dots
            \\
            &= \energyfunc{\slat} + \delta\energyfunc{\slat}
            + \dots,
        \end{align}
        where the \emph{first variation} in $\energyfunc{\slat}$ is given by
        \begin{align}
            \delta\energyfunc{\slat}
            \equiv
            \bra{\delta\slat}\hamil\ketslat + \braslat\hamil\ket{\delta\slat},
        \end{align}
        where we treat $\delta$ as a linear differential operator.
        Higher order variations are ignored and we are thus only interested in
        finding the Slater determinant, $\ketslat$, for which
        \begin{align}
            \delta\energyfunc{\slat} = 0,
        \end{align}
        i.e., the stationary point of the energy functional in terms of the
        function $\slat$.
        However the energy functional, $\energyfunc{\slat}$, does not
        incorporate the constraint that the molecular orbitals should be
        orthonormal.
        To ensure this constraing we use the method of Lagrange multipliers,
        with one multiplier for every constraint.
        We thus construct the Lagrangian functional
        \begin{align}
            \lagrangianfunc{\slat, \lambda}
            &= \energyfunc{\slat}
            - \lambda_{ji}\para{
                \braket{\phi_i}{\phi_j}
                - \delta_{ij}
            }.
        \end{align}
        As the Lagrangian functional, $\lagrangianfunc{\slat, \lambda}$, is
        real, and as the constraint is Hermitian, the Lagrange multipliers,
        $\lambda_{ji}$, can be taken to be Hermitian as well.

        \begin{proof}[%
                Proof that the Lagrange multipliers can be chosen Hermitian%
            ]
            Following the derivation done by Mayer \cite{mayer2003simple}, we
            start by noticing that the constraint is Hermitian, i.e.,
            \begin{align}
                \braket{\phi_i}{\phi_j} - \delta_{ij}
                = \braket{\phi_j}{\phi_i}^{*} - \delta_{ji}.
            \end{align}
            As of now we have two independent Lagrange multipliers; one for the
            overlap $\braket{\phi_i}{\phi_j}$ and another for the the complex
            conjugate $\braket{\phi_j}{\phi_i}$.
            We can formulate the constraint for the real and imaginary part
            separately.
            This yields
            \begin{align}
                \Re\brac{
                    \braket{\phi_i}{\phi_j}
                }
                &=
                \half\brac{
                    \braket{\phi_i}{\phi_j}
                    + \braket{\phi_j}{\phi_i}
                }
                = 0,
                \\
                \Im\brac{
                    \braket{\phi_i}{\phi_j}
                }
                &=
                \frac{1}{2i}\brac{
                    \braket{\phi_i}{\phi_j}
                    - \braket{\phi_j}{\phi_i}
                }
                = 0.
            \end{align}
            Introducing two separate Lagrange multipliers $\mu_{ij}$ and
            $\nu_{ij}$ for the two latter conditions, we get
            \begin{align}
                \mu_{ij}\Re\brac{
                    \braket{\phi_i}{\phi_j}
                }
                + \nu_{ij}\Im\brac{
                    \braket{\phi_i}{\phi_j}
                }
                &=
                \half\brak{
                    \mu_{ij} - i\nu_{ij}
                }
                \braket{\phi_i}{\phi_j}
                \nonumber
                \\
                &\qquad
                + \half\brak{
                    \mu_{ij} + i\nu_{ij}
                }
                \braket{\phi_j}{\phi_i}.
            \end{align}
            We now choose our combined Lagrange multipliers to be
            \begin{align}
                \lambda_{ji} &=
                -\half\brak{
                    \mu_{ij} - i\nu_{ij}
                }, \\
                \lambda_{ij} &=
                -\half\brak{
                    \mu_{ij} + i\nu_{ij}
                },
            \end{align}
            which implies that $\lambda_{ji} = \lambda_{ij}^{*}$, as was to be
            shown.
        \end{proof}
        We are now interested in finding a stationary point in the Lagrangian
        with respect to the molecular orbitals $\phi_k$ subject to the
        constraint that they are orthonormal.
        As we are trying to find a stationary point of the Lagrangian functional
        with respect to the molecular orbitals, we will use \emph{functional
        derivatives} as the orbitals are functions.
        % TODO: Read up on functional derivative theory
        The action of a functional derivative on the matrix elements in the
        Lagrangian functional is given by
        \begin{gather}
            \dfd{\phi_k^{*}}\braket{\phi_i}{\phi_j}
            = \delta_{ik}\ket{\phi_j}, \\
            \dfd{\phi_k^{*}}\bra{\phi_i}\hat{h}\ket{\phi_i}
            = \hat{h}\ket{\phi_k}, \\
            \dfd{\phi_k^{*}}\bra{\phi_i}\dpd{}{t}\ket{\phi_i}
            = \dpd{}{t}\ket{\phi_k}, \\
            \dfd{\phi_k^{*}}\bra{\phi_i\phi_j}\hat{u}\ket{\phi_i\phi_j}_{AS}
            = 2\bra{\cdot\phi_j}\hat{u}\ket{\phi_k\phi_j}_{AS},
        \end{gather}
        where the "dot" is defined in terms of the tensor product by
        \begin{align}
            \ket{\phi_i\phi_j} = \ket{\phi_i}\otimes\ket{\phi_j}
            \implies \ket{\cdot\phi_j} \equiv \1 \otimes\ket{\phi_j}.
        \end{align}
        The last derivative can be found by expanding the antisymmetrix matrix
        elements by \autoref{eq:antisymmetric_two_body} and applying the
        functional derivative two times on each term. Collecting the four terms
        and using the symmetric property of the two-body elements that
        \begin{align}
            \bra{\phi_i\phi_j}\hat{u}\ket{\phi_k\phi_l}
            = \bra{\phi_j\phi_i}\hat{u}\ket{\phi_l\phi_k}.
        \end{align}
        We treat $\phi_k(x)$ and $\phi_k^{*}(x)$ as independent
        variables and taking the functional derivative of the Lagrangian
        function with respect to either form yields the same equations but
        complex conjugated of one another, we will restrict our attention to
        finding the functional derivative of the Lagrangian functional with
        respect to $\phi_k^{*}(x)$.
        \begin{align}
            \dfd{\phi_k^{*}}\lagrangianfunc{\slat, \lambda}
            = \dfd{\phi_k^{*}}\energyfunc{\slat}
            - \lambda_{jk}\ket{\phi_{j}},
        \end{align}
        where the functional derivative of the energy functional is given by
        \begin{align}
            \dfd{\phi_k^{*}}\energyfunc{\slat}
            = \onehamil\ket{\phi_k}
            + \bra{\cdot\phi_j}\twohamil\ket{\phi_k\phi_j}_{AS}.
        \end{align}
        Inserted back into the functional derivative of the Lagrangian and
        equating to zero we get
        \begin{gather}
            \onehamil\ket{\phi_k}
            + \bra{\cdot\phi_j}\twohamil\ket{\phi_k\phi_j}_{AS}
            - \lambda_{jk}\ket{\phi_{j}} = 0.
            \label{eq:min_lagrangian_phi_k}
        \end{gather}
        To go from here we introduce the Fock operator
        \begin{align}
            \fock\ket{\phi_k}
            = \onehamil\ket{\phi_k}
            + \bra{\cdot\phi_j}\twohamil\ket{\phi_k\phi_j}_{AS},
        \end{align}
        where the states $\ket{\phi_j}$ are contained in the Slater determinant,
        whereas $\ket{\phi_k}$ can be any molecular orbital in the basis.
        % TODO: The point on the occupied orbitals needs to made more explicit.
        We insert the expression for the Fock operator into
        \autoref{eq:min_lagrangian_phi_k}.
        We also move the Lagrange multipliers to the right hand side of the
        equation.
        This yields the equation
        \begin{align}
            \fock\ket{\phi_k} = \lambda_{jk}\ket{\phi_j},
            \label{eq:non-canonical-hartree-fock}
        \end{align}
        known as the \emph{non-canonical Hartree-Fock equations}.
        This equation will yield the correct Hartree-Fock energy, but we are
        interested in an eigenvalue equation without a summation over the matrix
        of Lagrange multipliers.
        To reach this goal we define a new set of spin-orbitals
        $\brac{\ket{\psi_p}}$ which is given by
        \begin{align}
            \ket{\psi_p} = U_{qp}\ket{\phi_q},
        \end{align}
        where $\brac{\phi_q}$ is the set of spin-orbitals from the non-canonical
        Hartree-Fock equations and $U_{qp}$ is an element in the unitary matrix
        $\vfg{U}$.
        In other words, we perform a unitary transformation from the
        non-canonical spin-orbitals to the new set of spin-orbitals.
        As the transformation is unitary, this means that the orthonormality
        condition in the new spin-orbitals is preserved from the old
        spin-orbitals.
        \begin{proof}[%
                Proof that a unitary transformation preserves orthonormality%
        ]
            Given a set of orthonormal basis states
            $\brac{\ket{\chi_{\alpha}}}_{\alpha = 1}^{L}$,
            we perform a unitary transformation from this basis set to a new
            basis set defined by
            \begin{align}
                \ket{\phi_p} = U_{\alpha p}\ket{\chi_{\alpha}},
            \end{align}
            where $U_{\alpha p}$ is an element in the unitary matrix $\vfg{U} \in
            \mathbb{C}^{L \times L}$ satisfying
            \begin{align}
                \vfg{U}^{\dagger}\vfg{U} = \vfg{U}\vfg{U}^{\dagger} = \1.
            \end{align}
            Taking the inner product between two states in the new basis set we
            get
            \begin{align}
                \braket{\phi_p}{\phi_q}
                = U^{*}_{\alpha p}
                \braket{\chi_{\alpha}}{\chi_{\beta}}
                U_{\beta q}
                = U^{*}_{\alpha p} U_{\beta q} \delta_{\alpha \beta}
                = U^{*}_{\alpha p} U_{\alpha q}
                = \delta_{pq},
            \end{align}
            which shows that the new basis set $\brac{\ket{\phi_p}}_{p = 1}^L$
            preserves the orthonormality after the unitary transfomation.
            % TODO: This proof should probably be moved to the quantum-mechanics
            % section.
        \end{proof}
        Furthermore, the orthonormality of the Slater determinants are also
        preserved as shown in the proof in the subsection on Slater determinants.
        % TODO: Reference this properly
        As the Lagrange multipliers $\lambda_{ji}$ are Hermitian, we can
        construct a matrix $\vfg{\Lambda}$ of the multipliers which will be
        Hermitian.
        This means that we can construct a diagonal matrix $\vfg{\energy}$ from
        the Lagrange multipliers using the spectral theorem, viz.
        \begin{align}
            \vfg{\Lambda} = \vfg{U} \vfg{\energy} \vfg{U}^{\dagger}.
        \end{align}
        This procedure is also known as \emph{Schur decomposition}.
        In tensor notation we introduce a Kronkecker Delta of rank 3 defined as
        \begin{align}
            \delta^{i}_{jk} = \begin{cases}
                1 & i = j = k, \\
                0 & else.
            \end{cases}
        \end{align}
        The Schur decomposition of the Lagrange multipliers then take on the
        form
        \begin{align}
            \lambda_{ij} = U_{ik} E_{kl} U^{*}_{jl}
            = U_{ik} \delta^{m}_{kl} \epsilon_{m} U^{*}_{jl},
        \end{align}
        where $\epsilon_m$ are the diagonal entries in the matrix $\vfg{E}$.
        Starting from the non-canonical Hartree-Fock equations in
        \autoref{eq:non-canonical-hartree-fock} and inserting the transformed
        spin-orbitals we get
        \begin{gather}
            \fock\ket{\phi_k} = \lambda_{jk}\ket{\phi_j}
            \\
            \implies
            \fock U^{*}_{kl}\ket{\psi_{l}} = \lambda_{jk} U^{*}_{jl} \ket{\psi_l}
            \\
            \implies
            U_{km} U^{*}_{kl} \fock\ket{\psi_l}
            = U_{km} \lambda_{jk} U^{*}_{jl} \ket{\psi_l}
            \\
            \implies
            \delta_{ml} \fock \ket{\psi_l}
            = \delta^{n}_{ml} \epsilon_{n} \ket{\psi_l}
            \\
            \implies
            \fock\ket{\psi_m}
            = \epsilon_{m} \ket{\psi_m},
            \label{eq:canonical-hartree-fock}
        \end{gather}
        where we are left with the \emph{canonical Hartree-Fock equations}.
        These equations constitute an eigenvalue equation that only depends on
        the choice of basis.
        That is, they make no assumption on which orbitals are occupied or not.
        As mentioned earlier, the Hartree-Fock equations does not guarantee that
        we find a global minimum.
        In fact, it does not even guarantee that we find a minimum!
        We might just as well stumble upon a saddle point.
        % TODO: Discuss how to determine if we have found a minimum.

        \subsection{Brillouin's theorem}
            Brillouin's theorem states that given an orthonormal single-particle
            basis $\brac{\ket{\phi_p}}_{i = 1}^{L}$, which is used to build a basis of
            Slater determinants $\brac{\ket{\Phi_I}}_{I = 1}^{N_s}$, then
            \begin{align}
                \bra{\Phi}\hamil\ket{\Phi^{a}_{i}} = 0,
            \end{align}
            is true iff the single-particle basis is found from solving the
            Hartree-Fock equations and $\ket{\Phi^{a}_{i}}$ is any singly
            excited determinant from the reference determinant $\ket{\Phi}$
            \cite{kvaal2017notes}.
            An important consequence of this is that all single excitations,
            from the reference state, can be neglected if we choose the
            Hartree-Fock reference state as our reference determinant.
            \begin{proof}
                We prove Brillouin's theorem directly by evaluating the matrix
                element
                \begin{align}
                    \bra{\Phi}\hamil\ket{\Phi^{a}_{i}}
                    &= \bra{\Phi}\onehamil\ket{\Phi^{a}_{i}}
                    + \frac{1}{4}\bra{\Phi}\twohamil\ket{\Phi^{a}_{i}}
                    = \oneten^{a}_{i} + \twoten^{aj}_{ij}
                    = \fockten^{a}_{i},
                \end{align}
                where we've used the Slater-Condon rules to evaluate the matrix
                elements.
                As the single-particle basis is the molecular orbitals found
                from solving the Hartree-Fock equations
                \begin{align}
                    \fock\ket{\phi_p} = \varepsilon_p\ket{\phi_p},
                \end{align}
                the Fock matrix is diagonal.
                This means that
                \begin{align}
                    \fockten^{a}_{i}
                    \equiv
                    \bra{\phi_a}\fock\ket{\phi_i}
                    = \varepsilon_i\braket{\phi_a}{\phi_i}
                    = 0,
                \end{align}
                as the molecular orbitals are orthonormal by construction.
            \end{proof}

    \section{Solving the Hartree-Fock equations in a basis}
        Having found the canonical Hartree-Fock equations, we are interested
        in utilizing the method in order to find molecular orbitals
        $\brac{\ket{\phi_p}}_{p = 1}^{L}$ serving as an improvement to our
        known atomic orbital basis $\brac{\ket{\chi_{\alpha}}}_{\alpha =
        1}^{K}$.
        We will in the following demonstrate three different procedures that
        lets us find the molecular orbitals.
        These procedures are related to the restrictions put on the
        spin-orbitals as discussed in
        \autoref{subsec:restrictions-on-spin-orbitals}.
        In fact, each procedure provides a way to choose which restriction we
        want on our molecular orbitals.
        First we'll discuss a general Hartree-Fock method which puts no
        restrictions on the molecular orbitals.
        This method leads to general spin-orbitals as shown in
        \autoref{eq:general-spin-orbital}.
        The second method is known as the \emph{restricted Hartree-Fock} method
        as it assumes restricted spin-orbitals.
        This leads to molecular orbitals that are restricted spin-orbitals as
        shown in \autoref{eq:restricted-spin-orbital}.
        Finally, we'll demonstrate the \emph{unrestricted Hartree-Fock method}
        yielding unrestricted spin-orbitals for the molecular orbitals as
        shown in \autoref{eq:unrestricted-spin-orbital}.

        \subsection{Hartree-Fock with general spin-orbitals}
            Given an atomic orbital basis, e.g., harmonic oscillator basis,
            $\brac{\ket{\chi_{\alpha}}}_{\alpha = 1}^{K}$ we wish to find an
            orthonomal basis of molecular orbitals $\brac{\ket{\phi_{p}}}_{p =
            1}^{L}$ satisfying the canonical Hartree-Fock equations.
            We can transform from the known atomic orbital basis to the unknown
            molecular orbital basis by
            \begin{align}
                \ket{\phi_p} = C_{\alpha p}\ket{\chi_{\alpha}},
            \end{align}
            where $\vfg{C} \in \mathbb{C}^{K\times L}$ is now our unknown
            coefficient matrix.
            By left-projecting with a state from our atomic orbital basis onto
            the canonical Hartree-Fock equations, we can create a set of
            equations in order to find the coefficients.
            \begin{gather}
                \bra{\chi_{\alpha}}\fock\ket{\phi_q}
                = \epsilon_{q} \braket{\chi_{\alpha}}{\phi_q}
                \\
                \implies
                \bra{\chi_{\alpha}}\fock\ket{\chi_{\beta}} C_{\beta q}
                = \epsilon_q C_{\beta q} \braket{\chi_{\alpha}}{\chi_{\beta}}.
            \end{gather}
            We denote the matrix elements of the Fock operator in the atomic
            orbital basis by
            \begin{align}
                \bra{\chi_{\alpha}}\fock\ket{\chi_{\beta}}
                \equiv \fockten_{\alpha \beta},
            \end{align}
            and the overlap in the atomic orbital basis by
            \begin{align}
                \braket{\chi_{\alpha}}{\chi_{\beta}} = \overlapten_{\alpha\beta}.
            \end{align}
            In the case of an orthonormal basis of atomic orbitals, the overlap
            matrix $\overlapmat \in \mathbb{C}^{K \times K}$ reduces to the identity
            matrix.
            The projected Hartree-Fock equations can then be written
            \begin{gather}
                \fockten_{\alpha\beta} C_{\beta q}
                = \overlapten_{\alpha \beta} C_{\beta q} \epsilon_{q}
                \\
                \implies
                \fockmat \vfg{C} = \overlapmat \vfg{C} \vfg{\epsilon},
                \label{eq:roothan-hall-general}
            \end{gather}
            where $\fockmat \in \mathbb{C}^{K \times K}$ is the \emph{Fock
            matrix} with elements from the atomic orbital basis on the Fock
            operator.
            The diagonal matrix $\vfg{\epsilon} = \diag(\epsilon_1, \dots,
            \epsilon_L)$ is the matrix with the eigenenergies from the canonical
            Hartree-Fock equation.
            The equation in \autoref{eq:roothan-hall-general} is known as the
            \emph{Roothan-Hall} equations \cite{roothan, hall}.
            They constitute a formulation of the integro-differential equations
            that are the canonical Hartree-Fock equations to a generalized
            eigenvalue equation formulated as matrices.
            An important point to note is that the Fock operator is an operator
            that includes the molecular orbitals which are dependent on the
            coefficient matrix $\vfg{C}$.
            That is,
            \begin{align}
                \fockten_{\alpha\beta}
                &= \bra{\chi_{\alpha}}\fock\ket{\chi_{\beta}}
                = \bra{\chi_{\alpha}}\onehamil\ket{\chi_{\beta}}
                +
                \bra{\chi_{\alpha}\phi_j}\twohamil\ket{\chi_{\beta}\phi_j}_{AS},
            \end{align}
            where $j$ only sums over the $N$ occupied indices in the ground state
            Slater determinant.
            We see that only the anti-symmetric two-body elements depends on the
            coefficient matrix.
            Formulating the elements in terms of the known atomic orbitals and
            the coefficient matrix we get
            \begin{align}
                \bra{\chi_{\alpha}\phi_j}\twohamil\ket{\chi_{\beta}\phi_j}
                &=
                C^{*}_{\gamma j} C_{\delta j}
                \bra{\chi_{\alpha}\chi_{\gamma}}\twohamil\ket{\chi_{\beta}\chi_{\delta}},
                \\
                \bra{\chi_{\alpha}\phi_j}\twohamil\ket{\phi_j\chi_{\beta}}
                &=
                C^{*}_{\gamma j} C_{\delta j}
                \bra{\chi_{\alpha}\chi_{\gamma}}\twohamil\ket{\chi_{\delta}\chi_{\beta}}.
            \end{align}
            Introducing the denisty matrix of the occupied orbitals
            \begin{align}
                D_{\gamma\delta} \equiv
                C^{*}_{\gamma j} C_{\delta j},
            \end{align}
            we can write the matrix elements of the Fock operator in terms of
            the atomic orbitals and the density matrix as
            \begin{align}
                \fockten_{\alpha\beta}
                &= \bra{\chi_{\alpha}}\onehamil\ket{\chi_{\beta}}
                +
                D_{\gamma\delta}
                \bra{\chi_{\alpha}\chi_{\gamma}}\twohamil\ket{\chi_{\beta}\chi_{\delta}}_{AS}.
            \end{align}

        \subsection{The restricted Hartree-Fock method}
        \subsection{The unrestricted Hartree-Fock method}


\section{Time-dependent Hartree-Fock theory}
    Time-dependent Hartree-Fock continues with the ansatz that the full
    many-body wave function $\ket*{\Psi(t)}$ is described by a single Slater
    determinant of $N$ single-particle states, viz.
    \begin{align}
        \ket*{\Psi(t)} = \ket*{\slat(t)}
        = \ket*{\phi_1(t) \phi_2(t)\dots \phi_N(t)},
    \end{align}
    where $\brac{\phi_p}$ is a basis of time-dependent molecular orbitals
    subject to the constraint that they are orthonormal in time.
    We now use the time-dependent variational principle as discussed in
    \autoref{sec:tdvp} in order to find the equations of motion for the system.
    The Lagrangian of the system is given by
    \begin{align}
        \lagrangianfunc{\slat, \slat^{*}, \lambda}
        = \mel*{\slat(t)}{\para{
            i\hslash\partial_t
            - \hamil(t)
        }}{\slat(t)}
        - \lambda_{ji}\para{
            \braket*{\phi_i(t)}{\phi_j(t)}
            - \delta_{ij}
        },
        \label{eq:lagrangian-tdhf}
    \end{align}
    where we keep the orthonormality condition in the time-dependent case as
    well, and we have ignored the explicit time-dependence in the functional
    arguments to the Lagrangian.
    The action functional of the time-dependent variational principle is given
    by
    \begin{align}
        S[\slat, \slat^{*}]
        =
        \int \dd t \lagrangianfunc{\slat, \slat^{*}, \lambda},
    \end{align}
    with the stationary condition over the first-order variations found from
    \begin{align}
        \delta S
        = \int\dd t \delta \lagrangianfunc{\slat, \slat^{*}, \lambda}
        = 0
        \implies
        \delta\lagrangianfunc{\slat, \slat^{*}, \lambda}
        = 0.
    \end{align}
    Computing the expectation value of the Hamiltonian in
    \autoref{eq:lagrangian-tdhf} yields the same expression for the reference
    energy as in \autoref{subsec:reference-energy}, but now with time-dependent
    operators and molecular orbitals.
    As the time-derivative is a Hermitian, single-particle operator, we find the
    expectation value to be
    \begin{align}
        \mel*{\slat(t)}{\partial_t}{\slat(t)}
        = \mel*{\phi_i(t)}{\partial_t}{\phi_i(t)}.
    \end{align}
    % TODO: Prove this in the appendix.
    Performing variations over a single molecular orbital at a time we have
    \begin{align}
        \tilde{\phi}_i(x, t)
        = \phi_i(x, t) + \delta_{ik}\epsilon\eta(x, t),
    \end{align}
    where again $\epsilon \in \mathbb{R}$ and $\eta(x, t)$ is a complex
    function.
    As done in the time-independent case, we will restrict our attention to
    variations over the complex conjugate of $\phi_i(x, t)$.
    Variations over $\phi_i(x, t)$ will yield the adjoint equation.
    The only new term in the variation over the time-dependent Lagrangian from
    the time-independent case is the variation over the time-derivative.
    We find
    \begin{align}
        \mel*{\tilde{\slat}(t)}{\partial_t}{\slat(t)}
        = \mel*{\phi_i(t)}{\partial_t}{\phi_i(t)}
        + \epsilon\mel*{\eta(t)}{\partial_t}{\phi_k(t)}.
    \end{align}
    From the stationary conditions of the action functional we find
    \begin{gather}
        \delta\lagrangianfunc{\slat, \tilde{\slat}^{*}, \lambda}
        = \lagrangianfunc{\slat, \tilde{\slat}^{*}, \lambda}
        - \lagrangianfunc{\slat, \slat^{*}, \lambda}
        = 0 \\
        \implies
        i\hslash\mel*{\eta}{\partial_t}{\phi_k}
        -\mel*{\eta}{\fock}{\phi_k}
        = \lambda_{jk}\braket*{\eta}{\phi_j},
    \end{gather}
    where we now for the sake of brevity removed the explicit time-dependence of
    the orbitals and the operators.
    Furthermore, we jumped straight from the variation of the energy functional
    to include the Fock-operator.
    As $\eta(x, t)$ was arbitrary, this must be valid for all choices of
    $\eta(x, t)$ and we can formulate the stationary condition as
    \begin{align}
        i\hslash\partial_t\ket*{\phi_k}
        - \fock\ket*{\phi_k}
        = \lambda_{jk}\ket*{\phi_j}.
        \label{eq:stationary-condition-1}
    \end{align}
    Projecting onto $\phi_l(x, t)$ and applying the constraint that the
    molecular orbitals are orthonormal we find an equation for the Lagrange
    multipliers,
    \begin{align}
        \lambda_{lk}
        = i\hslash\mel*{\phi_l}{\partial_t}{\phi_k}
        - \mel*{\phi_l}{\fock}{\phi_k}.
    \end{align}
    We now insert this expression for the Lagrange multipliers back into the
    stationary condition in \autoref{eq:stationary-condition-1}.
    This yields
    \begin{align}
        \hat{P}\brak{
            i\hslash\partial_t\ket*{\phi_k}
            - \fock\ket*{\phi_k}
        }
        = 0,
    \end{align}
    where we have now defined the projection operator
    \begin{align}
        \hat{P} \equiv \1 - \dyad{\phi_i}{\phi_i},
    \end{align}
    and we see that we have gotten rid of the Lagrange multipliers.
    Solving for the time-derivative of the molecular orbitals we find
    \begin{align}
        i\hslash\hat{P}\partial_t\ket*{\phi_k}
        = \hat{P}\fock\ket*{\phi_k}.
    \end{align}
    We now define an arbitrary time-dependent, Hermitian operator $\hat{Q}$
    in terms of a unitary transformation \cite{hochstuhl2014time}
    \begin{align}
        i\hslash\mel*{\phi_i}{\partial_t}{\phi_j}
        \equiv
        \mel*{\phi_i}{\hat{Q}}{\phi_j}.
    \end{align}
    We know that  Slater determinants are invariant up to a complex phase under
    unitary transformations, as seen in
    \autoref{lemma:unitary-transformation-slater}, this unitary transformation
    will leave the time-dependent Lagrangian invariant.
    Since $\hat{Q}$ was an arbitary, Hermitian operator and the Fock operator is
    Hermitian, we can choose $\hat{Q} = \fock$.
    Expanding the projection operator and inserting this choice, we get the
    time-dependent Hartree-Fock equations
    \begin{align}
        i\hslash\partial_t\ket*{\phi_k(t)}
        = \fock(t)\ket*{\phi_k(t)},
        \label{eq:tdhf}
    \end{align}
    where the molecular orbitals stays orthonormal in time.


\clearemptydoublepage

