\chapter{Hartree-Fock theory}
    \label{chap:hf}
    One can not tackle the subject of many-body theory without a discussion of
    the Hartree-Fock method. It serves as an excellent initial approximation,
    and in many cases the \emph{only} approximation, to the many-body
    wavefunction for a given system. It is a rather cheap method, in terms of
    computational intensity, and explains much of the underlying physics of a
    given system of many particles.

    \section{Time-independent Hartree-Fock theory}
        \label{sec:hf}
        Starting from the time-independent Schrödinger equation,
        \begin{align}
            \hamil\ket{\Psi} = \energy\ket{\Psi},
        \end{align}
        where $\hamil$ is the electronic Hamiltonian with one- and two-body
        operators.
        We know that the ground state of the one-body Hamiltonian will be a
        single Slater determinant as demonstrated in
        \autoref{sec:slater-determinants}.
        If the two-body interactions are weak, we can treat these contributions
        perturbatively and a single Slater determinant will serve as a good
        approximation to the full many-body wave function.\footnote{%
            We will see that it does not take much before the two-body
            interaction becomes a little more than just a small perturbation.
        }
        This motivates the approximation that the many-body wave function
        $\ket{\Psi}$ can be approximated by a single Slater determinant, viz.
        \begin{align}
            \ket{\Psi} = \ket{\slat} = \ket{\phi_1 \phi_2 \dots \phi_N},
        \end{align}
        where the \emph{molecular orbitals}\footnote{%
            The term molecular orbitals will be used exclusively to denote the
            ``Hartree-Fock'' orbitals.
        } $\brac{\phi_i}_{i = 1}^{N}$ are the
        primary unknowns, subject to the constraint that they are orthonormal,
        \begin{align}
            \braket{\phi_i}{\phi_j} = \delta_{ij}
            \implies
            \braket{\slat} = 1.
        \end{align}

        \subsection{The non-canonical Hartree-Fock equations}
            Starting from the variational principle, we define the energy
            functional
            \begin{align}
                \energyfunc{\slat, \slat^{*}}
                \equiv \mel{\slat}{\hamil}{\slat}
                =
                \mel{\phi_i}{\onehamil}{\phi_i}
                + \half\mel{\phi_i\phi_j}{\twohamil}{\phi_i\phi_j}_{AS},
                \label{eq:energy_func_hf}
            \end{align}
            found from the definition of the reference energy in
            \autoref{subsec:reference-energy}.
            Our task is now to find the molecular orbitals $\brac{\phi_i}$ that
            minimizes the energy functional, i.e., we find the stationary points
            of the energy functional as discussed in
            \autoref{sec:variational-principle}.
            As the energy functional $\energyfunc{\slat, \slat^{*}}$ does not
            incorporate the constraint that the molecular orbitals should be
            orthonormal, we use Lagrange's method of undetermined multipliers.
            This yields the Lagrangian functional
            \begin{align}
                \lagrangianfunc{\slat, \slat^{*}, \lambda}
                &= \energyfunc{\slat, \slat^{*}}
                - \lambda_{ji}\para{
                    \braket{\phi_i}{\phi_j}
                    - \delta_{ij}
                },
            \end{align}
            where $\lambda_{ji}$ are Lagrange multipliers, one for each
            constraint.
            As the Lagrangian functional is real and the constraint is
            Hermitian, the Lagrange multipliers can be chosen Hermitian as well.

            \begin{proof}[%
                    Proof that the Lagrange multipliers can be chosen Hermitian%
                ]
                We follow closely the derivation done by Mayer
                \cite{mayer2003simple}, we start by noticing that the constraint is
                Hermitian, i.e.,
                \begin{align}
                    \braket{\phi_i}{\phi_j} - \delta_{ij}
                    = \braket{\phi_j}{\phi_i}^{*} - \delta_{ji}.
                \end{align}
                As of now we have two independent Lagrange multipliers; one for the
                overlap $\braket{\phi_i}{\phi_j}$ and another for the the complex
                conjugate $\braket{\phi_j}{\phi_i}$.
                We can formulate the constraint for the real and imaginary part
                separately,
                \begin{align}
                    \Re\brac{
                        \braket{\phi_i}{\phi_j}
                    }
                    &=
                    \half\brac{
                        \braket{\phi_i}{\phi_j}
                        + \braket{\phi_j}{\phi_i}
                    }
                    = 0,
                    \\
                    \Im\brac{
                        \braket{\phi_i}{\phi_j}
                    }
                    &=
                    \frac{1}{2i}\brac{
                        \braket{\phi_i}{\phi_j}
                        - \braket{\phi_j}{\phi_i}
                    }
                    = 0.
                \end{align}
                Introducing two separate Lagrange multipliers $\mu_{ij}$ and
                $\nu_{ij}$ for the two latter conditions, we get
                \begin{align}
                    \mu_{ij}\Re\brac{
                        \braket{\phi_i}{\phi_j}
                    }
                    + \nu_{ij}\Im\brac{
                        \braket{\phi_i}{\phi_j}
                    }
                    &=
                    \half\brak{
                        \mu_{ij} - i\nu_{ij}
                    }
                    \braket{\phi_i}{\phi_j}
                    \nonumber
                    \\
                    &\qquad
                    + \half\brak{
                        \mu_{ij} + i\nu_{ij}
                    }
                    \braket{\phi_j}{\phi_i}.
                \end{align}
                We now choose our combined Lagrange multipliers to be
                \begin{align}
                    \lambda_{ji} &=
                    -\half\brak{
                        \mu_{ij} - i\nu_{ij}
                    }, \\
                    \lambda_{ij} &=
                    -\half\brak{
                        \mu_{ij} + i\nu_{ij}
                    },
                \end{align}
                which implies that $\lambda_{ji} = \lambda_{ij}^{*}$, which is
                what we wanted to show.
            \end{proof}

            We are now interested in finding a stationary point of the
            Lagrangian with respect to small variations in functional
            dependency, that is, the molecular orbitals and the Lagrange
            multipliers.
            The stationary conditions for the Lagrange multipliers yield the
            constraint that the molecular orbitals should be orthonormal, viz.
            \begin{gather}
                \dpd{}{\lambda_{ji}}\lagrangianfunc{\slat, \slat^{*}, \lambda}
                = 0
                \implies
                \braket{\phi_i}{\phi_j} = \delta_{ij},
            \end{gather}
            which are included in the end when we find solutions that are
            orthonormal as the constraint is included in the variation over the
            molecular orbitals \cite{kvaal2017notes}.
            The variation over a specific orbital $k$ is given by
            \begin{align}
                \tilde{\phi}_i(x) = \phi_{i}(x) + \delta_{ik}\epsilon\eta(x),
            \end{align}
            where $\epsilon \in \mathbb{R}$ is a small number and $\eta(x)$ is a
            single-particle function over some coordinate $x$.
            Note the use of the Kronecker-Delta to ensure that the variation
            only occurs for a single orbital at a time.
            This variation is similar for the complex conjugate of the
            molecular orbitals, but with $\phi^{*}_i(x)$ and $\eta^{*}(x)$
            instead.
            To avoid too much clutter, we will denote a variation over the
            molecular orbitals by a variation over the Slater determinants.
            That is, we take $\tilde{\Phi}(\vfg{x})$ to mean a variation over a
            single orbital $\phi_i(x)$ in $\Phi(\vfg{x})$, and similarly for the
            complex conjugate.
            Taylor expanding the Lagrangian functional,
            \begin{align}
                \lagrangianfunc{\tilde{\slat}, \tilde{\slat}^{*}, \lambda}
                &=
                \lagrangianfunc{\slat, \slat^{*}, \lambda}
                +
                \left.
                \dpd{
                    \lagrangianfunc{\tilde{\slat}, \slat^{*}, \lambda}
                }{\epsilon}
                \right\lvert_{\epsilon = 0}
                \epsilon
                \nonumber \\
                &\qquad
                +
                \left.
                \dpd{
                    \lagrangianfunc{\slat, \tilde{\slat}^{*}, \lambda}
                }{\epsilon}
                \right\lvert_{\epsilon = 0}
                \epsilon
                + \dots,
            \end{align}
            where the dots represent variations beyond the first order
            stationary condition.
            As discussed in the section on the variational principle, the first
            order variations does not guarantee that we have found a minimum,
            but they do it \emph{often enough} for us to not bother with second
            order variations or more \cite{szabo1996modern}.
            In lieu of this comforting thought, we proceed on our merry way by
            finding the stationary points using the method of functional
            derivatives as dicussed in \autoref{sec:variational-principle}.
            We will restrict ourselves to the variation over the complex
            conjugated orbitals as both variations yield the same equation
            adjointed of one another.
            % TODO: Is this an allowed word? Rethink this sentence.
            This yields
            \begin{align}
                \lagrangianfunc{\Phi, \tilde{\Phi}^{*}, \lambda}
                =
                \energyfunc{\Phi, \tilde{\Phi}^{*}}
                - \lambda_{ji}\brak{
                    \braket{\phi_i}{\phi_j}
                    + \epsilon \delta_{ik}\braket{\eta}{\phi_j}
                    - \delta_{ij}
                },
            \end{align}
            where we used the linearity of the inner product to split ut the
            variation over the orbital $\phi^{*}_i(x)$ in the constraint term.
            Keeping only first order variations in the energy functional we find
            \begin{align}
                \energyfunc{\Phi, \tilde{\Phi}^{*}}
                = \energyfunc{\Phi, \Phi^{*}}
                + \epsilon\para{
                    \mel{\eta}{\onehamil}{\phi_k}
                    + \mel{\eta\phi_j}{\twohamil}{\phi_k\phi_j}_{AS}
                }
                + \mathcal{O}(\epsilon^2),
            \end{align}
            where we've used the antisymmetric properties of the two-body
            elements to collect the two variations over the left-hand side of
            the matrix elements.
            Furthermore, we've collapsed one of the sums over the
            Kronecker-Delta in the variation in all terms.
            The variation in the Lagrangian functional can now be found by
            \begin{align}
                \delta\lagrangianfunc{\Phi, \tilde{\Phi}^{*}, \lambda}
                &=
                \lagrangianfunc{\Phi, \tilde{\Phi}^{*}, \lambda}
                -
                \lagrangianfunc{\Phi, \Phi^{*}, \lambda}
                \\
                &=
                \epsilon\brak{
                    \mel{\eta}{\onehamil}{\phi_k}
                    +
                    \mel{\eta\phi_j}{\twohamil}{\phi_k\phi_j}_{AS}
                    - \lambda_{jk}
                    \braket{\eta}{\phi_j}
                }.
            \end{align}
            Having collapsed one of the orbital sums to yield $\phi_k$ we now
            restrict the variation over the molecular orbitals to
            \begin{align}
                \delta\phi^{*}_k(x)
                = \tilde{\phi}^{*}_k(x)
                - \phi^{*}_k(x)
                = \epsilon\eta^{*}(x).
            \end{align}
            Computing the stationary point of the Lagrangian with variations
            over $\phi^{*}_k(x)$ now gives
            \begin{gather}
                \left.
                \dpd{
                    \lagrangianfunc{\slat, \tilde{\slat}^{*}, \lambda}
                }{\epsilon}
                \right\lvert_{\epsilon = 0}
                =
                \int\dd x\frac{
                    \delta\lagrangianfunc{\slat, \tilde{\slat}^{*}, \lambda}
                }{\delta \phi^{*}_k(x)}
                \eta^{*}(x)
                = 0
                \\
                \implies
                \mel{\eta}{\onehamil}{\phi_k}
                + \mel{\eta\phi_j}{\twohamil}{\phi_k\phi_j}_{AS}
                = \lambda_{jk}\braket{\eta}{\phi_j},
                \label{eq:stationary-lagrangian}
            \end{gather}
            which according to the fundamental lemma of calculus of variations
            \cite{jost1998calculus} must be valid for all variations
            $\epsilon\eta^{*}(x)$.
            We now introduce the single-particle \emph{Fock operator} by
            its matrix elements
            \begin{align}
                \mel{\phi_p}{\fock}{\phi_q}
                \equiv
                \mel{\phi_p}{\onehamil}{\phi_q}
                +
                \mel{\phi_p\phi_j}{\twohamil}{\phi_q\phi_j}_{AS},
                \label{eq:fock-mel}
            \end{align}
            where we use the Fermi vacuum formalism for the sums.
            This lets us write the stationary point of the Lagrangian functional
            from \autoref{eq:stationary-lagrangian} as
            \begin{align}
                \fock\ket{\phi_k}
                = \lambda_{jk}\ket{\phi_j}.
                \label{eq:non-canonical-hartree-fock}
            \end{align}
            This equation is known as the \emph{non-canonical Hartree-Fock
            equations}, where the molecular orbitals $\phi_k$ are orthonormal.

        \subsection{Canonical Hartree-Fock equations}
            Now, \autoref{eq:non-canonical-hartree-fock} will yield the
            stationary Hartree-Fock energy, but from
            \autoref{lemma:unitary-transformation-slater} we know that a unitary
            transformation of the Slater determinant $\Phi(\vfg{x})$ leaves the
            determinant invariant up to a complex phase.
            We therefore look for a new set of spin-orbitals $\brac{\psi_p}$
            which will diagonalize the non-canonical Hartree-Fock equations.
            We define the unitary transformation from the non-canonical orbitals
            $\brac{\phi_p}$ to the new spin-orbitals by
            \begin{align}
                \ket{\psi_p} = U_{qp}\ket{\phi_q}
                \iff
                \ket{\phi_r} = U^{*}_{pr}\ket{\psi_p},
            \end{align}
            where $U_{qp}$ is an element in the unitary matrix $\vfg{U}$.
            From \autoref{lemma:unitary-transformation-orthonormality} we know
            that the new basis set $\brac{\psi_{p}}$ will preserve the
            orthonormality of the original basis set $\brac{\phi_q}$.
            We construct the Hermitian matrix $\vfg{\Lambda}$ from the Hermitian
            Lagrange multipliers $\lambda_{ji}$.
            From the Schur decomposition \cite{mat-inf4130} we can write
            \begin{align}
                \vfg{\Lambda} = \vfg{U} \vfg{\energy} \vfg{U}^{\dagger},
            \end{align}
            where we from the spectral theorem know that the matrix
            $\vfg{\energy} = \diag(\epsilon_1, \dots)$ will be diagonal with the
            eigenvalues $\epsilon_m$ of $\vfg{\Lambda}$ on the diagonal
            \cite{mat-inf4130}.
            Using the Kronecker-Delta of rank 3 defined in
            \autoref{eq:rank-3-kd}, the Schur decomposition of the Lagrange
            multipliers then take on the form
            \begin{align}
                \lambda_{jk} = U_{ji} E_{il} U^{*}_{kl}
                = U_{ji} \delta^{m}_{il} \epsilon_m U^{*}_{kl}
                \implies
                \delta^{m}_{il} \epsilon_m
                = U^{*}_{ji} \lambda_{jk} U_{kl}.
            \end{align}
            Transforming from the non-canonical molecular orbitals in
            \autoref{eq:non-canonical-hartree-fock} to the new spin-orbital
            basis, we get
            \begin{gather}
                \fock\ket{\phi_k} = \lambda_{jk}\ket{\phi_j}
                \implies
                \fock U^{*}_{lk}\ket{\psi_{l}}
                = \lambda_{jk} U^{*}_{lj} \ket{\psi_l}
                \\
                \implies
                U^{*}_{lk} U_{km} \fock\ket{\psi_l}
                = U^{*}_{lj} \lambda_{jk} U_{km} \ket{\psi_l}
                \implies
                \delta_{ml} \fock \ket{\psi_l}
                = \delta^{n}_{lm} \epsilon_{n} \ket{\psi_l}
                \\
                \implies
                \fock\ket{\psi_m}
                = \epsilon_{m} \ket{\psi_m},
                \label{eq:canonical-hartree-fock}
            \end{gather}
            where we are left with the \emph{canonical Hartree-Fock equations},
            which constitutes an eigenvalue equation.
            It is worth re-iterating that the equations were found from first
            order variations in the variational principle and therefore does not
            guarantee that we have found a minimum.
            In order to categorize the stationary point we can perform higher order
            variations to determine if we have find a minimum, saddle point, or
            a maximum.
            However, we will not explore higher order variations any further.

        \subsection{The mean-field approximation}
            It is worth discussing the Fock operator defined in
            \autoref{eq:fock-mel} in more detail.

            First, if we look at a non-interacting system, that is,
            $\twoten^{pq}_{rs} = 0$ for all indices, then the canonical
            Hartree-Fock equations in \autoref{eq:canonical-hartree-fock}
            reduces to the time-independent Schrödinger equation for the
            orbitals.
            Furthermore, creating a Slater determinant from the spin-orbitals
            with the lowest eigenenergy yield the ground state of the
            non-interacting many-body problem.

            Moving back to the interacting system we adopt a common notation
            used from quantum chemistry \cite{szabo1996modern} by introducing
            the Coulomb operator
            \begin{align}
                \hat{J}
                = \int \dd x_2
                \phi^{*}_j(x_2)\twohamil(x, x_2)
                \phi_j(x_2),
                \label{eq:coulomb-operator}
            \end{align}
            where we denote the two-body operator $\twohamil(x_1, x_2)$ with
            explicit coordinates to describe which coordinates are to be
            integrated.
            The matrix elements of the Coulomb operator is given by
            \begin{align}
                \mel{\phi_i}{\hat{J}}{\phi_k}
                &= \int\dd x_1\dd x_2
                \phi^{*}_i(x_1)\phi^{*}_j(x_2)
                \twohamil(x_1, x_2)
                \phi_k(x_1)\phi_j(x_2)
                \\
                &=
                \mel{\phi_i\phi_j}{\twohamil}{\phi_k\phi_j}.
            \end{align}
            That is, the Coulomb operator is just the two-body elements, but
            formulated as a single-particle operator as the inner integral is
            pre-computed.
            The interpretation of the Coulomb operator is that all
            single-particle states $\phi_j(x)$ creates an average potential
            which an individual particle will interact with, hence the name
            ``mean-field potential''.
            Now, the second term arising from the antisymmetric two-body
            elements yields nonlocal effects as it cannot be described as an
            average potential in the same way as the Coulomb operator.
            The exchange potential is therefore defined in terms of its action
            on a spin-orbital $\phi_k(x)$,
            \begin{align}
                \hat{K}\ket{\phi_k}
                = \brak{
                    \int\dd x_2 \phi^{*}_j(x_2)\twohamil(x, x_2)\phi_k(x_2)
                }\phi_j(x).
                \label{eq:exchange-operator}
            \end{align}
            The exchange operator represents a strong deviation from classical
            mechanics as the potential experienced by a single-particle state
            depends on the value of the single-particle state in all of
            coordinate space.
            The matrix elements of the exchange operator is given by
            \begin{align}
                \mel{\phi_i}{\hat{K}}{\phi_k}
                &= \int\dd x_1\dd x_2
                \phi^{*}_i(x_1)\phi^{*}_j(x_2)
                \twohamil(x_1, x_2)
                \phi_j(x_1)\phi_k(x_2)
                \\
                &=
                \mel{\phi_i\phi_j}{\twohamil}{\phi_j\phi_k}.
            \end{align}
            The Fock operator can now be written
            \begin{align}
                \fock
                = \onehamil
                + \hat{J} - \hat{K},
            \end{align}
            where the Coulomb and exchange operators have reduced the
            two-particle contributions to a single-particle mean-field
            potential.
            These terms are the reason why the Hartree-Fock method is called a
            mean-field approximation.
            In terms of three-particle interactions, the Hartree-Fock method can
            be used in a similar fashion where the mean-field approximation now
            incorporates the three-particle interactions.
            % TODO: Find references for this.

        \subsection{Brillouin's theorem}
            An important result from the canonical Hartree-Fock equations is
            Brillouin's theorem.
            \begin{theorem}
                \label{theorem:brillouin}
                Let $\brac{\phi_p}$ be an orthonormal set of single-particle
                states found by solving the canonical Hartree-Fock equations,
                and sorted such that the Fock energies $\epsilon_1 \leq
                \epsilon_2 \leq \dots$, then building the reference Slater
                determinant $\ket{\slat}$ from the $N$ first single-particle
                states, we have that
                \begin{align}
                    \mel*{\slat}{\hamil}{\slat^{a}_{i}} = 0,
                \end{align}
                where $\ket*{\slat^{a}_{i}}$ is a singly-excited determinant.
                The converse is also true.
            \end{theorem}
            This theorem is important because it tells us that all single
            excitations from the reference state can be neglected if we choose
            the Hartree-Fock reference state as our reference determinant.
            Note that the theorem does not hold for arbitrary excitations and
            neither for single-excitations from an already excited state.
            \begin{proof}
                We prove the converse of Brillouin's theorem directly by
                evaluating the matrix elements
                \begin{align}
                    \mel*{\slat}{\hamil}{\slat^{a}_{i}}
                    &= \mel*{\phi_i}{\onehamil}{\phi_a}
                    + \mel*{\phi_i\phi_j}{\twohamil}{\phi_a\phi_j}_{AS}
                    = \mel*{\phi_i}{\fock}{\phi_a},
                \end{align}
                where we've used the Slater-Condon rules from
                \autoref{lemma:slater-condon-one-body} and
                \autoref{lemma:slater-condon-two-body} to evaluate the matrix
                elements.
                As the single-particle basis is the molecular orbitals found
                from solving the canonical Hartree-Fock equations, that the Fock
                matrix is diagonal.
                This means that
                \begin{align}
                    \mel*{\phi_i}{\fock}{\phi_a}
                    = \varepsilon_a\braket*{\phi_a}{\phi_i}
                    = 0,
                \end{align}
                where we've used that the molecular orbitals are orthonormal by
                construction.
                Proving the opposite proposition can be done by starting from
                the canonical Hartree-Fock equations and then using the
                Slater-Condon rules in reverse to get the original formulation
                of Brillouin's theorem.
                % TODO: Can you?
            \end{proof}


\section{Time-dependent Hartree-Fock theory}
    Time-dependent Hartree-Fock continues with the ansatz that the full
    many-body wave function $\ket*{\Psi(t)}$ is described by a single Slater
    determinant of $N$ single-particle states, viz.
    \begin{align}
        \ket*{\Psi(t)} = \ket*{\slat(t)}
        = \ket*{\phi_1(t) \phi_2(t)\dots \phi_N(t)},
    \end{align}
    where $\brac{\phi_p}$ is a basis of time-dependent molecular orbitals
    subject to the constraint that they are orthonormal in time.
    We now use the time-dependent variational principle as discussed in
    \autoref{sec:tdvp} in order to find the equations of motion for the system.
    The Lagrangian of the system is given by
    \begin{align}
        \lagrangianfunc{\slat, \slat^{*}, \lambda}
        = \mel*{\slat(t)}{\para{
            i\hslash\partial_t
            - \hamil(t)
        }}{\slat(t)}
        - \lambda_{ji}\para{
            \braket*{\phi_i(t)}{\phi_j(t)}
            - \delta_{ij}
        },
        \label{eq:lagrangian-tdhf}
    \end{align}
    where we keep the orthonormality condition in the time-dependent case as
    well, and we have ignored the explicit time-dependence in the functional
    arguments to the Lagrangian.
    The action functional of the time-dependent variational principle is given
    by
    \begin{align}
        S[\slat, \slat^{*}]
        =
        \int \dd t \lagrangianfunc{\slat, \slat^{*}, \lambda},
    \end{align}
    with the stationary condition over the first-order variations found from
    \begin{align}
        \delta S
        = \int\dd t \delta \lagrangianfunc{\slat, \slat^{*}, \lambda}
        = 0
        \implies
        \delta\lagrangianfunc{\slat, \slat^{*}, \lambda}
        = 0.
    \end{align}
    Computing the expectation value of the Hamiltonian in
    \autoref{eq:lagrangian-tdhf} yields the same expression for the reference
    energy as in \autoref{subsec:reference-energy}, but now with time-dependent
    operators and molecular orbitals.
    As the time-derivative is a Hermitian, single-particle operator, we find the
    expectation value to be
    \begin{align}
        \mel*{\slat(t)}{\partial_t}{\slat(t)}
        = \mel*{\phi_i(t)}{\partial_t}{\phi_i(t)}.
    \end{align}
    % TODO: Prove this in the appendix.
    Performing variations over a single molecular orbital at a time we have
    \begin{align}
        \tilde{\phi}_i(x, t)
        = \phi_i(x, t) + \delta_{ik}\epsilon\eta(x, t),
    \end{align}
    where again $\epsilon \in \mathbb{R}$ and $\eta(x, t)$ is a complex
    function.
    As done in the time-independent case, we will restrict our attention to
    variations over the complex conjugate of $\phi_i(x, t)$.
    Variations over $\phi_i(x, t)$ will yield the adjoint equation.
    The only new term in the variation over the time-dependent Lagrangian from
    the time-independent case is the variation over the time-derivative.
    We find
    \begin{align}
        \mel*{\tilde{\slat}(t)}{\partial_t}{\slat(t)}
        = \mel*{\phi_i(t)}{\partial_t}{\phi_i(t)}
        + \epsilon\mel*{\eta(t)}{\partial_t}{\phi_k(t)}.
    \end{align}
    From the stationary conditions of the action functional we find
    \begin{gather}
        \delta\lagrangianfunc{\slat, \tilde{\slat}^{*}, \lambda}
        = \lagrangianfunc{\slat, \tilde{\slat}^{*}, \lambda}
        - \lagrangianfunc{\slat, \slat^{*}, \lambda}
        = 0 \\
        \implies
        i\hslash\mel*{\eta}{\partial_t}{\phi_k}
        -\mel*{\eta}{\fock}{\phi_k}
        = \lambda_{jk}\braket*{\eta}{\phi_j},
    \end{gather}
    where we now for the sake of brevity removed the explicit time-dependence of
    the orbitals and the operators.
    Furthermore, we jumped straight from the variation of the energy functional
    to include the Fock-operator.
    As $\eta(x, t)$ was arbitrary, this must be valid for all choices of
    $\eta(x, t)$ and we can formulate the stationary condition as
    \begin{align}
        i\hslash\partial_t\ket*{\phi_k}
        - \fock\ket*{\phi_k}
        = \lambda_{jk}\ket*{\phi_j}.
        \label{eq:stationary-condition-1}
    \end{align}
    Projecting onto $\phi_l(x, t)$ and applying the constraint that the
    molecular orbitals are orthonormal we find an equation for the Lagrange
    multipliers,
    \begin{align}
        \lambda_{lk}
        = i\hslash\mel*{\phi_l}{\partial_t}{\phi_k}
        - \mel*{\phi_l}{\fock}{\phi_k}.
    \end{align}
    We now insert this expression for the Lagrange multipliers back into the
    stationary condition in \autoref{eq:stationary-condition-1}.
    This yields
    \begin{align}
        \hat{P}\brak{
            i\hslash\partial_t\ket*{\phi_k}
            - \fock\ket*{\phi_k}
        }
        = 0,
    \end{align}
    where we have now defined the projection operator
    \begin{align}
        \hat{P} \equiv \1 - \dyad{\phi_i}{\phi_i},
    \end{align}
    and we see that we have gotten rid of the Lagrange multipliers.
    Solving for the time-derivative of the molecular orbitals we find
    \begin{align}
        i\hslash\hat{P}\partial_t\ket*{\phi_k}
        = \hat{P}\fock\ket*{\phi_k}.
    \end{align}
    We now define an arbitrary time-dependent, Hermitian operator $\hat{Q}$
    in terms of a unitary transformation \cite{hochstuhl2014time}
    \begin{align}
        i\hslash\mel*{\phi_i}{\partial_t}{\phi_j}
        \equiv
        \mel*{\phi_i}{\hat{Q}}{\phi_j}.
    \end{align}
    We know that  Slater determinants are invariant up to a complex phase under
    unitary transformations, as seen in
    \autoref{lemma:unitary-transformation-slater}, this unitary transformation
    will leave the time-dependent Lagrangian invariant.
    Since $\hat{Q}$ was an arbitary, Hermitian operator and the Fock operator is
    Hermitian, we can choose $\hat{Q} = \fock$.
    Expanding the projection operator and inserting this choice, we get the
    time-dependent Hartree-Fock equations
    \begin{align}
        i\hslash\partial_t\ket*{\phi_k(t)}
        = \fock(t)\ket*{\phi_k(t)},
        \label{eq:tdhf}
    \end{align}
    where the molecular orbitals stays orthonormal in time.


\clearemptydoublepage

