\chapter{Hartree-Fock theory}
    \label{chap:hf}
    One can not tackle the subject of many-body theory without a discussion of
    the Hartree-Fock method. It serves as an excellent initial approximation,
    and in many cases the \emph{only} approximation, to the many-body
    wavefunction for a given system. It is a rather cheap method, in terms of
    computational intensity, and explains much of the underlying physics of a
    given system of many particles.

    \section{Time-independent Hartree-Fock theory}
        Starting from the time-independent Schrödinger equation,
        \begin{align}
            \hamil\ket{\Psi} = \energy\ket{\Psi},
        \end{align}
        where $\hamil$ is the electronic Hamiltonian with one- and two-body
        operators.
        We know that the ground state of the one-body Hamiltonian will be a
        single Slater determinant as demonstrated in
        \autoref{sec:slater-determinants}.
        If the two-body interactions are weak, we can treat these contributions
        perturbatively and a single Slater determinant will serve as a good
        approximation to the full many-body wave function.\footnote{%
            We will see that it does not take much before the two-body
            interaction becomes a little more than just a small perturbation.
        }
        This motivates the approximation that the many-body wave function
        $\ket{\Psi}$ can be approximated by a single Slater determinant, viz.
        \begin{align}
            \ket{\Psi} = \ket{\slat} = \ket{\phi_1, \phi_2, \dots, \phi_N},
        \end{align}
        where the \emph{molecular orbitals} $\brac{\phi_i}_{i = 1}^{N}$ are the
        primary unknowns, subject to the constraint that they are orthonormal,
        \begin{align}
            \braket{\phi_i}{\phi_j} = \delta_{ij}
            \implies
            \braket{\slat} = 1.
        \end{align}

        \subsection{The non-canonical Hartree-Fock equations}
            Starting from the variational principle, we define the energy
            functional
            \begin{align}
                \energyfunc{\slat, \slat^{*}}
                \equiv \mel{\slat}{\hamil}{\slat}
                =
                \mel{\phi_i}{\onehamil}{\phi_i}
                + \half\mel{\phi_i\phi_j}{\twohamil}{\phi_i\phi_j}_{AS},
                \label{eq:energy_func_hf}
            \end{align}
            found from the definition of the reference energy in
            \autoref{subsec:reference-energy}.
            Our task is now to find the molecular orbitals $\brac{\phi_i}_{i =
            1}^{N}$ that minimizes the energy functional, i.e., we find the
            stationary points of the energy functional as discussed in
            \autoref{sec:variational-principle}.
            %By performing a variation in the Slater determinant,
            %\begin{align}
            %    \slat \to \slat + \delta\slat,
            %\end{align}
            %we find the that the energy functional is changed by
            %\begin{align}
            %    \energyfunc{\slat + \delta\slat}
            %    &=
            %    \bra{\slat + \delta\slat}\hamil\ket{\slat + \delta\slat}
            %    \\
            %    &= \energyfunc{\slat}
            %    + \bra{\delta\slat}\hamil\ketslat + \braslat\hamil\ket{\delta\slat}
            %    + \dots
            %    \\
            %    &= \energyfunc{\slat} + \delta\energyfunc{\slat}
            %    + \dots,
            %\end{align}
            %where the \emph{first variation} in $\energyfunc{\slat}$ is given by
            %\begin{align}
            %    \delta\energyfunc{\slat}
            %    \equiv
            %    \bra{\delta\slat}\hamil\ketslat + \braslat\hamil\ket{\delta\slat},
            %\end{align}
            %where we treat $\delta$ as a linear differential operator.
            %Higher order variations are ignored and we are thus only interested in
            %finding the Slater determinant, $\ketslat$, for which
            %\begin{align}
            %    \delta\energyfunc{\slat} = 0,
            %\end{align}
            %i.e., the stationary point of the energy functional in terms of the
            %function $\slat$.
            As the energy functional $\energyfunc{\slat, \slat^{*}}$ does not
            incorporate the constraint that the molecular orbitals should be
            orthonormal, we use Lagrange's method of undetermined multipliers.
            This yields the Lagrangian functional
            \begin{align}
                \lagrangianfunc{\slat, \slat^{*}, \lambda}
                &= \energyfunc{\slat, \slat^{*}}
                - \lambda_{ji}\para{
                    \braket{\phi_i}{\phi_j}
                    - \delta_{ij}
                },
            \end{align}
            where $\lambda_{ji}$ are Lagrange multipliers, one for each
            constraint.
            As the Lagrangian functional is real and the constraint is
            Hermitian, the Lagrange multipliers can be chosen Hermitian as well.

            \begin{proof}[%
                    Proof that the Lagrange multipliers can be chosen Hermitian%
                ]
                We follow closely the derivation done by Mayer
                \cite{mayer2003simple}, we start by noticing that the constraint is
                Hermitian, i.e.,
                \begin{align}
                    \braket{\phi_i}{\phi_j} - \delta_{ij}
                    = \braket{\phi_j}{\phi_i}^{*} - \delta_{ji}.
                \end{align}
                As of now we have two independent Lagrange multipliers; one for the
                overlap $\braket{\phi_i}{\phi_j}$ and another for the the complex
                conjugate $\braket{\phi_j}{\phi_i}$.
                We can formulate the constraint for the real and imaginary part
                separately,
                \begin{align}
                    \Re\brac{
                        \braket{\phi_i}{\phi_j}
                    }
                    &=
                    \half\brac{
                        \braket{\phi_i}{\phi_j}
                        + \braket{\phi_j}{\phi_i}
                    }
                    = 0,
                    \\
                    \Im\brac{
                        \braket{\phi_i}{\phi_j}
                    }
                    &=
                    \frac{1}{2i}\brac{
                        \braket{\phi_i}{\phi_j}
                        - \braket{\phi_j}{\phi_i}
                    }
                    = 0.
                \end{align}
                Introducing two separate Lagrange multipliers $\mu_{ij}$ and
                $\nu_{ij}$ for the two latter conditions, we get
                \begin{align}
                    \mu_{ij}\Re\brac{
                        \braket{\phi_i}{\phi_j}
                    }
                    + \nu_{ij}\Im\brac{
                        \braket{\phi_i}{\phi_j}
                    }
                    &=
                    \half\brak{
                        \mu_{ij} - i\nu_{ij}
                    }
                    \braket{\phi_i}{\phi_j}
                    \nonumber
                    \\
                    &\qquad
                    + \half\brak{
                        \mu_{ij} + i\nu_{ij}
                    }
                    \braket{\phi_j}{\phi_i}.
                \end{align}
                We now choose our combined Lagrange multipliers to be
                \begin{align}
                    \lambda_{ji} &=
                    -\half\brak{
                        \mu_{ij} - i\nu_{ij}
                    }, \\
                    \lambda_{ij} &=
                    -\half\brak{
                        \mu_{ij} + i\nu_{ij}
                    },
                \end{align}
                which implies that $\lambda_{ji} = \lambda_{ij}^{*}$, which is
                what we wanted to show.
            \end{proof}

            We are now interested in finding a stationary point of the
            Lagrangian with respect to small variations in functional
            dependency, that is, the molecular orbitals and the Lagrange
            multipliers.
            The stationary conditions for the Lagrange multipliers yield the
            constraint that the molecular orbitals should be orthonormal, viz.
            \begin{gather}
                \dpd{}{\lambda_{ji}}\lagrangianfunc{\slat, \slat^{*}, \lambda}
                = 0
                \implies
                \braket{\phi_i}{\phi_j} = \delta_{ij},
            \end{gather}
            which are included in the end when we find solutions that are
            orthonormal as the constraint is included in the variation over the
            molecular orbitals \cite{kvaal2017notes}.
            The variation over a specific orbital $k$ is given by
            \begin{align}
                \tilde{\phi}_i(x) = \phi_{i}(x) + \delta_{ik}\epsilon\eta(x),
            \end{align}
            where $\epsilon \in \mathbb{R}$ is a small number and $\eta(x)$ is a
            single-particle function over some coordinate $x$.
            Note the use of the Kronecker-Delta to ensure that the variation
            only occurs for a single orbital at a time.
            This variation is similar for the complex conjugate of the
            molecular orbitals, but with $\phi^{*}_i(x)$ and $\eta^{*}(x)$
            instead.
            To avoid too much clutter, we will denote a variation over the
            molecular orbitals by a variation over the Slater determinants.
            That is, we take $\tilde{\Phi}(\vfg{x})$ to mean a variation over a
            single orbital $\phi_i(x)$ in $\Phi(\vfg{x})$, and similarly for the
            complex conjugate.
            Taylor expanding the Lagrangian functional,
            \begin{align}
                \lagrangianfunc{\tilde{\slat}, \tilde{\slat}^{*}, \lambda}
                &=
                \lagrangianfunc{\slat, \slat^{*}, \lambda}
                +
                \left.
                \dpd{
                    \lagrangianfunc{\tilde{\slat}, \slat^{*}, \lambda}
                }{\epsilon}
                \right\lvert_{\epsilon = 0}
                \epsilon
                \nonumber \\
                &\qquad
                +
                \left.
                \dpd{
                    \lagrangianfunc{\slat, \tilde{\slat}^{*}, \lambda}
                }{\epsilon}
                \right\lvert_{\epsilon = 0}
                \epsilon
                + \dots,
            \end{align}
            where the dots represent variations beyond the first order
            stationary condition.
            As discussed in the section on the variational principle, the first
            order variations does not guarantee that we have found a minimum,
            but they do it \emph{often enough} for us to not bother with second
            order variations or more \cite{szabo1996modern}.
            In lieu of this comforting thought, we proceed on our merry way by
            finding the stationary points using the method of functional
            derivatives as dicussed in \autoref{sec:variational-principle}.
            We will restrict ourselves to the variation over the complex
            conjugated orbitals as both variations yield the same equation
            adjointed of one another.
            % TODO: Is this an allowed word? Rethink this sentence.
            This yields
            \begin{align}
                \lagrangianfunc{\Phi, \tilde{\Phi}^{*}, \lambda}
                =
                \energyfunc{\Phi, \tilde{\Phi}^{*}}
                - \lambda_{ji}\brak{
                    \braket{\phi_i}{\phi_j}
                    + \epsilon \delta_{ik}\braket{\eta}{\phi_j}
                    - \delta_{ij}
                },
            \end{align}
            where we used the linearity of the inner product to split ut the
            variation over the orbital $\phi^{*}_i(x)$ in the constraint term.
            Keeping only first order variations in the energy functional we find
            \begin{align}
                \energyfunc{\Phi, \tilde{\Phi}^{*}}
                = \energyfunc{\Phi, \Phi^{*}}
                + \epsilon\para{
                    \mel{\eta}{\onehamil}{\phi_k}
                    + \mel{\eta\phi_j}{\twohamil}{\phi_k\phi_j}_{AS}
                }
                + \mathcal{O}(\epsilon^2),
            \end{align}
            where we've used the antisymmetric properties of the two-body
            elements to collect the two variations over the left-hand side of
            the matrix elements.
            Furthermore, we've collapsed one of the sums over the
            Kronecker-Delta in the variation in all terms.
            The variation in the Lagrangian functional can be now be found by
            \begin{align}
                \delta\lagrangianfunc{\Phi, \tilde{\Phi}^{*}, \lambda}
                &=
                \lagrangianfunc{\Phi, \tilde{\Phi}^{*}, \lambda}
                -
                \lagrangianfunc{\Phi, \Phi^{*}, \lambda}
                \\
                &=
                \epsilon\brak{
                    \mel{\eta}{\onehamil}{\phi_k}
                    +
                    \mel{\eta\phi_j}{\twohamil}{\phi_k\phi_j}_{AS}
                    - \lambda_{jk}
                    \braket{\eta}{\phi_j}
                }.
            \end{align}
            Having collapsed one of the orbital sums to yield $\phi_k$ we now
            restrict the variation over the molecular orbitals to
            \begin{align}
                \delta\phi^{*}_k(x)
                = \tilde{\phi}^{*}_k(x)
                - \phi^{*}_k(x)
                = \epsilon\eta^{*}(x).
            \end{align}
            Computing the stationary point of the Lagrangian with variations
            over $\phi^{*}_k(x)$ now gives
            \begin{gather}
                \int\dd x\frac{
                    \delta\lagrangianfunc{\Phi, \tilde{\Phi}^{*}, \lambda}
                }{\delta \phi^{*}_k(x)}
                \eta^{*}(x)
                = 0
                \\
                \implies
                \mel{\eta}{\onehamil}{\phi_k}
                + \mel{\eta\phi_j}{\twohamil}{\phi_k\phi_j}_{AS}
                = \lambda_{jk}\braket{\eta}{\phi_j},
                \label{eq:stationary-lagrangian}
            \end{gather}
            which according to the fundamental lemma of calculus
            \cite{wiki:fundamental-lemma} must be valid for all variations
            $\epsilon\eta^{*}(x)$.
            We now introduce the single-particle \emph{Fock operator} by
            its matrix elements
            \begin{align}
                \mel{\phi_p}{\fock}{\phi_q}
                \equiv
                \mel{\phi_p}{\onehamil}{\phi_q}
                +
                \mel{\phi_p\phi_j}{\twohamil}{\phi_q\phi_j}_{AS},
            \end{align}
            where we use the Fermi vacuum formalism for the sums.
            This lets us write the stationary point of the Lagrangian functional
            from \autoref{eq:stationary-lagrangian} as
            \begin{align}
                \fock\ket{\phi_k}
                = \lambda_{jk}\ket{\phi_j}.
                \label{eq:non-canonical-hartree-fock}
            \end{align}
            This equation is known as the \emph{non-canonical Hartree-Fock
            equations}, where the molecular orbitals $\phi_k$ are orthonormal.

            % TODO: Discuss the Fock operator.

        \subsection{The mean-field approximation}


        \subsection{Canonical Hartree-Fock equations}
            This equation will yield the correct Hartree-Fock energy, but we are
            interested in an eigenvalue equation without a summation over the matrix
            of Lagrange multipliers.
            To reach this goal we define a new set of spin-orbitals
            $\brac{\psi_p}$ which is given by
            \begin{align}
                \ket{\psi_p} = U_{qp}\ket{\phi_q},
            \end{align}
            where $\brac{\phi_q}$ is the set of spin-orbitals from the non-canonical
            Hartree-Fock equations and $U_{qp}$ is an element in the unitary matrix
            $\vfg{U}$.
            In other words, we perform a unitary transformation from the
            non-canonical spin-orbitals to the new set of spin-orbitals.
            From \autoref{lemma:unitary-transformation-orthonormality} we know
            that the new basis set $\brac{\psi_{p}}$ will preserve the
            orthonormality of the original basis set $\brac{\phi_q}$.
            Furthermore, the orthonormality of the Slater determinants are also
            preserved as stated in
            \autoref{lemma:unitary-transformation-slater}.
            As the Lagrange multipliers $\lambda_{ji}$ are Hermitian, we can
            construct a matrix $\vfg{\Lambda}$ of the multipliers which will be
            Hermitian.
            This means that we can construct a diagonal matrix $\vfg{\energy}$ from
            the Lagrange multipliers using the spectral theorem, viz.
            \begin{align}
                \vfg{\Lambda} = \vfg{U} \vfg{\energy} \vfg{U}^{\dagger}.
            \end{align}
            This procedure is also known as \emph{Schur decomposition}.
            Using the Kronecker-Delta of rank 3 defined in
            \autoref{eq:rank-3-kd}, the Schur decomposition of the Lagrange
            multipliers then take on the form
            \begin{align}
                \lambda_{ij} = U_{ik} E_{kl} U^{*}_{jl}
                = U_{ik} \delta^{m}_{kl} \epsilon_{m} U^{*}_{jl},
            \end{align}
            where $\epsilon_m$ are the diagonal entries in the matrix $\vfg{E}$.
            Starting from the non-canonical Hartree-Fock equations in
            \autoref{eq:non-canonical-hartree-fock} and inserting the transformed
            spin-orbitals we get
            \begin{gather}
                \fock\ket{\phi_k} = \lambda_{jk}\ket{\phi_j}
                \\
                \implies
                \fock U^{*}_{kl}\ket{\psi_{l}} = \lambda_{jk} U^{*}_{jl} \ket{\psi_l}
                \\
                \implies
                U_{km} U^{*}_{kl} \fock\ket{\psi_l}
                = U_{km} \lambda_{jk} U^{*}_{jl} \ket{\psi_l}
                \\
                \implies
                \delta_{ml} \fock \ket{\psi_l}
                = \delta^{n}_{ml} \epsilon_{n} \ket{\psi_l}
                \\
                \implies
                \fock\ket{\psi_m}
                = \epsilon_{m} \ket{\psi_m},
                \label{eq:canonical-hartree-fock}
            \end{gather}
            where we are left with the \emph{canonical Hartree-Fock equations}.
            These equations constitute an eigenvalue equation that only depends on
            the choice of basis.
            That is, they make no assumption on which orbitals are occupied or not.
            As mentioned earlier, the Hartree-Fock equations does not guarantee that
            we find a global minimum.
            In fact, it does not even guarantee that we find a minimum!
            We might just as well stumble upon a saddle point.
            % TODO: Discuss how to determine if we have found a minimum.

        \subsection{Brillouin's theorem}
            Brillouin's theorem states that given an orthonormal single-particle
            basis $\brac{\ket{\phi_p}}_{i = 1}^{L}$, which is used to build a basis of
            Slater determinants $\brac{\ket{\Phi_I}}_{I = 1}^{N_s}$, then
            \begin{align}
                \bra{\Phi}\hamil\ket{\Phi^{a}_{i}} = 0,
            \end{align}
            is true iff the single-particle basis is found from solving the
            Hartree-Fock equations and $\ket{\Phi^{a}_{i}}$ is any singly
            excited determinant from the reference determinant $\ket{\Phi}$
            \cite{kvaal2017notes}.
            An important consequence of this is that all single excitations,
            from the reference state, can be neglected if we choose the
            Hartree-Fock reference state as our reference determinant.
            \begin{proof}
                We prove Brillouin's theorem directly by evaluating the matrix
                element
                \begin{align}
                    \bra{\Phi}\hamil\ket{\Phi^{a}_{i}}
                    &= \bra{\Phi}\onehamil\ket{\Phi^{a}_{i}}
                    + \frac{1}{4}\bra{\Phi}\twohamil\ket{\Phi^{a}_{i}}
                    = \oneten^{a}_{i} + \twoten^{aj}_{ij}
                    = \fockten^{a}_{i},
                \end{align}
                where we've used the Slater-Condon rules to evaluate the matrix
                elements.
                As the single-particle basis is the molecular orbitals found
                from solving the Hartree-Fock equations
                \begin{align}
                    \fock\ket{\phi_p} = \varepsilon_p\ket{\phi_p},
                \end{align}
                the Fock matrix is diagonal.
                This means that
                \begin{align}
                    \fockten^{a}_{i}
                    \equiv
                    \bra{\phi_a}\fock\ket{\phi_i}
                    = \varepsilon_i\braket{\phi_a}{\phi_i}
                    = 0,
                \end{align}
                as the molecular orbitals are orthonormal by construction.
            \end{proof}

    \section{Solving the Hartree-Fock equations in a basis}
        Having found the canonical Hartree-Fock equations, we are interested
        in utilizing the method in order to find molecular orbitals
        $\brac{\ket{\phi_p}}_{p = 1}^{L}$ serving as an improvement to our
        known atomic orbital basis $\brac{\ket{\chi_{\alpha}}}_{\alpha =
        1}^{K}$.
        We will in the following demonstrate three different procedures that
        lets us find the molecular orbitals.
        These procedures are related to the restrictions put on the
        spin-orbitals as discussed in
        \autoref{subsec:restrictions-on-spin-orbitals}.
        In fact, each procedure provides a way to choose which restriction we
        want on our molecular orbitals.
        First we'll discuss a general Hartree-Fock method which puts no
        restrictions on the molecular orbitals.
        This method leads to general spin-orbitals as shown in
        \autoref{eq:general-spin-orbital}.
        The second method is known as the \emph{restricted Hartree-Fock} method
        as it assumes restricted spin-orbitals.
        This leads to molecular orbitals that are restricted spin-orbitals as
        shown in \autoref{eq:restricted-spin-orbital}.
        Finally, we'll demonstrate the \emph{unrestricted Hartree-Fock method}
        yielding unrestricted spin-orbitals for the molecular orbitals as
        shown in \autoref{eq:unrestricted-spin-orbital}.

        \subsection{Hartree-Fock with general spin-orbitals}
            Given an atomic orbital basis, e.g., harmonic oscillator basis,
            $\brac{\ket{\chi_{\alpha}}}_{\alpha = 1}^{K}$ we wish to find an
            orthonomal basis of molecular orbitals $\brac{\ket{\phi_{p}}}_{p =
            1}^{L}$ satisfying the canonical Hartree-Fock equations.
            We can transform from the known atomic orbital basis to the unknown
            molecular orbital basis by
            \begin{align}
                \ket{\phi_p} = C_{\alpha p}\ket{\chi_{\alpha}},
            \end{align}
            where $\vfg{C} \in \mathbb{C}^{K\times L}$ is now our unknown
            coefficient matrix.
            The orthonormality condition of the molecular orbitals can no be
            formulated as
            \begin{align}
                \braket{\phi_p}{\phi_q}
                &= C^{*}_{\alpha p} C_{\beta q}
                \braket{\chi_{\alpha}}{\chi_{\beta}}
                = C^{*}_{\alpha p} \overlapten_{\alpha \beta}
                C_{\beta q}
                = \delta_{pq},
            \end{align}
            where we've introduced the overlap matrix $\overlapmat$ with
            elements
            \begin{align}
                \braket{\chi_{\alpha}}{\chi_{\beta}} = \overlapten_{\alpha\beta}.
            \end{align}
            By left-projecting with a state from our atomic orbital basis onto
            the canonical Hartree-Fock equations, we can create a set of
            equations in order to find the coefficients.
            \begin{gather}
                \mel{\chi_{\alpha}}{\fock}{\phi_q}
                = \epsilon_{q} \braket{\chi_{\alpha}}{\phi_q}
                \\
                \implies
                \mel{\chi_{\alpha}}{\fock}{\chi_{\beta}} C_{\beta q}
                = \epsilon_q C_{\beta q} \overlapten_{\alpha \beta}.
            \end{gather}
            We denote the matrix elements of the Fock operator in the atomic
            orbital basis by
            \begin{align}
                \mel{\chi_{\alpha}}{\fock}{\chi_{\beta}}
                \equiv \fockten_{\alpha \beta}.
            \end{align}
            In the case of an orthonormal basis of atomic orbitals, the overlap
            matrix $\overlapmat \in \mathbb{C}^{K \times K}$ reduces to the identity
            matrix.
            The projected Hartree-Fock equations can then be written
            \begin{gather}
                \fockten_{\alpha\beta} C_{\beta q}
                = \overlapten_{\alpha \beta} C_{\beta q} \epsilon_{q}
                \\
                \implies
                \fockmat \vfg{C} = \overlapmat \vfg{C} \vfg{\epsilon},
                \label{eq:roothan-hall-general}
            \end{gather}
            where $\fockmat \in \mathbb{C}^{K \times K}$ is the \emph{Fock
            matrix} with elements from the atomic orbital basis on the Fock
            operator.
            The diagonal matrix $\vfg{\epsilon} = \diag(\epsilon_1, \dots,
            \epsilon_L)$ is the matrix with the eigenenergies from the canonical
            Hartree-Fock equation.
            The equation in \autoref{eq:roothan-hall-general} is known as the
            \emph{Roothan-Hall} equations \cite{roothan, hall}.
            They constitute a formulation of the integro-differential equations
            that are the canonical Hartree-Fock equations to a generalized
            eigenvalue equation formulated as matrices.
            An important point to note is that the Fock operator is an operator
            that includes the molecular orbitals which are dependent on the
            coefficient matrix $\vfg{C}$.
            That is,
            \begin{align}
                \fockten_{\alpha\beta}
                &= \mel{\chi_{\alpha}}{\fock}{\chi_{\beta}}
                = \mel{\chi_{\alpha}}{\onehamil}{\chi_{\beta}}
                +
                \mel{\chi_{\alpha}\phi_j}{\twohamil}{\chi_{\beta}\phi_j}_{AS},
            \end{align}
            where $j$ only sums over the $N$ occupied indices in the ground state
            Slater determinant.
            We see that only the antisymmetric two-body elements depends on the
            coefficient matrix.
            Formulating the elements in terms of the known atomic orbitals and
            the coefficient matrix we get
            \begin{align}
                \mel{\chi_{\alpha}\phi_j}{\twohamil}{\chi_{\beta}\phi_j}
                &=
                C^{*}_{\gamma j} C_{\delta j}
                \mel{\chi_{\alpha}\chi_{\gamma}}{\twohamil}{\chi_{\beta}\chi_{\delta}},
                \\
                \mel{\chi_{\alpha}\phi_j}{\twohamil}{\phi_j\chi_{\beta}}
                &=
                C^{*}_{\gamma j} C_{\delta j}
                \mel{\chi_{\alpha}\chi_{\gamma}}{\twohamil}{\chi_{\delta}\chi_{\beta}}.
            \end{align}
            Introducing the density matrix of the occupied orbitals
            \begin{align}
                D_{\delta\gamma} \equiv
                C^{*}_{\gamma j} C_{\delta j},
            \end{align}
            where it is important to note the ordering of the indices.
            We can then write the matrix elements of the Fock operator in terms
            of the atomic orbitals and the density matrix as
            \begin{align}
                \fockten_{\alpha\beta}
                &= \mel{\chi_{\alpha}}{\onehamil}{\chi_{\beta}}
                +
                D_{\delta\gamma}
                \mel{\chi_{\alpha}\chi_{\gamma}}{\twohamil}{\chi_{\beta}\chi_{\delta}}_{AS}.
            \end{align}
            % TODO: Describe the SCF-procedure
            % TODO: Describe how to handle non-orthogonal atomic orbitals

            Having found the coefficients from the self-consistent field
            iterations and therefore the molecular orbitals and the ground state
            in the Hartree-Fock regime, we are at liberty to compute various
            observables.

            \subsubsection{General Hartree-Fock energy}
                The Hartree-Fock energy can be found by inserting the expansion
                of the molecular orbitals in the energy functional from
                \autoref{eq:energy_func_hf}.
                \begin{align}
                    \energy
                    &= \bra{\phi_i}\onehamil\ket{\phi_i}
                    + \half\bra{\phi_i\phi_j}\twohamil\ket{\phi_i\phi_j}_{AS}
                    \\
                    &=
                    C^{*}_{\alpha i} C_{\beta i}
                    \bra{\chi_{\alpha}}\onehamil\ket{\chi_{\beta}}
                    + \half
                    C^{*}_{\alpha i} C_{\gamma i}
                    C^{*}_{\beta j} C_{\delta j}
                    \bra{\chi_{\alpha}\chi_{\beta}}\twohamil
                    \ket{\chi_{\gamma}\chi_{\delta}}_{AS}
                    \\
                    &=
                    D^{\beta}_{\alpha} \oneten^{\alpha}_{\beta}
                    + \half
                    D^{\gamma}_{\alpha} D^{\delta}_{\beta}
                    \twoten^{\alpha \beta}_{\gamma \delta}.
                    \label{eq:general-hartree-fock-energy}
                \end{align}

            \subsubsection{General Hartree-Fock one-body density matrix}
                Due to the orthonormality of the molecular orbitals, the
                one-body density matrix is particularly comfortable to compute.
                \begin{align}
                    \densityten^{q}_{p}
                    = \braslat
                    \ccr{q}
                    \can{p}
                    \ketslat
                    = \delta_{p \in o}\delta_{pq},
                \end{align}
                where we have labelled the set of occupied indices in the Slater
                determinants by $o = \brac{1, \dots N}$.
                We can represent the one-body density matrix as a block matrix
                by
                \begin{align}
                    \vfg{\densityten}
                    = \begin{pmatrix}
                        \1_{N \times N} & \vfg{0}_{N \times M} \\
                        \vfg{0}_{M \times N} & \vfg{0}_{M \times M}
                    \end{pmatrix},
                \end{align}
                where $M = L - N$, i.e., the number of virtual basis states.

        \subsection{The restricted Hartree-Fock method}
            In the restricted Hartree-Fock method we make the assumption that
            each spin-direction is doubly occupied by an orbital.
            This can be a valid assumption if the Hamiltonian is
            spin-independent\footnote{%
                We write \emph{can} as there are situations where the
                Hamiltonian is spin-independent, but subject to conditions where
                the spin-symmetry of the restricted spin-orbitals break.
                % TODO: This needs to be explained properly
            }.
            To be even more specific, we will look at the \emph{closed-shell
            restricted Hartree-Fock} method, i.e., each spin-orbital \emph{must}
            be doubly occupied and each energy shell must be completely filled.
            This yields the spin-restricted spin-orbitals from
            \autoref{eq:restricted-spin-orbital}.
            For $L$ basis functions, we then get $L/2$ orbitals, where $L$ must
            be an even number.
            We label the states by
            \begin{align}
                \phi_{P}(x) = \varphi_p(\vf{r}) \sigma(m_s)
                \implies
                \ket{\phi_P} = \ket{\varphi_p\sigma},
            \end{align}
            where $P \in \brac{1, \dots, L}$ and $p \in \brac{1, \dots, L / 2}$.
            That is, we use capital letters to refer to composite indices and
            lowercase letters for the orbitals.
            We write the ground state Slater determinant as
            \begin{align}
                \ketslat = \ket{\phi_1 \phi_2 \dots \phi_{L - 1} \phi_L}
                = \ket{
                    (\varphi_1 \alpha)
                    (\varphi_1 \beta)
                    \dots
                    (\varphi_{L / 2} \alpha)
                    (\varphi_{L / 2}\beta)
                }.
            \end{align}
            The restricted molecular orbitals are orthonormal.
            As a consequence both the spin basis functions and the orbitals are
            orthonormal.
            \begin{align}
                \braket{\phi_P}{\phi_Q}
                = \braket{\sigma}{\tau}
                \braket{\varphi_p}{\varphi_q}
                = \delta_{\sigma \tau}
                \delta_{pq}
                = \delta_{PQ}.
            \end{align}
            We now insert the restricted spin-orbitals into the canonical
            Hartree-Fock equation.
            \begin{align}
                \fock\ket{\phi_P} = \epsilon_P\ket{\phi_P}
                \implies
                \fock\ket{\varphi_p\sigma}
                = \epsilon_P\ket{\varphi_p\sigma}.
            \end{align}
            By projecting onto another spin-orbital we demonstrate how we can
            construct the Fock matrix elements when the Hamiltonian is
            spin-independent.
            \begin{align}
                \bra{\phi_P}\fock\ket{\phi_Q}
                = \bra{\phi_P}\onehamil\ket{\phi_Q}
                + \bra{\phi_P\phi_J}\twohamil\ket{\phi_Q\phi_J}_{AS}.
                \label{eq:mo-fock-elements}
            \end{align}
            Looking at the one-body and the two-body parts separately we will
            demonstrate how the spin can be integrated out of the matrix
            elements.
            \begin{align}
                \bra{\phi_P}\onehamil\ket{\phi_Q}
                &= \braket{\sigma}{\tau}\bra{\varphi_p}\onehamil\ket{\varphi_q}
                = \delta_{\sigma\tau}\bra{\varphi_p}\onehamil\ket{\varphi_q}.
            \end{align}
            We split up the antisymmetric elements into its constituent parts
            and show the spin-dependence in each explicitly.
            \begin{align}
                \bra{\phi_P\phi_J}\twohamil\ket{\phi_Q\phi_J}
                &= \braket{\sigma}{\tau}\braket{\nu}{\nu}
                \bra{\varphi_p\varphi_j}\twohamil\ket{\varphi_q\varphi_j}
                = 2 \delta_{\sigma\tau}
                \bra{\varphi_p\varphi_j}\twohamil\ket{\varphi_q\varphi_j},
            \end{align}
            where we've summed over the spin-dependence $\ket{\nu}$ from the two
            occupied orbitals in the two-body elements.
            That is,
            \begin{align}
                \braket{\nu}{\nu} = \delta_{\nu\nu} = 2.
            \end{align}
            For the second integral in the antisymmetric two body elements we
            get
            \begin{align}
                \bra{\phi_P\phi_J}\twohamil\ket{\phi_J\phi_Q}
                &= \braket{\sigma}{\nu}\braket{\nu}{\tau}
                \bra{\varphi_p\varphi_j}\twohamil\ket{\varphi_j\varphi_q}
                = \delta_{\sigma\tau}
                \bra{\varphi_p\varphi_j}\twohamil\ket{\varphi_j\varphi_q},
            \end{align}
            where we've used the completness relation for the spin of the
            occupied molecular orbitals, viz.
            \begin{align}
                \ket{\nu}\bra{\nu}
                = \1 \in \mathbb{R}^{2 \times 2}.
            \end{align}
            Collecting terms, we get the Fock matrix elements
            \begin{align}
                \bra{\phi_P}\fock\ket{\phi_Q}
                &=
                \delta_{\sigma\tau}
                \para{
                    \bra{\varphi_p}\onehamil\ket{\varphi_q}
                    +
                    2
                    \bra{\varphi_p\varphi_j}\twohamil\ket{\varphi_q\varphi_j}
                    -
                    \bra{\varphi_p\varphi_j}\twohamil\ket{\varphi_j\varphi_q}
                },
            \end{align}
            where we see that the spin-dependence has been removed from the
            orbital integrals.
            We can therefore restrict ourselves to the orbital integrals for the
            Fock matrix elements.
            \begin{align}
                \bra{\varphi_p}\fock\ket{\varphi_q}
                &=
                \bra{\varphi_p}\onehamil\ket{\varphi_q}
                +
                2
                \bra{\varphi_p\varphi_j}\twohamil\ket{\varphi_q\varphi_j}
                -
                \bra{\varphi_p\varphi_j}\twohamil\ket{\varphi_j\varphi_q}.
            \end{align}
            This lets us look for coefficients for the orbitals that are
            independent of the spin.
            \begin{align}
                \ket{\varphi_p} = C_{\alpha p} \ket{\chi_{\alpha}},
            \end{align}
            where $\brac{\ket{\chi_{\alpha}}}_{\alpha = 1}^{K}$ is our basis of
            known atomic orbitals, without spin.
            By projecting the canonical Hartree-Fock equations for the orbitals
            onto the atomic orbitals we will again be left with the Roothan-Hall
            equations as in the case of the general Hartree-Fock method.
            However, the difference between the restricted and the general
            Hartree-Fock methods lies in our calculation of the Fock matrix
            elements in the atomic orbital basis.
            \begin{align}
                \fockten_{\alpha\beta}
                &\equiv \bra{\chi_{\alpha}}\fock\ket{\chi_{\beta}}
                =
                \bra{\chi_{\alpha}}\onehamil\ket{\chi_{\beta}}
                +
                2 \bra{\chi_{\alpha}\varphi_j}
                \twohamil
                \ket{\chi_{\beta}\varphi_j}
                - \bra{\chi_{\alpha}\varphi_j}
                \twohamil
                \ket{\varphi_j\chi_{\beta}}
                \\
                &=
                \bra{\chi_{\alpha}}\onehamil\ket{\chi_{\beta}}
                + 2 C^{*}_{j\gamma} C_{j\delta}
                \bra{\chi_{\alpha}\chi_{\gamma}}
                \twohamil
                \ket{\chi_{\beta}\chi_{\delta}}
                - C^{*}_{j\gamma} C_{j\delta}
                \bra{\chi_{\alpha}\chi_{\gamma}}
                \twohamil
                \ket{\chi_{\delta}\chi_{\beta}}
                \\
                &=
                \bra{\chi_{\alpha}}\onehamil\ket{\chi_{\beta}}
                + D_{\delta\gamma}
                \para{
                    \bra{\chi_{\alpha}\chi_{\gamma}}
                    \twohamil
                    \ket{\chi_{\beta}\chi_{\delta}}
                    -
                    \half
                    \bra{\chi_{\alpha}\chi_{\gamma}}
                    \twohamil
                    \ket{\chi_{\delta}\chi_{\beta}}
                },
                \label{eq:atomic-fock-rhf}
            \end{align}
            where we've introduced the restriced density matrix
            \begin{align}
                D_{\beta \alpha}
                = 2 C^{*}_{\alpha i} C_{\beta i},
            \end{align}
            as the index $i \in \brac{1, \dots, N / 2}$, where $N$ is the number
            of occupied states in the Hartree-Fock Slater determinant.

            By proceeding witht the self-consistent field iterations solving the
            Roothan-Hall equations with \autoref{eq:atomic-fock-rhf} as the
            definition of the Fock matrix elements, we find the orbital
            coefficient matrix $\vfg{C} \in \mathbb{C}^{K \times L/2}$ which we
            use to transform to the restricted molecular orbitals.
            Once transformed, we are at liberty to introduce spin-redundancy to
            open up for more general post Hartree-Fock methods.

            \subsubsection{Restricted Hartree-Fock energy}
                We can compute the ground-state restricted Hartree-Fock energy
                by inserting our expression for the restricted molecular
                orbitals into the energy functional in
                \autoref{eq:energy_func_hf}.
                \begin{align}
                    \energy
                    &= \bra{\Phi}\hamil\ket{\Phi}
                    = \bra{\phi_I}\onehamil\ket{\phi_I}
                    + \half
                    \bra{\phi_I\phi_J}
                    \twohamil
                    \ket{\phi_I\phi_J}_{AS}
                    \\
                    &=
                    \delta_{\sigma\sigma}
                    \bra{\varphi_i}\onehamil\ket{\varphi_i}
                    + \half\para{
                        \delta_{\sigma\sigma}\delta_{\tau\tau}
                        \bra{\varphi_i\varphi_j}
                        \twohamil
                        \ket{\varphi_i\varphi_j}
                        - \delta_{\sigma\tau}\delta_{\tau\sigma}
                        \bra{\varphi_i\varphi_j}
                        \twohamil
                        \ket{\varphi_j\varphi_i}
                    }
                    \\
                    &=
                    D^{\beta}_{\alpha}
                    \oneten^{\alpha}_{\beta}
                    + \half\para{
                        D^{\beta}_{\alpha}
                        D^{\delta}_{\gamma}
                        \twotensym^{\alpha\gamma}_{\beta\delta}
                        - \half
                        D^{\beta}_{\alpha}
                        D^{\delta}_{\gamma}
                        \twotensym^{\alpha\gamma}_{\delta\beta}
                    }
                    \\
                    &=
                    D^{\beta}_{\alpha}
                    \brac{
                        \oneten^{\alpha}_{\beta}
                        + \half D^{\delta}_{\gamma}
                        \para{
                            \twotensym^{\alpha\gamma}_{\beta\delta}
                            - \half
                            \twotensym^{\alpha\gamma}_{\delta\beta}
                        }
                    },
                    %&=
                    %D_{\beta \alpha}
                    %%C^{*}_{\alpha i} C_{\beta i}
                    %\bra{\chi_{\alpha}}\onehamil\ket{\chi_{\beta}}
                    %+ \half\para{
                    %    D_{\gamma\alpha} D_{\delta\beta}
                    %    %C^{*}_{\alpha i} C_{\gamma i}
                    %    %C^{*}_{\beta j} C_{\delta j}
                    %    -
                    %    \half
                    %    D_{\delta\alpha}
                    %    D_{\gamma\beta}
                    %    %C^{*}_{\alpha i} C_{\delta i}
                    %    %C^{*}_{\beta j} C_{\gamma j}
                    %}
                    %\bra{\chi_{\alpha}\chi_{\beta}}
                    %\twohamil
                    %\ket{\chi_{\gamma}\chi_{\delta}}.
                \end{align}
                where we've introduced the notation
                \begin{align}
                    \twotensym^{\alpha\beta}_{\gamma\delta}
                    \equiv \bra{\chi_{\alpha}\chi_{\beta}}
                    \twohamil
                    \ket{\chi_{\gamma}\chi_{\delta}},
                \end{align}
                for the two-body integrals.

        \subsection{The unrestricted Hartree-Fock method}
            The unrestricted Hartree-Fock method allows the molecular orbitals
            to have independent orbitals for each spin-direction.
            Hence, we assume that the molecular orbitals can be described by
            spin-unrestricted spin-orbitals as seen in
            \autoref{eq:unrestricted-spin-orbital}.
            Introducing indices for the different molecular orbitals, we denote
            the spin-unrestricted molecular orbitals by
            \begin{align}
                \phi_P(x)
                &=
                \varphi^{\sigma}_{p}(\vf{r})
                \sigma(m_s)
                \implies
                \ket{\phi_P}
                = \ket{\varphi^{\sigma}_{p}\sigma},
            \end{align}
            where $P \in \brac{1, \dots, L}$, $\sigma \in \brac{\alpha, \beta}$,
            and $p \in \brac{1, \dots, L_{\sigma}}$.
            We have that $L = L_{\alpha} + L_{\beta}$ and we have refrained from
            labelling the lower case orbital indices as they always occur with a
            spin index.
            Note that there is no implicit sum over the label $\sigma$ in the
            orbital $\varphi^{\sigma}_{p}$ and the spin-function $\sigma(m_s)$.
            We can collect the orbitals in two sets
            $\brac{\ket{\varphi^{\sigma}_{p}}}_{p = 1}^{L_{\sigma}}$, one for
            each spin-direction.
            The ground state Slater determinant can then be written
            \begin{align}
                \ketslat
                &=
                \ket{\phi_1 \phi_2 \dots \phi_{N - 1} \phi_N}
                =
                \ket{
                    (\varphi^{\alpha}_{1}\alpha)
                    \dots
                    (\varphi^{\alpha}_{N_{\alpha}}\alpha)
                    (\varphi^{\beta}_{1}\beta)
                    \dots
                    (\varphi^{\beta}_{N_{\beta}}\beta)
                },
            \end{align}
            where $N_{\sigma}$ is the number of particles with spin
            $\sigma(m_s)$.
            The orthonormality of the molecular orbitals is given by
            \begin{align}
                \braket{\phi_P}{\phi_Q}
                &= \delta^{P}_{Q}
                = \braket{\sigma}{\tau}
                \braket{\varphi^{\sigma}_{p}}{\varphi^{\tau}_{q}},
            \end{align}
            where the overlap between two orbitals with differing spin is not
            necessarily zero.
            However, if the two spin-directions are the same, i.e., $\sigma =
            \tau$, we get
            \begin{align}
                \braket{\varphi^{\sigma}_{p}}{\varphi^{\sigma}_{q}}
                = \delta^{p}_{q}.
            \end{align}
            Inserting the unrestricted spin-orbitals into the canonical
            Hartree-Fock equation yields
            \begin{align}
                \fock\ket{\phi_P}
                = \epsilon_P\ket{\phi_P}
                \implies
                \fock\ket{\varphi^{\sigma}_{p}\sigma}
                = \epsilon^{\sigma}_{p}\ket{\varphi^{\sigma}_{p} \sigma},
            \end{align}
            which demonstrates how each spin-component yields a different
            equation as the Fock eigenenergies $\epsilon^{\alpha}_{p}$ is in
            general different from $\epsilon^{\beta}_{p}$.
            By projecting onto another molecular orbital as in
            \autoref{eq:mo-fock-elements} we demonstrate how the spin yields two
            separate Fock matrices, one for each spin-direction.\footnote{%
                Note that this assumes a spin-independent Hamiltonian.
            }
            The one-body elements in the molecular orbital basis is given by
            \begin{align}
                \bra{\phi_P}\onehamil\ket{\phi_Q}
                &= \braket{\sigma}{\tau}
                \bra{\varphi^{\sigma}_{p}}
                \onehamil
                \ket{\varphi^{\tau}_{q}}
                = \delta^{\sigma}_{\tau}
                \bra{\varphi^{\sigma}_{p}}
                \onehamil
                \ket{\varphi^{\tau}_{q}}
            \end{align}
            The two-body elements yield
            \begin{align}
                \bra{\phi_P\phi_J}
                \twohamil
                \ket{\phi_Q\phi_J}
                &=
                \braket{\sigma}{\tau}
                \braket{\rho}{\rho}
                \bra{\varphi^{\sigma}_{p}\varphi^{\rho}_{j}}
                \twohamil
                \ket{\varphi^{\tau}_{q}\varphi^{\rho}_{j}}
                =
                \delta^{\sigma}_{\tau}
                \bra{\varphi^{\sigma}_{p}\varphi^{\rho}_{j}}
                \twohamil
                \ket{\varphi^{\tau}_{q}\varphi^{\rho}_{j}}
                \\
                &=
                \delta^{\sigma}_{\tau}
                \para{
                    \bra{\varphi^{\sigma}_{p}\varphi^{\alpha}_{j}}
                    \twohamil
                    \ket{\varphi^{\tau}_{q}\varphi^{\alpha}_{j}}
                    +
                    \bra{\varphi^{\sigma}_{p}\varphi^{\beta}_{j}}
                    \twohamil
                    \ket{\varphi^{\tau}_{q}\varphi^{\beta}_{j}}
                },
            \end{align}
            where, unlike in the restricted scheme, we get no factor $2$ from
            the Kronecker-Delta $\delta^{\rho}_{\rho}$ as the orbitals in the two
            different spin-directions are unequal.
            We note that this term provides a coupling between the orbitals in
            both spin-directions.
            The last term in the antisymmetric two-body elements is given by
            \begin{align}
                \bra{\phi_P\phi_J}
                \twohamil
                \ket{\phi_J\phi_Q}
                &=
                \braket{\sigma}{\rho}
                \braket{\rho}{\tau}
                \bra{\varphi^{\sigma}_{p}\varphi^{\rho}_{j}}
                \twohamil
                \ket{\varphi^{\rho}_{j}\varphi^{\tau}_{q}}
                =
                \delta^{\sigma}_{\tau}
                \bra{\varphi^{\sigma}_{p}\varphi^{\tau}_{j}}
                \twohamil
                \ket{\varphi^{\sigma}_{j}\varphi^{\tau}_{q}}.
            \end{align}
            Collecting terms we are then left with
            \begin{align}
                \bra{\phi_P}\fock\ket{\phi_Q}
                &=
                \delta^{\sigma}_{\tau}
                \para{
                    \bra{\varphi^{\sigma}_{p}}
                    \onehamil
                    \ket{\varphi^{\tau}_{q}}
                    +
                    \bra{\varphi^{\sigma}_{p}\varphi^{\rho}_{j}}
                    \twohamil
                    \ket{\varphi^{\tau}_{q}\varphi^{\rho}_{j}}
                    -
                    \bra{\varphi^{\sigma}_{p}\varphi^{\tau}_{j}}
                    \twohamil
                    \ket{\varphi^{\sigma}_{j}\varphi^{\tau}_{q}}
                },
            \end{align}
            where we've demonstrated how the spin yields two different Fock
            matrices from the canonical Hartree-Fock equations.
            That is,
            \begin{align}
                \bra{\varphi^{\sigma}_{p}}\fock\ket{\varphi^{\sigma}_{q}}
                &=
                \bra{\varphi^{\sigma}_{p}}
                \onehamil
                \ket{\varphi^{\sigma}_{q}}
                +
                \bra{\varphi^{\sigma}_{p}\varphi^{\rho}_{j}}
                \twohamil
                \ket{\varphi^{\sigma}_{q}\varphi^{\rho}_{j}}
                -
                \bra{\varphi^{\sigma}_{p}\varphi^{\sigma}_{j}}
                \twohamil
                \ket{\varphi^{\sigma}_{j}\varphi^{\sigma}_{q}},
            \end{align}
            We now look for a set of coefficients for the orbitals in each
            spin-direction in terms of our original atomic orbital basis.
            \begin{align}
                \ket{\varphi^{\sigma}_{p}}
                &= C^{\sigma}_{\kappa p} \ket{\chi_{\kappa}},
            \end{align}
            where we use the greek letters $\kappa$, $\lambda$, $\mu$, and $\nu$
            for the atomic orbitals to avoid confusion with the spin-functions
            $\alpha(m_s)$ and $\beta(m_s)$.
            Before we demonstrate how we can generate a set of equations in
            order to find the coefficient matrices $\vfg{C}^{\sigma}$, we
            motivate the spin-labelling of the Fock matrices in the atomic
            orbital basis.
            \begin{align}
                \fockten^{\sigma}_{\kappa\lambda}
                &\equiv
                \bra{\chi_{\kappa}}\fock^{\sigma}\ket{\chi_{\lambda}}
                =
                \oneten_{\kappa\lambda}
                + \bra{\chi_{\kappa}\varphi^{\rho}_{j}}
                \twohamil
                \ket{\chi_{\lambda}\varphi^{\rho}_{j}}
                - \bra{\chi_{\kappa}\varphi^{\sigma}_{j}}
                \twohamil
                \ket{\varphi^{\sigma}_{j}\chi_{\lambda}}
                \\
                &=
                \oneten_{\kappa\lambda}
                +
                (C^{\rho}_{\mu j})^{*}
                C^{\rho}_{\nu j}
                \twotensym^{\kappa\mu}_{\lambda\nu}
                -
                (C^{\sigma}_{\mu j})^{*}
                C^{\sigma}_{\nu j}
                \twotensym^{\kappa\mu}_{\nu\lambda}
                \\
                &=
                \oneten_{\kappa\lambda}
                +
                D^{\rho}_{\nu\mu}
                \twotensym^{\kappa\mu}_{\lambda\nu}
                -
                D^{\sigma}_{\nu\mu}
                \twotensym^{\kappa\mu}_{\nu\lambda},
            \end{align}
            where the density matrix $D^{\rho}_{\nu\mu}$ with the free
            spin-index $\rho$ is summed yielding
            \begin{align}
                D^{\rho}_{\nu\mu}
                = D^{\alpha}_{\nu\mu}
                + D^{\beta}_{\nu\mu},
            \end{align}
            as opposed to the density matrix with the same spin-index as the
            left-hand side, that is, $\sigma$.
            Left-projecting the atomic orbital basis on the canonical
            Hartree-Fock equations acting on an orbital in the unrestriced
            regime yields
            \begin{gather}
                \bra{\chi_{\kappa}}\fock^{\sigma}\ket{\varphi^{\sigma}_{p}}
                = \epsilon^{\sigma}_{p}
                \braket{\chi_{\kappa}}{\varphi^{\sigma}_{p}}
                \\
                \implies
                C^{\sigma}_{\lambda p}
                \bra{\chi_{\kappa}}\fock^{\sigma}\ket{\chi_{\lambda}}
                = C^{\sigma}_{\lambda p} \epsilon^{\sigma}_{p}
                \braket{\chi_{\kappa}}{\chi_{\lambda}}
                \\
                \implies
                \fockten^{\sigma}_{\kappa \lambda}
                C^{\sigma}_{\lambda p}
                =
                \overlapten_{\kappa\lambda}
                C^{\sigma}_{\lambda p}
                \epsilon^{\sigma}_{p}
                \\
                \implies
                \vfg{F}^{\sigma}
                \vfg{C}^{\sigma}
                =
                \overlapmat
                \vfg{C}^{\sigma}
                \vfg{\epsilon}^{\sigma}.
            \end{gather}
            These coupled equations constitute the \emph{Pople-Nesbet
            equations}.
            They resemble the Roothan-Hall equations seen in the two previous
            methods in that they are generalized eigenvalue equations, but now
            we solve two sets of eigenvalue equations simultaneously.

            \subsubsection{The unrestricted Hartree-Fock energy}
                Inserting our expression for the molecular orbitals into the
                energy functional we find the unrestricted Hartree-Fock energy.
                \begin{align}
                    \energy
                    &=
                    \bra{\slat}\hamil\ket{\slat}
                    =
                    \bra{\phi_I}\onehamil\ket{\phi_I}
                    +
                    \half
                    \bra{\phi_I\phi_J}
                    \twohamil
                    \ket{\phi_I\phi_J}_{AS}
                    \\
                    &=
                    \bra{\varphi^{\sigma}_{i}}
                    \onehamil
                    \ket{\varphi^{\sigma}_{i}}
                    + \half\para{
                        \bra{\varphi^{\sigma}_{i}\varphi^{\tau}_{j}}
                        \twohamil
                        \ket{\varphi^{\sigma}_{i}\varphi^{\tau}_{j}}
                        -
                        \bra{\varphi^{\sigma}_{i}\varphi^{\sigma}_{j}}
                        \twohamil
                        \ket{\varphi^{\sigma}_{j}\varphi^{\sigma}_{i}}
                    }
                    \\
                    &=
                    D^{\sigma}_{\beta\alpha}
                    \oneten^{\alpha}_{\beta}
                    + \half
                    D^{\sigma}_{\beta\alpha}
                    D^{\tau}_{\delta\gamma}
                    \twotensym^{\alpha\gamma}_{\beta\delta}
                    - \half
                    D^{\sigma}_{\beta\alpha}
                    D^{\sigma}_{\delta\gamma}
                    \twotensym^{\alpha\gamma}_{\delta\beta},
                \end{align}
                where we've carried out the spin-sums implicitly and skipped the
                step where we show the coefficient matrices that go into the
                density matrices.


\section{Time-dependent Hartree-Fock theory}
    Time-dependent Hartree-Fock continues with the ansatz that the full
    many-body wave function $\ket*{\Psi(t)}$ is described by a single Slater
    determinant of $N$ single-particle states, viz.
    \begin{align}
        \ket*{\Psi(t)} = \ket*{\slat(t)}
        = \ket*{\phi_1(t) \phi_2(t)\dots \phi_N(t)},
    \end{align}
    where $\brac{\phi_p}$ is a basis of time-dependent molecular orbitals
    subject to the constraint that they are orthonormal in time.
    We now use the time-dependent variational principle as discussed in
    \autoref{sec:tdvp} in order to find the equations of motion for the system.
    The Lagrangian of the system is given by
    \begin{align}
        \lagrangianfunc{\slat, \slat^{*}, \lambda}
        = \mel*{\slat(t)}{\para{
            i\hslash\partial_t
            - \hamil(t)
        }}{\slat(t)}
        - \lambda_{ji}\para{
            \braket*{\phi_i(t)}{\phi_j(t)}
            - \delta_{ij}
        },
        \label{eq:lagrangian-tdhf}
    \end{align}
    where we keep the orthonormality condition in the time-dependent case as
    well, and we have ignored the explicit time-dependence in the functional
    arguments to the Lagrangian.
    The action functional of the time-dependent variational principle is given
    by
    \begin{align}
        S[\slat, \slat^{*}]
        =
        \int \dd t \lagrangianfunc{\slat, \slat^{*}, \lambda},
    \end{align}
    with the stationary condition over the first-order variations found from
    \begin{align}
        \delta S
        = \int\dd t \delta \lagrangianfunc{\slat, \slat^{*}, \lambda}
        = 0
        \implies
        \delta\lagrangianfunc{\slat, \slat^{*}, \lambda}
        = 0.
    \end{align}
    Computing the expectation value of the Hamiltonian in
    \autoref{eq:lagrangian-tdhf} yields the same expression for the reference
    energy as in \autoref{subsec:reference-energy}, but now with time-dependent
    operators and molecular orbitals.
    As the time-derivative is a Hermitian, single-particle operator, we find the
    expectation value to be
    \begin{align}
        \mel*{\slat(t)}{\partial_t}{\slat(t)}
        = \mel*{\phi_i(t)}{\partial_t}{\phi_i(t)}.
    \end{align}
    % TODO: Prove this in the appendix.
    Performing variations over a single molecular orbital at a time we have
    \begin{align}
        \tilde{\phi}_i(x, t)
        = \phi_i(x, t) + \delta_{ik}\epsilon\eta(x, t),
    \end{align}
    where again $\epsilon \in \mathbb{R}$ and $\eta(x, t)$ is a complex
    function.
    As done in the time-independent case, we will restrict our attention to
    variations over the complex conjugate of $\phi_i(x, t)$.
    Variations over $\phi_i(x, t)$ will yield the adjoint equation.
    The only new term in the variation over the time-dependent Lagrangian from
    the time-independent case is the variation over the time-derivative.
    We find
    \begin{align}
        \mel*{\tilde{\slat}(t)}{\partial_t}{\slat(t)}
        = \mel*{\phi_i(t)}{\partial_t}{\phi_i(t)}
        + \epsilon\mel*{\eta(t)}{\partial_t}{\phi_k(t)}.
    \end{align}
    From the stationary conditions of the action functional we find
    \begin{gather}
        \delta\lagrangianfunc{\slat, \tilde{\slat}^{*}, \lambda}
        = \lagrangianfunc{\slat, \tilde{\slat}^{*}, \lambda}
        - \lagrangianfunc{\slat, \slat^{*}, \lambda}
        = 0 \\
        \implies
        i\hslash\mel*{\eta}{\partial_t}{\phi_k}
        -\mel*{\eta}{\fock}{\phi_k}
        = \lambda_{jk}\braket*{\eta}{\phi_j},
    \end{gather}
    where we now for the sake of brevity removed the explicit time-dependence of
    the orbitals and the operators.
    Furthermore, we jumped straight from the variation of the energy functional
    to include the Fock-operator.
    As $\eta(x, t)$ was arbitrary, this must be valid for all choices of
    $\eta(x, t)$ and we can formulate the stationary condition as
    \begin{align}
        i\hslash\partial_t\ket*{\phi_k}
        - \fock\ket*{\phi_k}
        = \lambda_{jk}\ket*{\phi_j}.
        \label{eq:stationary-condition-1}
    \end{align}
    Projecting onto $\phi_l(x, t)$ and applying the constraint that the
    molecular orbitals are orthonormal we find an equation for the Lagrange
    multipliers,
    \begin{align}
        \lambda_{lk}
        = i\hslash\mel*{\phi_l}{\partial_t}{\phi_k}
        - \mel*{\phi_l}{\fock}{\phi_k}.
    \end{align}
    We now insert this expression for the Lagrange multipliers back into the
    stationary condition in \autoref{eq:stationary-condition-1}.
    This yields
    \begin{align}
        \hat{P}\brak{
            i\hslash\partial_t\ket*{\phi_k}
            - \fock\ket*{\phi_k}
        }
        = 0,
    \end{align}
    where we have now defined the projection operator
    \begin{align}
        \hat{P} \equiv \1 - \dyad{\phi_i}{\phi_i},
    \end{align}
    and we see that we have gotten rid of the Lagrange multipliers.
    Solving for the time-derivative of the molecular orbitals we find
    \begin{align}
        i\hslash\hat{P}\partial_t\ket*{\phi_k}
        = \hat{P}\fock\ket*{\phi_k}.
    \end{align}
    We now define an arbitrary time-dependent, Hermitian operator $\hat{Q}$
    in terms of a unitary transformation \cite{hochstuhl2014time}
    \begin{align}
        i\hslash\mel*{\phi_i}{\partial_t}{\phi_j}
        \equiv
        \mel*{\phi_i}{\hat{Q}}{\phi_j}.
    \end{align}
    We know that  Slater determinants are invariant up to a complex phase under
    unitary transformations, as seen in
    \autoref{lemma:unitary-transformation-slater}, this unitary transformation
    will leave the time-dependent Lagrangian invariant.
    Since $\hat{Q}$ was an arbitary, Hermitian operator and the Fock operator is
    Hermitian, we can choose $\hat{Q} = \fock$.
    Expanding the projection operator and inserting this choice, we get the
    time-dependent Hartree-Fock equations
    \begin{align}
        i\hslash\partial_t\ket*{\phi_k(t)}
        = \fock(t)\ket*{\phi_k(t)},
        \label{eq:tdhf}
    \end{align}
    where the molecular orbitals stays orthonormal in time.


\clearemptydoublepage

