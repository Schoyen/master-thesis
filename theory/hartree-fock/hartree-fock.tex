\chapter{Hartree-Fock theory}
    One can not tackle the subject of many-body theory without a discussion of
    the Hartree-Fock method. It serves as an excellent initial approximation,
    and in many cases the \emph{only} approximation, to the many-body
    wavefunction for a given system. It is a rather cheap method, in terms of
    computational intensity, and explains much of the underlying physics of a
    given system of many particles.

    \section{Assumptions used in the Hartree-Fock method}
        In the Hartree-Fock method we make five assumptions in order to make the
        many-body problem tractable.
        \begin{enumerate}
            \item We assume that the \emph{Born-Oppenheimer approximation} is a
                good approximation.
                % TODO: Add theory on the Born-Oppenheimer approximation
            \item We assume that the motion of the electrons can be described
                non-relativistically.
            \item We assume that the solution to the variational problem can be
                represented as a linear combination of a finite number of basis
                functions.
            \item The energy eigenfunctions of the time-independent Schrödinger
                equation can be described by a single Slater determinant.
            \item We assume that correlation between particles can be described
                in the \emph{mean-field approximation}.
        \end{enumerate}
        These assumptions provide the basis for the Hartree-Fock method. We
        shall see later that we quickly reach a limit where these assumptions
        break apart thus motivating the use of \emph{post Hartree-Fock methods}
        such as the coupled cluster method.

    \section{Deriving the time-independent Hartree-Fock equations}
        Much of the theory shown in this section draws from the excellent
        lecture notes by \citeauthor{kvaal2017notes} \cite{kvaal2017notes} and
        the book \citetitle{szabo1996modern} by \citeauthor{szabo1996modern}
        \cite{szabo1996modern}.
        Starting from the Schrödinger equation
        \begin{align}
            \hamil\ket{\Psi} = \energy\ket{\Psi},
        \end{align}
        where $\hamil$ is the electronic Hamiltonian with at most two-body
        interactions on the form
        \begin{align}
            \hamil = \onehamil + \twohamil,
        \end{align}
        where $\onehamil$ is the one-body part of the Hamiltonian and
        $\twohamil$ the higher order correlations.
        We know that the ground state of $\onehamil$ will be a single Slater
        determinant.
        If the two-body interactions are ``small'' we can assume that there will
        exist a Slater determinant which will capture most of the true ground
        state of the full Hamiltonian\footnote{We will see that it does not take
        much before the two-body interaction becomes a little more than just a
        small perturbation.}.
        This motivates the approximation that our many-body wave function
        $\ket{\Psi}$ can be approximated by a single Slater determinant.
        \begin{align}
            \ket{\Psi} = \ketslat = \ket{\phi_1, \phi_2, \dots, \phi_N},
        \end{align}
        where the \emph{molecular orbitals} $\brac{\phi_i}_{i = 1}^{N}$ are the
        primary unknowns.
        They are subject to the constraint that they are orthonormal.
        That is,
        \begin{align}
            \braket{\phi_i}{\phi_j} = \delta_{ij}
            \implies \braket{\slat}{\slat} = 1.
            % TODO: Check if this is an actual implication
        \end{align}
        Defining the energy functional
        \begin{align}
            \energyfunc{\slat}
            \equiv \braslat\hamil\ketslat
            =
            \bra{\phi_i}\onehamil\ket{\phi_i}
            + \half\bra{\phi_i\phi_j}\twohamil\ket{\phi_i\phi_j}_{AS},
            \label{eq:energy_func_hf}
        \end{align}
        where the Einstein summation convention is understood.
        The variational principle\footnote{%
            Note that we treat the Slater determinants as orthonormal thus
            removing the need to include the normalization factor in the energy
            functional.
        }
        tells us that the true ground state energy, $\energygs$, will be a lower
        bound to the energy found from the energy functional in
        \autoref{eq:energy_func_hf} for any normalized trial wavefunction
        $\ketslat$.
        That is,
        \begin{align}
            \energygs
            \leq \energyfunc{\slat} = \braslat\hamil\ketslat.
        \end{align}
        Our task is now to find the molecular orbitals $\brac{\phi_i}_{i =
        1}^{N}$ that minimizes the energy functional%
        \footnote{%
            This is done by finding a stationary state for $\energyfunc{\slat}$,
            which does not guarantee that we have found a minimum, but often the
            stationary state will be a minimum.
        }.
        By performing a variation in the Slater determinant,
        \begin{align}
            \slat \to \slat + \delta\slat,
        \end{align}
        we find the that the energy functional is changed by
        \begin{align}
            \energyfunc{\slat + \delta\slat}
            &=
            \bra{\slat + \delta\slat}\hamil\ket{\slat + \delta\slat}
            \\
            &= \energyfunc{\slat}
            + \bra{\delta\slat}\hamil\ketslat + \braslat\hamil\ket{\delta\slat}
            + \dots
            \\
            &= \energyfunc{\slat} + \delta\energyfunc{\slat}
            + \dots,
        \end{align}
        where the \emph{first variation} in $\energyfunc{\slat}$ is given by
        \begin{align}
            \delta\energyfunc{\slat}
            \equiv
            \bra{\delta\slat}\hamil\ketslat + \braslat\hamil\ket{\delta\slat},
        \end{align}
        where we treat $\delta$ as a linear differential operator.
        Higher order variations are ignored and we are thus only interested in
        finding the Slater determinant, $\ketslat$, for which
        \begin{align}
            \delta\energyfunc{\slat} = 0,
        \end{align}
        i.e., the stationary point of the energy functional in terms of the
        function $\slat$.
        However the energy functional, $\energyfunc{\slat}$, does not
        incorporate the constraint that the molecular orbitals should be
        orthonormal.
        To ensure this constraing we use the method of Lagrange multipliers,
        with one multiplier for every constraint.
        We thus construct the Lagrangian functional
        \begin{align}
            \lagrangianfunc{\slat, \lambda}
            &= \energyfunc{\slat}
            - \lambda_{ji}\para{
                \braket{\phi_i}{\phi_j}
                - \delta_{ij}
            }.
        \end{align}
        As the Lagrangian functional, $\lagrangianfunc{\slat, \lambda}$, is
        real, and as the constraint is Hermitian, the Lagrange multipliers,
        $\lambda_{ji}$, can be taken to be Hermitian as well.

        \begin{proof}[%
                Proof that the Lagrange multipliers can be chosen Hermitian%
            ]
            Following the derivation done by Mayer \cite{mayer2003simple}, we
            start by noticing that the constraint is Hermitian, i.e.,
            \begin{align}
                \braket{\phi_i}{\phi_j} - \delta_{ij}
                = \braket{\phi_j}{\phi_i}^{*} - \delta_{ji}.
            \end{align}
            As of now we have two independent Lagrange multipliers; one for the
            overlap $\braket{\phi_i}{\phi_j}$ and another for the the complex
            conjugate $\braket{\phi_j}{\phi_i}$.
            We can formulate the constraint for the real and imaginary part
            separately.
            This yields
            \begin{align}
                \Re\brac{
                    \braket{\phi_i}{\phi_j}
                }
                &=
                \half\brac{
                    \braket{\phi_i}{\phi_j}
                    + \braket{\phi_j}{\phi_i}
                }
                = 0,
                \\
                \Im\brac{
                    \braket{\phi_i}{\phi_j}
                }
                &=
                \frac{1}{2i}\brac{
                    \braket{\phi_i}{\phi_j}
                    - \braket{\phi_j}{\phi_i}
                }
                = 0.
            \end{align}
            Introducing two separate Lagrange multipliers $\mu_{ij}$ and
            $\nu_{ij}$ for the two latter conditions, we get
            \begin{align}
                \mu_{ij}\Re\brac{
                    \braket{\phi_i}{\phi_j}
                }
                + \nu_{ij}\Im\brac{
                    \braket{\phi_i}{\phi_j}
                }
                &=
                \half\brak{
                    \mu_{ij} - i\nu_{ij}
                }
                \braket{\phi_i}{\phi_j}
                \nonumber
                \\
                &\qquad
                + \half\brak{
                    \mu_{ij} + i\nu_{ij}
                }
                \braket{\phi_j}{\phi_i}.
            \end{align}
            We now choose our combined Lagrange multipliers to be
            \begin{align}
                \lambda_{ji} &=
                -\half\brak{
                    \mu_{ij} - i\nu_{ij}
                }, \\
                \lambda_{ij} &=
                -\half\brak{
                    \mu_{ij} + i\nu_{ij}
                },
            \end{align}
            which implies that $\lambda_{ji} = \lambda_{ij}^{*}$, as was to be
            shown.
        \end{proof}
        We are now interested in finding a stationary point in the Lagrangian
        with respect to the molecular orbitals $\phi_i$ subject to the
        constraint that they are orthonormal.

\section{The time-dependent Hartree-Fock equations in a basis}
    In the time-dependent Hartree-Fock method we will evolve the molecular
    orbitals in time, but we keep the atomic orbitals fixed.
    This means that
    \begin{align}
        \ket{\phi_i(t)} = C_{\alpha i}(t)\ket{\chi_{\alpha}},
    \end{align}
    that is, the time-dependence occurs in the coefficients $C_{\alpha i}(t)$.
    Thus the time-dependent Hartree-Fock Slater determinant is given by
    \begin{align}
        \ket{\Phi(t)}
        &=
        \ket{
            \phi_1(t), \dots, \phi_N(t)
        }.
    \end{align}
    The time-dependent Slater determinants in a system with a time-dependent
    Hamiltonian $\hat{H}(t)$ satisfies the Schrödinger equation
    \begin{align}
        i\hslash\dpd[]{}{t}\ket{\Phi(t)}
        &= \hamil(t)\ket{\Phi(t)},
    \end{align}
    where $\hamil(t)$ is the time-dependent electronic Hamiltonian.

    \subsection{Deriving the time-dependent Hartree-Fock equations}
        For the sake of brevity we will remove the explicit time-dependence in
        the orbitals, the Slater determinants and the operators, though they are
        still present.
        We will be following closely the derivation done by
        \citeauthor{hochstuhl2014time} \cite{hochstuhl2014time}, but for a
        single Slater determinant.
        Furthermore, the time derivative of the Slater determinant is
        % TODO: How do you get this result?
        % Write out SD as sum, and use orthonormality of orbitals?
        % Seems like only one term "survives"
        \begin{align}
            \bra{\Phi}\dpd{}{t}\ket{\Phi}
            &= \bra{\phi_i}\dpd{}{t}\ket{\phi_i}.
        \end{align}
        We can now find the equations of motion by applying time-dependent
        variational principle, that is,
        \begin{align}
            \bra{\delta\Phi}\hat{H} - i\hslash\dpd{}{t}\ket{\Phi} = 0,
        \end{align}
        under the constraint that the time-dependent molecular orbitals are
        orthonormal, i.e.,
        \begin{align}
            \braket{\phi_i(t)}{\phi_j(t)} = \delta_{ij},
        \end{align}
        where we for the sake of emphasis temporarily added the explicit
        time-dependence in the orbitals. We now define the Lagrangian functional
        with the Lagrange multipliers $\lambda_{ij}$ to be
        \begin{align}
            \mathcal{L}
            &= \bra{\Phi}\hat{H} - i\hslash\dpd{}{t}\ket{\Phi}
            - \lambda_{ij}\para{
                \braket{\phi_i}{\phi_j} - \delta_{ij}
            }.
        \end{align}
        We now find an extremum of $\mathcal{L}$ by taking the functional
        derivative of the functional and equating it to zero.
        \begin{gather}
            \dfd{\phi_k^{*}}\mathcal{L} = 0, \\
            \implies
            \hat{h}\ket{\phi_k}
            + \bra{\cdot\phi_j}\hat{u}\ket{\phi_k\phi_j}_{AS}
            - i\hslash\dpd{}{t}\ket{\phi_k}
            - \lambda_{kj}\ket{\phi_j} = 0.
            \label{eq:extremum_lagrangian}
        \end{gather}
        Left-projecting with an orbital $\bra{\phi_l}$ and solving for the
        Lagrange multipliers, $\lambda_{kl}$, where we apply the constraint that
        the orbitals are orthonormal we get
        \begin{align}
            \lambda_{kl}
            =
            \bra{\phi_l}\hat{h}\ket{\phi_k}
            + \bra{\phi_l\phi_j}\hat{u}\ket{\phi_k\phi_j}_{AS}
            - i\hslash\bra{\phi_l}\dpd{}{t}\ket{\phi_k}.
            \label{eq:lagrange_multipliers}
        \end{align}
        Inserting the expression for the Lagrange multipliers in
        \autoref{eq:lagrange_multipliers} into \autoref{eq:extremum_lagrangian}
        we get
        \begin{align}
            \hat{P}\para{
                \hat{h}\ket{\phi_k}
                + \bra{\cdot\phi_j}\hat{u}\ket{\phi_k\phi_j}_{AS}
                - i\hslash\dpd{}{t}\ket{\phi_k}
            } = 0,
        \end{align}
        where we have defined the projection operator, $\hat{P}$, to be
        \begin{align}
            \hat{P} = \1 - \ket{\phi_i}\bra{\phi_i}.
        \end{align}
        Solving for the time-derivative of the molecular orbitals we get
        \begin{align}
            i\hslash\hat{P}\dpd{}{t}\ket{\phi_k}
            = \hat{P}\para{
                \hat{h}\ket{\phi_k}
                + \bra{\cdot\phi_j}\hat{u}\ket{\phi_k\phi_j}_{AS}
            }
            = \hat{P}\hat{f}\ket{\phi_k},
            \label{eq:tdhf_equations_p}
        \end{align}
        where we have defined the Fock operator to be
        \begin{align}
            \hat{f}
            \equiv \hat{h} + \bra{\cdot\phi_j}\hat{u}\ket{\cdot\phi_j}_{AS}.
        \end{align}
        We now define an arbitrary hermitian operator $\hat{Q}(t)$ in terms of
        a unitary transformation
        \begin{align}
            \bra{\phi_i}i\hslash\dpd{}{t}\ket{\phi_j}
            \equiv \bra{\phi_i}\hat{Q}(t)\ket{\phi_j}.
            \label{eq:unitary_transformation_time}
        \end{align}
        This is valid as any Slater determinant is invariant under unitary
        transformations. See section \ref{proof:slater_determinants_invariant}
        for a short proof of this. As the Slater determinant is invariant under
        the unitary transformation, by extension the Lagrangian functional will
        be invariant. Expanding $\hat{P}$ in
        \autoref{eq:tdhf_equations_p} and inserting
        \autoref{eq:unitary_transformation_time} we get
        \begin{gather}
            i\hslash\dpd{}{t}\ket{\phi_k}
            - i\hslash\ket{\phi_i}\bra{\phi_i}\dpd{}{t}\ket{\phi_k}
            = \hat{f}\ket{\phi_k}
            - \ket{\phi_i}\bra{\phi_i}\hat{f}\ket{\phi_k}
            \\
            \implies
            i\hslash\dpd{}{t}\ket{\phi_k}
            - \ket{\phi_i}\bra{\phi_i}\hat{Q}(t)\ket{\phi_k}
            = \hat{f}\ket{\phi_k}
            - \ket{\phi_i}\bra{\phi_i}\hat{f}\ket{\phi_k}.
        \end{gather}
        Since $\hat{Q}(t)$ was an arbitrary hermitian operator and
        $\hat{f}^{\dagger}(t) = \hat{f}(t)$ we can choose $\hat{Q}(t) =
        \hat{f}(t)$ yielding the time-dependent Hartree-Fock equations on the
        form
        \begin{align}
            i\hslash\dpd{}{t}\ket{\phi_k}
            = \hat{f}\ket{\phi_k}.
        \end{align}
        We will in the next section restore the explicit time-dependence in the
        molecular orbitals and the operators.

    \subsection{Evolving the coefficients in time}
        Having restricted the full wave function to a single Slater determinant,
        we can write the equations for the molecular orbitals in terms of the
        time-dependent Fock operator.
        \begin{align}
            i\hslash\dpd{}{t}\ket{\phi_i(t)}
            &= \hat{f}(t)\ket{\phi_i(t)}.
        \end{align}
        We can now insert the linear combination for the molecular orbitals in
        the atomic orbital basis with the time-dependent coefficients.
        \begin{align}
            i\hslash\dpd{}{t}C_{\alpha i}(t)\ket{\chi_{\alpha}}
            &= \hat{f}(t)C_{\alpha i}(t)\ket{\chi_{\alpha}}.
        \end{align}
        Left projecting with another state from the atomic orbitals we can rewrite
        the previous equation to
        \begin{gather}
            i\hslash\dpd{}{t}C_{\alpha i}(t)\braket{\chi_{\beta}}{\chi_{\alpha}}
            =
            C_{\alpha i}(t)\bra{\chi_{\beta}}\hat{f}(t)\ket{\chi_{\alpha}}
            \\
            \implies
            i\hslash \dot{C}_{\alpha i} S_{\beta\alpha}
            = C_{\alpha i}(t)f_{\beta\alpha}(t)
            \\
            \implies
            i\hslash \vf{S}\dot{\vf{C}}
            = \vf{F}(t)\vf{C}(t),
            \label{eq:tdhf-equations}
        \end{gather}
        where the overlap matrix $\vf{S}$ is the matrix with elements
        \begin{align}
            S_{\beta\alpha} \equiv \braket{\chi_{\beta}}{\chi_{\alpha}},
            \label{eq:overlap_ao}
        \end{align}
        and the time-dependent Fock matrix $\vf{F}(t)$ contains the entries
        \begin{align}
            f_{\beta\alpha}(t) = \bra{\chi_{\beta}}\hat{f}(t)\ket{\chi_\alpha}.
        \end{align}
        Often we will work with basis sets where the overlap matrix reduces to
        the identity. In this case, \autoref{eq:tdhf-equations} collapses to the
        equation
        \begin{align}
            i\hslash \dot{\vf{C}} = \vf{F}(t)\vf{C}(t).
            \label{eq:tdhf-orthogonal}
        \end{align}

    \subsection{Time evolution in a non-orthogonal basis of atomic orbitals}
        In order to solve \autoref{eq:tdhf-orthogonal} for non-orthogonal
        atomic orbitals and avoid having to solve \autoref{eq:tdhf-equations}
        as they stand, we orthogonalize the atomic orbitals. We will follow the
        derivation shown in the book \citetitle{szabo1996modern} by
        \citeauthor{szabo1996modern} \cite{szabo1996modern} closely.

        For the case when
        \begin{align}
            \braket{\chi_{\alpha}}{\chi_{\beta}} \equiv S_{\alpha\beta}
            \neq \delta_{\alpha\beta},
        \end{align}
        that is, a non-orthogonal atomic orbital basis
        $\brac{\ket{\chi_{\alpha}}}_{\alpha = 1}^{N}$, we move to a transformed
        orthonormal basis $\brac{\ket{\xi_{\mu}}}_{\mu = 1}^{K}$ by the
        transformation
        \begin{align}
            \ket{\xi_{\mu}} = X_{\alpha\mu}\ket{\chi_{\alpha}}.
        \end{align}
        This yields an equation on the form
        \begin{align}
            \braket{\xi_{\mu}}{\xi_{\nu}}
            &= \para{
                X_{\alpha\mu}^{*}\bra{\chi_{\alpha}}
            }\para{
                X_{\beta\nu}\ket{\chi_{\beta}}
            }
            = X_{\alpha\mu}^{*}S_{\alpha\beta}X_{\beta\nu}
            = \delta_{\mu\nu}.
        \end{align}
        As a matrix equation we get
        \begin{align}
            \vf{X}^{\dagger}\vf{S}\vf{X} = \1.
            \label{eq:orthonormal-transformation}
        \end{align}
        This, along with the requirement that $\vf{X}$ must be non-singular,
        forms the restrictions on the matrix $\vf{X}$ in order for
        $\brac{\ket{\xi_{\mu}}}_{\mu = 1}^{L}$ to form an orthonormal basis.
        Since $\vf{S}$ is Hermitian (by construction), there exists a unitary
        matrix $\vf{U}$ that diagonalizes $\vf{S}$ by
        \begin{align}
            \vf{U}^{\dagger}\vf{S}\vf{U} = \vf{s},
        \end{align}
        where $\vf{s}$ is a diagonal matrix with the eigenvalues of $\vf{S}$.
        The matrix $\vf{S}$ is positive definite as all the eigenvalues of
        $\vf{S}$ are positive.
        We can find an equation for $\vf{X}$ using \emph{symmetric
        orthogonalization} \cite{mayer2002lowdin}.
        This yields
        \begin{align}
            \vf{X} = \vf{S}^{-1/2} = \vf{U}\vf{s}^{-1/2}\vf{U}^{\dagger}.
        \end{align}
        We can now use $\vf{X}$ to transform to the new basis
        $\brac{\xi_{\mu}}_{\mu = 1}^{L}$ and solving the Roothan-Hall equations
        in this orthonormal basis. In practice we do this by transforming the
        coefficient matrix $\vf{C}$ from the atomic orbitals in terms of the
        orthonormal orbitals $\xi_{\mu}$. In terms of the original atomic
        orbitals we can formulate this by
        \begin{align}
            \ket{\phi_i}
            &= C_{\mu i}'\ket{\xi_{\mu}}
            = C_{\mu i}'X_{\alpha \mu}\ket{\chi_{\alpha}}
            = C_{\alpha i}\ket{\chi_{\alpha}},
        \end{align}
        where $C_{\mu i}'$ are the coefficients from the new orthonormal orbital
        basis $\ket{\xi_{\mu}}$ to the molecular orbitals. In terms of the
        coefficient matrices we then get
        \begin{align}
            \vf{C} = \vf{X}\vf{C}',
        \end{align}
        where $\vf{C}$ are the original coefficients from the original atomic
        orbital basis $\ket{\chi_{\alpha}}$ to the molecular orbital basis.
        Having eliminated the linear dependencies in $\vf{S}$ there will exist
        an inverse for $\vf{X}$. Thus we find
        \begin{align}
            \vf{C}' = \vf{X}^{-1}\vf{C}.
        \end{align}
        Inserting this expression into \autoref{eq:tdhf-equations} when the
        coefficient matrices are time-dependent\footnote{Note that $\vf{X}(t) =
        \vf{X}$ as we keep the atomic orbitals fixed and let the coefficients
        handle the time-dependence.} we get
        \begin{gather}
            i\hslash\vf{S}\dot{\vf{C}}(t) = \vf{F}(t)\vf{C}(t) \\
            \implies
            i\hslash\vf{S}\vf{X}\dot{\vf{C}}'(t) = \vf{F}(t)\vf{X}\vf{C}'(t)
            \\
            \implies
            i\hslash\vf{X}^{\dagger}\vf{S}\vf{X}\dot{\vf{C}}'(t)
            = \vf{X}^{\dagger}\vf{F}(t)\vf{X}\vf{C}'(t)
            \\
            \implies
            i\hslash \dot{\vf{C}}'(t) = \vf{F}'(t)\vf{C}'(t),
        \end{gather}
        where we've used the orthonormality of the transformation in
        \autoref{eq:orthonormal-transformation} and defined the transformed Fock
        matrix
        \begin{align}
            \vf{X}^{\dagger}\vf{F}(t)\vf{X} = \vf{F}'(t).
        \end{align}
        Doing this transformation simplifies the use of linear algebra libraries
        to solve the time-dependent Hartree Fock equations.

    \subsection{Time dependent overlap}
        When working with Slater determinants in a given basis we can compute
        the overlap between two wavefunctions in an efficient manner. We define
        the orbitals
        \begin{align}
            \ket{\phi_i} &= C_{\alpha i}\ket{\chi_{\alpha}}, \\
            \ket{\psi_i} &= D_{\alpha i}\ket{\chi_{\alpha}},
        \end{align}
        as two separate linear combinations of the original atomic orbitals. The
        overlap between these two molecular orbitals is then
        \begin{align}
            \braket{\psi_i}{\phi_j}
            &= \para{
                D_{\alpha i}^{*}
                \bra{\chi_{\alpha}}
            }\para{
                C_{\beta j}
                \ket{\chi_{\beta}}
            }
            = D_{\alpha i}^{*}S_{\alpha\beta} C_{\beta j}
            = T_{ij}.
        \end{align}
        In matrix notation this reduces to
        \begin{align}
            \vf{T} = \vf{D}^{\dagger}\vf{S}\vf{C},
        \end{align}
        where $\vf{C}$ and $\vf{D}$ are the coefficient matrices of the set of
        molecular orbitals $\ket{\phi_i}$ and $\ket{\psi_i}$ respectively and
        $\vf{S}$ is the overlap matrix for the atomic orbitals.  We can then
        write the two Slater determinants consisting of $N$ of the molecular
        orbitals as
        \begin{align}
            \ket{\Phi} &= \frac{1}{\sqrt{N!}}
            \sum_{\sigma \in S_N}(-1)^{|\sigma|}
            \ket{\phi_{\sigma(1)}}
            \otimes\dots\otimes
            \ket{\phi_{\sigma(N)}}, \\
            \ket{\Psi} &= \frac{1}{\sqrt{N!}}
            \sum_{\sigma \in S_N}(-1)^{|\sigma|}
            \ket{\psi_{\sigma(1)}}
            \otimes\dots\otimes
            \ket{\psi_{\sigma(N)}},
        \end{align}
        where $S_N$ is the permutation group of all permutations $\sigma$ of the
        labels $i \in \brac{1, \dots, N}$. Furthermore, $|\sigma|$, is the
        defined as the number of permutations performed from the original order
        of the labels. The overlap of the two Slater determinants is then
        \begin{align}
            \braket{\Psi}{\Phi}
            &= \frac{1}{N!}
            \sum_{\sigma, \tau \in S_{N}}
            (-1)^{|\sigma||\tau|}
            \prod_{i = 1}^{N}
            \braket{\psi_{\sigma(i)}}{\phi_{\tau(i)}}
            \\
            &= \frac{1}{N!}
            \sum_{\sigma, \tau \in S_{N}}
            (-1)^{|\sigma||\tau|}
            \prod_{i = 1}^{N}
            T_{\sigma(i)\tau(i)},
        \end{align}
        where we now have a double sum over all permutations $\sigma$ and
        $\tau$. To go from here we have to collapse one of the sums over the
        permutations. By keeping one of the permutation sums fixed we get all
        unique combinations of the terms in the products. By iterating over the
        second sum over permutations we create the same set of combinations, but
        with a different ordering. As the terms commute, we can just add it to
        the already existing set of unique combinations. In total there are $N!$
        permutations in $S_N$ thus giving a factor of $N!$ to the sum. This
        yields
        \begin{align}
            \braket{\Psi}{\Phi}
            &= \frac{1}{N!}N!
            \sum_{\sigma \in S_N}
            (-1)^{|\sigma|}
            \prod_{i = 1}^{N}T_{i\sigma(i)}
            \\
            &= \begin{vmatrix}
                T_{11} & \dots & T_{1N} \\
                \vdots & \ddots & \vdots \\
                T_{N1} & \dots & T_{NN}
            \end{vmatrix}
            = \det(\vf{T})
        \end{align}
        In particular in the time-dependent Hartree-Fock regime where we have
        the time-dependent coefficients $C_{\alpha i}(t)$ we get the overlap
        with the ground state to be
        \begin{align}
            \braket{\Phi(t)}{\Phi(0)}
            &= \det(\vf{C}^{\dagger}(t)\vf{S}\vf{C}(0))
            = \det(\vf{T}(t)),
        \end{align}
        where $\vf{C}(t)$ is the coefficient matrix of the time evolved states.

        \subsubsection{Overlap with non-orthogonal atomic orbitals}
            In the case that we have transformed from the non-orthogonal atomic
            orbital basis $\brac{\ket{\chi_{\alpha}}}_{\alpha = 1}^{N}$ to the
            orthonormal atomic orbital basis $\brac{\ket{\xi_{\mu}}}_{\mu =
            1}^{N}$ by,
            \begin{align}
                \ket{\xi_{\mu}} = X_{\alpha\mu}\ket{\chi_{\alpha}},
            \end{align}
            where the transformation to the molecular orbitals is given by,
            \begin{align}
                \ket{\phi_{i}(t)} = C_{\mu i}'(t)\ket{\xi_{\mu}}
                = C_{\mu i}'(t)X_{\alpha\mu}\ket{\chi_{\alpha}}
                = C_{\alpha i}(t)\ket{\chi_{\alpha}},
            \end{align}
            we get the time-dependent overlap to be
            \begin{align}
                \braket{\Phi(t)}{\Phi(0)}
                &= \det(\vf{C}^{\dagger}(t)\vf{S}\vf{C}(0))
                \\
                &= \det\Bigl[
                    (\vf{C}'(t))^{\dagger} \vf{X}^{\dagger}
                    \vf{S}\vf{X}\vf{C}'(t)
                \Bigr]
                \\
                &=
                \det\Bigl[
                    (\vf{C}'(t))^{\dagger}
                    \vf{C}'(t)
                \Bigr].
            \end{align}
            This simplifies the process of computing the overlap, when we've
            already transformed to the orthonormal basis.

