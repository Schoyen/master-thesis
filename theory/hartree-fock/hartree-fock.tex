\chapter{Hartree-Fock theory}
    One can not tackle the subject of many-body theory without a discussion of
    the Hartree-Fock approximation. It serves as an excellent initial
    approximation, and in many cases the \emph{only} approximation, to the
    many-body wavefunction for a given system. It is a rather cheap
    approximation, in terms of computational intensity, and explains much of the
    underlying physics of a given system of many particles.

    \section{Assumptions used in the Hartree-Fock method}
        In the Hartree-Fock method we make five assumptions in order to make the
        many-body problem tractable.
        \begin{enumerate}
            \item We assume that the \emph{Born-Oppenheimer approximation} is a
                good approximation.
                % TODO: Add theory on the Born-Oppenheimer approximation
            \item We assume that the motion of the electrons can be described
                non-relativistically.
            \item We assume that the solution to the variational problem can be
                represented as a linear combination of a finite number of basis
                functions.
            \item The energy eigenfunctions of the time-independent Schrödinger
                equation can be described by a single Slater determinant.
            \item We assume that correlation between particles can be described
                in the \emph{mean-field approximation}.
        \end{enumerate}
        These assumptions provides the basis for the Hartree-Fock method. We
        shall see later that we quickly reach a limit where these assumptions
        break apart thus motivating the use of \emph{post Hartree-Fock methods}
        such as the coupled cluster method.

    \section{Deriving the time-independent Hartree-Fock equations}
        Much of the theory shown in this section draws from the excellent
        lecture notes by \citeauthor{kvaal2017notes} and
        \citeauthor{szabo1996modern}.  We start by making the ansatz that the
        full many-body wavefunction is a single Slater determinant. For a given
        system Hamiltonian
        \begin{align}
            \hamil = \onehamil + \twohamil,
        \end{align}
        where $\onehamil$ is the one-body part of the Hamiltonian and
        $\twohamil$ the higher order correlations. In our case we will limit
        ourselves to Coulomb two-body interactions. We know that the ground
        state of $\onehamil$ will be a single Slater determinant. If the
        two-body interactions are ``small'' we can assume that there will exist
        a Slater determinant which will capture most of the true ground state of
        the full Hamiltonian\footnote{We will see that it does not take much
        before the two-body interaction becomes a little more than just a small
        perturbation.}.

        Our many-body wavefunction will now be represented as a single Slater
        determinant
        \begin{align}
            \ketslat = \ket{\phi_1, \phi_2, \dots, \phi_N},
        \end{align}
        where the \emph{molecular orbitals} $\brac{\phi_i}_{i = 1}^{N}$ are the
        primary unknowns subject to the constraint that they are orthonormal.
        That is,
        \begin{align}
            \braket{\phi_i}{\phi_j} = \delta_{ij}
            \implies \braket{\slat}{\slat} = 1.
            % TODO: Check if this is an actual implication
        \end{align}
        Defining the energy functional
        \begin{align}
            \energyfunc{\slat}
            = \braslat\hamil\ketslat,
            \label{eq:energy_func_hf}
        \end{align}
        the variational principle tells us that the true ground state energy,
        $\energygs$, will be a lower bound to the energy found from
        \autoref{eq:energy_func_hf} for any normalized trial wavefunction
        $\ketslat$. That is,
        \begin{align}
            \energygs
            \leq \energyfunc{\slat} = \braslat\hamil\ketslat.
        \end{align}
        Our task is now to find the molecular orbitals $\brac{\phi_i}_{i =
        1}^{N}$ such that $\energyfunc{\slat}$ becomes stationary\footnote{This
        does not guarantee that we have found a minimum, but often the
        stationary state will be a minimum.}.
