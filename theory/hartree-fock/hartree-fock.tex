\chapter{Hartree-Fock theory}
    \label{chap:hf}
    One can not tackle the subject of many-body theory without a discussion of
    the Hartree-Fock method.
    It serves as an excellent initial approximation, and in many cases the most
    the favored approximation, to the many-body wave function for a given system.
    It is a rather cheap method, in terms of computational intensity, and
    explains much of the underlying physics of a given system of many particles.

    \section{Time-independent Hartree-Fock theory}
        \label{sec:hf}
        We start from the time-independent Schrödinger equation,
        \begin{align}
            \hamil\ket*{\Psi} = \energy\ket*{\Psi},
        \end{align}
        where $\hamil$ is the electronic Hamiltonian with one- and two-body
        operators.
        We know that the ground state of the one-body Hamiltonian will be a
        single Slater determinant as demonstrated in
        \autoref{sec:slater-determinants}.
        If the two-body interactions are weak, we can treat these contributions
        perturbatively and a single Slater determinant will serve as a good
        approximation to the full many-body wave function.\footnote{%
            We will see that it does not take much before the two-body
            interaction becomes a little more than just a small perturbation.
        }
        This motivates the approximation that the many-body wave function
        $\ket*{\Psi}$ can be approximated by a single Slater determinant, viz.
        \begin{align}
            \ket*{\Psi} = \ket*{\slat} = \ket*{\phi_1 \phi_2 \dots \phi_N},
        \end{align}
        where the $N$ \emph{molecular orbitals}\footnote{%
            The term molecular orbitals will be used exclusively to denote the
            ``Hartree-Fock'' orbitals.
        } $\brac{\phi_i}$ are the primary unknowns, subject to the constraint
        that they are orthonormal,
        \begin{align}
            \braket*{\phi_i}{\phi_j} = \delta_{ij}
            \implies
            \braket*{\slat} = 1.
        \end{align}

        \subsection{The non-canonical Hartree-Fock equations}
            Starting from the variational principle, we define the energy
            functional
            \begin{align}
                \energyfunc{\slat, \slat^{*}}
                \equiv \mel*{\slat}{\hamil}{\slat}
                =
                \mel*{\phi_i}{\onehamil}{\phi_i}
                + \half\mel*{\phi_i\phi_j}{\twohamil}{\phi_i\phi_j}_{AS},
                \label{eq:energy_func_hf}
            \end{align}
            found from the definition of the reference energy in
            \autoref{subsec:reference-energy}.
            Our task is now to find the molecular orbitals $\brac{\phi_i}$ that
            minimize the energy functional, i.e., we find the stationary points
            of the energy functional as discussed in
            \autoref{sec:variational-principle}.
            As the energy functional $\energyfunc{\slat, \slat^{*}}$ does not
            incorporate the constraint that the molecular orbitals should be
            orthonormal, we use Lagrange's method of undetermined multipliers.
            This yields the Lagrangian functional
            \begin{align}
                \lagrangianfunc{\slat, \slat^{*}, \lambda}
                &= \energyfunc{\slat, \slat^{*}}
                - \lambda_{ji}\para{
                    \braket*{\phi_i}{\phi_j}
                    - \delta_{ij}
                },
            \end{align}
            where $\lambda_{ji}$ are Lagrange multipliers, one for each
            constraint.
            As the Lagrangian functional is real and the constraint is
            Hermitian, the Lagrange multipliers can be chosen Hermitian as well.

            \begin{proof}[%
                    Proof that the Lagrange multipliers can be chosen Hermitian%
                ]
                We follow closely the derivation done by Mayer
                \cite{mayer2003simple}, we start by noticing that the constraint is
                Hermitian, i.e.,
                \begin{align}
                    \braket*{\phi_i}{\phi_j} - \delta_{ij}
                    = \braket*{\phi_j}{\phi_i}^{*} - \delta_{ji}.
                \end{align}
                As of now we have two independent Lagrange multipliers; one for
                the overlap $\braket*{\phi_i}{\phi_j}$ and another for the the
                complex conjugate $\braket*{\phi_j}{\phi_i}$.
                We can formulate the constraint for the real and imaginary part
                separately,
                \begin{align}
                    \Re\brac{
                        \braket*{\phi_i}{\phi_j}
                    }
                    &=
                    \half\brac{
                        \braket*{\phi_i}{\phi_j}
                        + \braket*{\phi_j}{\phi_i}
                    }
                    = 0,
                    \\
                    \Im\brac{
                        \braket*{\phi_i}{\phi_j}
                    }
                    &=
                    \frac{1}{2i}\brac{
                        \braket*{\phi_i}{\phi_j}
                        - \braket*{\phi_j}{\phi_i}
                    }
                    = 0.
                \end{align}
                Introducing two separate Lagrange multipliers $\mu_{ij}$ and
                $\nu_{ij}$ for the two latter conditions, we get
                \begin{align}
                    \mu_{ij}\Re\brac{
                        \braket*{\phi_i}{\phi_j}
                    }
                    + \nu_{ij}\Im\brac{
                        \braket*{\phi_i}{\phi_j}
                    }
                    &=
                    \half\brak{
                        \mu_{ij} - i\nu_{ij}
                    }
                    \braket*{\phi_i}{\phi_j}
                    \nonumber
                    \\
                    &\qquad
                    + \half\brak{
                        \mu_{ij} + i\nu_{ij}
                    }
                    \braket*{\phi_j}{\phi_i}.
                \end{align}
                We now choose our combined Lagrange multipliers to be
                \begin{align}
                    \lambda_{ji} &=
                    -\half\brak{
                        \mu_{ij} - i\nu_{ij}
                    }, \\
                    \lambda_{ij} &=
                    -\half\brak{
                        \mu_{ij} + i\nu_{ij}
                    },
                \end{align}
                which implies that $\lambda_{ji} = \lambda_{ij}^{*}$, which is
                what we wanted to show.
            \end{proof}

            We are now interested in finding a stationary point of the
            Lagrangian with respect to small variations in functional
            dependency, that is, the molecular orbitals and the Lagrange
            multipliers.
            The stationary conditions for the Lagrange multipliers yield the
            constraint that the molecular orbitals should be orthonormal, viz.
            \begin{gather}
                \dpd{}{\lambda_{ji}}\lagrangianfunc{\slat, \slat^{*}, \lambda}
                = 0
                \implies
                \braket*{\phi_i}{\phi_j} = \delta_{ij},
            \end{gather}
            which are included in the end when we find solutions that are
            orthonormal as the constraint is included in the variation over the
            molecular orbitals \cite{kvaal2017notes}.
            The variation over a specific orbital $k$ is given by
            \begin{align}
                \tilde{\phi}_i(x) = \phi_{i}(x) + \delta_{ik}\epsilon\eta(x),
            \end{align}
            where $\epsilon \in \mathbb{R}$ is a small number and $\eta(x)$ is a
            single-particle function over some coordinate $x$.
            Note the use of the Kronecker-Delta to ensure that the variation
            only occurs for a single orbital at a time.
            This variation is similar for the complex conjugate of the
            molecular orbitals, but with $\phi^{*}_i(x)$ and $\eta^{*}(x)$
            instead.
            To avoid too much clutter, we will denote a variation over the
            molecular orbitals by a variation over the Slater determinants.
            That is, we take $\tilde{\Phi}(\vfg{x})$ to mean a variation over a
            single orbital $\phi_i(x)$ in $\Phi(\vfg{x})$, and similarly for the
            complex conjugate.
            Taylor expanding the Lagrangian functional,
            \begin{align}
                \lagrangianfunc{\tilde{\slat}, \tilde{\slat}^{*}, \lambda}
                &=
                \lagrangianfunc{\slat, \slat^{*}, \lambda}
                +
                \left.
                \dpd{
                    \lagrangianfunc{\tilde{\slat}, \slat^{*}, \lambda}
                }{\epsilon}
                \right\lvert_{\epsilon = 0}
                \epsilon
                \nonumber \\
                &\qquad
                +
                \left.
                \dpd{
                    \lagrangianfunc{\slat, \tilde{\slat}^{*}, \lambda}
                }{\epsilon}
                \right\lvert_{\epsilon = 0}
                \epsilon
                + \dots,
            \end{align}
            where the dots represent variations beyond the first-order
            stationary condition.
            As discussed in the section on the variational principle, the first
            order variations does not guarantee that we have found a minimum,
            but they do it \emph{often enough} for us to not bother with second
            order variations or more \cite{szabo1996modern}.
            In lieu of this comforting thought, we proceed on our merry way by
            finding the stationary points using the method of functional
            derivatives as dicussed in \autoref{sec:variational-principle}.
            We will restrict ourselves to the variation over the complex
            conjugated orbitals as both variations yield the same equation
            adjointed of one another.
            % TODO: Is this an allowed word? Rethink this sentence.
            This yields
            \begin{align}
                \lagrangianfunc{\Phi, \tilde{\Phi}^{*}, \lambda}
                =
                \energyfunc{\Phi, \tilde{\Phi}^{*}}
                - \lambda_{ji}\brak{
                    \braket*{\phi_i}{\phi_j}
                    + \epsilon \delta_{ik}\braket*{\eta}{\phi_j}
                    - \delta_{ij}
                },
            \end{align}
            where we used the linearity of the inner product to split ut the
            variation over the orbital $\phi^{*}_i(x)$ in the constraint term.
            Keeping only first-order variations in the energy functional we find
            \begin{align}
                \energyfunc{\Phi, \tilde{\Phi}^{*}}
                = \energyfunc{\Phi, \Phi^{*}}
                + \epsilon\para{
                    \mel*{\eta}{\onehamil}{\phi_k}
                    + \mel*{\eta\phi_j}{\twohamil}{\phi_k\phi_j}_{AS}
                }
                + \mathcal{O}(\epsilon^2),
            \end{align}
            where we have used the antisymmetric properties of the two-body
            elements to collect the two variations over the left-hand side of
            the matrix elements.
            Furthermore, we have collapsed one of the sums over the
            Kronecker-Delta in the variation in all terms.
            The variation in the Lagrangian functional can now be found by
            \begin{align}
                \delta\lagrangianfunc{\Phi, \tilde{\Phi}^{*}, \lambda}
                &=
                \lagrangianfunc{\Phi, \tilde{\Phi}^{*}, \lambda}
                -
                \lagrangianfunc{\Phi, \Phi^{*}, \lambda}
                \\
                &=
                \epsilon\brak{
                    \mel*{\eta}{\onehamil}{\phi_k}
                    +
                    \mel*{\eta\phi_j}{\twohamil}{\phi_k\phi_j}_{AS}
                    - \lambda_{jk}
                    \braket*{\eta}{\phi_j}
                }.
            \end{align}
            Having collapsed one of the orbital sums to yield $\phi_k$ we now
            restrict the variation over the molecular orbitals to
            \begin{align}
                \delta\phi^{*}_k(x)
                = \tilde{\phi}^{*}_k(x)
                - \phi^{*}_k(x)
                = \epsilon\eta^{*}(x).
            \end{align}
            Computing the stationary point of the Lagrangian with variations
            over $\phi^{*}_k(x)$ now gives
            \begin{gather}
                \left.
                \dpd{
                    \lagrangianfunc{\slat, \tilde{\slat}^{*}, \lambda}
                }{\epsilon}
                \right\lvert_{\epsilon = 0}
                =
                \int\dd x\frac{
                    \delta\lagrangianfunc{\slat, \tilde{\slat}^{*}, \lambda}
                }{\delta \phi^{*}_k(x)}
                \eta^{*}(x)
                = 0
                \\
                \implies
                \mel*{\eta}{\onehamil}{\phi_k}
                + \mel*{\eta\phi_j}{\twohamil}{\phi_k\phi_j}_{AS}
                = \lambda_{jk}\braket*{\eta}{\phi_j},
                \label{eq:stationary-lagrangian}
            \end{gather}
            which according to the fundamental lemma of calculus of variations
            \cite{jost1998calculus} must be valid for all variations
            $\epsilon\eta^{*}(x)$.
            We now introduce the single-particle \emph{Fock operator} by
            its matrix elements
            \begin{align}
                \mel*{\phi_p}{\fock}{\phi_q}
                \equiv
                \mel*{\phi_p}{\onehamil}{\phi_q}
                +
                \mel*{\phi_p\phi_j}{\twohamil}{\phi_q\phi_j}_{AS},
                \label{eq:fock-mel}
            \end{align}
            where we use the Fermi vacuum formalism for the sums.
            This lets us write the stationary point of the Lagrangian functional
            from \autoref{eq:stationary-lagrangian} as
            \begin{align}
                \fock\ket*{\phi_k}
                = \lambda_{jk}\ket*{\phi_j}.
                \label{eq:non-canonical-hartree-fock}
            \end{align}
            This equation is known as the \emph{non-canonical Hartree-Fock
            equations}, where the molecular orbitals $\phi_k$ are orthonormal.

        \subsection{Canonical Hartree-Fock equations}
            Now, \autoref{eq:non-canonical-hartree-fock} will yield the
            stationary Hartree-Fock energy, but from
            \autoref{lemma:unitary-transformation-slater} we know that a unitary
            transformation of the Slater determinant $\Phi(\vfg{x})$ leaves the
            determinant invariant up to a complex phase.
            We therefore look for a new set of spin-orbitals $\brac{\psi_p}$
            which will diagonalize the non-canonical Hartree-Fock equations.
            We define the unitary transformation from the non-canonical orbitals
            $\brac{\phi_p}$ to the new spin-orbitals by
            \begin{align}
                \ket*{\psi_p} = U_{qp}\ket*{\phi_q}
                \iff
                \ket*{\phi_r} = U^{*}_{pr}\ket*{\psi_p},
            \end{align}
            where $U_{qp}$ is an element in the unitary matrix $\vfg{U}$.
            From \autoref{lemma:unitary-transformation-orthonormality} we know
            that the new basis set $\brac{\psi_{p}}$ will preserve the
            orthonormality of the original basis set $\brac{\phi_q}$.
            We construct the Hermitian matrix $\vfg{\Lambda}$ from the Hermitian
            Lagrange multipliers $\lambda_{ji}$.
            From the Schur decomposition \cite{mat-inf4130} we can write
            \begin{align}
                \vfg{\Lambda} = \vfg{U} \vfg{\energy} \vfg{U}^{\dagger},
            \end{align}
            where we from the spectral theorem know that the matrix
            $\vfg{\energy} = \diag(\epsilon_1, \dots)$ will be diagonal with the
            eigenvalues $\epsilon_m$ of $\vfg{\Lambda}$ on the diagonal
            \cite{mat-inf4130}.
            Using the Kronecker-Delta of rank 3 defined in
            \autoref{eq:rank-3-kd}, the Schur decomposition of the Lagrange
            multipliers then take on the form
            \begin{align}
                \lambda_{jk} = U_{ji} E_{il} U^{*}_{kl}
                = U_{ji} \delta^{m}_{il} \epsilon_m U^{*}_{kl}
                \implies
                \delta^{m}_{il} \epsilon_m
                = U^{*}_{ji} \lambda_{jk} U_{kl}.
            \end{align}
            Transforming from the non-canonical molecular orbitals in
            \autoref{eq:non-canonical-hartree-fock} to the new spin-orbital
            basis, we get
            \begin{gather}
                \fock\ket*{\phi_k} = \lambda_{jk}\ket*{\phi_j}
                \implies
                \fock U^{*}_{lk}\ket*{\psi_{l}}
                = \lambda_{jk} U^{*}_{lj} \ket*{\psi_l}
                \\
                \implies
                U^{*}_{lk} U_{km} \fock\ket*{\psi_l}
                = U^{*}_{lj} \lambda_{jk} U_{km} \ket*{\psi_l}
                \implies
                \delta_{ml} \fock \ket*{\psi_l}
                = \delta^{n}_{lm} \epsilon_{n} \ket*{\psi_l}
                \\
                \implies
                \fock\ket*{\psi_m}
                = \epsilon_{m} \ket*{\psi_m},
                \label{eq:canonical-hartree-fock}
            \end{gather}
            where we are left with the \emph{canonical Hartree-Fock equations},
            which constitutes an eigenvalue equation.
            It is worth re-iterating that the equations were found from first
            order variations in the variational principle and therefore do not
            guarantee that we have found a minimum.
            In order to categorize the stationary point we can perform higher
            order variations to determine if we have found a minimum, saddle
            point, or a maximum.
            However, we will not explore higher-order variations any further.

        \subsection{The mean-field approximation}
            It is worth discussing the Fock operator defined in
            \autoref{eq:fock-mel} in more detail.

            First, if we look at a non-interacting system, that is,
            $\twoten^{pq}_{rs} = 0$ for all indices, then the canonical
            Hartree-Fock equations in \autoref{eq:canonical-hartree-fock}
            reduce to the time-independent Schrödinger equation for the
            orbitals.
            Furthermore, creating a Slater determinant from the spin-orbitals
            with the lowest eigenenergy yield the ground state of the
            non-interacting many-body problem.

            Moving back to the interacting system we adopt a common notation
            used from quantum chemistry \cite{szabo1996modern} by introducing
            the Coulomb operator
            \begin{align}
                \hat{J}
                = \int \dd x_2
                \phi^{*}_j(x_2)\twohamil(x, x_2)
                \phi_j(x_2),
                \label{eq:coulomb-operator}
            \end{align}
            where we denote the two-body operator $\twohamil(x_1, x_2)$ with
            explicit coordinates to describe which coordinates are to be
            integrated.
            The matrix elements of the Coulomb operator are given by
            \begin{align}
                \mel*{\phi_i}{\hat{J}}{\phi_k}
                &= \int\dd x_1\dd x_2
                \phi^{*}_i(x_1)\phi^{*}_j(x_2)
                \twohamil(x_1, x_2)
                \phi_k(x_1)\phi_j(x_2)
                \\
                &=
                \mel*{\phi_i\phi_j}{\twohamil}{\phi_k\phi_j}.
            \end{align}
            That is, the Coulomb operator is just the two-body elements, but
            formulated as a single-particle operator as the inner integral is
            pre-computed.
            The interpretation of the Coulomb operator is that all
            single-particle states $\phi_j(x)$ create an average potential
            which an individual particle will interact with, hence the name
            ``mean-field potential''.
            Now, the second term arising from the antisymmetric two-body
            elements yield nonlocal effects as it cannot be described as an
            average potential in the same way as the Coulomb operator.
            The exchange potential is therefore defined in terms of its action
            on a spin-orbital $\phi_k(x)$,
            \begin{align}
                \hat{K}\ket*{\phi_k}
                = \brak{
                    \int\dd x_2 \phi^{*}_j(x_2)\twohamil(x, x_2)\phi_k(x_2)
                }\phi_j(x).
                \label{eq:exchange-operator}
            \end{align}
            The exchange operator represents a strong deviation from classical
            mechanics as the potential experienced by a single-particle state
            depends on the value of the single-particle state in all of
            coordinate space.
            The matrix elements of the exchange operator are given by
            \begin{align}
                \mel*{\phi_i}{\hat{K}}{\phi_k}
                &= \int\dd x_1\dd x_2
                \phi^{*}_i(x_1)\phi^{*}_j(x_2)
                \twohamil(x_1, x_2)
                \phi_j(x_1)\phi_k(x_2)
                \\
                &=
                \mel*{\phi_i\phi_j}{\twohamil}{\phi_j\phi_k}.
            \end{align}
            The Fock operator can now be written as
            \begin{align}
                \fock
                = \onehamil
                + \hat{J} - \hat{K},
            \end{align}
            where the Coulomb and exchange operators have reduced the
            two-particle contributions to a single-particle mean-field
            potential.
            These terms are the reason why the Hartree-Fock method is called a
            mean-field approximation.
            In terms of three-particle interactions, the Hartree-Fock method can
            be used in a similar fashion where the mean-field approximation now
            incorporates the three-particle interactions.
            % TODO: Find references for this.

        \subsection{Brillouin's theorem}
            An important result from the canonical Hartree-Fock equations is
            Brillouin's theorem.
            \begin{theorem}
                \label{theorem:brillouin}
                Let $\brac{\phi_p}$ be an orthonormal set of single-particle
                states found by solving the canonical Hartree-Fock equations,
                and sorted such that the Fock energies $\epsilon_1 \leq
                \epsilon_2 \leq \dots$, then building the reference Slater
                determinant $\ket*{\slat}$ from the $N$ first single-particle
                states, we have that
                \begin{align}
                    \mel*{\slat}{\hamil}{\slat^{a}_{i}} = 0,
                \end{align}
                where $\ket*{\slat^{a}_{i}}$ is a singly-excited determinant.
                The converse is also true.
            \end{theorem}
            This theorem is important because it tells us that all single
            excitations from the reference state can be neglected if we choose
            the Hartree-Fock reference state as our reference determinant.
            Note that the theorem does not hold for arbitrary excitations and
            neither for single-excitations from an already excited state.
            \begin{proof}
                We prove the converse of Brillouin's theorem directly by
                evaluating the matrix elements
                \begin{align}
                    \mel*{\slat}{\hamil}{\slat^{a}_{i}}
                    &= \mel*{\phi_i}{\onehamil}{\phi_a}
                    + \mel*{\phi_i\phi_j}{\twohamil}{\phi_a\phi_j}_{AS}
                    = \mel*{\phi_i}{\fock}{\phi_a},
                \end{align}
                where we have used the Slater-Condon rules from
                \autoref{lemma:slater-condon-one-body} and
                \autoref{lemma:slater-condon-two-body} to evaluate the matrix
                elements.
                As the single-particle basis is represented by the molecular
                orbitals found from solving the canonical Hartree-Fock
                equations, the Fock matrix is diagonal.
                This means that
                \begin{align}
                    \mel*{\phi_i}{\fock}{\phi_a}
                    = \varepsilon_a\braket*{\phi_a}{\phi_i}
                    = 0,
                \end{align}
                where we have used that the molecular orbitals are orthonormal by
                construction.
                Proving the opposite proposition can be done by starting from
                the canonical Hartree-Fock equations and then using the
                Slater-Condon rules in reverse to get the original formulation
                of Brillouin's theorem.
                % TODO: Can you?
            \end{proof}


\section{The time-dependent Hartree-Fock equations in a basis}
    In the time-dependent Hartree-Fock method we will evolve the molecular
    orbitals in time, but we keep the atomic orbitals fixed.
    This means that
    \begin{align}
        \ket{\phi_i(t)} = C_{\alpha i}(t)\ket{\chi_{\alpha}},
    \end{align}
    that is, the time-dependence occurs in the coefficients $C_{\alpha i}(t)$.
    Thus the time-dependent Hartree-Fock Slater determinant is given by
    \begin{align}
        \ket{\Phi(t)}
        &=
        \ket{
            \phi_1(t), \dots, \phi_N(t)
        }.
    \end{align}
    The time-dependent Slater determinants in a system with a time-dependent
    Hamiltonian $\hat{H}(t)$ satisfies the Schrödinger equation
    \begin{align}
        i\hslash\dpd[]{}{t}\ket{\Phi(t)}
        &= \hamil(t)\ket{\Phi(t)},
    \end{align}
    where $\hamil(t)$ is the time-dependent electronic Hamiltonian.

    \subsection{Deriving the time-dependent Hartree-Fock equations}
        For the sake of brevity we will remove the explicit time-dependence in
        the orbitals, the Slater determinants and the operators, though they are
        still present.
        We will be following closely the derivation done by
        \citeauthor{hochstuhl2014time} \cite{hochstuhl2014time}, but for a
        single Slater determinant.
        Furthermore, the time derivative of the Slater determinant is
        % TODO: How do you get this result?
        % Write out SD as sum, and use orthonormality of orbitals?
        % Seems like only one term "survives"
        \begin{align}
            \bra{\Phi}\dpd{}{t}\ket{\Phi}
            &= \bra{\phi_i}\dpd{}{t}\ket{\phi_i}.
        \end{align}
        We can now find the equations of motion by applying time-dependent
        variational principle, that is,
        \begin{align}
            \bra{\delta\Phi}\hat{H} - i\hslash\dpd{}{t}\ket{\Phi} = 0,
        \end{align}
        under the constraint that the time-dependent molecular orbitals are
        orthonormal, i.e.,
        \begin{align}
            \braket{\phi_i(t)}{\phi_j(t)} = \delta_{ij},
        \end{align}
        where we for the sake of emphasis temporarily added the explicit
        time-dependence in the orbitals. We now define the Lagrangian functional
        with the Lagrange multipliers $\lambda_{ij}$ to be
        \begin{align}
            \mathcal{L}
            &= \bra{\Phi}\hat{H} - i\hslash\dpd{}{t}\ket{\Phi}
            - \lambda_{ij}\para{
                \braket{\phi_i}{\phi_j} - \delta_{ij}
            }.
        \end{align}
        We now find an extremum of $\mathcal{L}$ by taking the functional
        derivative of the functional and equating it to zero.
        \begin{gather}
            \dfd{\phi_k^{*}}\mathcal{L} = 0, \\
            \implies
            \hat{h}\ket{\phi_k}
            + \bra{\cdot\phi_j}\hat{u}\ket{\phi_k\phi_j}_{AS}
            - i\hslash\dpd{}{t}\ket{\phi_k}
            - \lambda_{kj}\ket{\phi_j} = 0.
            \label{eq:extremum_lagrangian}
        \end{gather}
        Left-projecting with an orbital $\bra{\phi_l}$ and solving for the
        Lagrange multipliers, $\lambda_{kl}$, where we apply the constraint that
        the orbitals are orthonormal we get
        \begin{align}
            \lambda_{kl}
            =
            \bra{\phi_l}\hat{h}\ket{\phi_k}
            + \bra{\phi_l\phi_j}\hat{u}\ket{\phi_k\phi_j}_{AS}
            - i\hslash\bra{\phi_l}\dpd{}{t}\ket{\phi_k}.
            \label{eq:lagrange_multipliers}
        \end{align}
        Inserting the expression for the Lagrange multipliers in
        \autoref{eq:lagrange_multipliers} into \autoref{eq:extremum_lagrangian}
        we get
        \begin{align}
            \hat{P}\para{
                \hat{h}\ket{\phi_k}
                + \bra{\cdot\phi_j}\hat{u}\ket{\phi_k\phi_j}_{AS}
                - i\hslash\dpd{}{t}\ket{\phi_k}
            } = 0,
        \end{align}
        where we have defined the projection operator, $\hat{P}$, to be
        \begin{align}
            \hat{P} = \1 - \ket{\phi_i}\bra{\phi_i}.
        \end{align}
        Solving for the time-derivative of the molecular orbitals we get
        \begin{align}
            i\hslash\hat{P}\dpd{}{t}\ket{\phi_k}
            = \hat{P}\para{
                \hat{h}\ket{\phi_k}
                + \bra{\cdot\phi_j}\hat{u}\ket{\phi_k\phi_j}_{AS}
            }
            = \hat{P}\hat{f}\ket{\phi_k},
            \label{eq:tdhf_equations_p}
        \end{align}
        where we have defined the Fock operator to be
        \begin{align}
            \hat{f}
            \equiv \hat{h} + \bra{\cdot\phi_j}\hat{u}\ket{\cdot\phi_j}_{AS}.
        \end{align}
        We now define an arbitrary hermitian operator $\hat{Q}(t)$ in terms of
        a unitary transformation
        \begin{align}
            \bra{\phi_i}i\hslash\dpd{}{t}\ket{\phi_j}
            \equiv \bra{\phi_i}\hat{Q}(t)\ket{\phi_j}.
            \label{eq:unitary_transformation_time}
        \end{align}
        This is valid as any Slater determinant is invariant under unitary
        transformations. See section \ref{proof:slater_determinants_invariant}
        for a short proof of this. As the Slater determinant is invariant under
        the unitary transformation, by extension the Lagrangian functional will
        be invariant. Expanding $\hat{P}$ in
        \autoref{eq:tdhf_equations_p} and inserting
        \autoref{eq:unitary_transformation_time} we get
        \begin{gather}
            i\hslash\dpd{}{t}\ket{\phi_k}
            - i\hslash\ket{\phi_i}\bra{\phi_i}\dpd{}{t}\ket{\phi_k}
            = \hat{f}\ket{\phi_k}
            - \ket{\phi_i}\bra{\phi_i}\hat{f}\ket{\phi_k}
            \\
            \implies
            i\hslash\dpd{}{t}\ket{\phi_k}
            - \ket{\phi_i}\bra{\phi_i}\hat{Q}(t)\ket{\phi_k}
            = \hat{f}\ket{\phi_k}
            - \ket{\phi_i}\bra{\phi_i}\hat{f}\ket{\phi_k}.
        \end{gather}
        Since $\hat{Q}(t)$ was an arbitrary hermitian operator and
        $\hat{f}^{\dagger}(t) = \hat{f}(t)$ we can choose $\hat{Q}(t) =
        \hat{f}(t)$ yielding the time-dependent Hartree-Fock equations on the
        form
        \begin{align}
            i\hslash\dpd{}{t}\ket{\phi_k}
            = \hat{f}\ket{\phi_k}.
        \end{align}
        We will in the next section restore the explicit time-dependence in the
        molecular orbitals and the operators.

    \subsection{Evolving the coefficients in time}
        Having restricted the full wave function to a single Slater determinant,
        we can write the equations for the molecular orbitals in terms of the
        time-dependent Fock operator.
        \begin{align}
            i\hslash\dpd{}{t}\ket{\phi_i(t)}
            &= \hat{f}(t)\ket{\phi_i(t)}.
        \end{align}
        We can now insert the linear combination for the molecular orbitals in
        the atomic orbital basis with the time-dependent coefficients.
        \begin{align}
            i\hslash\dpd{}{t}C_{\alpha i}(t)\ket{\chi_{\alpha}}
            &= \hat{f}(t)C_{\alpha i}(t)\ket{\chi_{\alpha}}.
        \end{align}
        Left projecting with another state from the atomic orbitals we can rewrite
        the previous equation to
        \begin{gather}
            i\hslash\dpd{}{t}C_{\alpha i}(t)\braket{\chi_{\beta}}{\chi_{\alpha}}
            =
            C_{\alpha i}(t)\bra{\chi_{\beta}}\hat{f}(t)\ket{\chi_{\alpha}}
            \\
            \implies
            i\hslash \dot{C}_{\alpha i} S_{\beta\alpha}
            = C_{\alpha i}(t)f_{\beta\alpha}(t)
            \\
            \implies
            i\hslash \vf{S}\dot{\vf{C}}
            = \vf{F}(t)\vf{C}(t),
            \label{eq:tdhf-equations}
        \end{gather}
        where the overlap matrix $\vf{S}$ is the matrix with elements
        \begin{align}
            S_{\beta\alpha} \equiv \braket{\chi_{\beta}}{\chi_{\alpha}},
            \label{eq:overlap_ao}
        \end{align}
        and the time-dependent Fock matrix $\vf{F}(t)$ contains the entries
        \begin{align}
            f_{\beta\alpha}(t) = \bra{\chi_{\beta}}\hat{f}(t)\ket{\chi_\alpha}.
        \end{align}
        Often we will work with basis sets where the overlap matrix reduces to
        the identity. In this case, \autoref{eq:tdhf-equations} collapses to the
        equation
        \begin{align}
            i\hslash \dot{\vf{C}} = \vf{F}(t)\vf{C}(t).
            \label{eq:tdhf-orthogonal}
        \end{align}

    \subsection{Time evolution in a non-orthogonal basis of atomic orbitals}
        In order to solve \autoref{eq:tdhf-orthogonal} for non-orthogonal
        atomic orbitals and avoid having to solve \autoref{eq:tdhf-equations}
        as they stand, we orthogonalize the atomic orbitals. We will follow the
        derivation shown in the book \citetitle{szabo1996modern} by
        \citeauthor{szabo1996modern} \cite{szabo1996modern} closely.

        For the case when
        \begin{align}
            \braket{\chi_{\alpha}}{\chi_{\beta}} \equiv S_{\alpha\beta}
            \neq \delta_{\alpha\beta},
        \end{align}
        that is, a non-orthogonal atomic orbital basis
        $\brac{\ket{\chi_{\alpha}}}_{\alpha = 1}^{N}$, we move to a transformed
        orthonormal basis $\brac{\ket{\xi_{\mu}}}_{\mu = 1}^{K}$ by the
        transformation
        \begin{align}
            \ket{\xi_{\mu}} = X_{\alpha\mu}\ket{\chi_{\alpha}}.
        \end{align}
        This yields an equation on the form
        \begin{align}
            \braket{\xi_{\mu}}{\xi_{\nu}}
            &= \para{
                X_{\alpha\mu}^{*}\bra{\chi_{\alpha}}
            }\para{
                X_{\beta\nu}\ket{\chi_{\beta}}
            }
            = X_{\alpha\mu}^{*}S_{\alpha\beta}X_{\beta\nu}
            = \delta_{\mu\nu}.
        \end{align}
        As a matrix equation we get
        \begin{align}
            \vf{X}^{\dagger}\vf{S}\vf{X} = \1.
            \label{eq:orthonormal-transformation}
        \end{align}
        This, along with the requirement that $\vf{X}$ must be non-singular,
        forms the restrictions on the matrix $\vf{X}$ in order for
        $\brac{\ket{\xi_{\mu}}}_{\mu = 1}^{L}$ to form an orthonormal basis.
        Since $\vf{S}$ is Hermitian (by construction), there exists a unitary
        matrix $\vf{U}$ that diagonalizes $\vf{S}$ by
        \begin{align}
            \vf{U}^{\dagger}\vf{S}\vf{U} = \vf{s},
        \end{align}
        where $\vf{s}$ is a diagonal matrix with the eigenvalues of $\vf{S}$.
        The matrix $\vf{S}$ is positive definite as all the eigenvalues of
        $\vf{S}$ are positive.
        We can find an equation for $\vf{X}$ using \emph{symmetric
        orthogonalization} \cite{mayer2002lowdin}.
        This yields
        \begin{align}
            \vf{X} = \vf{S}^{-1/2} = \vf{U}\vf{s}^{-1/2}\vf{U}^{\dagger}.
        \end{align}
        We can now use $\vf{X}$ to transform to the new basis
        $\brac{\xi_{\mu}}_{\mu = 1}^{L}$ and solving the Roothan-Hall equations
        in this orthonormal basis. In practice we do this by transforming the
        coefficient matrix $\vf{C}$ from the atomic orbitals in terms of the
        orthonormal orbitals $\xi_{\mu}$. In terms of the original atomic
        orbitals we can formulate this by
        \begin{align}
            \ket{\phi_i}
            &= C_{\mu i}'\ket{\xi_{\mu}}
            = C_{\mu i}'X_{\alpha \mu}\ket{\chi_{\alpha}}
            = C_{\alpha i}\ket{\chi_{\alpha}},
        \end{align}
        where $C_{\mu i}'$ are the coefficients from the new orthonormal orbital
        basis $\ket{\xi_{\mu}}$ to the molecular orbitals. In terms of the
        coefficient matrices we then get
        \begin{align}
            \vf{C} = \vf{X}\vf{C}',
        \end{align}
        where $\vf{C}$ are the original coefficients from the original atomic
        orbital basis $\ket{\chi_{\alpha}}$ to the molecular orbital basis.
        Having eliminated the linear dependencies in $\vf{S}$ there will exist
        an inverse for $\vf{X}$. Thus we find
        \begin{align}
            \vf{C}' = \vf{X}^{-1}\vf{C}.
        \end{align}
        Inserting this expression into \autoref{eq:tdhf-equations} when the
        coefficient matrices are time-dependent\footnote{Note that $\vf{X}(t) =
        \vf{X}$ as we keep the atomic orbitals fixed and let the coefficients
        handle the time-dependence.} we get
        \begin{gather}
            i\hslash\vf{S}\dot{\vf{C}}(t) = \vf{F}(t)\vf{C}(t) \\
            \implies
            i\hslash\vf{S}\vf{X}\dot{\vf{C}}'(t) = \vf{F}(t)\vf{X}\vf{C}'(t)
            \\
            \implies
            i\hslash\vf{X}^{\dagger}\vf{S}\vf{X}\dot{\vf{C}}'(t)
            = \vf{X}^{\dagger}\vf{F}(t)\vf{X}\vf{C}'(t)
            \\
            \implies
            i\hslash \dot{\vf{C}}'(t) = \vf{F}'(t)\vf{C}'(t),
        \end{gather}
        where we've used the orthonormality of the transformation in
        \autoref{eq:orthonormal-transformation} and defined the transformed Fock
        matrix
        \begin{align}
            \vf{X}^{\dagger}\vf{F}(t)\vf{X} = \vf{F}'(t).
        \end{align}
        Doing this transformation simplifies the use of linear algebra libraries
        to solve the time-dependent Hartree Fock equations.

    \subsection{Time dependent overlap}
        When working with Slater determinants in a given basis we can compute
        the overlap between two wavefunctions in an efficient manner. We define
        the orbitals
        \begin{align}
            \ket{\phi_i} &= C_{\alpha i}\ket{\chi_{\alpha}}, \\
            \ket{\psi_i} &= D_{\alpha i}\ket{\chi_{\alpha}},
        \end{align}
        as two separate linear combinations of the original atomic orbitals. The
        overlap between these two molecular orbitals is then
        \begin{align}
            \braket{\psi_i}{\phi_j}
            &= \para{
                D_{\alpha i}^{*}
                \bra{\chi_{\alpha}}
            }\para{
                C_{\beta j}
                \ket{\chi_{\beta}}
            }
            = D_{\alpha i}^{*}S_{\alpha\beta} C_{\beta j}
            = T_{ij}.
        \end{align}
        In matrix notation this reduces to
        \begin{align}
            \vf{T} = \vf{D}^{\dagger}\vf{S}\vf{C},
        \end{align}
        where $\vf{C}$ and $\vf{D}$ are the coefficient matrices of the set of
        molecular orbitals $\ket{\phi_i}$ and $\ket{\psi_i}$ respectively and
        $\vf{S}$ is the overlap matrix for the atomic orbitals.  We can then
        write the two Slater determinants consisting of $N$ of the molecular
        orbitals as
        \begin{align}
            \ket{\Phi} &= \frac{1}{\sqrt{N!}}
            \sum_{\sigma \in S_N}(-1)^{|\sigma|}
            \ket{\phi_{\sigma(1)}}
            \otimes\dots\otimes
            \ket{\phi_{\sigma(N)}}, \\
            \ket{\Psi} &= \frac{1}{\sqrt{N!}}
            \sum_{\sigma \in S_N}(-1)^{|\sigma|}
            \ket{\psi_{\sigma(1)}}
            \otimes\dots\otimes
            \ket{\psi_{\sigma(N)}},
        \end{align}
        where $S_N$ is the permutation group of all permutations $\sigma$ of the
        labels $i \in \brac{1, \dots, N}$. Furthermore, $|\sigma|$, is the
        defined as the number of permutations performed from the original order
        of the labels. The overlap of the two Slater determinants is then
        \begin{align}
            \braket{\Psi}{\Phi}
            &= \frac{1}{N!}
            \sum_{\sigma, \tau \in S_{N}}
            (-1)^{|\sigma||\tau|}
            \prod_{i = 1}^{N}
            \braket{\psi_{\sigma(i)}}{\phi_{\tau(i)}}
            \\
            &= \frac{1}{N!}
            \sum_{\sigma, \tau \in S_{N}}
            (-1)^{|\sigma||\tau|}
            \prod_{i = 1}^{N}
            T_{\sigma(i)\tau(i)},
        \end{align}
        where we now have a double sum over all permutations $\sigma$ and
        $\tau$. To go from here we have to collapse one of the sums over the
        permutations. By keeping one of the permutation sums fixed we get all
        unique combinations of the terms in the products. By iterating over the
        second sum over permutations we create the same set of combinations, but
        with a different ordering. As the terms commute, we can just add it to
        the already existing set of unique combinations. In total there are $N!$
        permutations in $S_N$ thus giving a factor of $N!$ to the sum. This
        yields
        \begin{align}
            \braket{\Psi}{\Phi}
            &= \frac{1}{N!}N!
            \sum_{\sigma \in S_N}
            (-1)^{|\sigma|}
            \prod_{i = 1}^{N}T_{i\sigma(i)}
            \\
            &= \begin{vmatrix}
                T_{11} & \dots & T_{1N} \\
                \vdots & \ddots & \vdots \\
                T_{N1} & \dots & T_{NN}
            \end{vmatrix}
            = \det(\vf{T})
        \end{align}
        In particular in the time-dependent Hartree-Fock regime where we have
        the time-dependent coefficients $C_{\alpha i}(t)$ we get the overlap
        with the ground state to be
        \begin{align}
            \braket{\Phi(t)}{\Phi(0)}
            &= \det(\vf{C}^{\dagger}(t)\vf{S}\vf{C}(0))
            = \det(\vf{T}(t)),
        \end{align}
        where $\vf{C}(t)$ is the coefficient matrix of the time evolved states.

        \subsubsection{Overlap with non-orthogonal atomic orbitals}
            In the case that we have transformed from the non-orthogonal atomic
            orbital basis $\brac{\ket{\chi_{\alpha}}}_{\alpha = 1}^{N}$ to the
            orthonormal atomic orbital basis $\brac{\ket{\xi_{\mu}}}_{\mu =
            1}^{N}$ by,
            \begin{align}
                \ket{\xi_{\mu}} = X_{\alpha\mu}\ket{\chi_{\alpha}},
            \end{align}
            where the transformation to the molecular orbitals is given by,
            \begin{align}
                \ket{\phi_{i}(t)} = C_{\mu i}'(t)\ket{\xi_{\mu}}
                = C_{\mu i}'(t)X_{\alpha\mu}\ket{\chi_{\alpha}}
                = C_{\alpha i}(t)\ket{\chi_{\alpha}},
            \end{align}
            we get the time-dependent overlap to be
            \begin{align}
                \braket{\Phi(t)}{\Phi(0)}
                &= \det(\vf{C}^{\dagger}(t)\vf{S}\vf{C}(0))
                \\
                &= \det\Bigl[
                    (\vf{C}'(t))^{\dagger} \vf{X}^{\dagger}
                    \vf{S}\vf{X}\vf{C}'(t)
                \Bigr]
                \\
                &=
                \det\Bigl[
                    (\vf{C}'(t))^{\dagger}
                    \vf{C}'(t)
                \Bigr].
            \end{align}
            This simplifies the process of computing the overlap, when we've
            already transformed to the orthonormal basis.

