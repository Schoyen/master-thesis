\section{Time-dependent Hartree-Fock theory}
    Time-dependent Hartree-Fock continues with the ansatz that the full
    many-body wave function $\ket*{\Psi(t)}$ is described by a single Slater
    determinant of $N$ single-particle states, viz.
    \begin{align}
        \ket*{\Psi(t)} = \ket*{\slat(t)}
        = \ket*{\phi_1(t) \phi_2(t)\dots \phi_N(t)},
    \end{align}
    where $\brac{\phi_p}$ is a basis of time-dependent molecular orbitals
    subject to the constraint that they are orthonormal in time.
    We now use the time-dependent variational principle as discussed in
    \autoref{sec:tdvp} in order to find the equations of motion for the system.
    The Lagrangian of the system is given by
    \begin{align}
        \lagrangianfunc{\slat, \slat^{*}, \lambda}
        = \mel*{\slat(t)}{\para{
            i\hslash\partial_t
            - \hamil(t)
        }}{\slat(t)}
        - \lambda_{ji}\para{
            \braket*{\phi_i(t)}{\phi_j(t)}
            - \delta_{ij}
        },
        \label{eq:lagrangian-tdhf}
    \end{align}
    where we keep the orthonormality condition in the time-dependent case as
    well, and we have ignored the explicit time-dependence in the functional
    arguments to the Lagrangian.
    The action functional of the time-dependent variational principle is given
    by
    \begin{align}
        S[\slat, \slat^{*}]
        =
        \int \dd t \lagrangianfunc{\slat, \slat^{*}, \lambda},
    \end{align}
    with the stationary condition over the first-order variations found from
    \begin{align}
        \delta S
        = \int\dd t \delta \lagrangianfunc{\slat, \slat^{*}, \lambda}
        = 0
        \implies
        \delta\lagrangianfunc{\slat, \slat^{*}, \lambda}
        = 0.
    \end{align}
    Computing the expectation value of the Hamiltonian in
    \autoref{eq:lagrangian-tdhf} yields the same expression for the reference
    energy as in \autoref{subsec:reference-energy}, but now with time-dependent
    operators and molecular orbitals.
    As the time-derivative is a Hermitian, single-particle operator, we find the
    expectation value to be
    \begin{align}
        \mel*{\slat(t)}{\partial_t}{\slat(t)}
        = \mel*{\phi_i(t)}{\partial_t}{\phi_i(t)}.
    \end{align}
    % TODO: Prove this in the appendix.
    Performing variations over a single molecular orbital at a time we have
    \begin{align}
        \tilde{\phi}_i(x, t)
        = \phi_i(x, t) + \delta_{ik}\epsilon\eta(x, t),
    \end{align}
    where again $\epsilon \in \mathbb{R}$ and $\eta(x, t)$ is a complex
    function.
    As done in the time-independent case, we will restrict our attention to
    variations over the complex conjugate of $\phi_i(x, t)$.
    Variations over $\phi_i(x, t)$ will yield the adjoint equation.
    The only new term in the variation over the time-dependent Lagrangian from
    the time-independent case is the variation over the time-derivative.
    We find
    \begin{align}
        \mel*{\tilde{\slat}(t)}{\partial_t}{\slat(t)}
        = \mel*{\phi_i(t)}{\partial_t}{\phi_i(t)}
        + \epsilon\mel*{\eta(t)}{\partial_t}{\phi_k(t)}.
    \end{align}
    From the stationary conditions of the action functional we find
    \begin{gather}
        \delta\lagrangianfunc{\slat, \tilde{\slat}^{*}, \lambda}
        = \lagrangianfunc{\slat, \tilde{\slat}^{*}, \lambda}
        - \lagrangianfunc{\slat, \slat^{*}, \lambda}
        = 0 \\
        \implies
        i\hslash\mel*{\eta}{\partial_t}{\phi_k}
        -\mel*{\eta}{\fock}{\phi_k}
        = \lambda_{jk}\braket*{\eta}{\phi_j},
    \end{gather}
    where we now for the sake of brevity removed the explicit time-dependence of
    the orbitals and the operators.
    Furthermore, we jumped straight from the variation of the energy functional
    to include the Fock-operator.
    As $\eta(x, t)$ was arbitrary, this must be valid for all choices of
    $\eta(x, t)$ and we can formulate the stationary condition as
    \begin{align}
        i\hslash\partial_t\ket*{\phi_k}
        - \fock\ket*{\phi_k}
        = \lambda_{jk}\ket*{\phi_j}.
        \label{eq:stationary-condition-1}
    \end{align}
    Projecting onto $\phi_l(x, t)$ and applying the constraint that the
    molecular orbitals are orthonormal we find an equation for the Lagrange
    multipliers,
    \begin{align}
        \lambda_{lk}
        = i\hslash\mel*{\phi_l}{\partial_t}{\phi_k}
        - \mel*{\phi_l}{\fock}{\phi_k}.
    \end{align}
    We now insert this expression for the Lagrange multipliers back into the
    stationary condition in \autoref{eq:stationary-condition-1}.
    This yields
    \begin{align}
        \hat{P}\brak{
            i\hslash\partial_t\ket*{\phi_k}
            - \fock\ket*{\phi_k}
        }
        = 0,
    \end{align}
    where we have now defined the projection operator
    \begin{align}
        \hat{P} \equiv \1 - \dyad{\phi_i}{\phi_i},
    \end{align}
    and we see that we have gotten rid of the Lagrange multipliers.
    Solving for the time-derivative of the molecular orbitals we find
    \begin{align}
        i\hslash\hat{P}\partial_t\ket*{\phi_k}
        = \hat{P}\fock\ket*{\phi_k}.
    \end{align}
    We now define an arbitrary time-dependent, Hermitian operator $\hat{Q}$
    in terms of a unitary transformation \cite{hochstuhl2014time}
    \begin{align}
        i\hslash\mel*{\phi_i}{\partial_t}{\phi_j}
        \equiv
        \mel*{\phi_i}{\hat{Q}}{\phi_j}.
    \end{align}
    We know that  Slater determinants are invariant up to a complex phase under
    unitary transformations, as seen in
    \autoref{lemma:unitary-transformation-slater}, this unitary transformation
    will leave the time-dependent Lagrangian invariant.
    Since $\hat{Q}$ was an arbitary, Hermitian operator and the Fock operator is
    Hermitian, we can choose $\hat{Q} = \fock$.
    Expanding the projection operator and inserting this choice, we get the
    time-dependent Hartree-Fock equations
    \begin{align}
        i\hslash\partial_t\ket*{\phi_k(t)}
        = \fock(t)\ket*{\phi_k(t)},
        \label{eq:tdhf}
    \end{align}
    where the molecular orbitals stays orthonormal in time.


\clearemptydoublepage
