\chapter{Formalism}
    In this chapter we'll describe some of the formalism used when describing
    many-body quantum mechanics.

    \section{Basis sets}
        In this document we will rely heavily on basis transformations and
        creating linear combinations from a basis to another. When creating
        many-body wavefunctions we will use a finite basis of \emph{orbitals}
        $\brac{\ket{\chi_{\alpha}}}_{\alpha = 1}^{L}$ as our \emph{atomic
        orbitals}. These represent known basis states, e.g., the harmonic
        oscillator eigenstates, etc. We can construct many-body wavefunctions
        from a linear combination of tensor products of single-particle states,
        \begin{align}
            \ket{\Psi}
            = \sum_{\sigma \in S} c_{\sigma}
            % TODO: Figure out how to label the coefficients c
            \ket{\chi_{\sigma(1)}}\otimes\dots\otimes\ket{\chi_{\sigma(L)}},
        \end{align}
        where we have assumed normalized single-particle functions
        $\braket{\chi_{\alpha}}{\chi_{\alpha}} = 1$, but not necessarily
        orthonormal.
        % TODO: Review these statements. Can the total wave function be
        % normalized if the single particle functions are normalized but not
        % orthonormal?

        \subsection{Fermi level}
            Given a basis of $L$ single particle functions $\ket{p}$ where
            \begin{align}
                \left\{\ket{p}\right\}_{p = 1}^{L}
                &=
                \left\{\ket{i}\right\}_{i = 1}^{N}
                \cup \left\{\ket{a}\right\}_{a = N + 1}^{L}.
            \end{align}
            Here $i, j, k, \dots$ represents the $N$ first occupied states of
            the reference Slater determinant whereas $a, b, c, \dots$ represent
            the remaining $M = L - N$ virtual states in the total basis $p, q,
            r, \dots$\footnote{Occupied and virtual states are also known as
            hole and particle states if we treat the reference Slater
            determinant as the \textit{Fermi level}}.

    \section{The normal-ordered Hamiltonian}
        A general electronic Hamiltonian with one- and two-body operators on
        second-quantized form is given by
        \begin{align}
            \hamil
            &= \oneten^{p}_{q}\ccr{p}\can{q}
            + \frac{1}{4}\twoten^{pq}_{rs}\ccr{p}\ccr{q}\can{s}\can{r},
        \end{align}
        where we've labelled the anti-symmetric two-body elements by
        $u^{pq}_{rs}$, and Einstein summation is assumed.
        We can write the electronic Hamiltonian on a normal ordered form using
        Wick's theorem.
