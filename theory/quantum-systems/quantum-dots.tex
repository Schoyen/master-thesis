\section{Quantum dots}
    Artificial atoms, or the so-called quantum dots, constitute a hot topic in
    condensed matter physics and material sciences. We will be exploring several
    types of quantum dots in both one and two dimensions in this thesis. The
    difference between the types of quantum dots is found in the one-body
    potential. All of the dots share the characteristic of being in an infinite
    well which makes the systems \emph{bound}. In our study of systems subject
    to intense laser fields, this will prove to be a bad approximation as the
    particles have no way of being excited, i.e., escape the potential well.
    Even so, for weak laser fields and for ground state calculations, they serve
    as excellent candidates for our methods.

    \subsection{One-dimensional harmonic oscillator}
        One of the simplest theoretical models studied. The one-dimensional
        harmonic oscillator is usually one of the first models studied in
        undergraduate quantum mechanics courses. The one-body Hamiltonian of the
        one-dimensional harmonic oscillator is given by
        \begin{align}
            \onehamil = -\frac{\momentum^2}{2m} + \half m\omega^2 \position^2,
        \end{align}
        where the latter term, i.e., the one-body potential, gives rise to the
        name of the system. Plugging this operator into the time-independent
        Schrödinger equation
        \begin{gather}
            \onehamil\ket{n} = \epsilon_n\ket{n},
        \end{gather}
        which results in an energy eigenvalue equation we wish to solve in order
        to find an expression for $\ket{n}$.
        % TODO: Discuss both the algebraic derivation and the solution using
        % diagonalization on a grid.
        % TODO: Include analytic solution to \ket{n}.
        % TODO: Include figues of the potential and the eigenstates.

    \subsection{One-dimensional double well quantum dot}
        A slightly more complicated model is the \emph{double well} quantum dot.
        Succintly named due to its ``bump'' in the bottom of the parabolic
        potential from the harmonic oscillator. The one-body hamiltonian is
        given by
        \begin{align}
            \onehamil = -\frac{\momentum^2}{2m}
            + \half m\omega^2\para{
                \position^2
                + \frac{1}{4}l^2
                - l|\position|
            },
        \end{align}
        where $l$ is the ``width'' of the potential barrier in the bottom of the
        parabola.
        % TODO: What is the width? RMS?

    \subsection{Two-dimensional harmonic oscillator}
        The two-dimensional harmonic oscillator

    \subsection{Two-dimensional double well quantum dot}
