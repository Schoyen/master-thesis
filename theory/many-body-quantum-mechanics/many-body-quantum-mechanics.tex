\chapter{Many-body quantum mechanics}
    In this chapter we'll describe some of the formalism used when describing
    many-body quantum mechanics.

    \section{Particle statistics}
        \label{sec:particle-statistics}
        The particles we concern ourselves with are \emph{identical}.
        This means that it is not possible to discern two particles of the
        same type from one another.
        This statement yields quite profound results in the sense that the
        ordering of the particles in a many-body wave function is in some sense
        arbitrary; the wave functions are the same up to a complex phase factor
        regardless of the ordering of their particles.
        As a consequence the probability density of our state must be
        permutation invariant since we are not able to distinguish between the
        ordering of identical particles.
        We define $\sigma \in S_{N}$ as a permutation of $N$ indices $\vf{x}
        = (x_1, \dots, x_N) \in X^{N}$ wherein $X^N$ is a coordinate space where
        both spin and position is incorporated.
        Furthermore, $S_{N}$ is the group of all permutations $\sigma$ where
        $S_{N}$ has $N!$ distinct permutations.
        We denote the permutation of the indices by
        \begin{align}
            \vf{x} = (x_1, \dots, x_N)
            \to \vf{x}_{\sigma} = (x_{\sigma(1)}, \dots, x_{\sigma(N)}).
            \label{eq:permutation-indices}
        \end{align}
        We can then formulate particle indistinguishability by
        \cite{leinaas1977, kvaal2017notes}
        \begin{align}
            \abs{
                \psi(\vf{x})
            }^2
            = \abs{
                \psi(\vf{x}_{\sigma})
            }^2.
            \label{eq:particle-indistinguishability}
        \end{align}
        This can be formulated as \cite{kvaal2017notes}
        \begin{align}
            \psi(\vf{x}) = \exp[i\alpha(\sigma)] \psi(\vf{x}_{\sigma}),
        \end{align}
        where $\alpha(\sigma) \in \mathbb{R}$ depends on $\sigma$.
        A transposition $\tau_{ij} \in S_{N}$ is a permutation exchanging a single
        pair $(i, j) \to (j, i)$ where $i \neq j$.
        We denote the transposition of two indices in a similar manner as in
        \autoref{eq:permutation-indices}, but where $\vf{x}_{ij}$ signifies that
        all indices in $\vf{x}$ are the same except for the pair $(i, j)$.
        We can construct any permutation $\sigma$ as a product of an even or an
        odd number of transpositions \cite{fraleigh2003first}
        \begin{align}
            \sigma = \prod_{k = 1}^{n} \tau_{i_k, j_k},
        \end{align}
        where $\abs{\sigma} \equiv n$ counts the number of transpositions and is
        always either even or odd.
        Stated differently, a permutation $\sigma$ created from a product of
        transpositions is not unique as the same permutation can be achieved
        from the a different product of transpositions, however, all products
        are either even or odd.
        A permutation created from a even product of transpositions is said to
        have \emph{even parity} whereas a permutation created from an odd
        product of transpositions is said to have \emph{odd parity}.
        We define the \emph{exchange operator} $\hat{P}_{ij}$ as the operator
        that interchanges a pair of indices by
        \begin{align}
            \hat{P}_{ij}\psi(\vf{x})
            = \psi(\vf{x}_{ij}).
        \end{align}
        From this definition we can then see that
        \begin{align}
            \hat{P}^2_{ij}\psi(\vf{x})
            = \hat{P}_{ij}\psi(\vf{x}_{ij})
            = \psi(\vf{x}),
        \end{align}
        which means that the eigenvalues of $\hat{P}_{ij}$ is $p_{ij} = \pm
        1$, that is,
        \begin{align}
            \hat{P}_{ij}\psi(\vf{x})
            = \psi(\vf{x}_{ij})
            = p_{ij} \psi(\vf{x}).
        \end{align}
        As pointed out by \citeauthor{leinaas1977} in their seminal paper
        \citetitle{leinaas1977} \cite{leinaas1977}, this is only part of the
        truth as two-dimensional systems allow an infinite amount of
        eigenstates for the exchange operator.
        However, we will not concern ourselves with other eigenvalues for the
        exchange operator.
        % TODO: Verify that this is what dem bois did find out.
        Now, it is possible to write a transposition $\tau_{ij}$ as a product of
        three transpositions via an arbitrary index $k \neq i \neq j$.
        An illustration of this fact can be seen in
        \autoref{fig:transpositions}.
        As we can replace the action of a single exchange operator with three
        other exchange operators, we have that
        \begin{align}
            \hat{P}_{jk} \psi(\vf{x})
            &= \hat{P}_{ij} \hat{P}_{ik} \hat{P}_{ij} \psi(\vf{x})
            = p_{ij} p_{ik} p_{ij} \psi(\vf{x})
            = p_{ij}^2 p_{ik} \psi(\vf{x})
            = p_{ik} \psi(\vf{x})
             = p_{jk} \psi(\vf{x}),
        \end{align}
        which tells us that the eigenvalues of all the exchange operators are
        the same.
        Thus we have $p_{ij} = p_{jk} \equiv p = \pm 1$.
        A consequence of this is that the eigenvalue of an exchange operator is
        a charateristic of the wave function $\psi(\vf{x})$.
        Furthermore, by constructing a \emph{permutation operator}
        $\hat{P}_{\sigma}$ as a product of exchange operators, that is,
        \begin{align}
            \hat{P}_{\sigma}
            = \prod_{k = 1}^{n} \hat{P}_{i_k j_k},
        \end{align}
        we find that the eigenvalue of the permutation operator is
        \begin{align}
            \hat{P}_{\sigma}\psi(\vf{x})
            = p^{\abs{\sigma}}\psi(\vf{x}).
        \end{align}
        Now, depending on the wave function we can get two different situations.
        If $p = +1$, then we call $\psi(\vf{x})$ symmetric as $p^{\abs{\sigma}}
        = 1$ regardless of the parity of the permutation operator.
        A symmetric wave function describes a system of \emph{bosons}, e.g.,
        photons.
        On the other hand, if $p = -1$, then $\psi(\vf{x})$ is anti-symmetric
        and
        \begin{align}
            p^{\abs{\sigma}} =
            \begin{cases}
                +1 & \text{even parity}, \\
                -1 & \text{odd parity}.
            \end{cases}
        \end{align}
        Particles described by an anti-symmetric wave function are called
        \emph{fermions}, e.g., electrons.


        \begin{figure}
            \centering
            \begin{tikzpicture}
                \begin{scope}
                    \draw (-4, 0) node[anchor=east]
                    {$\hat{P}_{13}\psi(x_i, x_j, x_k)$} -- (4, 0);
                    \node[
                        draw,
                        circle,
                        black,
                        thick,
                        fill=alice,
                        text=white
                    ]
                    (1) at (-3, 0) {$i$};
                    \node[
                        draw,
                        circle,
                        black,
                        thick,
                        fill=darkerindigo,
                        text=white
                    ]
                    (2) at (0, 0) {$j$};
                    \node[
                        draw,
                        circle,
                        black,
                        thick,
                        fill=darkerruby,
                        text=white
                    ]
                    (3) at (3, 0) {$k$};
                    \path[->, thick] (1) edge[bend right] node [left] {} (3);
                    \path[->, thick] (3) edge[bend right] node [left] {} (1);
                \end{scope}
                \begin{scope}[yshift=-2.5cm]
                    \draw (-4, 0) node[anchor=east]
                    {$\hat{P}_{12}\psi(x_k, x_j, x_i)$} -- (4, 0);
                    \node[
                        draw,
                        circle,
                        black,
                        thick,
                        fill=darkerruby,
                        text=white
                    ]
                    (1) at (-3, 0) {$k$};
                    \node[
                        draw,
                        circle,
                        black,
                        thick,
                        fill=darkerindigo,
                        text=white
                    ]
                    (2) at (0, 0) {$j$};
                    \node[
                        draw,
                        circle,
                        black,
                        thick,
                        fill=alice,
                        text=white
                    ]
                    (3) at (3, 0) {$i$};
                    \path[->, thick] (1) edge[bend right] node [left] {} (2);
                    \path[->, thick] (2) edge[bend right] node [left] {} (1);
                \end{scope}
                \begin{scope}[yshift=-5cm]
                    \draw (-4, 0) node[anchor=east]
                    {$\hat{P}_{13}\psi(x_j, x_k, x_i)$} -- (4, 0);
                    \node[
                        draw,
                        circle,
                        black,
                        thick,
                        fill=darkerindigo,
                        text=white
                    ]
                    (1) at (-3, 0) {$j$};
                    \node[
                        draw,
                        circle,
                        black,
                        thick,
                        fill=darkerruby,
                        text=white
                    ]
                    (2) at (0, 0) {$k$};
                    \node[
                        draw,
                        circle,
                        black,
                        thick,
                        fill=alice,
                        text=white
                    ]
                    (3) at (3, 0) {$i$};
                    \path[->, thick] (1) edge[bend right] node [left] {} (3);
                    \path[->, thick] (3) edge[bend right] node [left] {} (1);
                \end{scope}
                \begin{scope}[yshift=-7.5cm]
                    \draw (-4, 0) node[anchor=east]
                    {$\psi(x_i, x_k, x_j)$} -- (4, 0);
                    \node[draw, circle, black, thick, fill=alice, text=white] at
                    (-3, 0) {$i$};
                    \node[draw, circle, black, thick, fill=darkerruby, text=white] at
                    (0, 0) {$k$};
                    \node[draw, circle, black, thick, fill=darkerindigo, text=white] at
                    (3, 0) {$j$};
                \end{scope}
            \end{tikzpicture}
            \caption{Here we illustrate how a single transposition
            $\tau_{jk}$ can be replaced by a product of three transpositions
            $\tau_{ij}\tau_{ik}\tau_{ij}$.
            In this figure the numbers represent a position inside the wave
            function arguments, that is, the exchange operators shuffle the
            ordering, and the labels $i$, $j$, and $k$ along with their colored
            spheres are there to distinguish the three coordinates from another.
            The arrows represent the action of the exchange operator from one
            line to the next (before the exchange has occured).
            Take care not to mistake the circles to mean particles, the
            particles are indistinguishable after all and such a labelling is
            not possible.}
            \label{fig:transpositions}
        \end{figure}

        \subsection{Pauli exclusion principle}
            An interesting result of the anti-symmetric property of fermions is
            that we can never have a system containing two fermions \emph{in the
            same state}.
            We demonstrate this by considering an anti-symmetric wave function
            $\psi(\vf{x})$ where $\vf{x} = (x_1, \dots, x_N) \in X^{N}$ is the
            coordinates of the $N$ particles in the system.
            We now that the eigenvalue of $\hat{P}_{ij}$ on $\psi(\vf{x})$ will
            be $p = -1$, that is,
            \begin{align}
                \hat{P}_{ij}\psi(\vf{x})
                = \psi(\vf{x}_{ij})
                = -1\psi(\vf{x}),
            \end{align}
            where we interpret $\vf{x}_{ij}$ as the same collection of
            coordinates as $\vf{x}$ but with $x_i$ and $x_j$ interchanged.
            Now, if $x_i = x_j$ in $\vf{x}$ then get
            \begin{gather}
                \hat{P}_{ij}\psi(\vf{x})
                = \psi(\vf{x}_{ij})
                = \psi(\vf{x})
                = -1\psi(\vf{x}),
            \end{gather}
            which means that $\psi(\vf{x}) = 0$ for the equation to be
            satisified.
            This observation is known as the \emph{Pauli exclusion
            principle}.
            It is not so much that fermions aren't \emph{allowed} to be in the
            same state, but a fact of life that such things does not exist as
            they would immediately annihilate one another.


    \section{Second quantization}
        So far, we have worked with quantum mechanics formulated in first
        quantization where observables are operators acting on states that are
        functions defined on some space.
        Moving to second quantization we will express wave functions as products
        of creation and annihilation operators acting on the vacuum state.
        These operators provide a way to express many-body wave
        functions\footnote{%
            Or more precisely, configurations of many-body wave functions.%
        } as a product of operators where each operator represents a particle.
        Depending on the algebra for the operators, we can look at antisymmetric
        fermions or symmetric bosons.
        Expressing the first quantized observables in the same creation and
        annihilation operators, we are able to unify much of quantum mechanics
        to a single set of elementary operators \cite{helgaker-molecular}.

        \subsection{Fock space}
            In second quantization, we express a general many-body wave function
            as a vector in the abstract linear vector space called \emph{Fock
            space}.

            \begin{definition}
                For a given single-particle Hilbert space $H$, we define Fock
                space as the direct sum of the tensor products of copies $H$,
                viz. \cite{fock-space}
                \begin{align}
                    F_{\nu}
                    = \bigoplus_{n = 0}^{\infty}
                    S_{\nu} H^{\otimes n}
                    = \mathbb{C}
                    \oplus H
                    \oplus \brak{
                        S_{\nu}
                        \para{
                            H \otimes H
                        }
                    }
                    \oplus \dots,
                \end{align}
                where $S_{\nu}$ is an operator which symmetrizes or
                antisymmetrizes the tensor product of Hilbert spaces depending
                on whether or not the system consists of bosons $\nu \equiv +$
                or fermions $\nu \equiv -$ respectively.
            \end{definition}

            We will be working with a finite basis set of $L$ orthonormal
            single-particle states, $\brac{\ket{p}}_{p = 1}^{L}$, which means
            that we truncate the infinite direct sum over the tensor products of
            $H$ constructing the truncated Fock space $F_{\nu}(L)$.
            A Fock state\footnote{%
                Also known as an occupation number vector
                \cite{helgaker-molecular}.
            } $\ket{\vfg{n}} \in F_{\nu}(L)$ is denoted
            \begin{align}
                \ket{\vfg{n}}
                &\equiv \ket{n_1, n_2, \dots, n_L},
            \end{align}
            where $n_i$ denotes how many of state $\ket{i}$ are contained in
            $\ket{\vfg{n}}$.\footnote{%
                This must not be confused with a \emph{product state} as
                $\ket{\vfg{n}}$ is either a symmetric ($\nu = +$) or an
                anti-symmetric ($\nu = -$) state.%
            }
            We've defined the set $\vf{n}$ as the set containing all $n_i$.
            This formulation of the Fock states lets us incorporate the particle
            statistics depending on the system we are exploring by choosing the
            allowed values of $n_i$.
            \begin{align}
                n_i =
                \begin{cases}
                    0, 1 & \text{fermions} \iff \nu = -, \\
                    0, 1, 2, \dots & \text{bosons} \iff \nu = +,
                \end{cases}
            \end{align}
            where a $0$ means that the state is unoccupied and a $1$, or higher,
            means that state $\ket{i}$ is occupied by that amount of particles.
            We can decompose the truncated Fock space $F_{\nu}(L)$ into a direct
            sum of subspaces $F_{\nu}(L, N)$
            \begin{align}
                F_{\nu}(L)
                = \bigoplus_{n = 0}^{L}
                F_{\nu}(L, n),
            \end{align}
            where $F_{\nu}(L, N)$ is the Fock space with all Fock states
            $\ket{\vfg{n}}$ where $\abs{\vfg{n}} = N$, i.e., all states where
            all $N$ particles have been distributed among all $L$ basis states
            \cite{helgaker-molecular}.
            Due to the particles being indistinguishable we have
            \begin{gather}
                \dim[F_{+}(L, N)] = \binom{L + N}{N},
                \\
                \dim[F_{-}(L, N)] = \binom{L}{N},
            \end{gather}
            where the bosonic Fock states are in a much greater number as they
            are not subject to the Pauli exclusion principle.
            The inner product between two Fock states $\ket{\vfg{n}},
            \ket{\vfg{m}} \in F_{\nu}(L)$ is given by
            \begin{align}
                \braket{\vfg{n}}{\vfg{m}} = \prod_{i = 1}^{M} \delta_{pq}
                \equiv \delta_{\vfg{n}\vfg{m}}.
                \label{eq:inner-product-fock}
            \end{align}
            One of the convenient consequences of the Fock space formulation is
            that it opens up for a description of the inner product between two
            Fock states with an unequal amount of particles.\footnote{%
                The two Fock states must however be defined from the same basis
                set of $L$ basis functions.
            }
            From \autoref{eq:inner-product-fock} we can see that two Fock states
            with an unequal number of particles will yield a zero-overlap.
            Furthermore, we can construct the identity by
            \begin{align}
                \1 = \dyad{\vfg{n}}{\vfg{n}},
                \label{eq:fock-identity}
            \end{align}
            where we sum over over all $\ket{\vfg{n}} \in F_{\nu}(L)$.
            Of particular interest is the \emph{vacuum state} $\ket{\vac} \in
            F_{\nu}(L, 0)$ as the state with no particles.
            \begin{align}
                \ket{\vac} \equiv \ket{\vfg{0}}
                = \ket{0_1, 0_2, \dots, 0_L}.
            \end{align}
            From \autoref{eq:inner-product-fock} we have that the vacuum state
            is normalized to unity.

        \subsection{Creation and annihilation operators}
            We now introduce the \emph{creation} and \emph{annihilation}
            operators which acts on Fock states by adding or removing single
            particle states.

            \begin{definition}
                \label{def:creation_1}
                A \emph{creation} operator $\acr{p}$ is an operator acting on
                Fock states in $F_{\nu}(L)$.
                It is a mapping $\acr{p}: F_{\nu}(L, N) \mapsto F_{\nu}(L, N +
                1)$.
                Its action on a Fock state $\ket{\vfg{n}} \in F_{\nu}(L, N)$ is
                defined as
                \begin{align}
                    \acr{p}\ket{\vfg{n}}
                    = (\Gamma_{\nu})^{\vfg{n}}_{p}
                    \sqrt{n_p + 1}
                    \ket{n_1, \dots, n_p + 1, \dots, n_L}
                    = (\Gamma_{\nu})^{\vfg{n}}_{p}
                    \sqrt{n_p + 1}
                    \ket{\vfg{n}'},
                    \label{eq:general_creation}
                \end{align}
                where $(\Gamma_{\nu})^{\vfg{n}}_{p}$ is a phase factor which
                depends on the particle type $\nu$.
                \begin{gather}
                    (\Gamma_{+})^{\vfg{n}}_{p} = 1, \\
                    (\Gamma_{-})^{\vfg{n}}_{p}
                    = \prod_{q = 1}^{p - 1} (-1)^{n_q}.
                \end{gather}
                We will denote fermionic creation operators by $\ccr{p}$ and the
                corresponding bosonic creation operators by $\bcr{p}$.
            \end{definition}
            The phase factor is again a way of incorporating the particle
            statistics of the system we are exploring.
            A bosonic wave function is symmetric with respect to its particles
            and its phase is therefore constant regardles of the ordering of the
            creation of its particles.
            For a fermionic wave function, the interchange of two particles will
            incur a change sign as discussed in
            \autoref{sec:particle-statistics}.
            The fact that the ordering of the indices in a fermionic Fock state
            will depend on the overall sign of the state can lead to ambiguity.
            We therefore introduce \emph{canonical ordering} with respect to the
            particle indices.
            \begin{align}
                \ket{\vfg{n}}
                \equiv \brak{
                    \prod_{p = 1}^{L}
                    \para{\ccr{p}}^{n_p}
                }\ket{\vac},
                \label{eq:canonical-fock-state}
            \end{align}
            that is, we order the states in an increasing order.
            Recalling the Pauli exclusion principle that $n_p \in \brac{0, 1}$
            for a fermionic wave function, we have that
            \begin{align}
                \ccr{p}\ccr{p} = 0,
            \end{align}
            and the normalization factor $\sqrt{n_p + 1}$ will always be $1$ as
            long as the state is not annihilated by inserting an already
            existing state.
            From the phase in \autoref{def:creation_1} we have the commutation
            relation
            \begin{gather}
                \com{\bcr{p}}{\bcr{q}} = \bcr{p}\bcr{q} - \bcr{q}\bcr{p} = 0,
            \end{gather}
            for the bosonic system as the ordering of the creation of particles
            in a bosonic wave function is arbitrary.
            Fermionic creation operators satisfy the anticommutation relation
            \begin{gather}
                \acom{\ccr{p}}{\ccr{q}} = \ccr{p}\ccr{q} + \ccr{q}\ccr{p} = 0,
            \end{gather}
            as $p > q$ or $q > p$ and the phase factor will give a sign change
            by inserting one before the other thus yielding the same state with
            an opposite sign.

            We define the \emph{annihilation operator} as the adjoint of the
            creation operator.
            The action can be inferred by utilizing the spectral decomposition
            of the identity in \autoref{eq:fock-identity}.
            \begin{align}
                \aan{p}\ket{\vfg{n}}
                = \ket{\vfg{m}}\mel{\vfg{m}}{\aan{p}}{\vfg{n}}
                =
                \ket{\vfg{m}}
                \para{
                    \mel{\vfg{n}}{\acr{p}}{\vfg{m}}
                }^{*}
                = \para{\Gamma_{\nu}}^{\vfg{m}}_{p}
                \delta_{\vfg{n}\vfg{m}'}
                \sqrt{m_p + 1}
                \ket{\vfg{m}},
            \end{align}
            where
            \begin{align}
                \acr{p}\ket{\vfg{m}}
                = \para{\Gamma_{\nu}}^{\vfg{m}}_{p}
                \sqrt{m_p + 1}
                \ket{\vfg{m}'}.
            \end{align}
            In order for the overlap between $\ket{\vfg{n}}$ and
            $\acr{p}\ket{\vfg{m}}$ to be non-zero we have to have $\abs{\vfg{n}
            - \vfg{m}} = 1$, that is, the two states $\ket{\vfg{n}}$ and
            $\ket{\vfg{m}}$ can only differ by a single state.
            Futhermore, as we are acting on $\ket{\vfg{m}}$ with $\acr{p}$ we
            have that $n_p = m_p + 1$, i.e., $\ket{\vfg{n}}$ has one more
            particle in state $\ket{p}$ than $\ket{\vfg{m}}$.
            As all other states in $\ket{\vfg{n}}$ and $\ket{\vfg{m}}$ are the
            same, we have that $(\Gamma_{\nu})^{\vfg{m}}_{p} =
            (\Gamma_{\nu})^{\vfg{n}}_{p}$
            and $\sqrt{n_p} = \sqrt{m_p + 1}$.
            This means that
            \begin{align}
                (\Gamma_{\nu})^{\vfg{m}}_{p}
                \delta_{\vfg{n}\vfg{m}'}
                \sqrt{m_p + 1}
                =
                \para{\mel{\vfg{n}}{\acr{p}}{\vfg{m}}}^{*}
                =
                \mel{\vfg{m}}{\aan{p}}{\vfg{n}}
                =
                (\Gamma_{\nu})^{\vfg{n}}_{p}
                \sqrt{n_p}
                \delta_{\vfg{n}'\vfg{m}},
            \end{align}
            where $\ket{\vfg{n}'}$ is the state $\ket{\vfg{n}}$ with state
            $\ket{p}$ removed.
            We are thus left with our annihilation operator.
            \begin{definition}
                \label{def:annihilation_1}
                An \emph{annihilation} operator $\aan{p}$ is an operator acting
                on Fock states in $F_{\nu}(L)$.
                It is a mapping $\aan{p}: F_{\nu}(L, N) \mapsto F_{\nu}(L, N -
                1)$.
                Its action on a Fock state $\ket{\vfg{n}} \in F_{\nu}(L, N)$ is
                defined as
                \begin{align}
                    \aan{p}\ket{\vfg{n}}
                    = (\Gamma_{\nu})^{\vfg{n}}_{p}
                    \sqrt{n_p}
                    \ket{n_1, \dots, n_p - 1, \dots, n_L}
                    = (\Gamma_{\nu})^{\vfg{n}}_{p}
                    \sqrt{n_p}
                    \ket{\vfg{n}'},
                \end{align}
                where $(\Gamma_{\nu})^{\vfg{n}}_{p}$ is the same phase factor as
                in \autoref{def:creation_1}.
                We will denote fermionic annihilation operators by $\can{p}$ and
                the corresponding bosonic anniliation operators by $\ban{p}$.
            \end{definition}
            From the definition of the annihilation operator we have that
            \begin{align}
                \can{p}\can{p} = 0,
            \end{align}
            as there can only be one particle of type $p$ in a given fermionic
            state.
            Furthermore, we have the commutation relation
            \begin{align}
                \com{\ban{p}}{\ban{q}} = \ban{p}\ban{q} - \ban{q}\ban{p} = 0,
            \end{align}
            for the bosonic states and the anticommutation relation
            \begin{align}
                \acom{\can{p}}{\can{q}} = \can{p}\can{q} + \can{q}\can{p} = 0,
            \end{align}
            for the fermionic system.
            These relations can be seen by taking the Hermitian conjugate of the
            commutation and anitcommutation relation for the creation operators.
            If the annihilation operator $\aan{p}$ acts on a Fock state
            $\ket{\vfg{n}}$ which does not contain the state $\ket{p}$, then
            \begin{align}
                \aan{p}\ket{\vfg{n}}
                = \aan{p}\ket{n_1, \dots, 0_p, \dots, n_L}
                = 0.
            \end{align}
            As a consequence, any annihilation operator will annihilate the
            vacuum.

            Now, in order to derive the fundamental properties of the second
            quantized operators we look at the combined action of a creation and
            an annihilation operator acting on a state in combination.
            For the bosonic operators we have
            \begin{align}
                \begin{drcases}
                    \ban{p}\bcr{p}\ket{\vfg{n}}
                    &= (n_p + 1)\ket{\vfg{n}}
                    \\
                    \bcr{p}\ban{p}\ket{\vfg{n}}
                    &= n_p \ket{\vfg{n}}
                \end{drcases}
                \implies
                \com{\ban{p}}{\bcr{p}}
                = (n_p + 1) - n_p = 1,
            \end{align}
            when we create and remove the same particle in state $\ket{p}$.
            When $p \neq q$ we have
            \begin{align}
                \begin{drcases}
                    \ban{p}\bcr{q}\ket{\vfg{n}}
                    &= \sqrt{n_p}\sqrt{n_q + 1}\ket{\vfg{n}'}
                    \\
                    \bcr{q}\ban{p}\ket{\vfg{n}}
                    &= \sqrt{n_p}\sqrt{n_q + 1}\ket{\vfg{n}'}
                \end{drcases}
                \implies
                \com{\ban{p}}{\bcr{q}}
                = 0.
            \end{align}
            Collecting all three commutation relations for the bosonic states we
            find
            \begin{align}
                \com{\ban{p}}{\ban{q}} &= 0, \\
                \com{\bcr{p}}{\bcr{q}} &= 0, \\
                \com{\ban{p}}{\bcr{q}} &= \delta_{pq}.
            \end{align}
            Repeating this exercise for the fermionic states we start by noting
            that the fermionic phase factor squared is $1$.
            \begin{align}
                \begin{drcases}
                    \can{p}\ccr{p}\ket{\vfg{n}}
                    &= \delta_{n_p0}\ket{\vfg{n}}
                    \\
                    \ccr{p}\can{p}\ket{\vfg{n}}
                    &= \delta_{n_p1}\ket{\vfg{n}}
                \end{drcases}
                \implies
                \acom{\can{p}}{\ccr{p}}
                = \delta_{n_p 0} + \delta_{n_p 1}
                = 1,
            \end{align}
            where $n_p$ in the Kronecker-Deltas following the anticommutator
            expression are dummy indices.
            When $p \neq q$ we have
            \begin{align}
                \can{p}\ccr{q}\ket{\vfg{n}}
                &=
                (\Gamma_{+})^{\vfg{n}^{(1)}}_{p}
                (\Gamma_{+})^{\vfg{n}}_{q}
                \delta_{n_p 1} \delta_{n_q 0}
                \ket{\vfg{n}^{(3)}}
                \\
                \ccr{q}\can{p}\ket{\vfg{n}}
                &=
                (\Gamma_{+})^{\vfg{n}^{(2)}}_{q}
                (\Gamma_{+})^{\vfg{n}}_{p}
                \delta_{n_q 0} \delta_{n_p 1}
                \ket{\vfg{n}^{(3)}}
            \end{align}
            where we've defined the state $\ket{\vfg{n}^{(3)}}$ as state
            $\ket{\vfg{n}}$ with the single-particle states $\ket{q}$ added and
            $\ket{p}$ removed.
            Collecting the creation and annihilation pair into an anticommutator
            relation we have
            \begin{align}
                \acom{\can{p}}{\ccr{q}}
                =
                \delta_{n_q 0} \delta_{n_p 1}
                \brak{
                    (\Gamma_{+})^{\vfg{n}^{(1)}}_{p}
                    (\Gamma_{+})^{\vfg{n}}_{q}
                    +
                    (\Gamma_{+})^{\vfg{n}^{(2)}}_{q}
                    (\Gamma_{+})^{\vfg{n}}_{p}
                }.
            \end{align}
            In order to get a final expression for this anticommutator need to
            find expressions for the phases $(\Gamma_{+})^{\vfg{n}^{(1)}}_{p}$
            and $(\Gamma_{+})^{\vfg{n}^{(2)}}_{q}$ relative to the original
            state $\ket{\vfg{n}}$.
            We have two situations that needs to be explored, namely when $p >
            q$ and the opposite situation when $p < q$.
            Now, when $p > q$ we have
            \begin{align}
                (\Gamma_{+})^{\vfg{n}^{(1)}}_{p}
                (\Gamma_{+})^{\vfg{n}}_{q}
                =
                -
                (\Gamma_{+})^{\vfg{n}}_{p}
                (\Gamma_{+})^{\vfg{n}}_{q},
            \end{align}
            as $\ccr{q}$ inserts an extra state $\ket{q}$ between $n_1$ and
            $n_p$.
            Conversely, this sign change does not occur for the reverse
            situation when $\ccr{q}\can{p}$ acts on $\ket{\vfg{n}}$ and we get
            $(\Gamma_{+})^{\vfg{n}^{(2)}}_{p} = (\Gamma_{+})^{\vfg{n}}_{p}$.
            Thus, in total we have
            \begin{align}
                \acom{\can{p}}{\ccr{q}} = 0,
            \end{align}
            for $p > q$.
            When $p < q$ we get the same result as can be seen by taking the
            Hermitian conjugate of the anticommutator and reversing the dummy
            indices $p$ and $q$.
            In total we have the fundamental


        \subsection{Slater determinants}
            As we are working with fermions the total many-body wave function must
            be \emph{antisymmetric} with respect to interchange of
            particles\footnote{We will restrict ourselves to the interchange of a
            full spin-orbital.}. Mathematically the full wave function of $N$
            fermions should satisfy
            \begin{align}
                \Psi(1, \dots, i, \dots, j, \dots, N)
                =
                -\Psi(1, \dots, j, \dots, i, \dots, N),
            \end{align}
            where we for brevity have introduced the notation $i \equiv x_i =
            (\vf{r}_i, \sigma_i)$. A fully antisymmetric, normalized, $N$-body wave
            function is given by the \emph{Slater determinant} built from $N$
            individual spin-orbitals with $N$ generalized coordinates.
            \begin{align}
                \Psi(1, \dots, N)
                &= \frac{1}{\sqrt{N!}}
                \begin{vmatrix}
                    \psi_1(1) & \dots & \psi_N(1) \\
                    \vdots & \ddots & \vdots \\
                    \psi_1(N) & \dots & \psi_N(N)
                \end{vmatrix},
                \label{eq:coord_slater}
            \end{align}
            where $\psi_i(j) \equiv \psi_i(\vf{r}_j, \sigma_j)$ for the $i$'th
            spin-orbital in generalized coordinate $j$ \footnote{Note that the index
            on the spin-orbitals has nothing to do with which spin component is used
            as in the previous section on spin-orbitals.}. Exchanging a row in the
            determinant results in a sign change as we required from the
            antiysmmetric full wave function for the many-body problem. Furthermore,
            exchanging a column in the determinant also results in a sign-change.
            This means that there can not be two, or more, of the same spin-orbitals
            in the wave function perfectly encapsulating the Pauli principle as two
            of the same spin-orbitals translates to two particles being in the same
            state. Generally, the spin-orbitals in a Slater determinant are linearly
            independent, otherwise $\Psi = 0$. This is a consequence of the
            properties of a determinant that adding a scalar multiple of a column to
            another column does not change the determinant.

            The factor in front of the Slater determinant makes the many-body wave
            function normalized to unity as long as the spin-orbitals are
            orthonormalized.

            \begin{lemma}
                Given a Slater determinant $\ket{\Phi}$ built from an
                orthonormal basis $\brac{\ket{\phi_i}}_{i = 1}^{N}$ that spans
                the $N$-dimensional Hilbert space, we can then perform a unitary
                transformation to a new orthonormal basis
                $\brac{\ket{\psi_i}}_{i = 1}^{N}$ building a new Slater
                determinant $\ket{\Psi}$.
                This unitary transformation will then preserve the normalization
                of the original Slater determinant.
                % TODO: Clean-up formulation.
            \end{lemma}

            % TODO: Introduce proof
            \begin{proof}
                \label{proof:slater_determinants_invariant}
                Given a basis $\brac{\ket{\phi_i}}_{i = 1}^{N}$ that spans the
                $N$-dimensional Hilbert space. We can then do a unitary
                transformation from this basis to a new basis
                $\brac{\ket{\psi_i}}_{i = 1}^{N}$ in the same Hilbert space by
                \begin{align}
                    \ket{\psi_i} = U_{ji}\ket{\phi_j},
                    \label{eq:unitary_transformation}
                \end{align}
                where $U_{ji}$ is an element in the unitary matrix $\vfg{U}$.
                For a set of coordinates
                $\brac{x_1, \dots, x_N}$, we write
                \begin{align}
                    \psi_{ij} \equiv \psi_j(x_i)
                    \equiv \braket{x_i}{\psi_j},
                \end{align}
                and equivalently for $\phi_{ij}$. Projecting onto the coordinate
                basis we can write \autoref{eq:unitary_transformation} as
                \begin{gather}
                    \braket{x_k}{\psi_i}
                    = U_{ji}\braket{x_k}{\phi_j}
                    \implies
                    \psi_{ki} = U_{ji}\phi_{kj} = \phi_{kj}U_{ji}
                    \implies \vfg{\Psi} = \vfg{\Phi}\vfg{U},
                \end{gather}
                where the matrices $\vfg{\Psi}$, $\vfg{\Phi}$ and $\vfg{U}$ are
                the matrices with elements
                \begin{align}
                    \vfg{\Psi}
                    &= \begin{pmatrix}
                        \psi_{11} & \psi_{12} & \dots & \psi_{1N} \\
                        \vdots & \vdots & \ddots & \vdots \\
                        \psi_{N1} & \psi_{N2} & \dots & \psi_{NN}
                    \end{pmatrix}, \\
                    \vfg{\Phi}
                    &= \begin{pmatrix}
                        \phi_{11} & \phi_{12} & \dots & \phi_{1N} \\
                        \vdots & \vdots & \ddots & \vdots \\
                        \phi_{N1} & \phi_{N2} & \dots & \phi_{NN}
                    \end{pmatrix}, \\
                    \vfg{U}
                    &= \begin{pmatrix}
                        U_{11} & U_{12} & \dots & U_{1N} \\
                        \vdots & \vdots & \ddots & \vdots \\
                        U_{N1} & U_{N2} & \dots & U_{NN}
                    \end{pmatrix},
                \end{align}
                and where $\vfg{U}^{\dagger}\vfg{U} = \1$.
                Creating the fully antisymmetrized normalized wave function, i.e.,
                the Slater determinant, from $\vfg{\Psi}$ and $\vfg{\Phi}$ we see
                that
                % TODO: Fix description, the determinant of the "Slater matrices"
                % are not necessarily normalized without the pre-factor.
                \begin{gather}
                    \det(\vfg{\Psi}) = \det(\vfg{\Phi}\vfg{U})
                    = \det(\vfg{\Phi})\det(\vfg{U})
                \end{gather}
                We now take the squared norm on both sides.
                \begin{align}
                    \norm{\det(\vfg{\Psi})}^2
                    = \norm{\det(\vfg{\Phi})\det(\vfg{U})}^2
                    \leq
                    \norm{\det(\vfg{\Phi})}^2\norm{\det(\vfg{U})}^2
                    = \norm{\det(\vfg{\Phi})}^2,
                    \label{eq:squared_determinant}
                \end{align}
                where we have used that
                \begin{align}
                    \norm{\det(\vfg{U})}^2
                    &= \norm{\det(\vfg{U^{\dagger}})\det(\vfg{U})}
                    = \norm{\det(\vfg{U}^{\dagger}\vfg{U})}
                    = \norm{\det(\1)} = 1.
                \end{align}
                Now, in \autoref{eq:squared_determinant}, since the Slater
                determinant of $\vfg{\Phi}$ is orthonormalized and a unitary
                transformation preserves the normalization this means that the norm
                of $\det(\vfg{\Phi})$ and the norm of $\det(\vfg{\Psi})$ must be
                unity.  Thus, the equality in the splitting of the norm is preserved
                thus yielding
                \begin{gather}
                    \norm{\det(\vfg{\Psi})}^2 = \norm{\det(\vfg{\Phi})}^2.
                \end{gather}
            \end{proof}

            \subsubsection{Antisymmetrizer}
                To avoid having to write the Slater determinants as in
                \autoref{eq:coord_slater} we introduce the anti-symmetrizer
                $\hat{A}$ which is given by
                \begin{align}
                    \hat{A}
                    \equiv
                    \frac{1}{N!}
                    \sum_{\sigma \in S_N}
                    (-1)^{\abs{\sigma}}
                    \hat{P}_{\sigma},
                \end{align}
                where $\sigma$ is a permutation from the set of all permutations
                $S_N$ of $N$ indices, $\hat{P}_{\sigma}$ is the permutation
                operator permuting particles and $\abs{\sigma}$ is the number of
                transpositions in the permutation.
                There are $N!$ permutations in $S_N$ which is why we include the
                normalization factor.
                A Slater determinant can thus be represented by
                \begin{align}
                    \Phi(\vf{x})
                    &=
                    \sqrt{N!}\hat{A}\phi_1(x_1)\dots\phi_N(x_N)
                    \\
                    &= \frac{1}{\sqrt{N!}}
                    \sum_{\sigma \in S_N} (-1)^{\abs{\sigma}}
                    \phi_1(x_{\sigma(1)})\dots\phi_N(x_{\sigma(N)}).
                \end{align}
                The anti-symmetrizer opens up the possibility of representing a
                Slater determinant in a coordinate-independent manner.
                % TODO: Continue on this topic

            \subsubsection{Operator representation}

        \subsection{Time-dependent second quantized operators}


    \section{Spin-orbitals}
        In this thesis we will be looking at many-particle systems.
        Wave functions of many-particle systems depend on the coordinates of all
        the particles contained in the system.
        We will typically build the wave functions of the full system from
        \emph{single particle functions}.
        In this work we'll limit our attention to particles with spin one half ,
        i.e., fermions, which means that the total wave function must satisfy
        the \emph{Pauli exclusion principle} due to the particles being
        indistinguishable.
        The single particle functions will as a consequence depend on the
        particle's spatial orientation, i.e., the \emph{orbital} part, and the
        spin.
        We call these wave functions \emph{spin-orbitals}.
        \begin{align}
            \psi(x) \equiv \psi(\vf{r}, m_s)
        \end{align}
        where $x = (\vf{r}, m_s)$ is a generalized coordinate of both
        position, $\vf{r}$, and spin quantum number, $m_s$.
        For fermions we have only two allowed spin states
        \begin{align}
            s_z = \pm \half\hslash,
        \end{align}
        where $s_z$ is the spin along $z$-direction.
        As there are only two allowed states we have that $m_s \in
        \brac{\upspin, \downspin}$, where a positive value for $s_z$ corresponds
        to $m_s = \upspin$, i.e., spin up, and a negative value for $s_z$ to
        $m_s = \downspin$, i.e., spin down.
        We denote
        \begin{align}
            \psi_1(\vf{r}) \equiv \psi\para{\vf{r}, \upspin},
            \qquad
            \psi_2(\vf{r}) \equiv \psi\para{\vf{r}, \downspin},
        \end{align}
        for the two different spin-directions.
        We can thus represent the generalized spin-orbital $\psi(x)$ as a
        two-dimensional vector
        \begin{align}
            \psi(x) = \begin{pmatrix}
                \psi_1(\vf{r}) \\
                \psi_2(\vf{r})
            \end{pmatrix}.
        \end{align}
        We separate the spin dependence from the spatial part of the
        spin-orbitals by introducing separate ``spin functions'' for spin-up and
        spin-down.
        For example, choosing the basis
        \begin{align}
            \alpha \equiv \alpha(m_s) = \begin{pmatrix}
                1 \\
                0
            \end{pmatrix},
            \qquad
            \beta \equiv \beta(m_s) = \begin{pmatrix}
                0 \\
                1
            \end{pmatrix},
            \label{eq:spin-basis}
        \end{align}
        we use the same convention as in much of the many-body quantum mechanics
        litterature in labelling $\alpha$ as spin up and $\beta$ as spin down in
        the $z$-direction.
        Evaluating the spin functions thus yields
        \begin{gather}
            \alpha\para{\upspin} = 1, \qquad \alpha\para{\downspin} = 0, \\
            \beta\para{\upspin} = 0, \qquad \beta\para{\downspin} = 1.
        \end{gather}
        Using \autoref{eq:spin-basis} we see that we write the
        generalized spin-orbital as a linear combination of the spin basis
        functions by
        \begin{align}
            \psi(x)
            = \psi_1(\vf{r})\alpha(m_s)
            + \psi_2(\vf{r})\beta(m_s).
            \label{eq:general-spin-orbital-1}
        \end{align}
        % TODO: Consider adding theory on spin as in Mayer.
        We will for the most part work with spin-orbitals in a definite
        spin-direction, viz.
        \begin{align}
            \psi(x) = \phi(\vf{r})\sigma(m_s),
            \label{eq:spin-orbital}
        \end{align}
        where $\sigma(m_s)$ is either spin up $\alpha(m_s)$ or spin down
        $\beta(m_s)$ and we've denoted the spatial orbital by $\phi(\vf{r})$.
        The reason we avoid using a general spin-orbital as in
        \autoref{eq:general-spin-orbital-1} is convenience.
        It is more convenient to generate a basis of spin-orbitals where we use
        a definite basis of analytic orbitals, e.g., harmonic oscillator basis
        functions, Gauss functions, etc, before deciding the spin direction of
        each orbital.\footnote{
            For those that might find this frustrating, note that there is
            nothing stopping us from using orbitals and spin-functions which are
            linear combinations of an underlying basis of orbitals and
            spin-functions as well.
            There's linear combinations all the way down!
        }
        When labelling the different spin-orbitals from
        \autoref{eq:spin-orbital} we use the notation
        \begin{align}
            \psi_P(x) = \phi_p(\vf{r})\sigma(m_s),
        \end{align}
        where $P = (p, \sigma)$ is a composite index with $p$ labelling a
        specific orbital and $\sigma$ a specific spin-function.
        In the spin basis shown in \autoref{eq:spin-basis} we have that the
        inner product of two spin functions
        \begin{align}
            \braket{\sigma}{\tau}
            &=
            \sigma(\upspin)\tau(\upspin)
            + \sigma(\downspin)\tau(\downspin)
            = \delta_{\sigma\tau},
        \end{align}
        that is, they are orthonormal.
        The inner product of two spin-orbitals from \autoref{eq:spin-orbital} is
        then separated into an orbital and a spin inner product.
        \begin{align}
            \braket{\psi_P}{\psi_Q}
            &= \braket{\phi_p}{\phi_q}\braket{\sigma}{\tau}
            = \delta_{\sigma\tau}
            \int\dd\vf{r}\phi^{*}_{p}(\vf{r})\phi_{q}(\vf{r}).
            \label{eq:inner-spin-orbital}
        \end{align}
        This equation is quite suggestive in the sense that we can see that the
        spin decouples from the spatial part of the wave function.
        Spin is two-level system that can be represented as a vector in a
        two-dimensional Hilbert space.
        Unless an operator contains a spin-coupling, the two spin-directions are
        completely independent of one another.
        For example, given a spin-independent operator, $\hat{O}$, the
        matrix element between two spin-orbitals become
        \begin{align}
            \bra{\psi_P}\hat{O}\ket{\psi_Q}
            &= \bra{\phi_p}\hat{O}\ket{\phi_q}\braket{\sigma}{\tau}
            = \delta_{\sigma\tau}
            \int\dd\vf{r}\phi^{*}_{p}(\vf{r})\hat{O}\phi_q(\vf{r}).
            \label{eq:example-spin-decoupling}
        \end{align}
        This motivates the coordinate representation independent notation
        \begin{align}
            \ket{\psi_P}
            &= \ket{\phi_p}\otimes\ket{\sigma},
        \end{align}
        where the tensor product combines the two-dimensional Hilbert space from
        the spin $\ket{\sigma}$ and the $L$-dimensional Hilbert space
        containing the orbital basis function.
        A general operator on the combined Hilbert space can thus be represented
        by
        \begin{align}
            \hat{O}
            &= \sum_{ij} \hat{R}_i \otimes \hat{S}_j,
        \end{align}
        where $\hat{R}_i$ is an operator on the Hilbert space containing the
        orbitals and $\hat{S}_j$ an operator on the two-dimensional Hilbert
        space containing the spin functions.
        The important point is now that the elements of this operator in terms
        of spin-orbitals is given by
        \begin{align}
            \bra{\psi_{P}}\hat{O}\ket{\psi_{Q}}
            &=
            \sum_{ij}\bra{\phi_p}\hat{R}_i\ket{\phi_q}
            \bra{\sigma}\hat{S}_j\ket{\tau}.
        \end{align}
        Looking back at \autoref{eq:example-spin-decoupling} we see that a
        spin-independent operator $\hat{O}$ corresponds to $\hat{S}_j = \1$.

        \subsection{Restrictions on the choice of spin-orbitals}
            \label{subsec:restrictions-on-spin-orbitals}
            As mentioned earlier, choosing completely general spin-orbitals,
            i.e,. spin-orbitals that are linear combinations of the two
            spin-directions, can become problematic.
            Amongst other things, due to a phenomenom called
            \emph{spin-contamination}, which we'll discuss later in the section
            on Hartree-Fock theory.
            % TODO: Discuss spin-contamination in relation to the completely
            % unrestriced Hartree-Fock method.
            Some common choices of spin-orbitals can be categorized in the
            following manner.
            \begin{enumerate}
                \item The general spin-orbital is on the form
                    \begin{align}
                        \psi(x, t)
                        = \psi_1(\vf{r}, t) \alpha(m_s)
                        + \psi_2(\vf{r}, t) \beta(m_s).
                        \label{eq:general-spin-orbital}
                    \end{align}
                    These spin-orbitals can lead to mixing between both
                    spin-directions, known as spin-contamination, that is, the
                    spin-orbitals are no longer eigen states of the
                    spin-projection operator $\spinproj$.
                    This is often an unwanted effect as this greatly limits some
                    of the optimizations that can be employed for some of the
                    computational methods.
                \item The spin-unrestricted spin-orbital is on the form
                    \begin{align}
                        \psi(x, t)
                        = \psi_{\sigma}(\vf{r}, t) \sigma(m_s),
                        \label{eq:unrestricted-spin-orbital}
                    \end{align}
                    where the orbital $\psi_{\sigma}$ depends on the
                    spin-direction $\sigma$.
                    Often this type of spin-orbitals are dubbed unrestricted
                    spin-orbitals, which is somewhat of a misnomer as the
                    spin-orbitals are restricted in the choice of spin-direction
                    as opposed to the (truly) unrestricted generalized
                    spin-orbitals shown in \autoref{eq:general-spin-orbital}.
                \item The spin-restricted spin-orbital is given by
                    \begin{align}
                        \psi(x, t)
                        = \psi(\vf{r}, t) \sigma(m_s),
                        \label{eq:restricted-spin-orbital}
                    \end{align}
                    where each orbital is doubly occupied.
            \end{enumerate}

        \subsection{Orbital rotations}
            % TODO: Include Thouless' theorem.

        \subsection{Basis sets}
            Also known as \emph{atomic orbitals}.
            In this document we will rely heavily on basis transformations and
            creating linear combinations from a basis to another. When creating
            many-body wavefunctions we will use a finite basis of \emph{orbitals}
            $\brac{\ket{\chi_{\alpha}}}_{\alpha = 1}^{L}$ as our \emph{atomic
            orbitals}. These represent known basis states, e.g., the harmonic
            oscillator eigenstates, etc. We can construct many-body wavefunctions
            from a linear combination of tensor products of single-particle states,
            \begin{align}
                \ket{\Psi}
                = \sum_{\sigma \in S} c_{\sigma}
                % TODO: Figure out how to label the coefficients c
                \ket{\chi_{\sigma(1)}}\otimes\dots\otimes\ket{\chi_{\sigma(L)}},
            \end{align}
            where we have assumed normalized single-particle functions
            $\braket{\chi_{\alpha}}{\chi_{\alpha}} = 1$, but not necessarily
            orthonormal.
            % TODO: Review these statements. Can the total wave function be
            % normalized if the single particle functions are normalized but not
            % orthonormal?
        \subsection{Summation convention}
            We will throughout this thesis just a summation convention that
            resembles the Einstein summation convention, but with slight
            variations.


    \section{Wick's theorem}
        \subsection{Normal ordering}
        \subsection{Generalized Wick's theorem}
        \subsection{Fermi vacuum}
            \label{sec:fermi-vacuum}
            Given a basis of $L$ single particle functions $\ket{p}$ where
            \begin{align}
                \left\{\ket{p}\right\}_{p = 1}^{L}
                &=
                \left\{\ket{i}\right\}_{i = 1}^{N}
                \cup \left\{\ket{a}\right\}_{a = N + 1}^{L}.
            \end{align}
            Here $i, j, k, \dots$ represents the $N$ first occupied states of
            the reference Slater determinant whereas $a, b, c, \dots$ represent
            the remaining $M = L - N$ virtual states in the total basis $p, q,
            r, \dots$\footnote{Occupied and virtual states are also known as
            hole and particle states if we treat the reference Slater
            determinant as the \textit{Fermi vacuum}}.

    \section{Excited determinants}
        \subsection{Overlap with the reference state}
            % TODO: Remember to introduce the concept of the reference state
        \subsection{Overlap when using time-dependent orbitals}

    \section{The many-body Hamiltonian}
        A general many-body Hamiltonian can be described by
        \begin{align}
            \hamil \equiv \sum_{i = 0}^{\infty} \hamil_i,
        \end{align}
        where we denote the $N$-body Hamiltonian by $\hamil_i$.
        From the postulates of quantum mechanics we require that
        $\hamil^{\dagger} = \hamil$, that is, the Hamiltonian should be
        Hermitian.
        This in turn implies that $\hamil_i^{\dagger} = \hamil_i$ for all
        $N$-body terms.
        We can compute the energy, i.e., the expectation value of the
        Hamiltonian, by
        \begin{align}
            \mel{\psi}{\hamil}{\psi}
            &= \sum_{i = 0}^{\infty} \mel{\psi}{\hamil_i}{\psi}
            = \mel{\psi}{\hamil_0}{\psi}
            + \mel{\psi}{\hamil_1}{\psi}
            + \mel{\psi}{\hamil_2}{\psi}
            + \dots,
        \end{align}
        where each term involves the \emph{interaction} between $i$ particles.
        For $i = 0$ we have $\hamil_0$ which we take to signify a constant term,
        e.g., the nuclear charge of an atomic system.
        This term contributes with a constant energy shift in the expectation
        value.
        The one-body term, $\hamil_1$, is the familiar Hamiltonian from
        introductory courses in quantum mechanics.
        This term describes one-particle contribution, e.g., the kinetic
        energy, external potentials of the system, external fields, etc.
        Higher order terms describe interactions between particles in the
        system.
        For example, the Coulomb interaction is a two-body interaction between
        particle pairs and is described by $\hamil_2$.
        More exotic higher order interactions can occur in the nuclear physics
        and the sort.
        However, it is common to truncate the interactions in the many-body
        Hamiltonian as high order terms leads to complicated equations which
        become intractable unless clever approximations are introduced.

        Luckily, much of solid-state physiscs, and almost all of quantum
        chemistry, can be described by the Coulomb interaction as the system are
        mainly governed by the electrons.
        This means that we will truncate the many-body Hamiltonian to
        second-order interactions.
        Furthermore, we ignore the constant term in the Hamiltonian as this can
        always be added when we compute the energy and will not contribute when
        we look for a solution of the many-body problem.
        We will denote the general electronic Hamiltonian by
        \begin{align}
            \hamil = \onehamil + \twohamil,
        \end{align}
        where $\onehamil$ is the one-body term and $\twohamil$ describes
        two-body interactions, which in our case is the Coulomb interaction.

        \subsection{One-body Hamiltonian}
            The one-body Hamiltonian describes the non-interacting system we are
            examining.
            In the time-independent case we typically describe the system by a
            sum of kinetic and potential terms.
            \begin{align}
                \onehamil
                &= \sum_{i = 1}^{N}\onehamil_i
                = \sum_{i = 1}^{N}\para{
                    \kinetic_i + \potential_i
                }
                = \sum_{i = 1}^{N}\para{
                    \frac{\momentumvec^2_i}{2m_i} + \potential_i
                },
            \end{align}
            where the sum runs over all particles in the system separately.
            The potential term can come from many different sources, e.g., an
            external field, mean-field approximations, the electron-nucleon
            interaction, etc.
            We will discsuss these more in depth when we start looking at
            specific systems.
            % TODO: Perhaps include the grid representation of the operators?
            Often when solving the many-body problem we seek a solution to the
            one-body Hamiltonian in terms of a reference Slater determinant.
            This will in many cases provide us with a good starting guess for
            the system before using more sophisticated methods such as the
            Hartree-Fock method to improve on our reference state.

            \subsubsection{Time-dependency}
                In general the time-dependency can occur in all the terms in the
                full many-body Hamiltonian, but the most common scenario is an
                external interaction occuring in the one-body Hamiltonian, e.g.,
                a laser field, time-varying potential, etc.
                In this thesis we will only concern ourselves with an external
                laser field.
                We can then write the time-dependent one-body Hamiltonian on the
                form
                \begin{align}
                    \onehamil(t)
                    &= \sum_{i = 1}^{N} \onehamil_i(t)
                    = \sum_{i = 1}^{N}\para{
                        \kinetic_i
                        + \potential_i
                        + \laserfield_i(t)
                    },
                \end{align}
                where the time-dependence is represented by the time-dependent
                operator $\laserfield_i(t)$.


        \subsection{Two-body Hamiltonian}
            The two-body Hamiltonian provides interactions between pairs of
            particles.
            This complicates the matter by introducing a double sum over all
            particles to include all pairs.
            \begin{align}
                \twohamil
                = \sum_{i = 1}^{N} \sum_{j = 1}^{N} \twohamil_{ij}.
            \end{align}
            Often some care must be taken when $i = j$ as this results in
            self-interaction which can yield un-physical results, but we have
            for the sake of generality written the full sum.
            % TODO: Verify this.
            % Remember that the overlap occurs in the integral.
            The two-body operator is often the bottle-neck in terms of
            complexity for the electronic Hamiltonian.

            \subsubsection{Coulomb interaction}
                We will in this thesis concern ourselves with electronic
                interactions and will therefore limit ourselves to the Coulomb
                interaction as our two-body operator.
                This operator is given by
                \begin{align}
                    \twohamil
                    &= \sum_{i = 1}^{N}\sum_{j = i + 1}^{N}
                    \twohamil_{ij}
                    = \sum_{i = 1}^{N}\sum_{j = 1}^{N}
                    \frac{1}{4\epsilon_0\pi}\frac{e^2}{\abs{\positionvec_i -
                    \positionvec_j}},
                \end{align}
                The Coulomb interaction is a quantized version of the Coulomb
                potential with $\epsilon_0$ representing the vacuum
                permittivity and $e^2$ the charge of the particles, which we
                assume to be of the same type.
                % TODO: Expand on the physical interpretation of Coulomb's law
                % and the quantized version.

        \subsection{The Born-Oppenheimer approximation}
            The Born-Oppenheimer approximation first occured in a seminal paper
            by \citeauthor{born1927quantentheorie}
            \cite{born1927quantentheorie}.
            It is procedure describing how we can solve a molecular system
            consisting of electrons and nucleons where we assume that the total
            wave function can be separated into an electronic and a nuclear
            part, viz.
            \begin{align}
                \ket{\Psi} = \ket{\Psi_e}\otimes\ket{\Psi_n},
            \end{align}
            where we denote the electronic wave function by $\ket{\Psi_e}$ and
            the nuclear counterpart as $\ket{\Psi_n}$.
            % TODO: This might not be relevant unless we include atomic systems.

        \subsection{Second quantized formulation}
            Having introduced the second quantized formulation of many-body
            quantum mechanics we wish to represent the Hamiltonian in this
            formalism.
            This introduces a generalization which will prove very useful for
            the later many-body methods as it removes the explicit grid
            representation of the operators in introduces abstract matrix
            elements as coefficients to the second quantized operators.

            The expectation value of the Hamiltonian in terms of a Slater
            determinant is then
            \begin{align}
                \bra{\Phi}\hat{H}\ket{\Phi}
                &= \bra{\phi_i}\hat{h}\ket{\phi_i}
                + \frac{1}{2}\bra{\phi_i\phi_j}\hat{u}\ket{\phi_i\phi_j}_{AS},
            \end{align}
            where the antisymmetric matrix elements are given by
            \begin{align}
                \bra{\phi_i\phi_j}\hat{u}\ket{\phi_i\phi_j}_{AS}
                &= \bra{\phi_i\phi_j}\hat{u}\ket{\phi_i\phi_j}
                - \bra{\phi_i\phi_j}\hat{u}\ket{\phi_j\phi_i}.
                \label{eq:antisymmetric_two_body}
            \end{align}

    \section{The reference energy}
        Given a basis of orthonormal single-particle state
        $\brac{\ket{\phi_p}}_{p = 1}^{L}$ and constructing a Slater determinant
        from the $N$ first states, viz.
        \begin{align}
            \ket{\refslat} = \ket{\phi_1, \dots, \phi_N},
        \end{align}
        that is, we have $N$ occupied states and $M = L - N$ virtual states
        using the same convention for the indices as discussed in
        \autoref{sec:fermi-vacuum} on the Fermi vacuum.
        We dub this state the \emph{reference state} for reasons which will
        become clear when we start working on the many-body methods.
        For a general electronic Hamiltonian with one- and two-body operators,
        we can compute the expectation value of the energy from the reference
        state.
        \begin{align}
            \bra{\refslat}\hamil\ket{\refslat}
            &=
            \bra{\refslat}\onehamil\ket{\refslat}
            + \bra{\refslat}\twohamil\ket{\refslat}
            \\
            &=
            \oneten^{p}_{q}
            \bra{\refslat}\ccr{p}\can{q}\ket{\refslat}
            + \frac{1}{4}
            \twoten^{pq}_{rs}
            \bra{\refslat}\ccr{p}\ccr{q}\can{s}\can{r}\ket{\refslat},
        \end{align}
        where we use the anti-symmetric two-body elements.
        Using Wick's theorem\footnote{%
            Or the anti-commutation rules manually.
        }%
        we can evaluate the overlap strings on the reference state.
        This yields the \emph{reference energy} given by
        \begin{align}
            \energyref
            &\equiv
            \bra{\refslat}\hamil\ket{\refslat}
            =
            \oneten^{i}_{i}
            + \frac{1}{2}\twoten^{ij}_{ij}.
            \label{eq:reference-energy}
        \end{align}
        The derivation of this expression can be found in
        \autoref{sec:deriving-the-reference-energy}.

    \section{The normal-ordered Hamiltonian}
        A general electronic Hamiltonian with one- and two-body operators on
        a second-quantized form is given by
        \begin{align}
            \hamil
            &= \oneten^{p}_{q}\ccr{p}\can{q}
            + \frac{1}{4}\twoten^{pq}_{rs}\ccr{p}\ccr{q}\can{s}\can{r},
        \end{align}
        where we've labelled the anti-symmetric two-body elements by
        $u^{pq}_{rs}$, and Einstein summation is assumed.
        We can write the electronic Hamiltonian on a normal ordered form using
        Wick's theorem.



    \section{Many-body density matrices}
        In a seminal paper by \citeauthor{lowdin-density-matrices}
        \cite{lowdin-density-matrices}, the concept of a many-body density
        matrix in terms of the orbitals of a Slater determinant is discussed.
        These are dubbed $N$-body density matrices, where $N$ depends on the
        $N$-body interaction, that is, the number of particles included in the
        interaction.
        Consider a Hermitian operator $\hat{Q}$ which can be represented as a
        sum of $N$-body operators
        \begin{align}
            \hat{Q}
            &\equiv
            \sum_{i = 0}^{N} \hat{Q}_i
            = Q_0\1 + Q^{p}_{q}\ccr{p}\can{q}
            + \frac{1}{2!}Q^{pq}_{rs}\ccr{p}\ccr{q}\can{s}\can{r}
            + \dots,
        \end{align}
        where each $N$-body operator is symmetrical in its indices, and includes
        a $(N!)^{-1}$ factor to account for double counting.
        % TODO: Does the indidices need to symmetric?
        For a normalized, pure state $\ket{\psi}$, we are able to compute the
        expectation value of the operator $\hat{Q}$ by
        \begin{align}
            \expv{Q} &= \mel{\psi}{\hat{Q}}{\psi}
            = \tr(\density \hat{Q}),
        \end{align}
        where we've introduced the density operator $\ket{\psi}$ by
        \begin{align}
            \density = \dyad{\psi}{\psi}.
        \end{align}
        Now, if $\ket{\psi}$ is a wave function describing a system of $N$
        particles, we can write the expectation value of the operator $\hat{Q}$
        as
        \begin{align}
            \expv{Q}
            &= \tr(\density \hat{Q})
            = \braket{\psi}\mel{\psi}{\hat{Q}}{\psi}
            = \sum_{i = 0}^{N} \mel{\psi}{\hat{Q}_i}{\psi}
            \\
            &=
            Q_0 \mel{\psi}{\1}{\psi}
            + Q^{p}_{q}\mel{\psi}{\ccr{p}\can{q}}{\psi}
            + \frac{1}{2!}
            Q^{pq}_{rs}
            \mel{\psi}{\ccr{p}\ccr{q}\can{s}\can{r}}{\psi}
            + \dots
            \\
            &= Q_0 + Q^{p}_{q}\densityten^{q}_{p}
            + \frac{1}{2!} Q^{pq}_{rs} \densityten^{rs}_{pq}
            + \dots,
        \end{align}
        where we ask the reader to direct special attention to the index
        ordering of the $N$-body density matrix indices $\densityten$ as the
        ``opposite'' direction of the index ordering from the $N$-body matrix
        elements.
        % TODO: Prove this ordering from Szabo and Ostlund.
        We thus observe that all information about the wave function that is
        needed for the evaluation of the expectation value $\expv{Q}$ is
        contained in the $N$-body density matrices, where the elements are given
        by
        \begin{gather}
            \rho^{q}_{p}
            = \mel{\psi}{\ccr{p}\can{q}}{\psi},
            \label{eq:one-body-density-elements}
            \\
            \rho^{rs}_{pq}
            = \mel{\psi}{\ccr{p}\ccr{q}\can{s}\can{r}}{\psi},
            \label{eq:two-body-density-elements}
            \\
            \vdots
        \end{gather}
        Of all the $N$-body density matrices, the one- and two-body density
        matrices are the ones that are most applicable for our work as the
        many-body Hamiltonian we are looking at is limited to two-body
        interactions and we are only interested in observables expressable as
        one-body operators.

        \subsection{One-body density matrix}
            We will denote the one-body density operator by $\density_1$ where
            the elements are given by \autoref{eq:one-body-density-elements}.
            Comparing the one-body density operator with the general density
            operator as discussed in \autoref{sec:density-operators} we have
            that $\density^{\dagger}_1 = \density_1$, i.e., it is Hermitian.
            This can be seen from the elements by
            \begin{align}
                (\densityten^{q}_{p})^{*}
                = \mel{\psi}{\ccr{p}\can{q}}{\psi}^{*}
                = \mel{\psi}{\ccr{q}\can{p}}{\psi}
                = \densityten^{p}_{q}.
            \end{align}
            The one-body density is positive semidefinite, which means it
            satisifies the positivity condition for density operators.
            % TODO: Back this up.
            The normalization of the one-body density is however a little
            different from the general density operator.
            We have that
            \begin{align}
                \tr(\density_1)
                &= \densityten^{p}_{p}
                = \mel{\psi}{\ccr{p}\can{p}}{\psi}
                = N,
            \end{align}
            where $N$ is the number of particles contained in $\ket{\psi}$.
            The formalism of the one-body density operator can also be extended
            to a mixed many-body states similar to the general density operator.

            \subsubsection{Particle density}
                A quantity we will be concerned with is the one-particle density
                -- also known as the first-order reduced density matrix -- which
                describes the simulatanous distribution of all the particles in
                the system.
                As an integral it is defined as
                \begin{align}
                    \densityten(x_1)
                    &= N\para{\prod_{i = 2}^{N} \int \dd x_i}
                    \abs{\psi(x_1, x_2, \dots, x_N)}^2,
                \end{align}
                where the integral runs over all generalized coordinates $x_i$
                except for one \cite{lowdin-density-matrices,
                hogberget2013quantum}.
                Due to the indistinguishability of the particles the particle
                density is a measure of where any particle is located in space.
                The one-body density provides insight as to where particles in a
                system will be positioned, but not which particles are where nor
                how they behave relative to each other.
                Using the one-body density matrix we are able to find the
                distribution of all the particles by the single-particle
                functions.
                This is given by
                \begin{align}
                    \densityten(x)
                    =
                    \phi^{*}_q(x) \densityten^{q}_{p} \phi_p(x).
                \end{align}
                % TODO: Check the indexing.

        \subsection{Two-body density matrix}
            We denote the two-body density operator by $\density_2$ with
            elements from \autoref{eq:two-body-density-elements}.
            Due to the anticommutation relations between the second quantized
            operators we have the anti-symmetry
            \begin{align}
                \densityten^{rs}_{pq}
                = -\densityten^{sr}_{pq}
                = -\densityten^{rs}_{qp}
                = \densityten^{sr}_{qp}.
            \end{align}
            As a consequence, the Pauli principle is baked into the elements by
            \begin{align}
                \densityten^{rr}_{pq}
                = \densityten^{rs}_{pp}
                = \densityten^{rr}_{p}
                = 0,
            \end{align}
            in the same way as the two-body Hamiltonian.
            The two-body density operator is also Hermitian in the sense that
            \begin{align}
                (\densityten^{rs}_{pq})^{*}
                = \mel{\psi}{\ccr{p}\ccr{q}\can{s}\can{r}}{\psi}^{*}
                = \mel{\psi}{\ccr{r}\ccr{s}\can{q}\can{p}}{\psi}
                = \densityten^{pq}_{rs}.
            \end{align}
            The two-body density operator is also positive semidefinite in the
            same manner as the one-body density operator.
            The normalization of the two-body density operator can be found from
            \begin{align}
                \densityten^{pq}_{pq}
                &= \mel{\psi}{\ccr{p}\ccr{q}\can{q}\can{p}}{\psi}
                = -\mel{\psi}{\ccr{p}\ccr{q}\can{p}\can{q}}{\psi}
                = -\mel{\psi}{
                    \ccr{p}(\delta_{pq} - \can{p}\ccr{q})\can{q}
                }{\psi}
                \\
                &=
                \mel{\psi}{\nop{p}\nop{q}}{\psi}
                - \mel{\psi}{\nop{p}}{\psi}
                = N^2 - N
                = N(N - 1).
            \end{align}
            % TODO: Check that this is correct.
            % Helgaker operates with N(N - 1) / 2, but this might be due to the
            % counting of the matrix version of the elements.
            The two-body operator is not something we will use a lot as there
            are no two-body observables we will observe in this thesis.
            However, it will be used when we move to the orbital-adaptive
            time-dependent coupled cluster method as it is needed in the orbital
            equations.

    \section{Dipole moments}
        % TODO: Read "Atoms in intense laser fields"-book.
        \subsection{Transitions}

    \section{The Harmonic potential theorem}
        A remarkable result for quantum dots trapped in parabolic quantum wells
        is that the system behaves as a single large harmonic oscillator
        independent of the number of particles \cite{kohn, brey}.
        This means that we are unable to see a ``many-body effect'' when the
        system of quantum dots are trapped in an harmonic oscillator potential
        well as all inter-particle interactions are not observable, and the
        system behaves as a single particle.
