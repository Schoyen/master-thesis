\chapter{Many-body quantum mechanics}
    In this chapter we'll describe some of the formalism used when describing
    many-body quantum mechanics.

    \section{Spin-orbitals}
        In this thesis we will be looking at many-particle systems.
        Wave functions of many-particle systems depend on the coordinates of all
        the particles contained in the system.
        We will typically build the wave functions of the full system from
        \emph{single particle functions}.
        In this work we'll limit our attention to particles with spin one half ,
        i.e., fermions, which means that the total wave function must satisfy
        the \emph{Pauli exclusion principle} due to the particles being
        indistinguishable.
        The single particle functions will as a consequence depend on the
        particle's spatial orientation, i.e., the \emph{orbital} part, and the
        spin.
        We call these wave functions \emph{spin-orbitals}.
        \begin{align}
            \psi(x) \equiv \psi(\vf{r}, m_s)
        \end{align}
        where $x = (\vf{r}, m_s)$ is a generalized coordinate of both
        position, $\vf{r}$, and spin quantum number, $m_s$.
        For fermions we have only two allowed spin states
        \begin{align}
            s_z = \pm \half\hslash,
        \end{align}
        where $s_z$ is the spin along $z$-direction.
        As there are only two allowed states we have that $m_s \in
        \brac{\upspin, \downspin}$, where a positive value for $s_z$ corresponds
        to $m_s = \upspin$, i.e., spin up, and a negative value for $s_z$ to
        $m_s = \downspin$, i.e., spin down.
        We denote
        \begin{align}
            \psi_1(\vf{r}) \equiv \psi\para{\vf{r}, \upspin},
            \qquad
            \psi_2(\vf{r}) \equiv \psi\para{\vf{r}, \downspin},
        \end{align}
        for the two different spin-directions.
        We can thus represent the generalized spin-orbital $\psi(x)$ as a
        two-dimensional vector
        \begin{align}
            \psi(x) = \begin{pmatrix}
                \psi_1(\vf{r}) \\
                \psi_2(\vf{r})
            \end{pmatrix}.
        \end{align}
        We separate the spin dependence from the spatial part of the
        spin-orbitals by introducing separate ``spin functions'' for spin-up and
        spin-down.
        For example, choosing the basis
        \begin{align}
            \alpha \equiv \alpha(m_s) = \begin{pmatrix}
                1 \\
                0
            \end{pmatrix},
            \qquad
            \beta \equiv \beta(m_s) = \begin{pmatrix}
                0 \\
                1
            \end{pmatrix},
            \label{eq:spin-basis}
        \end{align}
        we use the same convention as in much of the many-body quantum mechanics
        litterature in labelling $\alpha$ as spin up and $\beta$ as spin down in
        the $z$-direction.
        Evaluating the spin functions thus yields
        \begin{gather}
            \alpha\para{\upspin} = 1, \qquad \alpha\para{\downspin} = 0, \\
            \beta\para{\upspin} = 0, \qquad \beta\para{\downspin} = 1.
        \end{gather}
        Using \autoref{eq:spin-basis} we see that we write the
        generalized spin-orbital as a linear combination of the spin basis
        functions by
        \begin{align}
            \psi(x)
            = \psi_1(\vf{r})\alpha(m_s)
            + \psi_2(\vf{r})\beta(m_s).
            \label{eq:general-spin-orbital-1}
        \end{align}
        % TODO: Consider adding theory on spin as in Mayer.
        We will for the most part work with spin-orbitals in a definite
        spin-direction, viz.
        \begin{align}
            \psi(x) = \phi(\vf{r})\sigma(m_s),
            \label{eq:spin-orbital}
        \end{align}
        where $\sigma(m_s)$ is either spin up $\alpha(m_s)$ or spin down
        $\beta(m_s)$ and we've denoted the spatial orbital by $\phi(\vf{r})$.
        The reason we avoid using a general spin-orbital as in
        \autoref{eq:general-spin-orbital-1} is convenience.
        It is simpler to generate a basis of spin-orbitals where we use a
        definite basis of analytic orbitals, e.g., harmonic oscillator basis
        functions, Gauss functions, etc, before deciding the spin direction of
        each orbital\footnote{
            For those that might find this frustrating, note that there is
            nothing stopping us from using orbitals and spin-functions which are
            linear combinations of an underlying basis of orbitals and
            spin-functions as well.
            There's linear combinations all the way down!
        }.
        When labelling the different spin-orbitals from
        \autoref{eq:spin-orbital} we use the notation
        \begin{align}
            \psi_P(x) = \phi_p(\vf{r})\sigma(m_s),
        \end{align}
        where $P = (p, \sigma)$ is a composite index with $p$ labelling a
        specific orbital and $\sigma$ a specific spin-function.
        In the spin basis shown in \autoref{eq:spin-basis} we have that the
        inner product of two spin functions
        \begin{align}
            \braket{\sigma}{\tau}
            &=
            \sigma(\upspin)\tau(\upspin)
            + \sigma(\downspin)\tau(\downspin)
            = \delta_{\sigma\tau},
        \end{align}
        that is, they are orthonormal.
        The inner product of two spin-orbitals from \autoref{eq:spin-orbital} is
        then separated into an orbital and a spin inner product.
        \begin{align}
            \braket{\psi_P}{\psi_Q}
            &= \braket{\phi_p}{\phi_q}\braket{\sigma}{\tau}
            = \delta_{\sigma\tau}
            \int\dd\vf{r}\phi^{*}_{p}(\vf{r})\phi_{q}(\vf{r}).
            \label{eq:inner-spin-orbital}
        \end{align}
        This equation is quite suggestive in the sense that we can see that the
        spin decouples from the spatial part of the wave function.
        Spin is two-level system that can be represented as a vector in a
        two-dimensional Hilbert space.
        Unless an operator contains a spin-coupling, the two spin-directions are
        completely independent of one another.
        For example, given a spin-independent operator, $\hat{O}$, the
        matrix element between two spin-orbitals become
        \begin{align}
            \bra{\psi_P}\hat{O}\ket{\psi_Q}
            &= \bra{\phi_p}\hat{O}\ket{\phi_q}\braket{\sigma}{\tau}
            = \delta_{\sigma\tau}
            \int\dd\vf{r}\phi^{*}_{p}(\vf{r})\hat{O}\phi_q(\vf{r}).
            \label{eq:example-spin-decoupling}
        \end{align}
        This motivates the coordinate representation independent notation
        \begin{align}
            \ket{\psi_P}
            &= \ket{\phi_p}\otimes\ket{\sigma},
        \end{align}
        where the tensor product combines the two-dimensional Hilbert space from
        the spin $\ket{\sigma}$ and the $L$-dimensional Hilbert space
        containing the orbital basis function.
        A general operator on the combined Hilbert space can thus be represented
        by
        \begin{align}
            \hat{O}
            &= \sum_{ij} \hat{R}_i \otimes \hat{S}_j,
        \end{align}
        where $\hat{R}_i$ is an operator on the Hilbert space containing the
        orbitals and $\hat{S}_j$ an operator on the two-dimensional Hilbert
        space containing the spin functions.
        The important point is now that the elements of this operator in terms
        of spin-orbitals is given by
        \begin{align}
            \bra{\psi_{P}}\hat{O}\ket{\psi_{Q}}
            &=
            \sum_{ij}\bra{\phi_p}\hat{R}_i\ket{\phi_q}
            \bra{\sigma}\hat{S}_j\ket{\tau}.
        \end{align}
        Looking back at \autoref{eq:example-spin-decoupling} we see that a
        spin-independent operator $\hat{O}$ corresponds to $\hat{S}_j = \1$.

        \subsection{Restrictions on the choice of spin-orbitals}
            \label{subsec:restrictions-on-spin-orbitals}
            As mentioned earlier, choosing completely general spin-orbitals,
            i.e,. spin-orbitals that are linear combinations of the two
            spin-directions, can become problematic.
            Amongst other things, due to a phenomenom called
            \emph{spin-contamination}, which we'll discuss later in the section
            on Hartree-Fock theory.
            % TODO: Discuss spin-contamination in relation to the completely
            % unrestriced Hartree-Fock method.
            Some common choices of spin-orbitals can be categorized in the
            following manner.
            \begin{enumerate}
                \item The general spin-orbital is on the form
                    \begin{align}
                        \psi(x, t)
                        = \psi_1(\vf{r}, t) \alpha(m_s)
                        + \psi_2(\vf{r}, t) \beta(m_s).
                        \label{eq:general-spin-orbital}
                    \end{align}
                    These spin-orbitals can lead to mixing between both
                    spin-directions, known as spin-contamination, that is, the
                    spin-orbitals are no longer eigen states of the
                    spin-projection operator $\spinproj$.
                    This is often an unwanted effect as this greatly limits some
                    of the optimizations that can be employed for some of the
                    computational methods.
                \item The spin-unrestricted spin-orbital is on the form
                    \begin{align}
                        \psi(x, t)
                        = \psi_{\sigma}(\vf{r}, t) \sigma(m_s),
                        \label{eq:unrestricted-spin-orbital}
                    \end{align}
                    where the orbital $\psi_{\sigma}$ depends on the
                    spin-direction $\sigma$.
                    Often this type of spin-orbitals are dubbed unrestricted
                    spin-orbitals, which is somewhat of a misnomer as the
                    spin-orbitals are restricted in the choice of spin-direction
                    as opposed to the (truly) unrestricted generalized
                    spin-orbitals shown in \autoref{eq:general-spin-orbital}.
                \item The spin-restricted spin-orbital is given by
                    \begin{align}
                        \psi(x, t)
                        = \psi(\vf{r}, t) \sigma(m_s),
                        \label{eq:restricted-spin-orbital}
                    \end{align}
                    where each orbital is doubly occupied.
            \end{enumerate}

        \subsection{Orbital rotations}

        \subsection{Basis sets}
            Also known as \emph{atomic orbitals}.
            In this document we will rely heavily on basis transformations and
            creating linear combinations from a basis to another. When creating
            many-body wavefunctions we will use a finite basis of \emph{orbitals}
            $\brac{\ket{\chi_{\alpha}}}_{\alpha = 1}^{L}$ as our \emph{atomic
            orbitals}. These represent known basis states, e.g., the harmonic
            oscillator eigenstates, etc. We can construct many-body wavefunctions
            from a linear combination of tensor products of single-particle states,
            \begin{align}
                \ket{\Psi}
                = \sum_{\sigma \in S} c_{\sigma}
                % TODO: Figure out how to label the coefficients c
                \ket{\chi_{\sigma(1)}}\otimes\dots\otimes\ket{\chi_{\sigma(L)}},
            \end{align}
            where we have assumed normalized single-particle functions
            $\braket{\chi_{\alpha}}{\chi_{\alpha}} = 1$, but not necessarily
            orthonormal.
            % TODO: Review these statements. Can the total wave function be
            % normalized if the single particle functions are normalized but not
            % orthonormal?
        \subsection{Summation convention}
            We will throughout this thesis just a summation convention that
            resembles the Einstein summation convention, but with slight
            variations.


    \section{Second quantization}
        So far, we have worked with quantum mechanics formulated in first
        quantization where observables are operators and states are functions
        defined in some space.
        Moving to second quantization we will express wave functions as creation
        and annihilation operators working on the vacuum state.
        Depending on the algebra for the operators, we can look at antisymmetric
        fermions or symmetric bosons.
        Expressing the first quantized observables in the same creation and
        annihilation operators, we are able to unify much of quantum mechanics
        to a single set of elementary operators \cite{helgaker-molecular}.

        \subsection{Creating and annihilating particles}

        \subsection{Bosons and fermions}
            % TODO: Discuss the two types of particles and the use of
            % commutators to differentiate these.

            \subsubsection{Pauli exclusion principle}
                % TODO: Include the spin-statistics theorem.

        \subsection{Slater determinants}
            As we are working with fermions the total many-body wave function must
            be \emph{antisymmetric} with respect to interchange of
            particles\footnote{We will restrict ourselves to the interchange of a
            full spin-orbital.}. Mathematically the full wave function of $N$
            fermions should satisfy
            \begin{align}
                \Psi(1, \dots, i, \dots, j, \dots, N)
                =
                -\Psi(1, \dots, j, \dots, i, \dots, N),
            \end{align}
            where we for brevity have introduced the notation $i \equiv x_i =
            (\vf{r}_i, \sigma_i)$. A fully antisymmetric, normalized, $N$-body wave
            function is given by the \emph{Slater determinant} built from $N$
            individual spin-orbitals with $N$ generalized coordinates.
            \begin{align}
                \Psi(1, \dots, N)
                &= \frac{1}{\sqrt{N!}}
                \begin{vmatrix}
                    \psi_1(1) & \dots & \psi_N(1) \\
                    \vdots & \ddots & \vdots \\
                    \psi_1(N) & \dots & \psi_N(N)
                \end{vmatrix},
                \label{eq:coord_slater}
            \end{align}
            where $\psi_i(j) \equiv \psi_i(\vf{r}_j, \sigma_j)$ for the $i$'th
            spin-orbital in generalized coordinate $j$ \footnote{Note that the index
            on the spin-orbitals has nothing to do with which spin component is used
            as in the previous section on spin-orbitals.}. Exchanging a row in the
            determinant results in a sign change as we required from the
            antiysmmetric full wave function for the many-body problem. Furthermore,
            exchanging a column in the determinant also results in a sign-change.
            This means that there can not be two, or more, of the same spin-orbitals
            in the wave function perfectly encapsulating the Pauli principle as two
            of the same spin-orbitals translates to two particles being in the same
            state. Generally, the spin-orbitals in a Slater determinant are linearly
            independent, otherwise $\Psi = 0$. This is a consequence of the
            properties of a determinant that adding a scalar multiple of a column to
            another column does not change the determinant.

            The factor in front of the Slater determinant makes the many-body wave
            function normalized to unity as long as the spin-orbitals are
            orthonormalized.

            % TODO: Introduce proof
            \begin{proof}[Proof that Slater determinants are invariant under unitary
                transformations]
                \label{proof:slater_determinants_invariant}
                Given a basis $\brac{\ket{\phi_i}}_{i = 1}^{N}$ that spans the
                $N$-dimensional Hilbert space. We can then do a unitary
                transformation from this basis to a new basis
                $\brac{\ket{\psi_i}}_{i = 1}^{N}$ in the same Hilbert space by
                \begin{align}
                    \ket{\psi_i} = U_{ji}\ket{\phi_j},
                    \label{eq:unitary_transformation}
                \end{align}
                where $U_{ji}$ is an element in the unitary matrix $\vfg{U}$.
                For a set of coordinates
                $\brac{x_1, \dots, x_N}$, we write
                \begin{align}
                    \psi_{ij} \equiv \psi_j(x_i)
                    \equiv \braket{x_i}{\psi_j},
                \end{align}
                and equivalently for $\phi_{ij}$. Projecting onto the coordinate
                basis we can write \autoref{eq:unitary_transformation} as
                \begin{gather}
                    \braket{x_k}{\psi_i}
                    = U_{ji}\braket{x_k}{\phi_j}
                    \implies
                    \psi_{ki} = U_{ji}\phi_{kj} = \phi_{kj}U_{ji}
                    \implies \vfg{\Psi} = \vfg{\Phi}\vfg{U},
                \end{gather}
                where the matrices $\vfg{\Psi}$, $\vfg{\Phi}$ and $\vfg{U}$ are
                the matrices with elements
                \begin{align}
                    \vfg{\Psi}
                    &= \begin{pmatrix}
                        \psi_{11} & \psi_{12} & \dots & \psi_{1N} \\
                        \vdots & \vdots & \ddots & \vdots \\
                        \psi_{N1} & \psi_{N2} & \dots & \psi_{NN}
                    \end{pmatrix}, \\
                    \vfg{\Phi}
                    &= \begin{pmatrix}
                        \phi_{11} & \phi_{12} & \dots & \phi_{1N} \\
                        \vdots & \vdots & \ddots & \vdots \\
                        \phi_{N1} & \phi_{N2} & \dots & \phi_{NN}
                    \end{pmatrix}, \\
                    \vfg{U}
                    &= \begin{pmatrix}
                        U_{11} & U_{12} & \dots & U_{1N} \\
                        \vdots & \vdots & \ddots & \vdots \\
                        U_{N1} & U_{N2} & \dots & U_{NN}
                    \end{pmatrix},
                \end{align}
                and where $\vfg{U}^{\dagger}\vfg{U} = \1$.
                Creating the fully antisymmetrized normalized wave function, i.e.,
                the Slater determinant, from $\vfg{\Psi}$ and $\vfg{\Phi}$ we see
                that
                % TODO: Fix description, the determinant of the "Slater matrices"
                % are not necessarily normalized without the pre-factor.
                \begin{gather}
                    \det(\vfg{\Psi}) = \det(\vfg{\Phi}\vfg{U})
                    = \det(\vfg{\Phi})\det(\vfg{U})
                \end{gather}
                We now take the squared norm on both sides.
                \begin{align}
                    \norm{\det(\vfg{\Psi})}^2
                    = \norm{\det(\vfg{\Phi})\det(\vfg{U})}^2
                    \leq
                    \norm{\det(\vfg{\Phi})}^2\norm{\det(\vfg{U})}^2
                    = \norm{\det(\vfg{\Phi})}^2,
                    \label{eq:squared_determinant}
                \end{align}
                where we have used that
                \begin{align}
                    \norm{\det(\vfg{U})}^2
                    &= \norm{\det(\vfg{U^{\dagger}})\det(\vfg{U})}
                    = \norm{\det(\vfg{U}^{\dagger}\vfg{U})}
                    = \norm{\det(\1)} = 1.
                \end{align}
                Now, in \autoref{eq:squared_determinant}, since the Slater
                determinant of $\vfg{\Phi}$ is orthonormalized and a unitary
                transformation preserves the normalization this means that the norm
                of $\det(\vfg{\Phi})$ and the norm of $\det(\vfg{\Psi})$ must be
                unity.  Thus, the equality in the splitting of the norm is preserved
                thus yielding
                \begin{gather}
                    \norm{\det(\vfg{\Psi})}^2 = \norm{\det(\vfg{\Phi})}^2.
                \end{gather}
            \end{proof}
            \subsubsection{Antisymmetrizer}


        \subsection{Fock space}
            In second quantization, we express a general many-body wave function
            as a vector in the abstract linear vector space called \emph{Fock
            space}.
            In a basis of $L$ single-particle function, $\brac{\phi_p}_{p =
            1}^{L}$, this \emph{occupation number vector} can be represented by
            \begin{align}
                \ket{\Psi}
                &= \ket{n_1, n_2, \dots, n_L}
                \equiv \ket{\vf{n}},
            \end{align}
            where $n_i$ signifies how many of state $i$ is contained in
            $\ket{\Psi}$ and we've defined the set $\vf{n}$ as the set
            containing all $n_i$.
            Already, we can choose the particle type we are working with based
            on the allowed values of $n_i$.
            We have that
            \begin{align}
                n_i =
                \begin{cases}
                    0, 1 & \text{fermions}, \\
                    0, 1, 2, \dots & \text{bosons},
                \end{cases}
            \end{align}
            where a $0$ means that the state is unoccupied and a $1$, or higher,
            means that the state is occupied by that amount of particles.
            We thus see that for fermions we can only have a state occupied by
            single particle, whereas for bosons there are no limitiations to the
            amount of particles that can be in the same state.

            If the basis of spin-orbitals are orthonormal, we have that the
            overlap between two vectors in Fock space is given by
            \begin{align}
                \braket{\Psi}{\Phi}
                = \braket{\vf{n}}{\vf{m}}
                = \delta_{\vf{n}\vf{m}}
                \equiv \prod_{i = 1}^{L} \delta_{n_i m_i},
            \end{align}
            that is, the two states must share the exact same occupied states.
            However, if the basis is non-orthogonal, viz.
            \begin{align}
                \braket{\phi_p}{\phi_q} = S_{pq} \neq \delta_{pq},
            \end{align}
            with $S_{pq}$ the overlap integral between the two spin-orbitals.
            We then have that the overlap between two occupation numbers vectors
            is given by
            \begin{align}
                \braket{\Psi}{\Phi}
                = \delta_{\vf{n}\vf{m}}\det\para{\vf{S}},
            \end{align}
            where $\vf{S}$ is the matrix with all the overlap integrals between
            the spin-orbitals.
        \subsection{Time-dependent second quantized operators}


    \section{Wick's theorem}
        \subsection{Normal ordering}
        \subsection{Generalized Wick's theorem}
        \subsection{Fermi vacuum}
            \label{sec:fermi-vacuum}
            Given a basis of $L$ single particle functions $\ket{p}$ where
            \begin{align}
                \left\{\ket{p}\right\}_{p = 1}^{L}
                &=
                \left\{\ket{i}\right\}_{i = 1}^{N}
                \cup \left\{\ket{a}\right\}_{a = N + 1}^{L}.
            \end{align}
            Here $i, j, k, \dots$ represents the $N$ first occupied states of
            the reference Slater determinant whereas $a, b, c, \dots$ represent
            the remaining $M = L - N$ virtual states in the total basis $p, q,
            r, \dots$\footnote{Occupied and virtual states are also known as
            hole and particle states if we treat the reference Slater
            determinant as the \textit{Fermi vacuum}}.

    \section{Excited determinants}
        \subsection{Overlap with the reference state}
            % TODO: Remember to introduce the concept of the reference state
        \subsection{Overlap when using time-dependent orbitals}

    \section{The many-body Hamiltonian}
        We denote the general electronic many-body Hamiltonian by
        \begin{align}
            \hamil(t) &= \onehamil(t) + \twohamil(t),
        \end{align}
        where $\onehamil(t)$ and $\twohamil(t)$ is the time-dependent one-body
        and two-body operators respectively.

        \subsection{The Born-Oppenheimer approximation}

        \subsection{One-body Hamiltonian}
            \subsubsection{Time-dependency}
        \subsection{Two-body Hamiltonian}
            \subsubsection{Coulomb interaction}
        \subsection{Second quantized formulation}
        The expectation value of the Hamiltonian in terms of a Slater
        determinant is then
        \begin{align}
            \bra{\Phi}\hat{H}\ket{\Phi}
            &= \bra{\phi_i}\hat{h}\ket{\phi_i}
            + \frac{1}{2}\bra{\phi_i\phi_j}\hat{u}\ket{\phi_i\phi_j}_{AS},
        \end{align}
        where the antisymmetric matrix elements are given by
        \begin{align}
            \bra{\phi_i\phi_j}\hat{u}\ket{\phi_i\phi_j}_{AS}
            &= \bra{\phi_i\phi_j}\hat{u}\ket{\phi_i\phi_j}
            - \bra{\phi_i\phi_j}\hat{u}\ket{\phi_j\phi_i}.
            \label{eq:antisymmetric_two_body}
        \end{align}

    \section{The reference energy}
        Given a basis of orthonormal single-particle state
        $\brac{\ket{\phi_p}}_{p = 1}^{L}$ and constructing a Slater determinant
        from the $N$ first states, viz.
        \begin{align}
            \ket{\refslat} = \ket{\phi_1, \dots, \phi_N},
        \end{align}
        that is, we have $N$ occupied states and $M = L - N$ virtual states
        using the same convention for the indices as discussed in
        \autoref{sec:fermi-vacuum} on the Fermi vacuum.
        We dub this state the \emph{reference state} for reasons which will
        become clear when we start working on the many-body methods.
        For a general electronic Hamiltonian with one- and two-body operators,
        we can compute the expectation value of the energy from the reference
        state.
        \begin{align}
            \bra{\refslat}\hamil\ket{\refslat}
            &=
            \bra{\refslat}\onehamil\ket{\refslat}
            + \bra{\refslat}\twohamil\ket{\refslat}
            \\
            &=
            \oneten^{p}_{q}
            \bra{\refslat}\ccr{p}\can{q}\ket{\refslat}
            + \frac{1}{4}
            \twoten^{pq}_{rs}
            \bra{\refslat}\ccr{p}\ccr{q}\can{s}\can{r}\ket{\refslat},
        \end{align}
        where we use the anti-symmetric two-body elements.
        Using Wick's theorem\footnote{%
            Or the anti-commutation rules manually.
        }%
        we can evaluate the overlap strings on the reference state.
        This yields the \emph{reference energy} given by
        \begin{align}
            \energyref
            &\equiv
            \bra{\refslat}\hamil\ket{\refslat}
            =
            \oneten^{i}_{i}
            + \frac{1}{2}\twoten^{ij}_{ij}.
            \label{eq:reference-energy}
        \end{align}
        The derivation of this expression can be found in
        \autoref{sec:deriving-the-reference-energy}.

    \section{The normal-ordered Hamiltonian}
        A general electronic Hamiltonian with one- and two-body operators on
        a second-quantized form is given by
        \begin{align}
            \hamil
            &= \oneten^{p}_{q}\ccr{p}\can{q}
            + \frac{1}{4}\twoten^{pq}_{rs}\ccr{p}\ccr{q}\can{s}\can{r},
        \end{align}
        where we've labelled the anti-symmetric two-body elements by
        $u^{pq}_{rs}$, and Einstein summation is assumed.
        We can write the electronic Hamiltonian on a normal ordered form using
        Wick's theorem.



    \section{Many-body density matrices}
        In a seminal paper by Löwdin \cite{lowdin-density-matrices}, the
        concept of a many-body density matrix in terms of the orbitals of a
        Slater determinant is discussed. These are dubbed $N$-body density
        matrices, where $N$ depends on the $N$-body interaction, that is,
        the number of particles included in the interaction. Of the most
        useful for our work, we have the one- and two-body density matrices.
        As we will work almost exclusively in second quantization, we will
        follow the derivation of the one- and two-body density matrices done
        by \citeauthor{helgaker-molecular}. Löwdin's paper
        \cite{lowdin-density-matrices} did not employ second quantization,
        and all matrices are expressed in the coordinate representation. We
        will list these as they arrive.
        \subsection{One-body density matrix}
            \subsubsection{Particle density}
        \subsection{Two-body density matrix}

    \section{Dipole moments}
        \subsection{Transitions}

    \section{The Harmonic potential theorem}
