\chapter{Many-body quantum mechanics}
    In this chapter we'll describe some of the formalism used when describing
    many-body quantum mechanics.

    \section{Particle statistics}
        The particles we concern ourselves with are \emph{identical}.
        This means that it is not possible to discern two particles of the
        same type from one another.
        This statement yields quite profound results in the sense
        that the ordering of the particles in a many-body wave function does
        not yield a new state and is in some sense arbitrary.
        As a consequence the probability density of our state must be
        permutation invariant.
        We define $\sigma \in S_{N}$ as a permutation of $N$ indices $\vf{x}
        = (x_1, \dots, x_N) \in X^{N}$ wherein both spin and position is
        incorporated.
        We denote the permutation of the indices by
        \begin{align}
            \vf{x} = (x_1, \dots, x_N)
            \to \vf{x}_{\sigma} = (x_{\sigma(1)}, \dots, x_{\sigma(N)}).
        \end{align}
        We can then formulate particle indistinguishability by
        \cite{leinaas1977, kvaal2017notes}
        \begin{align}
            \abs{
                \Psi(\vf{x})
            }^2
            = \abs{
                \Psi(\vf{x}_{\sigma})
            }^2.
            \label{eq:particle-indistinguishability}
        \end{align}
        We define the \emph{exchange operatore} $\hat{P}_{ij}$ as the
        operator that interchanges indices $(i, j) \to (j, i)$ where $i \neq
        j$.
        That is,
        \begin{align}
            \hat{P}_{ij}\Psi(x_1, \dots, x_i, \dots, x_j, \dots, x_N)
            = \Psi(x_1, \dots, x_j, \dots, x_i, \dots, x_N),
        \end{align}
        From \autoref{eq:particle-indistinguishability} we find that the
        exchange operator is idempotent as exchanging the same pair twice
        yields the original pair in return.
        However, as has been stated, the two states
        $\hat{P}_{ij}\Psi(\vf{x})$ and $\Psi(\vf{x})$ consist of
        indistinguishable particles.
        This means that we are unable to tell the two states apart when
        exchanging particles.
        The only way the two states can differ from one another is by a
        complex phase factor $s_{ij}$, where $s_{ij}^2 = 1$ due the exchange
        operator being idempotent.
        We then find $s_{ij} = \pm 1$.
        Depending on the sign of the phase we denote the state as either
        symmetric or anti-symmetric.
        If the eigenvalue is positive we call it a \emph{symmetric} state
        whereas a negative eigenvalue signifies an \emph{anti-symmetric}
        state.
        A wave function which is symmetric under exchange consists of
        \emph{bosons} whereas an anti-symmetric wave function denotes
        \emph{fermions}.


        We define the permutation operator $\hat{P}_{\sigma}$ as the
        operator that interchanges the position of the particles in
        $\Psi(\vf{x})$ by \emph{transpositions} of pairs.
        This yields
        \begin{align}
            \hat{P}_{\sigma}\Psi(x_1, \dots, x_N)
            &= s_{\sigma}\Psi(x_{\sigma(1)}, \dots, x_{\sigma(N)}),
        \end{align}
        where $s_{\sigma}$ is an eigenvalue of the permutation operator
        depending on the type of particles.

        \subsection{Pauli exclusion principle}
            % TODO: Include the spin-statistics theorem.
            An interesting result of the anti-symmetric property of
            fermions is that we can never have a system containing two
            fermions \emph{in the same state}.
            One way of demonstrating this phenomenom is through the use of
            the exchange operator $\hat{P}_{ij}$ as described above.
            Consider a fully anti-symmetric wave function $\ket{\Psi}$
            that is, a wave function describing a system of fermions, where
            two of the particles are in the same state.
            Let $\hat{P}_{ij}$ be the exchange operator that exchanges these
            two particles.
            We then get
            \begin{align}
                \ket{\Psi}
                =
                \hat{P}_{ij}\ket{\Psi}
                = -\ket{\Psi}
                = 0,
            \end{align}
            where we see that the exchange operator leaves the state
            unchanged, but introduces a sign change which necessarily means
            that the state is not permitted.
            This observation is known as the \emph{Pauli exclusion
            principle}.



    \section{Spin-orbitals}
        In this thesis we will be looking at many-particle systems.
        Wave functions of many-particle systems depend on the coordinates of all
        the particles contained in the system.
        We will typically build the wave functions of the full system from
        \emph{single particle functions}.
        In this work we'll limit our attention to particles with spin one half ,
        i.e., fermions, which means that the total wave function must satisfy
        the \emph{Pauli exclusion principle} due to the particles being
        indistinguishable.
        The single particle functions will as a consequence depend on the
        particle's spatial orientation, i.e., the \emph{orbital} part, and the
        spin.
        We call these wave functions \emph{spin-orbitals}.
        \begin{align}
            \psi(x) \equiv \psi(\vf{r}, m_s)
        \end{align}
        where $x = (\vf{r}, m_s)$ is a generalized coordinate of both
        position, $\vf{r}$, and spin quantum number, $m_s$.
        For fermions we have only two allowed spin states
        \begin{align}
            s_z = \pm \half\hslash,
        \end{align}
        where $s_z$ is the spin along $z$-direction.
        As there are only two allowed states we have that $m_s \in
        \brac{\upspin, \downspin}$, where a positive value for $s_z$ corresponds
        to $m_s = \upspin$, i.e., spin up, and a negative value for $s_z$ to
        $m_s = \downspin$, i.e., spin down.
        We denote
        \begin{align}
            \psi_1(\vf{r}) \equiv \psi\para{\vf{r}, \upspin},
            \qquad
            \psi_2(\vf{r}) \equiv \psi\para{\vf{r}, \downspin},
        \end{align}
        for the two different spin-directions.
        We can thus represent the generalized spin-orbital $\psi(x)$ as a
        two-dimensional vector
        \begin{align}
            \psi(x) = \begin{pmatrix}
                \psi_1(\vf{r}) \\
                \psi_2(\vf{r})
            \end{pmatrix}.
        \end{align}
        We separate the spin dependence from the spatial part of the
        spin-orbitals by introducing separate ``spin functions'' for spin-up and
        spin-down.
        For example, choosing the basis
        \begin{align}
            \alpha \equiv \alpha(m_s) = \begin{pmatrix}
                1 \\
                0
            \end{pmatrix},
            \qquad
            \beta \equiv \beta(m_s) = \begin{pmatrix}
                0 \\
                1
            \end{pmatrix},
            \label{eq:spin-basis}
        \end{align}
        we use the same convention as in much of the many-body quantum mechanics
        litterature in labelling $\alpha$ as spin up and $\beta$ as spin down in
        the $z$-direction.
        Evaluating the spin functions thus yields
        \begin{gather}
            \alpha\para{\upspin} = 1, \qquad \alpha\para{\downspin} = 0, \\
            \beta\para{\upspin} = 0, \qquad \beta\para{\downspin} = 1.
        \end{gather}
        Using \autoref{eq:spin-basis} we see that we write the
        generalized spin-orbital as a linear combination of the spin basis
        functions by
        \begin{align}
            \psi(x)
            = \psi_1(\vf{r})\alpha(m_s)
            + \psi_2(\vf{r})\beta(m_s).
            \label{eq:general-spin-orbital-1}
        \end{align}
        % TODO: Consider adding theory on spin as in Mayer.
        We will for the most part work with spin-orbitals in a definite
        spin-direction, viz.
        \begin{align}
            \psi(x) = \phi(\vf{r})\sigma(m_s),
            \label{eq:spin-orbital}
        \end{align}
        where $\sigma(m_s)$ is either spin up $\alpha(m_s)$ or spin down
        $\beta(m_s)$ and we've denoted the spatial orbital by $\phi(\vf{r})$.
        The reason we avoid using a general spin-orbital as in
        \autoref{eq:general-spin-orbital-1} is convenience.
        It is simpler to generate a basis of spin-orbitals where we use a
        definite basis of analytic orbitals, e.g., harmonic oscillator basis
        functions, Gauss functions, etc, before deciding the spin direction of
        each orbital\footnote{
            For those that might find this frustrating, note that there is
            nothing stopping us from using orbitals and spin-functions which are
            linear combinations of an underlying basis of orbitals and
            spin-functions as well.
            There's linear combinations all the way down!
        }.
        When labelling the different spin-orbitals from
        \autoref{eq:spin-orbital} we use the notation
        \begin{align}
            \psi_P(x) = \phi_p(\vf{r})\sigma(m_s),
        \end{align}
        where $P = (p, \sigma)$ is a composite index with $p$ labelling a
        specific orbital and $\sigma$ a specific spin-function.
        In the spin basis shown in \autoref{eq:spin-basis} we have that the
        inner product of two spin functions
        \begin{align}
            \braket{\sigma}{\tau}
            &=
            \sigma(\upspin)\tau(\upspin)
            + \sigma(\downspin)\tau(\downspin)
            = \delta_{\sigma\tau},
        \end{align}
        that is, they are orthonormal.
        The inner product of two spin-orbitals from \autoref{eq:spin-orbital} is
        then separated into an orbital and a spin inner product.
        \begin{align}
            \braket{\psi_P}{\psi_Q}
            &= \braket{\phi_p}{\phi_q}\braket{\sigma}{\tau}
            = \delta_{\sigma\tau}
            \int\dd\vf{r}\phi^{*}_{p}(\vf{r})\phi_{q}(\vf{r}).
            \label{eq:inner-spin-orbital}
        \end{align}
        This equation is quite suggestive in the sense that we can see that the
        spin decouples from the spatial part of the wave function.
        Spin is two-level system that can be represented as a vector in a
        two-dimensional Hilbert space.
        Unless an operator contains a spin-coupling, the two spin-directions are
        completely independent of one another.
        For example, given a spin-independent operator, $\hat{O}$, the
        matrix element between two spin-orbitals become
        \begin{align}
            \bra{\psi_P}\hat{O}\ket{\psi_Q}
            &= \bra{\phi_p}\hat{O}\ket{\phi_q}\braket{\sigma}{\tau}
            = \delta_{\sigma\tau}
            \int\dd\vf{r}\phi^{*}_{p}(\vf{r})\hat{O}\phi_q(\vf{r}).
            \label{eq:example-spin-decoupling}
        \end{align}
        This motivates the coordinate representation independent notation
        \begin{align}
            \ket{\psi_P}
            &= \ket{\phi_p}\otimes\ket{\sigma},
        \end{align}
        where the tensor product combines the two-dimensional Hilbert space from
        the spin $\ket{\sigma}$ and the $L$-dimensional Hilbert space
        containing the orbital basis function.
        A general operator on the combined Hilbert space can thus be represented
        by
        \begin{align}
            \hat{O}
            &= \sum_{ij} \hat{R}_i \otimes \hat{S}_j,
        \end{align}
        where $\hat{R}_i$ is an operator on the Hilbert space containing the
        orbitals and $\hat{S}_j$ an operator on the two-dimensional Hilbert
        space containing the spin functions.
        The important point is now that the elements of this operator in terms
        of spin-orbitals is given by
        \begin{align}
            \bra{\psi_{P}}\hat{O}\ket{\psi_{Q}}
            &=
            \sum_{ij}\bra{\phi_p}\hat{R}_i\ket{\phi_q}
            \bra{\sigma}\hat{S}_j\ket{\tau}.
        \end{align}
        Looking back at \autoref{eq:example-spin-decoupling} we see that a
        spin-independent operator $\hat{O}$ corresponds to $\hat{S}_j = \1$.

        \subsection{Restrictions on the choice of spin-orbitals}
            \label{subsec:restrictions-on-spin-orbitals}
            As mentioned earlier, choosing completely general spin-orbitals,
            i.e,. spin-orbitals that are linear combinations of the two
            spin-directions, can become problematic.
            Amongst other things, due to a phenomenom called
            \emph{spin-contamination}, which we'll discuss later in the section
            on Hartree-Fock theory.
            % TODO: Discuss spin-contamination in relation to the completely
            % unrestriced Hartree-Fock method.
            Some common choices of spin-orbitals can be categorized in the
            following manner.
            \begin{enumerate}
                \item The general spin-orbital is on the form
                    \begin{align}
                        \psi(x, t)
                        = \psi_1(\vf{r}, t) \alpha(m_s)
                        + \psi_2(\vf{r}, t) \beta(m_s).
                        \label{eq:general-spin-orbital}
                    \end{align}
                    These spin-orbitals can lead to mixing between both
                    spin-directions, known as spin-contamination, that is, the
                    spin-orbitals are no longer eigen states of the
                    spin-projection operator $\spinproj$.
                    This is often an unwanted effect as this greatly limits some
                    of the optimizations that can be employed for some of the
                    computational methods.
                \item The spin-unrestricted spin-orbital is on the form
                    \begin{align}
                        \psi(x, t)
                        = \psi_{\sigma}(\vf{r}, t) \sigma(m_s),
                        \label{eq:unrestricted-spin-orbital}
                    \end{align}
                    where the orbital $\psi_{\sigma}$ depends on the
                    spin-direction $\sigma$.
                    Often this type of spin-orbitals are dubbed unrestricted
                    spin-orbitals, which is somewhat of a misnomer as the
                    spin-orbitals are restricted in the choice of spin-direction
                    as opposed to the (truly) unrestricted generalized
                    spin-orbitals shown in \autoref{eq:general-spin-orbital}.
                \item The spin-restricted spin-orbital is given by
                    \begin{align}
                        \psi(x, t)
                        = \psi(\vf{r}, t) \sigma(m_s),
                        \label{eq:restricted-spin-orbital}
                    \end{align}
                    where each orbital is doubly occupied.
            \end{enumerate}

        \subsection{Orbital rotations}
            % TODO: Include Thouless' theorem.

        \subsection{Basis sets}
            Also known as \emph{atomic orbitals}.
            In this document we will rely heavily on basis transformations and
            creating linear combinations from a basis to another. When creating
            many-body wavefunctions we will use a finite basis of \emph{orbitals}
            $\brac{\ket{\chi_{\alpha}}}_{\alpha = 1}^{L}$ as our \emph{atomic
            orbitals}. These represent known basis states, e.g., the harmonic
            oscillator eigenstates, etc. We can construct many-body wavefunctions
            from a linear combination of tensor products of single-particle states,
            \begin{align}
                \ket{\Psi}
                = \sum_{\sigma \in S} c_{\sigma}
                % TODO: Figure out how to label the coefficients c
                \ket{\chi_{\sigma(1)}}\otimes\dots\otimes\ket{\chi_{\sigma(L)}},
            \end{align}
            where we have assumed normalized single-particle functions
            $\braket{\chi_{\alpha}}{\chi_{\alpha}} = 1$, but not necessarily
            orthonormal.
            % TODO: Review these statements. Can the total wave function be
            % normalized if the single particle functions are normalized but not
            % orthonormal?
        \subsection{Summation convention}
            We will throughout this thesis just a summation convention that
            resembles the Einstein summation convention, but with slight
            variations.


    \section{Second quantization}
        So far, we have worked with quantum mechanics formulated in first
        quantization where observables are operators acting on states that are
        functions defined in some space.
        Moving to second quantization we will express wave functions as creation
        and annihilation operators working on the vacuum state.
        Depending on the algebra for the operators, we can look at antisymmetric
        fermions or symmetric bosons.
        Expressing the first quantized observables in the same creation and
        annihilation operators, we are able to unify much of quantum mechanics
        to a single set of elementary operators \cite{helgaker-molecular}.

        \subsection{Fock space}
            In second quantization, we express a general many-body wave function
            as a vector in the abstract linear vector space called \emph{Fock
            space}.
            In a basis of $L$ single-particle function, $\brac{\phi_p(x)}_{p =
            1}^{L}$, this \emph{occupation number vector} can be represented by
            \begin{align}
                \ket{\Psi}
                &= \ket{n_1, n_2, \dots, n_L}
                \equiv \ket{\vf{n}},
            \end{align}
            where $n_i$ signifies how many states $\phi_i(x)$ are contained in
            $\ket{\Psi}$.
            We've defined the set $\vf{n}$ as the set containing all $n_i$.

            Already, we can choose the particle type we are working with based
            on the allowed values of $n_i$.
            We have that
            \begin{align}
                n_i =
                \begin{cases}
                    0, 1 & \text{fermions}, \\
                    0, 1, 2, \dots & \text{bosons},
                \end{cases}
            \end{align}
            where a $0$ means that the state is unoccupied and a $1$, or higher,
            means that the state is occupied by that amount of particles.
            We thus see that for fermions we can only have a state occupied by a
            single particle, whereas for bosons there are no limitiations to the
            amount of particles that can be in the same state.

            If the basis of spin-orbitals are orthonormal, we have that the
            overlap between two vectors of equal dimension in Fock space is
            given by
            \begin{align}
                \braket{\Psi}{\Phi}
                = \braket{\vf{n}}{\vf{m}}
                = \delta_{\vf{n}\vf{m}}
                \equiv \prod_{i = 1}^{L} \delta_{n_i m_i},
            \end{align}
            that is, the two states must share the exact same occupied states.

            % TODO: Find a natural way to introduce non-orthogonal spin-orbitals
            % in the Fock state.
            % This solution should be indepedent of bosons and fermions.
            %However, if the basis is non-orthogonal, viz.
            %\begin{align}
            %    \braket{\phi_p}{\phi_q} = S_{pq} \neq \delta_{pq},
            %\end{align}
            %with $S_{pq}$ the overlap integral between the two spin-orbitals.
            %We then have that the overlap between two occupation numbers vectors
            %is given by
            %\begin{align}
            %    \braket{\Psi}{\Phi}
            %    = \delta_{\vf{n}\vf{m}}\det\para{\vf{S}},
            %\end{align}
            %where $\vf{S}$ is the matrix with all the overlap integrals between
            %the spin-orbitals.


        \subsection{Creating and annihilating particles}
            From the occupation number representation introduced by the Fock
            space we introduce the \emph{vacuum state} as the state containing
            no particles.
            \begin{align}
                \ket{\vac} = \ket{\vfg{0}} = \ket{0, 0, \dots, 0},
            \end{align}
            where all positions in the occupation number vector are zero.
            This state is normalized such that
            \begin{align}
                \braket{\vac} = 1.
            \end{align}
            We now introduce \emph{creation} and \emph{annihilation} operators
            which lets us build filled states from the vacuum.
            Our initial definition of the second quantized operators will be
            quite short as we will open up for both fermions and bosons.
            After we have defined the particle type we are working with, we'll
            expand on the definitions.

            \begin{definition}
                \label{def:creation_1}
                For a given basis of normalized single-particle states
                $\brac{\ket{\phi_p}}_{p = 1}^{L}$ we define the creation
                operator by its action on the vacuum state.
                \begin{align}
                    \ccr{p}\ket{\vac}
                    = \ccr{p}\ket{0_1, \dots, 0_L}
                    = \ket{0_1, \dots, 1_p, \dots, 0_L}
                    = \ket{\phi_p},
                \end{align}
                that is, the operator populates the vacuum with a
                single-particle state.
            \end{definition}
            Having defined the creation operator as the operator that populates
            the vacuum with a single-particle state, we look for the reverse
            action, that is, the operator that removes a single-particle state.

            \begin{definition}
                \label{def:annihilation_1}
                For a given basis of normalized single-particle states
                $\brac{\ket{\phi_p}}_{p = 1}^{L}$ we define the annihilation
                operator as the Hermitian conjugate of the creation operator.
                That is,
                \begin{align}
                    \can{p} = \para{\ccr{p}}^{\dagger},
                \end{align}
                where $\can{p}$ is the annihilation operator.
            \end{definition}
            The annihilation operator will remove a single-particle state from
            the wave function we are examining.
            This can be seen from
            \begin{align}
                1 &= \braket{\phi_p}
                = \mel{\phi_p}{\ccr{p}}{\vac}
                = \mel{\vac}{\can{p}}{\phi_p}
                = \braket{\vac}
                = 1.
            \end{align}

            Having defined the creation and annihilation operators as in
            \autoref{def:creation_1} and \autoref{def:annihilation_1} we still
            have not said how we should go about building wave functions of many
            particles.
            The reason for this is that it depends on the particle type we are
            using.
            One of the major advantages of using the second quantization
            formalism is that we can incorporate the symmetry requirement of
            bosons and fermions in the commutation relation between the creation
            and annihilation operators.


        \subsection{Slater determinants}
            As we are working with fermions the total many-body wave function must
            be \emph{antisymmetric} with respect to interchange of
            particles\footnote{We will restrict ourselves to the interchange of a
            full spin-orbital.}. Mathematically the full wave function of $N$
            fermions should satisfy
            \begin{align}
                \Psi(1, \dots, i, \dots, j, \dots, N)
                =
                -\Psi(1, \dots, j, \dots, i, \dots, N),
            \end{align}
            where we for brevity have introduced the notation $i \equiv x_i =
            (\vf{r}_i, \sigma_i)$. A fully antisymmetric, normalized, $N$-body wave
            function is given by the \emph{Slater determinant} built from $N$
            individual spin-orbitals with $N$ generalized coordinates.
            \begin{align}
                \Psi(1, \dots, N)
                &= \frac{1}{\sqrt{N!}}
                \begin{vmatrix}
                    \psi_1(1) & \dots & \psi_N(1) \\
                    \vdots & \ddots & \vdots \\
                    \psi_1(N) & \dots & \psi_N(N)
                \end{vmatrix},
                \label{eq:coord_slater}
            \end{align}
            where $\psi_i(j) \equiv \psi_i(\vf{r}_j, \sigma_j)$ for the $i$'th
            spin-orbital in generalized coordinate $j$ \footnote{Note that the index
            on the spin-orbitals has nothing to do with which spin component is used
            as in the previous section on spin-orbitals.}. Exchanging a row in the
            determinant results in a sign change as we required from the
            antiysmmetric full wave function for the many-body problem. Furthermore,
            exchanging a column in the determinant also results in a sign-change.
            This means that there can not be two, or more, of the same spin-orbitals
            in the wave function perfectly encapsulating the Pauli principle as two
            of the same spin-orbitals translates to two particles being in the same
            state. Generally, the spin-orbitals in a Slater determinant are linearly
            independent, otherwise $\Psi = 0$. This is a consequence of the
            properties of a determinant that adding a scalar multiple of a column to
            another column does not change the determinant.

            The factor in front of the Slater determinant makes the many-body wave
            function normalized to unity as long as the spin-orbitals are
            orthonormalized.

            \begin{lemma}
                Given a Slater determinant $\ket{\Phi}$ built from an
                orthonormal basis $\brac{\ket{\phi_i}}_{i = 1}^{N}$ that spans
                the $N$-dimensional Hilbert space, we can then perform a unitary
                transformation to a new orthonormal basis
                $\brac{\ket{\psi_i}}_{i = 1}^{N}$ building a new Slater
                determinant $\ket{\Psi}$.
                This unitary transformation will then preserve the normalization
                of the original Slater determinant.
                % TODO: Clean-up formulation.
            \end{lemma}

            % TODO: Introduce proof
            \begin{proof}
                \label{proof:slater_determinants_invariant}
                Given a basis $\brac{\ket{\phi_i}}_{i = 1}^{N}$ that spans the
                $N$-dimensional Hilbert space. We can then do a unitary
                transformation from this basis to a new basis
                $\brac{\ket{\psi_i}}_{i = 1}^{N}$ in the same Hilbert space by
                \begin{align}
                    \ket{\psi_i} = U_{ji}\ket{\phi_j},
                    \label{eq:unitary_transformation}
                \end{align}
                where $U_{ji}$ is an element in the unitary matrix $\vfg{U}$.
                For a set of coordinates
                $\brac{x_1, \dots, x_N}$, we write
                \begin{align}
                    \psi_{ij} \equiv \psi_j(x_i)
                    \equiv \braket{x_i}{\psi_j},
                \end{align}
                and equivalently for $\phi_{ij}$. Projecting onto the coordinate
                basis we can write \autoref{eq:unitary_transformation} as
                \begin{gather}
                    \braket{x_k}{\psi_i}
                    = U_{ji}\braket{x_k}{\phi_j}
                    \implies
                    \psi_{ki} = U_{ji}\phi_{kj} = \phi_{kj}U_{ji}
                    \implies \vfg{\Psi} = \vfg{\Phi}\vfg{U},
                \end{gather}
                where the matrices $\vfg{\Psi}$, $\vfg{\Phi}$ and $\vfg{U}$ are
                the matrices with elements
                \begin{align}
                    \vfg{\Psi}
                    &= \begin{pmatrix}
                        \psi_{11} & \psi_{12} & \dots & \psi_{1N} \\
                        \vdots & \vdots & \ddots & \vdots \\
                        \psi_{N1} & \psi_{N2} & \dots & \psi_{NN}
                    \end{pmatrix}, \\
                    \vfg{\Phi}
                    &= \begin{pmatrix}
                        \phi_{11} & \phi_{12} & \dots & \phi_{1N} \\
                        \vdots & \vdots & \ddots & \vdots \\
                        \phi_{N1} & \phi_{N2} & \dots & \phi_{NN}
                    \end{pmatrix}, \\
                    \vfg{U}
                    &= \begin{pmatrix}
                        U_{11} & U_{12} & \dots & U_{1N} \\
                        \vdots & \vdots & \ddots & \vdots \\
                        U_{N1} & U_{N2} & \dots & U_{NN}
                    \end{pmatrix},
                \end{align}
                and where $\vfg{U}^{\dagger}\vfg{U} = \1$.
                Creating the fully antisymmetrized normalized wave function, i.e.,
                the Slater determinant, from $\vfg{\Psi}$ and $\vfg{\Phi}$ we see
                that
                % TODO: Fix description, the determinant of the "Slater matrices"
                % are not necessarily normalized without the pre-factor.
                \begin{gather}
                    \det(\vfg{\Psi}) = \det(\vfg{\Phi}\vfg{U})
                    = \det(\vfg{\Phi})\det(\vfg{U})
                \end{gather}
                We now take the squared norm on both sides.
                \begin{align}
                    \norm{\det(\vfg{\Psi})}^2
                    = \norm{\det(\vfg{\Phi})\det(\vfg{U})}^2
                    \leq
                    \norm{\det(\vfg{\Phi})}^2\norm{\det(\vfg{U})}^2
                    = \norm{\det(\vfg{\Phi})}^2,
                    \label{eq:squared_determinant}
                \end{align}
                where we have used that
                \begin{align}
                    \norm{\det(\vfg{U})}^2
                    &= \norm{\det(\vfg{U^{\dagger}})\det(\vfg{U})}
                    = \norm{\det(\vfg{U}^{\dagger}\vfg{U})}
                    = \norm{\det(\1)} = 1.
                \end{align}
                Now, in \autoref{eq:squared_determinant}, since the Slater
                determinant of $\vfg{\Phi}$ is orthonormalized and a unitary
                transformation preserves the normalization this means that the norm
                of $\det(\vfg{\Phi})$ and the norm of $\det(\vfg{\Psi})$ must be
                unity.  Thus, the equality in the splitting of the norm is preserved
                thus yielding
                \begin{gather}
                    \norm{\det(\vfg{\Psi})}^2 = \norm{\det(\vfg{\Phi})}^2.
                \end{gather}
            \end{proof}

            \subsubsection{Antisymmetrizer}
                To avoid having to write the Slater determinants as in
                \autoref{eq:coord_slater} we introduce the anti-symmetrizer
                $\hat{A}$ which is given by
                \begin{align}
                    \hat{A}
                    \equiv
                    \frac{1}{N!}
                    \sum_{\sigma \in S_N}
                    (-1)^{\abs{\sigma}}
                    \hat{P}_{\sigma},
                \end{align}
                where $\sigma$ is a permutation from the set of all permutations
                $S_N$ of $N$ indices, $\hat{P}_{\sigma}$ is the permutation
                operator permuting particles and $\abs{\sigma}$ is the number of
                transpositions in the permutation.
                There are $N!$ permutations in $S_N$ which is why we include the
                normalization factor.
                A Slater determinant can thus be represented by
                \begin{align}
                    \Phi(\vf{x})
                    &=
                    \sqrt{N!}\hat{A}\phi_1(x_1)\dots\phi_N(x_N)
                    \\
                    &= \frac{1}{\sqrt{N!}}
                    \sum_{\sigma \in S_N} (-1)^{\abs{\sigma}}
                    \phi_1(x_{\sigma(1)})\dots\phi_N(x_{\sigma(N)}).
                \end{align}
                The anti-symmetrizer opens up the possibility of representing a
                Slater determinant in a coordinate-independent manner.
                % TODO: Continue on this topic

            \subsubsection{Operator representation}

        \subsection{Time-dependent second quantized operators}


    \section{Wick's theorem}
        \subsection{Normal ordering}
        \subsection{Generalized Wick's theorem}
        \subsection{Fermi vacuum}
            \label{sec:fermi-vacuum}
            Given a basis of $L$ single particle functions $\ket{p}$ where
            \begin{align}
                \left\{\ket{p}\right\}_{p = 1}^{L}
                &=
                \left\{\ket{i}\right\}_{i = 1}^{N}
                \cup \left\{\ket{a}\right\}_{a = N + 1}^{L}.
            \end{align}
            Here $i, j, k, \dots$ represents the $N$ first occupied states of
            the reference Slater determinant whereas $a, b, c, \dots$ represent
            the remaining $M = L - N$ virtual states in the total basis $p, q,
            r, \dots$\footnote{Occupied and virtual states are also known as
            hole and particle states if we treat the reference Slater
            determinant as the \textit{Fermi vacuum}}.

    \section{Excited determinants}
        \subsection{Overlap with the reference state}
            % TODO: Remember to introduce the concept of the reference state
        \subsection{Overlap when using time-dependent orbitals}

    \section{The many-body Hamiltonian}
        A general many-body Hamiltonian can be described by
        \begin{align}
            \hamil \equiv \sum_{i = 0}^{\infty} \hamil_i,
        \end{align}
        where we denote the $N$-body Hamiltonian by $\hamil_i$.
        From the postulates of quantum mechanics we require that
        $\hamil^{\dagger} = \hamil$, that is, the Hamiltonian should be
        Hermitian.
        This in turn implies that $\hamil_i^{\dagger} = \hamil_i$ for all
        $N$-body terms.
        We can compute the energy, i.e., the expectation value of the
        Hamiltonian, by
        \begin{align}
            \mel{\psi}{\hamil}{\psi}
            &= \sum_{i = 0}^{\infty} \mel{\psi}{\hamil_i}{\psi}
            = \mel{\psi}{\hamil_0}{\psi}
            + \mel{\psi}{\hamil_1}{\psi}
            + \mel{\psi}{\hamil_2}{\psi}
            + \dots,
        \end{align}
        where each term involves the \emph{interaction} between $i$ particles.
        For $i = 0$ we have $\hamil_0$ which we take to signify a constant term,
        e.g., the nuclear charge of an atomic system.
        This term contributes with a constant energy shift in the expectation
        value.
        The one-body term, $\hamil_1$, is the familiar Hamiltonian from
        introductory courses in quantum mechanics.
        This term describes one-particle contribution, e.g., the kinetic
        energy, external potentials of the system, external fields, etc.
        Higher order terms describe interactions between particles in the
        system.
        For example, the Coulomb interaction is a two-body interaction between
        particle pairs and is described by $\hamil_2$.
        More exotic higher order interactions can occur in the nuclear physics
        and the sort.
        However, it is common to truncate the interactions in the many-body
        Hamiltonian as high order terms leads to complicated equations which
        become intractable unless clever approximations are introduced.

        Luckily, much of solid-state physiscs, and almost all of quantum
        chemistry, can be described by the Coulomb interaction as the system are
        mainly governed by the electrons.
        This means that we will truncate the many-body Hamiltonian to
        second-order interactions.
        Furthermore, we ignore the constant term in the Hamiltonian as this can
        always be added when we compute the energy and will not contribute when
        we look for a solution of the many-body problem.
        We will denote the general electronic Hamiltonian by
        \begin{align}
            \hamil = \onehamil + \twohamil,
        \end{align}
        where $\onehamil$ is the one-body term and $\twohamil$ describes
        two-body interactions, which in our case is the Coulomb interaction.

        \subsection{One-body Hamiltonian}
            The one-body Hamiltonian describes the non-interacting system we are
            examining.
            In the time-independent case we typically describe the system by a
            sum of kinetic and potential terms.
            \begin{align}
                \onehamil
                &= \sum_{i = 1}^{N}\onehamil_i
                = \sum_{i = 1}^{N}\para{
                    \kinetic_i + \potential_i
                }
                = \sum_{i = 1}^{N}\para{
                    \frac{\momentumvec^2_i}{2m_i} + \potential_i
                },
            \end{align}
            where the sum runs over all particles in the system separately.
            The potential term can come from many different sources, e.g., an
            external field, mean-field approximations, the electron-nucleon
            interaction, etc.
            We will discsuss these more in depth when we start looking at
            specific systems.
            % TODO: Perhaps include the grid representation of the operators?
            Often when solving the many-body problem we seek a solution to the
            one-body Hamiltonian in terms of a reference Slater determinant.
            This will in many cases provide us with a good starting guess for
            the system before using more sophisticated methods such as the
            Hartree-Fock method to improve on our reference state.

            \subsubsection{Time-dependency}
                In general the time-dependency can occur in all the terms in the
                full many-body Hamiltonian, but the most common scenario is an
                external interaction occuring in the one-body Hamiltonian, e.g.,
                a laser field, time-varying potential, etc.
                In this thesis we will only concern ourselves with an external
                laser field.
                We can then write the time-dependent one-body Hamiltonian on the
                form
                \begin{align}
                    \onehamil(t)
                    &= \sum_{i = 1}^{N} \onehamil_i(t)
                    = \sum_{i = 1}^{N}\para{
                        \kinetic_i
                        + \potential_i
                        + \laserfield_i(t)
                    },
                \end{align}
                where the time-dependence is represented by the time-dependent
                operator $\laserfield_i(t)$.


        \subsection{Two-body Hamiltonian}
            The two-body Hamiltonian provides interactions between pairs of
            particles.
            This complicates the matter by introducing a double sum over all
            particles to include all pairs.
            \begin{align}
                \twohamil
                = \sum_{i = 1}^{N} \sum_{j = 1}^{N} \twohamil_{ij}.
            \end{align}
            Often some care must be taken when $i = j$ as this results in
            self-interaction which can yield un-physical results, but we have
            for the sake of generality written the full sum.
            % TODO: Verify this.
            % Remember that the overlap occurs in the integral.
            The two-body operator is often the bottle-neck in terms of
            complexity for the electronic Hamiltonian.

            \subsubsection{Coulomb interaction}
                We will in this thesis concern ourselves with electronic
                interactions and will therefore limit ourselves to the Coulomb
                interaction as our two-body operator.
                This operator is given by
                \begin{align}
                    \twohamil
                    &= \sum_{i = 1}^{N}\sum_{j = i + 1}^{N}
                    \twohamil_{ij}
                    = \sum_{i = 1}^{N}\sum_{j = i + 1}^{N}
                    \frac{1}{4\epsilon_0\pi}\frac{e^2}{\abs{\positionvec_i -
                    \positionvec_j}},
                \end{align}
                where we note that the unconstrained double-sum has been
                replaced by two sums where $j > i$.
                This removes the singularity arising from self-interaction and
                exploits the symmetry of the Coulomb interaction as
                $\twohamil_{ij} = \twohamil_{ji}$.
                The Coulomb interaction is a quantized version of the Coulomb
                potential with $\epsilon_0$ representing the vacuum
                permittivity and $e^2$ the charge of the particles, which we
                assume to be of the same type.
                % TODO: Expand on the physical interpretation of Coulomb's law
                % and the quantized version.

        \subsection{The Born-Oppenheimer approximation}
            The Born-Oppenheimer approximation first occured in a seminal paper
            by \citeauthor{born1927quantentheorie}
            \cite{born1927quantentheorie}.
            It is procedure describing how we can solve a molecular system
            consisting of electrons and nucleons where we assume that the total
            wave function can be separated into an electronic and a nuclear
            part, viz.
            \begin{align}
                \ket{\Psi} = \ket{\Psi_e}\otimes\ket{\Psi_n},
            \end{align}
            where we denote the electronic wave function by $\ket{\Psi_e}$ and
            the nuclear counterpart as $\ket{\Psi_n}$.
            % TODO: This might not be relevant unless we include atomic systems.

        \subsection{Second quantized formulation}
            Having introduced the second quantized formulation of many-body
            quantum mechanics we wish to represent the Hamiltonian in this
            formalism.
            This introduces a generalization which will prove very useful for
            the later many-body methods as it removes the explicit grid
            representation of the operators in introduces abstract matrix
            elements as coefficients to the second quantized operators.

            The expectation value of the Hamiltonian in terms of a Slater
            determinant is then
            \begin{align}
                \bra{\Phi}\hat{H}\ket{\Phi}
                &= \bra{\phi_i}\hat{h}\ket{\phi_i}
                + \frac{1}{2}\bra{\phi_i\phi_j}\hat{u}\ket{\phi_i\phi_j}_{AS},
            \end{align}
            where the antisymmetric matrix elements are given by
            \begin{align}
                \bra{\phi_i\phi_j}\hat{u}\ket{\phi_i\phi_j}_{AS}
                &= \bra{\phi_i\phi_j}\hat{u}\ket{\phi_i\phi_j}
                - \bra{\phi_i\phi_j}\hat{u}\ket{\phi_j\phi_i}.
                \label{eq:antisymmetric_two_body}
            \end{align}

    \section{The reference energy}
        Given a basis of orthonormal single-particle state
        $\brac{\ket{\phi_p}}_{p = 1}^{L}$ and constructing a Slater determinant
        from the $N$ first states, viz.
        \begin{align}
            \ket{\refslat} = \ket{\phi_1, \dots, \phi_N},
        \end{align}
        that is, we have $N$ occupied states and $M = L - N$ virtual states
        using the same convention for the indices as discussed in
        \autoref{sec:fermi-vacuum} on the Fermi vacuum.
        We dub this state the \emph{reference state} for reasons which will
        become clear when we start working on the many-body methods.
        For a general electronic Hamiltonian with one- and two-body operators,
        we can compute the expectation value of the energy from the reference
        state.
        \begin{align}
            \bra{\refslat}\hamil\ket{\refslat}
            &=
            \bra{\refslat}\onehamil\ket{\refslat}
            + \bra{\refslat}\twohamil\ket{\refslat}
            \\
            &=
            \oneten^{p}_{q}
            \bra{\refslat}\ccr{p}\can{q}\ket{\refslat}
            + \frac{1}{4}
            \twoten^{pq}_{rs}
            \bra{\refslat}\ccr{p}\ccr{q}\can{s}\can{r}\ket{\refslat},
        \end{align}
        where we use the anti-symmetric two-body elements.
        Using Wick's theorem\footnote{%
            Or the anti-commutation rules manually.
        }%
        we can evaluate the overlap strings on the reference state.
        This yields the \emph{reference energy} given by
        \begin{align}
            \energyref
            &\equiv
            \bra{\refslat}\hamil\ket{\refslat}
            =
            \oneten^{i}_{i}
            + \frac{1}{2}\twoten^{ij}_{ij}.
            \label{eq:reference-energy}
        \end{align}
        The derivation of this expression can be found in
        \autoref{sec:deriving-the-reference-energy}.

    \section{The normal-ordered Hamiltonian}
        A general electronic Hamiltonian with one- and two-body operators on
        a second-quantized form is given by
        \begin{align}
            \hamil
            &= \oneten^{p}_{q}\ccr{p}\can{q}
            + \frac{1}{4}\twoten^{pq}_{rs}\ccr{p}\ccr{q}\can{s}\can{r},
        \end{align}
        where we've labelled the anti-symmetric two-body elements by
        $u^{pq}_{rs}$, and Einstein summation is assumed.
        We can write the electronic Hamiltonian on a normal ordered form using
        Wick's theorem.



    \section{Many-body density matrices}
        In a seminal paper by \citeauthor{lowdin-density-matrices}
        \cite{lowdin-density-matrices}, the concept of a many-body density
        matrix in terms of the orbitals of a Slater determinant is discussed.
        These are dubbed $N$-body density matrices, where $N$ depends on the
        $N$-body interaction, that is, the number of particles included in the
        interaction.
        Consider a Hermitian operator $\hat{Q}$ which can be represented as a
        sum of $N$-body operators
        \begin{align}
            \hat{Q}
            &\equiv
            \sum_{i = 0}^{N} \hat{Q}_i
            = Q_0\1 + Q^{p}_{q}\ccr{p}\can{q}
            + \frac{1}{2!}Q^{pq}_{rs}\ccr{p}\ccr{q}\can{s}\can{r}
            + \dots,
        \end{align}
        where each $N$-body operator is symmetrical in its indices, and includes
        a $(N!)^{-1}$ factor to account for double counting.
        For a normalized, pure state $\ket{\psi}$, we are able to compute the
        expectation value of the operator $\hat{Q}$ by
        \begin{align}
            \expv{Q} &= \mel{\psi}{\hat{Q}}{\psi}
            = \tr(\density \hat{Q}),
        \end{align}
        where we've introduced the density operator $\ket{\psi}$ by
        \begin{align}
            \density = \dyad{\psi}{\psi}.
        \end{align}
        Now, if $\ket{\psi}$ is a wave function describing a system of $N$
        particles, we can write the expectation value of the operator $\hat{Q}$
        as
        \begin{align}
            \expv{Q}
            &= \tr(\density \hat{Q})
            = \braket{\psi}\mel{\psi}{\hat{Q}}{\psi}
            = \sum_{i = 0}^{N} \mel{\psi}{\hat{Q}_i}{\psi}
            \\
            &=
            Q_0 \mel{\psi}{\1}{\psi}
            + Q^{p}_{q}\mel{\psi}{\ccr{p}\can{q}}{\psi}
            + \frac{1}{2!}
            Q^{pq}_{rs}
            \mel{\psi}{\ccr{p}\ccr{q}\can{s}\can{r}}{\psi}
            + \dots
            \\
            &= Q_0 + Q^{p}_{q}\densityten^{q}_{p}
            + \frac{1}{2!} Q^{pq}_{rs} \densityten^{rs}_{pq}
            + \dots,
        \end{align}
        where we ask the reader to direct special attention to the index
        ordering of the $N$-body density matrices $\densityten$ as the
        ``opposite'' direction of the index ordering from the $N$-body matrix
        elements.

        %Of the most useful for our work, we have the one- and two-body density
        %matrices.
        %As we will work almost exclusively in second quantization, we will
        %follow the derivation of the one- and two-body density matrices done
        %by \citeauthor{helgaker-molecular} \cite{helgaker-molecular}.
        %Löwdin's paper \cite{lowdin-density-matrices} did not employ second
        %quantization, and all matrices are expressed in the coordinate
        %representation.
        %We will list these as they arrive.

        \subsection{One-body density matrix}
            \subsubsection{Particle density}
        \subsection{Two-body density matrix}

    \section{Particle density}

    \section{Dipole moments}
        \subsection{Transitions}

    \section{The Harmonic potential theorem}
