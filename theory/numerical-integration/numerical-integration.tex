\chapter{Numerical Integration}
    In this chapther we'll review a select few time-integration schemes for
    solving time-dependent ordinary differential equations and we'll discuss the
    applicability of each scheme.

    \section{Problem statement}
    \section{Runge-Kutta}
    \section{Preservation of unitarity and conservation of energy}
    \section{Crank-Nicolson}
        Continuing the derivation from section 4.5 in
        \citeauthor{ullrich2011time} \cite{ullrich2011time}, we look at how we
        can numerically propagate in time a wave function with a time-dependent
        Hamiltonian.
        Luckily, due to the property of the time evolution operators that they
        can be composed of several intermediate time steps, discretizing the
        operators becomes quite natural.
        That is, if we define $\tau_1 = t_0$, and $\tau_N = t$ with all
        intermediate times given by
        \begin{align}
            \tau_{j + 1} = \tau_j + \Delta \tau,
        \end{align}
        where the time step $\Delta \tau$ is given by
        \begin{align}
            \Delta \tau = \frac{t - t_0}{N - 1},
        \end{align}
        and $N$ is the number of discrete time steps. This means that we can
        write
        \begin{align}
            \ket{\Psi(\tau_{j + 1})}
            = \hat{U}(\tau_{j + 1}, \tau_{j})\ket{\Psi(\tau_j)},
        \end{align}
        where the equality is maintained. For a time-dependent Hamiltonian
        (which is what we will be working on), the time evolution operator is
        given by \autoref{eq:td-evolution}, but this is intractable in a
        numerical scheme, to say the least. We therefore approximate the
        operator by
        \begin{align}
            \hat{U}(\tau_{j + 1}, \tau_{j})
            \approx \exp\brac{
                -\frac{i}{\hslash}\hat{H}(\tau_{j} + \Delta\tau/2)\Delta\tau
            }
            \equiv
            \exp\brac{
                -\frac{i}{\hslash}\hat{H}(\tau_{j + 1/2})\Delta\tau
            },
        \end{align}
        that is, we evaluate the Hamiltonian at the midpoint between $\tau_{j +
        1}$ and $\tau_{j}$. The choice of evaluating the time step at the
        midpoint comes about in order to preserve the unitarity of the
        approximated time evolution operator. That is,
        \begin{align}
            \ket{\Psi(\tau_j)}
            = \hat{U}(\tau_j, \tau_{j + 1})\ket{\Psi(\tau_{j + 1})}
            = \hat{U}(\tau_j, \tau_{j + 1})\hat{U}(\tau_{j + 1}, \tau_j)
            \ket{\Psi(\tau_j)},
        \end{align}
        which means that
        \begin{align}
            \hat{U}(\tau_j, \tau_{j + 1})
            \hat{U}(\tau_{j + 1}, \tau_j)
            = \hat{U}^{\dagger}(\tau_{j + 1}, \tau_j)
            \hat{U}(\tau_{j + 1}, \tau_j)
            = \1.
        \end{align}
        Using the Crank-Nicolson algorithm, we can approximate the exponential
        in the approximated time evolution operator to be
        \begin{align}
            \hat{U}(\tau_{j + 1}, \tau_j)
            \approx \exp\brac{
                -\frac{i}{\hslash}\hat{H}(\tau_{j + 1/2})\Delta \tau
            }
            \approx
            \frac{1 - i\hat{H}(\tau_{j + 1/2})\Delta\tau/(2\hslash)}
            {1 + i\hat{H}(\tau_{j + 1/2})\Delta\tau/(2\hslash)},
        \end{align}
        which is correct to second order in $\Delta \tau$ and preserves
        unitarity. Expressed as system of linear equations, we then have
        \begin{align}
            \para{
                1 + \frac{i}{2\hslash}\hat{H}(\tau_{j + 1/2})\Delta\tau
            }\ket{\Psi(\tau_{j + 1})}
            =
            \para{
                1 - \frac{i}{2\hslash}\hat{H}(\tau_{j + 1/2})\Delta\tau
            }\ket{\Psi(\tau_{j})}.
        \end{align}
        We can either solve this equation by inversion, or as a system of linear
        equations. The latter will be more stable.

    \section{Gauss-Legendre}
