\chapter{Configuration interaction}
    \label{chap:ci}
    The most natural approach to take in quantum mechanics when creating a wave
    function is to create a linear combination of all possible states contained
    in the space one is exploring.
    In configuration interaction theory the many-body wave function is written
    as a linear combination of all possible Slater determinants in the basis of
    spin-orbitals,
    \begin{align}
        \ket*{\Psi}
        &= C\ket*{\slat}
        + C^a_i\ket*{\slat^{a}_{i}}
        + \frac{1}{4}C^{ab}_{ij}\ket*{\slat^{ab}_{ij}}
        + \dots.
        \label{eq:ci_wavefunction}
    \end{align}
    The factor $4$ in the doubles sum is included to avoid overcounting as both
    the coefficients and the excited determinants are antisymmetric.
    By including all excited Slater determinants of $N$-particles, we get a
    method which is exact within the given space of single-particle states.
    When this is the case we call the method \emph{full configuration
    interaction} (FCI) theory.

    There is however a significant catch to the configuration interaction
    method, and that is its computational scaling.
    The number of Slater determinants $N_{s}$ for a given basis of $L$
    spin-orbitals with $N$ occupied particles will grow as \cite{kvaal2017notes}
    \begin{align}
        N_{s} = \binom{L}{N}.
    \end{align}
    This is such a significant roadblock that the method very quickly becomes
    completely infeasible for systems of interest.
    Significant attempts at lowering the cost of configuration interaction has
    been explored, but the factorial scaling quickly becomes a bottleneck for
    large systems.

    A word on notation, we will in the following refrain from using explicit
    excitation indices $a, b, \dots$ and $i, j, \dots$ when labelling excited
    Slater determinants, but rather label the coefficients and the Slater
    determinants by capital letters $I, J, K, \dots$.
    That is, we can write \autoref{eq:ci_wavefunction} on the short form
    \begin{align}
        \ket*{\Psi}
        &= C_{I}\ket*{\Phi_I}
        = C_{I}\hat{X}_I\ket*{\slat}.
    \end{align}
    The capital indices will then run over the total number of Slater
    determinants $N_s$ in the full Slater determinant basis.

    \section{Time-independent configuration interaction}
        We start with the time-independent Schrödinger equation
        \begin{align}
            \hamil\ket*{\Psi_J} = \energy_J\ket*{\Psi_J},
            \label{eq:ci_tise}
        \end{align}
        where $(E_J, \Psi_J)$ is an eigenpair of $\hamil$.
        Expanding the CI wavefunction in a Slater determinant basis,
        \begin{align}
            \ket*{\Psi_J} = C_{KJ}\ket*{\Phi_K},
            \label{eq:expanded_ci_wavefunction}
        \end{align}
        where $C_{KJ}$ are the amplitudes for a certain excitation $K$ for a
        specific energy level $J$.
        Inserting \autoref{eq:expanded_ci_wavefunction} into
        \autoref{eq:ci_tise} and left projecting on a state $\ket*{\Phi_I}$ we
        get
        \begin{align}
            \mel*{\Phi_I}{\hamil}{\Phi_K}C_{KJ}
            = E_J\braket*{\Phi_I}{\Phi_K}C_{KJ}.
        \end{align}
        We denote the matrix elements of the Hamiltonian by
        \begin{align}
            \hamilten_{IK}
            = \mel*{\Phi_I}{\hamil}{\Phi_J}.
        \end{align}
        The Hamiltonian matrix will have a dimensionality of $\hamilmat \in
        \mathbb{C}^{N_s \times N_s}$.
        If the underlying basis of spin-orbitals is non-orthogonal, the overlap
        between the Slater determinants will be given by the determinant of the
        overlap integrals between the occupied spin-orbitals in both
        determinants.
        We denote the overlap between the determinants by
        \begin{align}
            S_{IK} = \braket*{\Phi_I}{\Phi_K}.
        \end{align}
        The overlap matrix will have the same dimensionality as the Hamiltonian,
        that is, $\overlapmat \in \mathbb{C}^{N_s \times N_s}$.
        We can thus formulate the generalized eigenvalue equation
        \begin{gather}
            \hamilten_{IK}C_{KJ} = \energy_J S_{IK}C_{KJ}
            \\
            \implies
            \hamilmat \vfg{C} = \vfg{\energy} \overlapmat \vfg{C},
            \label{eq:ci-generalized}
        \end{gather}
        where $\vfg{\energy} = \diag(E_1, \dots, E_{N_s})$.
        We recognize this generalized eigenvalue equation as the same equation
        found in the variational method for a trial wave function expanded as a
        linear combination of an underlying basis shown in
        \autoref{subsec:variational-method}.
        Solving this eigenvalue equation corresponds to a variational minization
        procedure with \autoref{eq:ci-generalized} as the stationary condition.
        In fact, the configuration interaction method is a direct implementation
        of the variational method using a linear combination of Slater
        determinants.

        Now, to avoid the cost of storing and computing the overlap matrix
        $\overlapmat$ we restrict ourselves to an orthonormal basis of
        single-particle functions which means that $S_{IK} = \delta_{IK}$.
        If the original atomic orbital basis is non-orthogonal, we can make the
        basis orthonormal by either transforming to the Hartree-Fock basis or by
        performing an orthogonalization procedure, for example, symmetric
        orthogonalization \cite{mayer2002lowdin, szabo1996modern}.
        Thus, in the remainder of this chapter, we restrict ourselves to
        orthonormal Slater determinants.
        The generalized eigenvalue equation therefore reduces to the eigenvalue
        equation
        \begin{align}
            \hamilmat\vfg{C} = \vfg{\energy}\vfg{C}.
        \end{align}
        The configuration interaction method can thus be summarized as follows:
        construct the Hamiltonian matrix $\hamilmat$ from the matrix elements
        $\hamilten_{IJ}$, then diagonalize the matrix to get the coefficients of
        the eigenstates $\vfg{C}_I$ and the eigenenergies $E_I$
        \cite{karwowski}.
        As the diagonalization yields the full spectrum, configuration
        interaction is a method which yields higher-order states unlike methods
        such as the coupled-cluster method which requires extra steps in order
        to get anything but the ground state.

        \subsection{Truncated configuration interaction}
            \label{sub:truncated-configuration-interaction}
            As discussed in the beginning of this chapter, the full
            configuration interaction method quickly encounters exponential
            scaling and becomes intractable.
            A common technique utilized to lower the cost of the method is to
            only include Slater determinants based on their excitation.
            For example, by only including the doubly excited Slater
            determinants, we can construct the configuration interaction doubles
            (CID) wave function by
            \begin{align}
                \ket*{\Psi} = C\ket*{\slat}
                + \frac{1}{4} C^{ab}_{ij} \ket*{\slat^{ab}_{ij}},
            \end{align}
            which reduces the number of Slater determinants to
            \begin{align}
                N_s = 1 + \left\lfloor\frac{N(N - 1)(L - N)(L - N -
                1)}{4}\right\rfloor,
            \end{align}
            where the division is floored to the nearest integer.
            Defining $M = L - N$ as the number of virtual spin-orbitals, the
            scaling of the number of Slater determinants in the configuration
            interaction doubles method is given by
            \begin{align}
                N_s = \mathcal{O}(N^2 M^2),
            \end{align}
            which is a significant decrease in computational complexity.
            However, due to the truncation of the wave function not only do we
            decrease the quality of the method in terms of closeness to the
            exact energy solution given by the full configuration interaction
            method, we also lose size-extensivity and size-consistency
            \cite{crawford2000introduction, helgaker-molecular}.

            %\subsubsection{The singles approximation on a small basis}
            %    As an example of <size extensivity/size consistency> we demonstrate
            %    the troubles arising when we restrict ourselves to a system of two
            %    particles and four basis functions.
            %    Our basis thus consists of the spin-orbitals $\brac{\ket*{\phi_p},
            %    }_{p = 1}^{4}$, with the reference Slater determinant
            %    \begin{align}
            %        \ket*{\Phi}
            %        &= \frac{1}{\sqrt{2}}\brac{
            %            \ket*{\phi_1\phi_2}
            %            - \ket*{\phi_2\phi_1}
            %        }.
            %        % TODO: It might not be necessary to show what the determinant
            %        % looks like.
            %    \end{align}
            %    Using truncated configuration interaction with singles excitations
            %    only, we get the many-body wave function
            %    \begin{align}
            %        \ket*{\Psi}
            %        &= C\ket*{\Phi}
            %        + C^{a}_{i}\ket*{\Phi^{a}_{i}}
            %        = C\ket*{\Phi}
            %        + C^{3}_{1}\ket*{\Phi^{3}_{1}}
            %        + C^{4}_{1}\ket*{\Phi^{4}_{1}}
            %        + C^{3}_{2}\ket*{\Phi^{3}_{2}}
            %        + C^{4}_{2}\ket*{\Phi^{4}_{2}}.
            %        \label{eq:cis_wave_function}
            %    \end{align}
            %    Graphically we can represent the full space of the Slater determinants
            %    by \autoref{fig:tiny-slater-basis}.
            %    Comparing with \autoref{eq:cis_wave_function} we see that the only
            %    determinant missing in the truncated wave function is the doubly
            %    excited determinant $\ket*{\Phi^{34}_{12}}$.
            %    Stated a little differently, truncated configuration interaction is
            %    not able to exploit the full space of Slater determinants.
            %    The method also fails in ``removing all trace`` of the reference
            %    state.
            %    This means that truncated configuration interaction will not be able
            %    to fully excite the reference state.
            %    Note that the example shown in \autoref{eq:cis_wave_function} and
            %    \autoref{fig:tiny-slater-basis} extends for more particles and
            %    higher truncation levels for configuration interaction.
            %    For example for a system with four particles and truncated
            %    configuration interaction with singles and doubles excitations this
            %    same behaviour is exhibited.
            %    % TODO: Explain how this problem becomes worse for more basis
            %    % functions.
            %    % TODO: Explain how coupled cluster fixes this problem in the
            %    % chapter on coupled cluster.
            %    % TODO: Explain size consistency and size extensivity. See Crawford
            %    % & Schaefer pages 42 and 95.
            %    \begin{figure}
            %        \begin{center}
            %            \begin{tikzpicture}
            %                % State 1
            %                \begin{scope}
            %                    \foreach \i in {1,...,4} {
            %                        \draw (-1, \i - 1) node[anchor=east]
            %                        {$\ket*{\phi_{\i}}$} -- (1, \i - 1);
            %                    }
            %                    \filldraw (0,4.2) node[anchor=north]
            %                        {$\ket*{\Phi}$};
            %                    \filldraw (0, 0) circle (0.25cm);
            %                    \filldraw (0, 1) circle (0.25cm);
            %                \end{scope}

            %                % State 2
            %                \begin{scope}[xshift=5cm]
            %                    \foreach \i in {1,...,4} {
            %                        \draw (-1, \i - 1) node[anchor=east]
            %                        {$\ket*{\phi_{\i}}$} -- (1, \i - 1);
            %                    }
            %                    \filldraw (0,4.2) node[anchor=north]
            %                        {$\ket*{\Phi^{3}_{1}}$};
            %                    \filldraw (0, 2) circle (0.25cm);
            %                    \filldraw (0, 1) circle (0.25cm);
            %                \end{scope}

            %                % State 3
            %                \begin{scope}[xshift=10cm]
            %                    \foreach \i in {1,...,4} {
            %                        \draw (-1, \i - 1) node[anchor=east]
            %                        {$\ket*{\phi_{\i}}$} -- (1, \i - 1);
            %                    }
            %                    \filldraw (0,4.2) node[anchor=north]
            %                        {$\ket*{\Phi^{4}_{1}}$};
            %                    \filldraw (0, 3) circle (0.25cm);
            %                    \filldraw (0, 1) circle (0.25cm);
            %                \end{scope}

            %                % State 4
            %                \begin{scope}[yshift=-5.5cm]
            %                    \foreach \i in {1,...,4} {
            %                        \draw (-1, \i - 1) node[anchor=east]
            %                        {$\ket*{\phi_{\i}}$} -- (1, \i - 1);
            %                    }
            %                    \filldraw (0,4.2) node[anchor=north]
            %                        {$\ket*{\Phi^{3}_{2}}$};
            %                    \filldraw (0, 0) circle (0.25cm);
            %                    \filldraw (0, 2) circle (0.25cm);
            %                \end{scope}

            %                % State 5
            %                \begin{scope}[yshift=-5.5cm, xshift=5cm]
            %                    \foreach \i in {1,...,4} {
            %                        \draw (-1, \i - 1) node[anchor=east]
            %                        {$\ket*{\phi_{\i}}$} -- (1, \i - 1);
            %                    }
            %                    \filldraw (0,4.2) node[anchor=north]
            %                        {$\ket*{\Phi^{4}_{2}}$};
            %                    \filldraw (0, 0) circle (0.25cm);
            %                    \filldraw (0, 3) circle (0.25cm);
            %                \end{scope}

            %                % State 6
            %                \begin{scope}[yshift=-5.5cm, xshift=10cm]
            %                    \foreach \i in {1,...,4} {
            %                        \draw (-1, \i - 1) node[anchor=east]
            %                        {$\ket*{\phi_{\i}}$} -- (1, \i - 1);
            %                    }
            %                    \filldraw (0,4.2) node[anchor=north]
            %                        {$\ket*{\Phi^{34}_{12}}$};
            %                    \filldraw (0, 3) circle (0.25cm);
            %                    \filldraw (0, 2) circle (0.25cm);
            %                \end{scope}
            %            \end{tikzpicture}
            %        \end{center}
            %        \caption{In this figure we can see the six possible Slater
            %        determinants, i.e., all the basis Slater determinants, made from
            %        the four basis functions $\brac{\ket*{\phi_1}, \ket*{\phi_2},
            %        \ket*{\phi_3}, \ket*{\phi_4}}$ with two particles.}
            %        \label{fig:tiny-slater-basis}
            %    \end{figure}

        \subsection{Size-consistency}
            \label{subsec:size-consistency}
            The concept of size-consistency is defined as
            \cite{pople-size-consistency}
            \begin{align}
                E(AB) = E(A) + E(B),
                \label{eq:size-consistency}
            \end{align}
            where $A$ and $B$ are two systems which do not interact with one
            another.\footnote{%
                Note that the particles in system $A$ and system $B$ internally
                can include interations.
                There is just no interaction between the particles in system $A$
                with particles in system $B$, and vice versa.
            }
            This tells us that the sum of the energy of the two constituent
            parts should equal the total energy of the combined system.
            A similar concept is size-extensivity \cite{shavitt2009many}, which
            is often a consequence of size-consistency.
            In fact these two concepts tend to be discussed interchangeably
            \cite{size-extensivity, helgaker-molecular}.
            We will limit our attention to the case of size-consistency.
            We can then write the idealized compound Hamiltonian as
            \begin{align}
                \hamil_{AB}
                = \hamil_A + \hamil_B,
                \label{eq:additively-separable}
            \end{align}
            where it is understood that the second quantized operators in the
            two subsystems anticommute, that is,
            \begin{align}
                \acom{\can{{p_A}}}{\ccr{{q_B}}} = 0,
            \end{align}
            and that the Hamiltonian of a given subsystem only contain operators
            for that system.
            We can now formulate \autoref{eq:size-consistency} from the
            expectation value of the Hamiltonian by
            \begin{align}
                \mel*{\Psi^{AB}}{\hamil_{AB}}{\Psi^{AB}}
                = \mel*{\Psi^{A}}{\hamil_{A}}{\Psi^{A}}
                + \mel*{\Psi^{B}}{\hamil_{B}}{\Psi^{B}},
            \end{align}
            where we assume that both the compound wave function
            $\ket*{\Psi^{AB}}$ and the wave functions of each subsystem
            $\ket*{\Psi^{A}}$ and $\ket*{\Psi^{B}}$ are normalized.
            Furthermore, we assume that the energy of the specific systems is
            given by
            \begin{align}
                E(A) = \mel*{\Psi^{A}}{\hamil_{A}}{\Psi^{A}},
            \end{align}
            for both the systems.
            We can thus formulate the requirements necessary for a
            size-consistent method: the energy must be \emph{additively
            separable}, viz. \autoref{eq:size-consistency}; the wave function
            must be \emph{multiplicately separable} \cite{helgaker-molecular}.
            The latter requirement is formulated by
            \begin{align}
                \ket*{\Psi^{AB}}
                = \ket*{\Psi^{A}}\ket*{\Psi^{B}},
                \label{eq:multiplicately-separable}
            \end{align}
            where the product of the two states is understood as each state
            acting on a subsystem.
            This last requirement is enough to demonstrate that truncated
            configuration interaction is \emph{not size-consistent}
            \cite{helgaker-molecular}.
            Consider the truncated doubles wave function
            \begin{align}
                \ket*{\Psi} = \para{\1 + \hat{C}}\ket*{\slat},
            \end{align}
            where we've assumed intermediate normalization of the wave function
            and we've denoted the doubles operator by
            \begin{align}
                \hat{C} \equiv \frac{1}{4}C^{ab}_{ij}\hat{X}^{ab}_{ij},
            \end{align}
            for the sake of brevity.
            Now, looking at the right-hand side of
            \autoref{eq:multiplicately-separable} we get
            \begin{align}
                \ket*{\Psi^{A}}\ket*{\Psi^{B}}
                &= \para{\1^{A} + \hat{C}^{A}}\ket*{\slat^A}
                \para{\1^{B} + \hat{C}^{B}}\ket*{\slat^B}
                \\
                &=
                \ket*{\slat^{A}}\ket*{\slat^{B}}
                \pm \ket*{\slat^{A}}\hat{C}^{B}\ket*{\slat^{B}}
                + \hat{C}^{A}\ket*{\slat^{A}}\ket*{\slat^{B}}
                \nonumber \\
                &\qquad
                \pm \hat{C}^{A}\ket*{\slat^{A}}\hat{C}^{B}\ket*{\slat^{B}},
            \end{align}
            where we repeat that the second quantized operators of the two
            subsystems anticommute and we've included a possible sign change due
            to an odd number of particles in subsystem $A$ and hence an odd
            number of creation operators in this determinant.
            What is important to note in the last equation is that the last term
            consists of a \emph{quadruply} excited state as both determinants in
            each subsystem are excited at the same time.
            For the left-hand side of \autoref{eq:multiplicately-separable} we
            have that the doubles wave function for the combined subsystem is
            restricted to
            \begin{align}
                \ket*{\Psi^{AB}}
                &= \para{\1^{AB} + \hat{C}^{AB}}\ket*{\slat^{AB}}
                = \ket*{\slat^{A}}\ket*{\slat^{B}}
                + \para{\hat{C}^{A} + \hat{C}^{B}}
                \ket*{\slat^{A}}\ket*{\slat^{B}}
                \\
                &=
                \ket*{\slat^{A}}\ket*{\slat^{B}}
                + \hat{C}^{A}\ket*{\slat^{A}}\ket*{\slat^{B}}
                \pm \ket*{\slat^{A}}\hat{C}^{B}\ket*{\slat^{B}},
            \end{align}
            where we've used that the compound doubles excitation operator can
            at most be a sum of doubles operators for each subsystem.
            We note that the left-hand side of
            \autoref{eq:multiplicately-separable} lacks the expected quadruples
            excitation coming from the doubles excitation of both subsystems at
            the same time.
            This situation extends beyond the doubles excitations
            \cite{size-extensivity}, and we conclude that truncated
            configuration interaction is not size-consistent.
            However, if we use a full configuration interaction ansatz
            corresponding to a full excitation of the combined system, we then
            recover size consistency as both sides of
            \autoref{eq:multiplicately-separable} will be fully excited.

    \section{Time-dependent configuration interaction}
        Starting from the time-dependent Schrödinger equation, we can formulate
        the time evolution of the configuration interaction wave function as
        \begin{align}
            i\hslash \dod[]{}{t}\ket*{\Psi(t)}
            = \hamil(t)\ket*{\Psi(t)},
            \label{eq:ci_tdse}
        \end{align}
        Here the time-dependent wave function $\ket*{\Psi(t)}$ is expanded as a
        linear combination of a finite number of Slater determinants
        \begin{align}
            \ket*{\Psi(t)} = c_{I}(t) \ket*{\slat_I},
            \label{eq:ci_td_wave}
        \end{align}
        where the orbitals in the Slater determinants are time-independent.
        Our choice of initial state $\ket*{\Psi(0)}$ is to a large degree
        arbitrary as long as the coefficients are normalized and the
        determinants span the entire space we are looking at.
        For example, we can choose a single Slater determinant from our basis of
        determinants as an initial guess.
        In this thesis we will choose $\ket*{\Psi(0)} = \ket*{\Psi_J}$, where
        $\ket*{\Psi_J}$ is an eigenstate from the time-independent Schrödinger
        equation in \autoref{eq:ci_tise}.
        Specifically we will choose the ground state as our initial state, that
        is, $J = 0$.
        This corresponds to the first column in the coefficient matrix
        $\vfg{C} \in \mathbb{C}^{N_s \times N_s}$ in the diagonalization of the
        Hamiltonian matrix.
        The coefficient vector is then $\vfg{c} \in \mathbb{C}^{N_s}$.

        Inserting \autoref{eq:ci_td_wave} into the time-dependent Schrödinger
        equation we get an equation for the time evolution of the coefficients,
        \begin{align}
            i\hslash \dod[]{}{t}c_{J}(t)\ket*{\slat_J}
            = \hamil(t) c_{J}(t)\ket*{\slat_J}.
        \end{align}
        Left-projecting with a Slater determinant $\ket*{\slat_I}$ then yields
        \begin{gather}
            i\hslash \dod[]{}{t}c_{J}(t)
            = \hamilten_{IJ}(t) c_{J}(t),
            \label{eq:tdci}
        \end{gather}
        where we've used that the Slater determinants are orthonormal.
        The matrix elements of the time-dependent Hamiltonian is denoted by
        $\hamilten_{IJ}(t)$.
        As the initial coefficients $\vfg{c}(0)$ are known, we need to compute
        the matrix elements of the time-dependent Hamiltonian at each time step
        before using a time evolution scheme to solve \autoref{eq:tdci}.

\clearemptydoublepage
