\chapter{Configuration interaction}
     A popular post Hartree-Fock method is \textit{configuration interaction}.
     It consists of expressing the wavefunction as a linear combination of
     excited Slater determinants in a truncated single-particle and Slater
     determinant basis.
     \begin{align}
         \ket{\Psi_{\text{CI}}}
         &= A_0\ket{\Phi_0}
         \sum_{ai}A^a_i\ket{\Phi^a_i}
         + \frac{1}{4}\sum_{abij}A^{ab}_{ij}\ket{\Phi^{ab}_{ij}}
         + \dots,
     \end{align}
     where we have divided by a factor $4$ in the double sum to avoid over
     counting as both the coefficients and the excited determinants are
     antisymmetric. By generating all the possible Slater determinants from the
     $L$ single-particle functions we employ the \textit{full configuration
     interaction} method. This will give the most accurate value of the energy
     for the system, but quickly becomes computationally impossible as the FCI
     space grows in dimensions as $\binom{L}{N}$.
