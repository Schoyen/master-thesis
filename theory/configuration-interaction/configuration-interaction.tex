\chapter{Configuration interaction}
     A popular post Hartree-Fock method is \textit{configuration interaction}.
     % TODO: Discuss that FCI is "exact" for any choice of basis functions
     It consists of expressing the wavefunction as a linear combination of
     excited Slater determinants in a truncated single-particle and Slater
     determinant basis.
     \begin{align}
         \ket{\Psi}
         &= C_0\ketslat
         + \sum_{ai}C^a_i\ketslate{a}{i}
         + \frac{1}{4}\sum_{abij}C^{ab}_{ij}\ketslate{ab}{ij}
         + \dots,
     \end{align}
     where we have divided by a factor $4$ in the double sum to avoid over
     counting as both the coefficients and the excited determinants are
     antisymmetric. By generating all the possible Slater determinants from the
     $L$ single-particle functions we employ the \textit{full configuration
     interaction} method. This will give the most accurate value of the energy
     for the system, but quickly becomes computationally impossible as the FCI
     space grows in dimensions as $\binom{L}{N}$.
     \cite{kvaal2017notes}

     \section{Time-independent configuration interaction theory}
        We start with the time-independent Schrödinger equation
        \begin{align}
            \hamil\ket{\Psi_J} = E_J\ket{\Psi_J},
            \label{eq:ci_tise}
        \end{align}
        where $(E_J, \ket{\Psi_J})$ is an eigenpair for $\hat{H}$. Expanding the
        CI wavefunction in a Slater determinant basis.
        \begin{align}
            \ket{\Psi_J} = \sum_{K} C_{KJ}\ket{\Phi_K},
            \label{eq:expanded_ci_wavefunction}
        \end{align}
        where $C_{KJ}$ are the amplitudes for a certain excitation $K$ for a
        specific energy level $J$. Inserting
        \autoref{eq:expanded_ci_wavefunction} into \autoref{eq:ci_tise} and left
        projecting on a state $\ket{\Phi_I}$ we get
        \begin{align}
            \sum_{K}\bra{\Phi_I}\hat{H}\ket{\Phi_K}C_{KJ}
            = E_J\sum_{K}\braket{\Phi_I}{\Phi_K}C_{KJ}.
        \end{align}
        We now define the Hamiltonian matrix $H_{IK} =
        \bra{\Phi_I}\hat{H}\ket{\Phi_K}$ and the overlap matrix $S_{IK} =
        \braket{\Phi_I}{\Phi_K}$. We can thus formulate the generalized
        eigenvalue equation
        \begin{gather}
            \sum_{K}H_{IK}C_{KJ} = E_J\sum_{K}S_{IK}C_{KJ}
            \\
            \implies
            HC = ESC,
        \end{gather}
        where $S_{IK} = 1 \iff \braket{\Phi_I}{\Phi_K} = \delta_{IK}$. We will
        in this text only care about systems where the Slater determinants are
        orthonormal. Thus the eigenvalue equation we will solve will be
        \begin{align}
            HC = EC,
        \end{align}
        which means our job is to construct $H_{IJ}$ and diagonalize the
        matrix \cite{karwowski}. The elements $H_{IJ}$ are computed by
        \begin{align}
            \bra{\Phi_I}\hat{H}\ket{\Phi_J}
            &= \sum_{pq}h^{p}_{q}
            \bra{\Phi_I}\ccr{p}\can{q}\ket{\Phi_J}
            + \frac{1}{4}\sum_{pqrs}u^{pq}_{rs}
            \bra{\Phi_I}\ccr{p}\ccr{q}\can{s}\can{r}\ket{\Phi_J}.
        \end{align}
        Having constructed the one- and two-body elements $h^{p}_{q}$ and
        $u^{pq}_{rs}$ what remains is for us to evaluate the integrals
        $\bra{\Phi_I}\ccr{p}\can{q}\ket{\Phi_J}$ and
        $\bra{\Phi_I}\ccr{p}\ccr{q}\can{s}\can{r}\ket{\Phi_J}$. The total
        process of evaluating the matrix elements
        $\bra{\Phi_I}\hamil\ket{\Phi_J}$ can be done using the so-called
        \emph{Slater-Condon rules} which we'll describe below.

        \subsection{One-body density matrix}

        \subsection{The Slater-Condon rules}

    \section{Time-dependent configuration interaction}
        Having found the eigenstates $\ket{\Psi_J}$ given by
        \begin{align}
            \ket{\Psi_J} = C_{KJ}\ket{\Phi_K},
        \end{align}
        where $C_{KJ}$ are the coefficients for the Slater determinants
        $\ket{\Phi_K}$,
        we can use these states as our initial guess in a time-propagation of
        the system.
        Inserted into the time-dependent Schrödinger equation we can formulate
        the time-evolution of the coefficients.
        \begin{align}
            i\hslash \dpd[]{}{t}\ket{\Psi_J(t)}
            = \hamil(t)\ket{\Psi_J(t)},
            \label{eq:ci_tdse}
        \end{align}
        where $\ket{\Psi_J(0)} \equiv \ket{\Psi_J}$, that is, the eigenstates of
        the time-independent Schrödinger equation from \autoref{eq:ci_tise}.
        In time-dependent configuration interaction we treat our initial basis
        of orbitals, and by extension our basis of Slater determinants, as a
        \emph{static basis}.
        By that we mean that the orbitals are time-independent.
        It is only the coefficients that change in time, viz.
        \begin{align}
            \ket{\Psi_J(t)} = C_{KJ}(t)\ket{\Phi_K}.
        \end{align}
        This means that we can write \autoref{eq:ci_tdse} on the form
        \begin{align}
            i\hslash \dpd[]{}{t}C_{KJ}(t)\ket{\Phi_K}
            = \hamil(t) C_{KJ}(t)\ket{\Phi_K}.
        \end{align}
        Left-projecting with a Slater determinant $\ket{\Phi_I}$ then yields
        \begin{gather}
            i\hslash \dpd[]{}{t}C_{KJ}(t)\braket{\Phi_I}{\Phi_K}
            = \bra{\Phi_I}\hamil(t)\ket{\Phi_K} C_{KJ}(t) \\
            \implies
            i\hslash \dpd[]{}{t}C_{KJ}(t) S_{IK}
            = H_{IK}(t) C_{KJ}(t),
        \end{gather}
        where we've again introduced the matrix $S_{IK}$ with overlaps between
        the Slater determinants and the time-dependent Hamiltonian matrix
        $H_{IK}(t)$.
        We'll again limit our attention to systems where the Slater
        determinants are orthonormal, thus leaving us with the equation
        \begin{align}
            i\hslash \dpd[]{}{t}C_{IJ}(t)
            = H_{IK}(t) C_{KJ}(t).
            \label{eq:ci_tdse_full_coeff}
        \end{align}
        In \autoref{eq:ci_tdse_full_coeff} demonstrate how we evolve any of the
        eigenstates $\ket{\Psi_J}$ to a later state $\ket{\Psi_J(t)}$.
        We will for the most part limit ourselves to using the ground state as
        our initial guess, that is, we set $J = 0$.
