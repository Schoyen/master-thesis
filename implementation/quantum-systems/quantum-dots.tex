\section{One-dimensional quantum dots}
    \label{sec:one-dim-qd}
    Artificial atoms, or the so-called quantum dots, constitute a hot topic in
    condensed matter physics and material sciences \cite{hjorth2017advanced,
    lohne, fei, anisimovas1998energy}.
    We will be exploring several types of quantum dots in both one and two
    dimensions in this thesis.
    The difference between the types of quantum dots is found in the
    one-body potential.
    All of the dots share the characteristic of being in an infinite well which
    makes the systems \emph{bound}.
    In our study of systems subject to intense laser fields, this will prove to
    be a bad approximation when the laser becomes very strong as the particles
    have no way of being ionized, i.e., escaping the potential well.
    Even so, for weak laser fields and for ground state calculations, they serve
    as excellent candidates for our methods.

    The time-independent one-dimensional quantum dot can be described by the
    one-body Hamiltonian
    \begin{align}
        \oneten(x)
        = -\frac{\hslash^2}{2m} \dod[2]{}{x}
        + v(x),
        \label{eq:one-body-odqd}
    \end{align}
    where $v(x)$ is a potential function.
    In one dimension it is quite cheap to define a grid and use finite
    differences to solve the one-dimensional Schrödinger equation.
    Furthermore, for the systems we explore, numerically solving the Coulomb
    integrals is also feasible.
    This solution has its drawbacks in that we approximate an infinite integral
    on a finite line, but it also has the advantage of being ``blind'' to the
    choice of potential.
    This makes the solution completely general and opens up for a wide variety
    of potentials without the need for any extra mathematics.

    \subsection{Discretizing the one-dimensional quantum dot}
        \label{subsec:discretizing-the-odqd}
        For a given one-body Hamiltonian $h(x)$ on the form described in
        \autoref{eq:one-body-odqd}, we wish to find a solution to the
        time-indepedent Schrödinger equation
        \begin{align}
            \oneten(x)\psi(x) = \epsilon \psi(x),
        \end{align}
        where we've ignored labels on the eigenpair $(\epsilon, \psi(x))$ for
        now to avoid clutter.
        However, be aware that we are looking for the spectrum of the one-body
        Hamiltonian.
        We use the central finite difference scheme for the kinetic term in the
        one-body Hamiltonian, that is,
        \begin{align}
            \dod[2]{\psi(x)}{x}
            = \frac{
                \psi(x + \Delta x) - 2\psi(x) + \psi(x - \Delta x)
            }{(\Delta x)^2}
            + \mathcal{O}\para{(\Delta x)^2}.
        \end{align}
        Now we can write the time-independent Schrödinger equation in the fully
        discretized form
        \begin{align}
            \oneten(x) \psi(x)
            &= -\frac{1}{2}
            \frac{
                \psi(x + \Delta x) - 2\psi(x) + \psi(x - \Delta x)
            }{(\Delta x)^2}
            + v(x)\psi(x)
            = \epsilon \psi(x),
        \end{align}
        where we use atomic units.
        Introducing the uniformly discretized grid $x_i$ where $i \in \brac{1,
        n}$ and $n$ is the number of grid points, we label the wave function by
        \begin{align}
            \psi_i \equiv \psi(x_i),
        \end{align}
        and similarly for the one-body Hamiltonian and the potential.
        By including the entire grid we denote the eigenstate as a vector
        $\vfg{\psi}$.
        We have that
        \begin{align}
            x_{i + 1} = x_i + \Delta x,
        \end{align}
        with $\Delta x$ being the step-size between the grid points.
        There are smarter grid choices than the uniform grid, but for our
        simulations the uniform grid is sufficient.
        Collecting the wave function evaluated at the same grid points in the
        time-independent Schrödinger equation we can write the equation as
        \begin{align}
            \oneten \psi_i
            &=
            \para{
                \frac{1}{(\Delta x)^2}
                + v_i
            }\psi_i
            - \frac{1}{2 (\Delta x)^2}
            \para{
                \psi_{i + 1}
                + \psi_{i - 1}
            }
            = \epsilon \psi_i.
        \end{align}
        We are now in a position to formulate this equation as an eigenvalue
        equation on the form
        \begin{align}
            \vfg{\oneten}\vfg{\psi}
            = \vfg{\epsilon}\vfg{\psi},
            \label{eq:one-dim-qd-eigh}
        \end{align}
        where the one-body Hamiltonian matrix is a tridiagonal matrix with
        diagonal elements
        \begin{align}
            \oneten_{ii}
            =
            \frac{1}{(\Delta x)^2}
            + v_i
        \end{align}
        and off-diagonal elements
        \begin{align}
            \oneten_{ij}
            = -\frac{1}{2(\Delta x)^2}\para{
                \delta_{(i + 1) j}
                + \delta_{(i - 1) j}
            }.
        \end{align}
        We now wish to diagonalize \autoref{eq:one-dim-qd-eigh} and find all
        eigenpairs $(\epsilon_k, \vfg{\psi}_k)$.
        As $\onemat$ is Hermitian we have that \cite{mat-inf4130}
        \begin{align}
            \onemat = \vfg{P}^{-1}\vfg{D}\vfg{P}
            = \vfg{P}^{\dagger}\vfg{D}\vfg{P},
        \end{align}
        where $\vfg{P} \in \mathbb{C}^{n \times n}$ and $\vfg{D} = \diag(\epsilon_1,
        \dots, \epsilon_n)$.
        The columns of $\vfg{P}$ will now be the eigenstates $\vfg{\psi}_k$ with
        the elements indexed by
        \begin{align}
            P_{ik} = \psi_k(x_i).
        \end{align}
        The quality of the eigenpairs is dependent on the number of grid points
        $n$, and ideally we should include many grid points such that $\Delta x
        \to 0$.
        However, we are not in a position to utilize all $n$ eigenstates in the
        later analysis and we will therefore truncate the number of orbitals to
        the number of basis functions $L$ that we wish to include.
        Furthermore, the systems we are exploring are spin-independent which
        means that we only use $L / 2$ orbitals from the spectrum of $\vfg{h}$,
        thus halving the number of single-particle functions.

        We use the function \pyth{np.linalg.eigh} from NumPy \cite{numpy} to
        solve the eigenvalue equation in \autoref{eq:one-dim-qd-eigh}.
        This means that the eigenstates $\vfg{\psi}_k$ are normalized to unity
        with respect to the dot-product, that is,
        \begin{align}
            \vfg{\psi}_i^{\dagger} \vfg{\psi}_j = \delta_{ij}.
        \end{align}
        However, we wish to normalize the states to unity with respect to the
        integral inner product.
        To solve this we compute the normalized eigenstates $\tilde{\vfg{P}}$ by
        \begin{align}
            \tilde{\vfg{P}} = \frac{\vfg{P}}{\Delta x},
        \end{align}
        where an elementwise division is assumed.
        Lastly, we have not discussed the boundary conditions for the
        eigenstates.
        We require that $\psi_p(-\infty) = \psi_p(\infty) = 0$.
        We can realize this by introducing two extra grid points $i = 0$ and $i
        = n + 1$ after we have diagonalized the one-body Hamiltonian matrix and
        set
        \begin{align}
            \psi_p(x_0) = \psi_p(x_{n + 1}) = 0.
        \end{align}

        Having found the eigenpairs for the one-dimensional Hamiltonian, we
        construct a new diagonal, one-dimensional Hamiltonian from the
        eigenenergies.
        The normalized eigenstates from the eigenvalue equation will be
        orthonormal.


    \subsection{Constructing the dipole moments}
        The matrix elements of the dipole moment for the one-dimensional quantum
        dot is given by
        \begin{align}
            d_{pq}
            = \mel*{\psi_{p}}{\position}{\psi_{q}}
            = \int_{-\infty}^{\infty}\dd x
            \psi^{*}_{p}(x) x \psi_{q}(x).
        \end{align}
        As we are using a grid based solution, we approximate this integral
        using a numerical integration scheme on the grid.
        In our implementation we use the trapezoidal rule.
        By defining
        \begin{align}
            f_{pq}(x) = \psi^{*}_{p}(x) x \psi_{q}(x),
        \end{align}
        and given a uniform grid for $x_i \in \brak{a, b}$ where $i \in \brac{1,
        n}$ such that $x_1 = a$ and $x_n = b$ with grid spacing $\Delta x$, we
        approximate the dipole elements by
        \begin{align}
            d_{pq}
            \approx
            \frac{\Delta x}{2}\sum_{i = 2}^{n}\brak{
                f_{pq}(x_{i}) + f_{pq}(x_{i - 1})
            }.
        \end{align}
        The reason for choosing the trapezoidal rule is for its ease of usage
        and implementation, as well as the method being quite fast and precise.
        Furthermore, as our implementation is written in pure Python, we use
        Numba \cite{numba} to speed up explicit for-loops, but Numba does not
        support the usage of already implemented integrators.
        Thus, by writing the trapezoidal rule by hand and using Numba to
        \emph{just-in-time}-compile\footnote{%
            Just-in-time compilation, or just ``jit'' compilation, is a
            technique where code is compiled during execution.
            As Python is an interpreted language Numba \cite{numba} can be used
            to compile certain functions to LLVM (Low Level Virtual Machine)
            \cite{llvm} once they are encountered.
            After the initial hold-up from the compilation, this can lead to a
            truly impressive speed-up of the original code.
            For a small demonstration of this see:
            \url{https://github.com/Schoyen/Mandelbrot}.
        } the function, we gain a significant speed-up
        over regular Python.
        It is however, not the best integrator there is and more precise
        solutions should be considered.


    \subsection{Integrating the Coulomb elements}
        The arguably heaviest computation in setting up a one-dimensional
        quantum dot system, is the Coulomb integrals.
        In atomic units the one-dimensional Coulomb interaction can be
        represented by
        \begin{align}
            \twoten(x_i, x_j) = \frac{1}{\abs{x_i - x_j}}.
        \end{align}
        Now, there are a few drawbacks of using this interaction.
        For example, when integrating on a grid, we get numerical instabilities
        at the singularity when $x_1 = x_2$.
        This motivates the introduction of a \emph{shielded Coulomb interaction}
        \cite{suq} given by
        \begin{align}
            \twoten(x_i, x_j)
            &= \frac{\alpha}{\sqrt{(x_i - x_j)^2 + a^2}},
            \label{eq:shielded-coulomb}
        \end{align}
        where $\alpha$ is a dimensionless constant and the screening parameter
        $a$ removes the singularity while retaining the asymptotic behavior when
        $(x_i - x_j) \to \infty$ \cite{suq}.
        The integral we wish to solve is then
        \begin{align}
            \mel*{\psi_{p}\psi_{q}}{\twohamil}{\psi_{r}\psi_{s}}
            &=
            \int\dd x_1 \dd x_2
            \psi^{*}_{p}(x_1)\psi^{*}_{q}(x_2)
            \twoten(x_1, x_2)
            \psi_{r}(x_1)
            \psi_{s}(x_2),
        \end{align}
        where $\twoten(x_1, x_2)$ is the shielded Coulomb interaction.
        We introduce the inner integral $W_{qs}(x_1)$ given by
        \begin{align}
            W_{qs}(x_1)
            &= \int\dd x_2
            \psi^{*}_{q}(x_2) \twoten(x_1, x_2)
            \psi_{s}(x_2),
        \end{align}
        that is, we integrate over one of the two spatial integrals in the
        Coulomb interaction.
        We can thus write the Coulomb elements as
        \begin{align}
            \mel*{\psi_{p}\psi_{q}}{\twohamil}{\psi_{r}\psi_{s}}
            &=
            \mel*{\psi_{p}}{\hat{W}_{qs}}{\psi_{r}}
            =
            \int\dd x_1
            \psi^{*}_{p}(x_1)
            W_{qs}(x_1)
            \psi_{r}(x_1),
        \end{align}
        where we initially compute the inner integral $W_{qs}(x_1)$ on the
        entire grid and store it as a vector.
        We define the functions
        \begin{gather}
            g_{qs}(x_i, x) = \psi^{*}_{q}(x) \twoten(x_i, x) \psi_s(x),
            \\
            v^{pq}_{rs}(x_i)
            = \psi^{*}_{p}(x_i) W_{qs}(x_i) \psi_{r}(x_i),
        \end{gather}
        that is, the functions that occur under the integrals in the inner, and
        the outer Coulomb integral respectively.
        We can now compute the inner integrals using the trapezoidal rule in the
        same manner as for the dipole elements.
        That is, for a given point $x_i$ we have
        \begin{align}
            W_{qs}(x_i)
            \approx \frac{\Delta x}{2}
            \sum_{j = 2}^{n}
            \brak{
                g_{qs}(x_i, x_j) + g_{qs}(x_i, x_{j - 1})
            }.
        \end{align}
        From this we can compute the Coulomb elements by
        \begin{align}
            \mel*{\psi_{p}\psi_{q}}{\twohamil}{\psi_{r}\psi_{s}}
            \approx
            \frac{\Delta x}{2}
            \sum_{i = 2}^{n}
            \brak{
                v^{pq}_{rs}(x_i) + v^{pq}_{rs}(x_{i - 1})
            }.
        \end{align}
        Again, we emphasize that the choice of using the trapezoidal rule is
        based on convenience and can be replaced by a better integrator if it
        turns out that the integrals are too erroneous.


%\section{Two-dimensional quantum dot}
%    Extending the formalism of the one-dimensional quantum dot to two dimensions
%    we look at a system of particles confined to the plane.
%    Due to there existing analytical solutions to the Coulomb elements for the
%    two-dimensional harmonic oscillator in polar coordinates
%    \cite{anisimovas1998energy}, this avoids the need of evaluating the two-body
%    elements on a grid as in the one-dimensional case.
%    The one-body Hamiltonian in coordinate representation can thus be
%    represented by
%    \begin{align}
%        \oneten(r, \phi)
%        = -\frac{\hslash^2}{2m}\brak{
%            \frac{1}{r}\dpd{}{r}\para{
%                r\dpd{}{r}
%            }
%            + \frac{1}{r^2}\dpd[2]{}{\phi}
%        }
%        + v(r, \phi),
%    \end{align}
%    where we do not necessarily assume that the potential is circular symmetric
%    unless specified.
%
%    \subsection{Two-dimensional harmonic oscillator}
%        \label{subsec:two-dim-ho}
%        In the two-dimensional harmonic oscillator we have a circular symmetric
%        potential on the form,
%        \begin{align}
%            v(r, \phi) = v(r) = \half m \omega^2 r^2.
%        \end{align}
%        Before solving the one-dimensional problem, we will make the system
%        dimensionless in the same manner as done by
%        \citeauthor{anisimovas1998energy} \cite{anisimovas1998energy} as we use
%        their solution for the Coulomb elements.
%        The full two-dimensional Hamiltonian with Coulomb interaction in
%        coordinate representation is given by,
%        \begin{align}
%            \hamilten(\vf{r})
%            = \sum_{i = 1}^{N} \brak{
%                -\frac{\hslash^2}{2m} \vfg{\nabla}^2_i
%                + \half m \omega \vfg{r}^2_i
%            }
%            + \sum_{i < j}^{N}
%            \frac{e^2}{4\pi \epsilon_0 \abs{\vfg{r}_i - \vfg{r}_j}}.
%            \label{eq:two-dim-ho-hamiltonian}
%        \end{align}
%        We introduce a dimensionless position $\vfg{r}'$
%        \begin{align}
%            \vfg{r} = a \vfg{r}',
%        \end{align}
%        where the scaling radius $a$ is given by,
%        \begin{align}
%            a \equiv \sqrt{\frac{\hslash}{m\omega}}.
%            \label{eq:bohr-radius}
%        \end{align}
%        The complete derivation of the scaling is shown in
%        \autoref{app:two-dim-ho-dimensionless}.
%        For the sake of brevity we relabel the dimensionless position $\vfg{r}'
%        \to \vfg{r}$.
%        We then have the dimensionless form for the two-dimensionless harmonic
%        oscillator Hamiltonian
%        \begin{align}
%            \hamilten(\vfg{r})
%            = \frac{\hslash\omega}{2} \sum_{i = 1}^{N}\brak{
%                -\vfg{\nabla}^2_i
%                + \vfg{r}^2_i
%            }
%            + \hslash\omega\lambda
%            \sum_{i < j}^{N}
%            \frac{1}{\abs{\vfg{r}_i - \vfg{r}_j}},
%        \end{align}
%        where we have
%        \begin{align}
%            \lambda \equiv \frac{e^2}{
%                4\pi \epsilon_0 \hslash \omega a_0
%            }.
%            \label{eq:two-dim-ho-lambda}
%        \end{align}
%        In the article by \citeauthor{anisimovas1998energy}
%        \cite{anisimovas1998energy} they go the extra step of measuring the
%        energy in units of $\hslash\omega$ as this makes it more natural to
%        define the Coulomb interaction as a perturbation to the one-body
%        Hamiltonian.
%        Now, we do not share the goal of expanding the Hamiltonian as a
%        perturbation and we will therefore keep $\hslash\omega$ as they are.
%        Moving to atomic units we are left with the expression
%        \begin{align}
%            \hamilten(\vfg{r})
%            &= \omega \sum_{i = 1}^{N}
%            \oneten(\vfg{r}_i)
%            + \sqrt{\omega} \sum_{i < j}^{N} \twoten(\vfg{r}_i, \vfg{r}_j),
%            \label{eq:tdho-hamiltonian}
%        \end{align}
%        which reduces the scaling of the system down to the single parameter
%        $\omega$ as the strength of the harmonic oscillator well.
%        Furthermore, we've introduced the one-body Hamiltonian
%        $\oneten(\vfg{r})$ which contains the kinetic and potential energy
%        terms\footnote{%
%            This is where the factor $1/2$ went.
%        } and the two-body Hamiltonian $\twoten(\vfg{r}_i, \vfg{r}_j)$ for the
%        Coulomb potential.
%
%        Now, finding the spectrum of the full many-body Hamiltonian is
%        computationally extreme for $N > 2$ and we will therefore use our
%        repertoir of many-body methods as discussed in the previous chapters.
%        However, we still need a set of atomic orbitals to start our
%        calculations and a natural starting point is the analytical solutions to
%        the one-body Hamiltonian.
%        If $\lambda$ is small, this can turn out to be a very good approximation
%        as we are close to the non-interacting case where the spectrum of the
%        Hamiltonian is the spectrum of the one-body Hamiltonian.
%        If $\lambda$ becomes large this approximation will turn out to be poor
%        and there exists better orbital sets, e.g., the molecular orbitals from
%        the Hartree-Fock method.
%        Still, both solutions requires an initial guess and we might as well
%        choose the spectrum of the one-body Hamiltonian as there exists closed
%        form solutions for these single particle functions.
%
%        \subsubsection{The spectrum of the two-dimensional harmonic oscillator}
%            For the non-interacting case the Hamiltonian in
%            \autoref{eq:tdho-hamiltonian} reduces to the one-body Hamiltonian
%            which on coordinate form is given by
%            \begin{align}
%                \oneten(\vfg{r})
%                = -\half\vfg{\nabla}^2 + \half\vfg{r}^2,
%                \label{eq:tdho-one-body-hamiltonian}
%            \end{align}
%            where we've dropped the indices on the particles and left the energy
%            scaling in the full Hamiltonian.
%            We now look for a solution to the time-independent Schrödinger equation
%            \begin{align}
%                \oneten(\vfg{r})\psi(\vfg{r}) = \epsilon \psi(\vfg{r}),
%            \end{align}
%            where $(\epsilon, \psi(\vfg{r}))$ are the eigenpairs of the
%            non-interacting two-dimensional harmonic oscillator.
%            The eigenstates to this equation in polar coordinates is
%            \begin{align}
%                \psi_{nm}(r, \phi)
%                &= \sqrt{\frac{(2n)!}{2\pi(\abs{m} + n)!}}
%                \exp[i m \phi]
%                r^{\abs{m}}
%                \exp[-r^2 / 2]
%                L^{\abs{m}}_{n}(r^2),
%                \label{eq:eigenstate-tdho}
%            \end{align}
%            where $L^{\abs{m}}_{n}(x)$ are the \emph{associated Laguerre
%            polynomials} \cite{anisimovas1998energy}.
%            Furthermore, we have the principal quantum number $n \in \mathbb{N}$
%            and the azimuthal quantum number $m \in \mathbb{Z}$.
%            We collect the eigenstates in its separable parts,
%            \begin{align}
%                \psi_{nm}(r, \phi)
%                = N_{nm} R_{nm}(r) \Phi_{m}(\phi),
%            \end{align}
%            where the normalization is given by
%            \begin{align}
%                N_{nm}
%                &= \sqrt{\frac{(2n)!}{2\pi(\abs{m} + n)!}},
%            \end{align}
%            the radial functions
%            \begin{align}
%                R_{nm}
%                &=
%                r^{\abs{m}}
%                \exp[-r^2 / 2]
%                L^{\abs{m}}_{n}(r^2),
%                \label{eq:radial-function-tdho}
%            \end{align}
%            and the azimuthal part
%            \begin{align}
%                \Phi_{m}(\phi) = \exp[i m \phi],
%            \end{align}
%            for ease of lookup.
%            The eigenstates are orthonormal in both their principal and
%            azimuthal quantum numbers, that is,
%            \begin{align}
%                \braket*{\psi_{n_i m_i}}{\psi_{n_j m_j}}
%                = \delta_{n_i n_j}\delta_{m_i m_j}.
%            \end{align}
%            Furthermore, the orthonormality carries to the spin quantum number
%            $m_s$ as well.
%            The eigenenergies are given by
%            \begin{align}
%                \epsilon_{nm}
%                = 2n + \abs{m} + 1,
%                \label{eq:eigenenergy-tdho}
%            \end{align}
%            where we re-iterate that $\hslash\omega$ is factored out of the
%            one-body Hamiltonian and gets included in the full Hamiltonian
%            energy expression.
%            The derivation of the spectrum of the two-dimensional harmonic
%            oscillator is included in \autoref{app:tdho-spectrum}.
%
%            Looking at the expression for the eigenenergy in
%            \autoref{eq:eigenenergy-tdho} we see that we get degenerate
%            energies.
%            As the Hamiltonian in \autoref{eq:tdho-hamiltonian} is independent
%            of spin, we will in the case of fermions with $m_s = \pm 1/2$ get
%            twice as many degenerate states at each energy level.
%            An illustration of this is shown in
%            \autoref{fig:tdho-energy-levels}.
%            \begin{figure}
%                \begin{center}
%                    \begin{tikzpicture}
%                        \begin{scope}
%                            \foreach \i in {1, 2, 3} {
%                                \draw(-1, \i - 1) node[anchor=east]
%                                {$\epsilon = \i$};
%                            }
%
%                            % Highest energy level
%                            \foreach \i in {0, 3, 6} {
%                                \draw (\i, 2) -- (\i + 2, 2);
%                                \node at (\i + 0.75, 2) {$\uparrow$};
%                                \node at (\i + 1.25, 2) {$\downarrow$};
%                            }
%                            \node[below, inner sep=.2cm] at (1, 2)
%                            {$n = 0, m = -2$};
%                            \node[below, inner sep=.2cm] at (4, 2)
%                            {$n = 1, m = 0$};
%                            \node[below, inner sep=.2cm] at (7, 2)
%                            {$n = 0, m = 2$};
%
%                            % Middle energy level
%                            \foreach \i in {1.5, 4.5} {
%                                \draw (\i, 1) -- (\i + 2, 1);
%                                \node at (\i + 0.75, 1) {$\uparrow$};
%                                \node at (\i + 1.25, 1) {$\downarrow$};
%                            }
%                            \node[below, inner sep=.2cm] at (2.5, 1)
%                            {$n = 0, m = -1$};
%                            \node[below, inner sep=.2cm] at (5.5, 1)
%                            {$n = 0, m = 1$};
%
%                            % Lowest energy level
%                            \draw (3, 0) -- (5, 0);
%                            \node at (3 + 0.75, 0) {$\uparrow$};
%                            \node at (3 + 1.25, 0) {$\downarrow$};
%                            \node[below, inner sep=.2cm] at (4, 0)
%                            {$n = 0, m = 0$};
%                        \end{scope}
%                    \end{tikzpicture}
%                \end{center}
%                \caption{In this plot we can see the energy degeneracy of the lowest
%                three energy levels in the two-dimensional quantum dot.
%                Each arrow representes a spin up or a spin down state with the
%                quantum numbers $n$ and $m$ as listed below.
%                This pattern goes on indefinitly with the addition of one bar
%                (two oscillators) per level.}
%                \label{fig:tdho-energy-levels}
%            \end{figure}
%
%
%        \subsubsection{Mapping the quantum numbers to a single value}
%            When working with second quantized operators, we distinguish single
%            particle functions in a Slater determinant by a single index.
%            Including spin in the two-dimensional harmonic oscillator
%            eigenstates we have to deal with three quantum numbers $n$, $m$, and
%            $m_s$ per state.
%            We are thus interested in finding a mapping $(n, m) \mapsto p$ and
%            the inverse mapping $p \mapsto (n, m)$.\footnote{%
%                The spin quantum number is easiest to deal with in the end as we
%                can double each dimension in every matrix and label all odd or
%                even indices by a spin direction.
%            }
%            Due to the degeneracy of the eigenenergies such a mapping will not
%            be unique.
%            We thus have to decide in advance how we should count the basis
%            states.
%            In this work we choose the convention that we start from the
%            lowest energy level and move up by counting from left to right.
%            Tabulating \autoref{fig:tdho-energy-levels} in this convetion we get
%            the results shown in \autoref{tab:tdho-mapping}.
%            In \autoref{alg:nm-to-p} we demonstrate the algorithm we use to
%            compute $(n, m) \mapsto p$.
%            The inverse algorithm for $p \mapsto (n, m)$ is shown in
%            \autoref{alg:p-to-nm}.
%            We are now able to find indices $p$ such that
%            \begin{align}
%                (\epsilon_{nm}, \psi_{nm}) \mapsto (\epsilon_p, \psi_p).
%            \end{align}
%
%            \begin{table}
%                \centering
%                \caption{In this table we show an example of how our mapping
%                convention will index the states shown in
%                \autoref{fig:tdho-energy-levels}.}
%                \renewcommand{\arraystretch}{1.3}
%                \begin{tabular}{@{}lll@{}}
%                    \toprule
%                    $\epsilon_{nm}$ & $(n, m)$ & $p$ \\
%                    \midrule
%                    $1$ & $(0, 0)$ & $0$ \\
%                    $2$ & $(0, -1)$ & $1$ \\
%                    $2$ & $(0, 1)$ & $2$ \\
%                    $3$ & $(0, -2)$ & $3$ \\
%                    $3$ & $(1, 0)$ & $4$ \\
%                    $3$ & $(0, 2)$ & $5$ \\
%                    \bottomrule
%                \end{tabular}
%                \label{tab:tdho-mapping}
%            \end{table}
%
%            \begin{algorithm}
%                \inputpython{implementation/quantum-systems/get_index_p.py}{0}{35}
%                \caption{In this algorithm we describe how we can find $(n, m)
%                \mapsto p$ relatively quick without having to tabulate all
%                states up to some level.}
%                \label{alg:nm-to-p}
%            \end{algorithm}
%
%            \begin{algorithm}
%                \inputpython{implementation/quantum-systems/get_indices_nm.py}{0}{48}
%                \caption{In this algorithm we sketch how we can find $p \mapsto
%                (n, m)$, i.e., the inverse of \autoref{alg:nm-to-p}.}
%                \label{alg:p-to-nm}
%            \end{algorithm}
%
%        \subsubsection{Computing the dipole moments}
%            The matrix elements of the dipole moment of the two-dimensional
%            harmonic oscillator quantum dot is given by
%            \begin{align}
%                \vf{d}_{ij}
%                \equiv \mel*{\psi_i}{\positionvec}{\psi_j},
%            \end{align}
%            where we get a dipole moment for each coordinate in $\vf{r}$.
%            As we've expressed the single particle functions in polar
%            coordinates, but wish to express the dipole moments in a cartesian
%            coordinate system, we have to compute the two integrals
%            \begin{align}
%                \vf{d}_{ij}
%                &= \vf{i}\mel*{\psi_i}{\position}{\psi_j}
%                + \vf{j}\mel*{\psi_i}{\position[y]}{\psi_j}
%                = \vf{i}\mel*{\psi_i}{\hat{r} \cosine(\hat{\phi})}{\psi_j}
%                + \vf{j}\mel*{\psi_i}{\hat{r} \sine(\hat{\phi})}{\psi_j},
%                \label{eq:dipole_elements}
%            \end{align}
%            where $\vf{i}$ and $\vf{j}$ are the unit vectors along the $x$- and
%            $y$-axis respectively.
%            Using \autoref{eq:eigenstate-tdho} we are able to find analytical
%            expressions for the angular integral.
%            The radial integral must be evaluated from $0$ to $\infty$, but we
%            choose to use SymPy \cite{sympy}, which handles this for us.
%            Note that we use the notation $i \mapsto (n_i, m_i)$ from the
%            mapping algorithms described above.
%            We then get the following integrals
%            \begin{align}
%                \mel*{\psi_i}{\hat{r} \cosine(\hat{\phi})}{\psi_j}
%                &= N^{*}_{n_i m_i} N_{n_j m_j}
%                \mathcal{R}_{ij}
%                \mathcal{C}_{ij},
%                \\
%                \mel*{\psi_i}{\hat{r} \sine(\hat{\phi})}{\psi_j}
%                &= N^{*}_{n_i m_i} N_{n_j m_j}
%                \mathcal{R}_{ij}
%                \mathcal{S}_{ij},
%            \end{align}
%            where no summation is implied in the index labels.
%            The integrals are given by
%            \begin{gather}
%                \mathcal{R}_{ij}
%                =
%                \int_{0}^{\infty} \dd r r^2
%                R_{n_i m_i}^{*}(r) R_{n_j m_j}(r),
%                \label{eq:radial-integral-tdho}
%                \\
%                \mathcal{C}_{ij}
%                =
%                \int_{0}^{2\pi}
%                \dd \phi
%                \cos(\phi)
%                \Phi_{m_i}^{*}(\phi)
%                \Phi_{m_j}(\phi),
%                \label{eq:cos-integral-tdho}
%                \\
%                \mathcal{S}_{ij}
%                =
%                \int_{0}^{2\pi}
%                \dd \phi
%                \sin(\phi)
%                \Phi_{m_i}^{*}(\phi)
%                \Phi_{m_j}(\phi).
%                \label{eq:sin-integral-tdho}
%            \end{gather}
%            We solve the radial integral using SymPy \cite{sympy}.
%            We start by constructing the radial functions $R_{n_i m_i}(r)$.
%            This is done by the Python function shown in
%            \autoref{alg:radial-function-tdho}, where the mapping from $i
%            \mapsto (n_i, m_i)$ is done by \autoref{alg:p-to-nm}.
%            \begin{algorithm}
%                \inputpython{implementation/quantum-systems/spf_radial_function.py}{0}{10}
%                \caption{Python function constructing the radial functions from
%                \autoref{eq:radial-function-tdho}.}
%                \label{alg:radial-function-tdho}
%            \end{algorithm}
%            Having constructed the two radial functions we can perform the
%            integration using the function in \autoref{alg:radial-integral-tdho}.
%            \begin{algorithm}
%                \inputpython{implementation/quantum-systems/radial_integral.py}{0}{7}
%                \caption{Python function performing the radial integral in
%                \autoref{eq:radial-integral-tdho} using SymPy \cite{sympy} to
%                evaluate the integral from $r \in [0, \infty)$.}
%                \label{alg:radial-integral-tdho}
%            \end{algorithm}
%
%            The two angular integrals in \autoref{eq:cos-integral-tdho} and
%            \autoref{eq:sin-integral-tdho} have closed form solutions which we
%            derive in \autoref{app:tdho-dipole}.
%            Defining the difference in the azimuthal quantum number by,
%            \begin{align}
%                \Delta m_{ij} \equiv m_i - m_j,
%                \label{eq:diff-m-tdqd}
%            \end{align}
%            where $\Delta m_{ij} \in \mathbb{Z}$.
%            We then have
%            \begin{align}
%                \mathcal{C}_{ij}
%                = \begin{cases}
%                    \pi, & \Delta m_{ij} = \pm 1, \\
%                    0, & \text{else},
%                \end{cases}
%            \end{align}
%            for the cosine integral in \autoref{eq:cos-integral-tdho}.
%            For the sine integral we get
%            \begin{align}
%                \mathcal{S}_{ij}
%                = \begin{cases}
%                    \mp i\pi, & \Delta m_{ij} = \pm 1, \\
%                    0, & \text{else}.
%                \end{cases}
%            \end{align}
%            The results for the angular integrals yield the selection rules for
%            the dipole approximation for the two-dimensional harmonic
%            oscillator.
%            Unless $\Delta m_{ij} = \pm 1$, the transition between the states is
%            not allowed in the dipole approximation, i.e., the integral
%            vanishes.
%            The selection rules for the dipole approximation for the
%            two-dimensional harmonic oscillator is directly linked to the
%            harmonic potential theorem \cite{kohn, brey}.
%
%
%        \subsubsection{Computing the Coulomb elements}
%            Having found the basis functions and the elements of the one-body
%            hamiltonian, we are left with the task of finding the two-body
%            elements from the Coulomb interaction.
%            Luckily, there exists an analytic formula finding these elements for
%            the two-dimensional harmonic oscillator in polar coordinates.
%            This formula is shown in appendix A in the article
%            \citetitle{anisimovas1998energy} by
%            \citeauthor{anisimovas1998energy} \cite{anisimovas1998energy}.
%            Note that \citeauthor{anisimovas1998energy} interchanges the indices
%            in the ket-part of their two-body integrals as opposed to our
%            convention. That is, in our notation we would write
%            \begin{align}
%                \mel*{\psi_i\psi_j}{\twohamil}{\psi_k\psi_l}
%                &=
%                \int\dd x_1\dd x_2
%                \psi_{i}^{*}(x_1) \psi_{j}^{*}(x_2)
%                \twoten(x_1, x_2)
%                \psi_{k}(x_1)\psi_{l}(x_2)
%                \\
%                &\equiv
%                \mel*{\psi_i\psi_j}{\twohamil}{\psi_l\psi_k}_{AM},
%            \end{align}
%            where $ \mel*{\psi_i\psi_j}{\twohamil}{\psi_l\psi_k}_{AM}$ is the convention used by
%            \citeauthor{anisimovas1998energy}.
%            There are two significant symmetries included in the calculation of
%            the two-body elements.
%            The first occurs as the Hamiltonian is spin-independent and the
%            second comes from the azimuthal quantum number $m$.
%            This can be expressed as
%            \begin{align}
%                \mel*{\psi_i\psi_j}{\twohamil}{\psi_k\psi_l}
%                = \delta_{\sigma_i \sigma_k} \delta_{\sigma_j \sigma_l}
%                \int\dd \vf{r}_1\dd \vf{r}_2
%                \psi_{i}^{*}(\vf{r}_i) \psi_{j}^{*}(\vf{r}_2)
%                \twoten
%                \psi_{k}(\vf{r}_1)\psi_{l}(\vf{r}_2),
%            \end{align}
%            where the symmetry that $m_i + m_j = m_k + m_l$ is baked into the
%            integral.
%            The formula for computing the integral
%            \begin{align}
%                \mathcal{I}^{ij}_{kl}
%                =
%                \int\dd \vf{r}_1\dd \vf{r}_2
%                \psi_{i}^{*}(\vf{r}_1) \psi_{j}^{*}(\vf{r}_2)
%                \twoten
%                \psi_{k}(\vf{r}_1)\psi_{l}(\vf{r}_2),
%            \end{align}
%            is rather involved and we've pushed it to
%            \autoref{app:coulomb-elements}.
%
%
%    \subsection{Two-dimensional double well quantum dot}
%        \label{subsec:tddw}
%        Another interesting system to explore is the double well quantum dot,
%        i.e., a quantum dot with a finite barrier in the potential trap.
%        Building the barrier on top of the two-dimensional harmonic oscillator
%        lets us reuse much of the machinery from the previous section.
%        There are a multitude of ways to create a confining double well
%        potential.
%        In the following we'll be focusing on a double well with a sharp
%        boundary for the barrier in either $x$- or $y$-direction.
%        The one-body Hamiltonian of the double well with the barrier in the
%        $x$-direction is given by
%        \begin{align}
%            \onehamil
%            &=
%            \frac{\momentum^2}{2m}
%            + \half m \omega^2 \position[r]^2
%            + \half m \omega^2 \para{
%                \frac{1}{4} l^2 - l\abs{\position}
%            }.
%            \label{eq:one-body-2ddw}
%        \end{align}
%        % TODO: Give motivation for the shape of the double-well.
%        % Discuss the constant term.
%        In \autoref{eq:one-body-2ddw} we recognize the two first terms as the
%        kinetic energy and the harmonic oscillator confining potential.
%        Instead of finding an analytical expression for the full one-body
%        Hamiltonian, as we did for the harmonic oscillator, we'll instead use
%        the single-particle functions in \autoref{eq:eigenstate-tdho} as trial
%        wave functions and use the variational method to find as good an
%        estimate as possible to the underlying exact eigenpairs.
%        The one-body matrix elements can then be found from
%        \begin{align}
%            \oneten^{p}_{q}
%            &= \epsilon_{p}\delta^{p}_{q}
%            + \half m \omega^2\mel*{\phi_p}{
%                \para{
%                    \frac{1}{4} l^2 - l \abs{\position}
%                }
%            }{\phi_q}
%            \\
%            &= \epsilon_{p}\delta^{p}_{q}
%            + \frac{1}{8} m \omega^2 l^2 \delta^{p}_{q}
%            - \half m \omega^2 l \mel*{\phi_p}{\abs{\position}}{\phi_q},
%        \end{align}
%        where we have used the mapping $(n, m) \to p$ from
%        \autoref{alg:nm-to-p} described above and where
%        the eigenenergies $\epsilon_p$ are given by
%        \autoref{eq:eigenenergy-tdho}.
%        The two first terms give a shifted harmonic oscillator potential, viz.
%        \begin{align}
%            \epsilon_p' = \epsilon_p + \frac{1}{8} m \omega^2 l^2.
%        \end{align}
%        The last term is the double well barrier and resembles the dipole matrix
%        elements from \autoref{eq:dipole_elements} except for the absolute
%        value.
%        We rewrite the potential barrier in polar coordinates by
%        \begin{gather}
%            \abs{x} = r\abs{\cos(\phi)}, \\
%            \abs{y} = r\abs{\sin(\phi)},
%        \end{gather}
%        where we've included the barrier in $y$-direction for the sake of
%        generality.
%        We can thus compute the barrier integrals using the eigenstates from the
%        harmonic oscillator,
%        \begin{gather}
%            \mel*{\phi_p}{\abs{\position}}{\phi_q}
%            =
%            N^{*}_{n_p m_p} N_{n_q m_q}
%            \mathcal{R}_{pq} \tilde{\mathcal{C}}_{pq},
%            \\
%            \mel*{\phi_p}{\abs{\position[y]}}{\phi_q}
%            =
%            N^{*}_{n_p m_p} N_{n_q m_q}
%            \mathcal{R}_{pq} \tilde{\mathcal{S}}_{pq},
%        \end{gather}
%        where the normalization and the radial integral is the same as for the
%        dipole moment in the harmonic oscillator system.
%        Note again that no summation is implied in the indices for the
%        integrals.
%        The only difference is the absolute value on the periodic functions in
%        the polar integrals.
%        These integrals are given by
%        \begin{gather}
%            \tilde{\mathcal{C}}_{pq}
%            =
%            \int_{0}^{2\pi} \dd \phi
%            \abs{\cos(\phi)}
%            \Phi^{*}_{m_p}(\phi)
%            \Phi_{m_q}(\phi),
%            \\
%            \tilde{\mathcal{S}}_{pq}
%            =
%            \int_{0}^{2\pi} \dd \phi
%            \abs{\sin(\phi)}
%            \Phi^{*}_{m_p}(\phi)
%            \Phi_{m_q}(\phi).
%        \end{gather}
%        For $k \in \mathbb{Z}$ and $\Delta m_{pq} \equiv m_p - m_q$, the
%        solution to the cosine integral is given by
%        \begin{align}
%            \tilde{\mathcal{C}}_{pq}
%            =
%            \frac{4}{1 - (\Delta m_{pq})^2}
%            \begin{cases}
%                0 & \Delta m_{pq} = 2k + 1, \\
%                1 & \Delta m_{pq} = 4k, \\
%                -1 & \Delta m_{pq} = 4k + 2,
%            \end{cases}
%        \end{align}
%        and for the sine integral we get
%        \begin{align}
%            \tilde{\mathcal{S}}_{pq}
%            &=
%            \frac{4}{1 - (\Delta m_{pq})^2}
%            \begin{cases}
%                0 & \Delta m_{pq} = 2k + 1, \\
%                1 & \Delta m_{pq} = 2k.
%            \end{cases}
%        \end{align}
%        For a full derivation, see \autoref{app:barrier-integrals}.
%
%        Having constructed the full one-body Hamiltonian matrix $\onemat$, we
%        diagonalize and find the variationally optimized eigenvectors $\vf{C}$
%        and single-particle eigenenergies $\vfg{\varepsilon}$.
%        We can now use the coefficients to transform to the new double-well
%        basis.
%        The process of changing basis will be discussed in
%        \autoref{sec:change-of-basis}.
