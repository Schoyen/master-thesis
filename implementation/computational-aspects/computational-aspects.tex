\chapter{Computational aspects}
    Along with this thesis we have implemented several quantum-mechanical
    solvers that are separated into different Github repositiories.\footnote{%
        Due to ongoing publications using the code most of the repositiories are
        at the time of writing private and access are therefore limited to
        collaborators.
        However, please do request access by sending a mail to:
        \href{mailto:o.s.schoyen@fys.uio.no}{o.s.schoyen@fys.uio.no}, and we'll
        set you up.
    }
    \begin{itemize}
        \item Quantum-systems is a Python package containing modules to set up
            matrix elements, time-evolution operators, and single-particle
            states to be used by the many-body solvers.
            Quantum-systems provides interfaces to the PySCF \cite{pyscf} and
            Psi4 \cite{psi4} systems.
            The code is located at
            \url{https://github.com/Schoyen/quantum-systems} with the
            documentation at
            \url{https://schoyen.github.io/quantum-systems/}.
        \item Coupled-cluster is a Python package with modules containing ground
            state and time-dependent coupled cluster solvers.
            Currently this packages contains doubles, singles-and-doubles ,
            non-orthogonal coupled cluster doubles, and the orbital adaptive
            time-dependent coupled cluster doubles methods.
            The module uses quantum-systems to get access to matrix elements.
            The code is located at
            \url{https://github.com/Schoyen/coupled-cluster} with the
            documentation at
            \url{https://schoyen.github.io/coupled-cluster/}.
        \item Configuration-interaction is a package containing ground state and
            time-dependent configuration interaction solvers.
            This code supports arbitrary levels of excitations, e.g.,
            singles-and-doubles, doubles-and-triples, etc, and full
            configuration interaction.
            The module uses quantum-systems to get access to matrix elements.
            The code is located at
            \url{https://github.com/Schoyen/configuration-interaction} with
            the documentation at
            \url{https://schoyen.github.io/configuration-interaction/}.
        \item Hartree-Fock is a package containing ground state and
            time-dependent Hartree-Fock solvers.
            We've implemented Hartree-Fock for general, restricted and
            unrestricted spin-orbitals.
            The module uses quantum-systems to get access to matrix elements.
            The code is located at
            \url{https://github.com/Schoyen/hartree-fock} with
            the documentation at
            \url{https://schoyen.github.io/hartree-fock/}.
    \end{itemize}
    We will in the rest of this chapter discuss various aspects of the
    implementation that we deem important to elaborate on, but we will leave
    specific implementation details and usage of the solvers to the
    repositiories and the documentation.

    \section{Why Python?}
        In working with this thesis we have developed a large computational
        framework for performing real-time quantum mechanics simulations for
        many-body problems in the programming language Python \cite{python}.
        The choice of using Python comes with a list of pros and cons.
        \begin{itemize}
            \item The development time is much lower when using Python as
                opposed to more verbose, but efficient languages such as C/C++,
                and Fortran.
            \item Integration with other Python libraries are relatively easy.
            \item Libraries such as SciPy \cite{scipy}, NumPy \cite{numpy}, and
                SymPy \cite{sympy} provides fast, and efficient interfaces to
                BLAS and LAPACK along with convenient array abstractions.
        \end{itemize}

    \section{Computing tensor contractions}
        Quite a significant amount of computational resources will go into the
        evaluation of tensor contractions\footnote{%
            We tend to call the matrix elements for tensors, but they more
            resemble numerical $N$-dimensional arrays.%
            % TODO: Check this footnote.
        } and we will therefore spend some time discussing how these
        contractions are performed and how we can speed up the contractions.

        Consider the antisymmetric, two-body, Coulomb elements given by
        \begin{align}
            \twoten^{pq}_{rs}
            = \mel{\phi_p\phi_q}{\twohamil}{\phi_r\phi_s}_{AS},
        \end{align}
        which, along with the two-body density matrix, is the largest tensor in
        use.
        This tensor often represents the bottleneck both in terms of memory and
        contraction time.
        When we represent these tensors mathematically, the labelling of the
        indices are in some sense arbitrary and depends on the context that the
        tensors are used.
        On a computer we however need to decide on a specific way of storing
        memory, and often this choice is related to speed concerns where some
        storage schemes show vast improvements in terms of cache hits as opposed
        to other schemes.
        However, tensor contractions are notoriously difficult to handle in
        terms of memory as they often involve change of dimensionality,
        re-ordering of indices resulting in the need of reshapes, and summation
        along axes that are inefficiently laid out in memory.
        % TODO: Add demonstration of this from CCD
        For the sake of generality we therefore ignore much of these problems
        and have decided a fixed layout of the memory and absorb the cost of
        reshapes and memory allocation.
        In our programs we read the axis from top-left and moving right before
        starting on bottom-left and going right.
        That is, to access element $\twoten^{pq}_{rs}$ we use the ordering
        \pyth{u[p, q, r, s]} in NumPy-arrays \cite{numpy}.
        This ordering is used consistenly in all solver implementations.

        \subsection{Intermediates}
            It is common in the coupled cluster litterature to talk about
            \emph{intermediates} \cite{hjorth2017advanced, gauss1995coupled}, or
            intermediate computations, as a technique for speeding up tensor
            contractions involving three or more tensors.

    \section{Optimization techniques}
        \subsection{Newton's method}
        \subsection{DIIS acceleration}
    \section{Numerical integration}
        In this section we'll review a select few time-integration schemes for
        solving time-dependent ordinary differential equations and we'll discuss the
        applicability of each scheme.

        \subsection{Problem statement}
            The problem we are trying to solve can be formulated as the
            two coupled time-dependent Schrödinger equations on the form
            \begin{gather}
                \dpd{}{t}\ket{\psi(t)}
                = -i\hamil(t)\ket{\psi(t)},
                \\
                \bra{\psi(t)}\dpd{}{t}
                = i\bra{\psi(t)}\hamil(t),
            \end{gather}
            where we've assumed atomic units, and moved $\pm i$ to the right-hand
            side.
            In the case of the time-dependent Hartree-Fock and the
            time-dependent configuration interaction methods, we only need to
            evaluate the former Schrödinger equation as the wave functions are
            variational and $\bra{\psi(t)}$ will be the adjoint of
            $\ket{\psi(t)}$.
            For the coupled cluster methods we need to solve both equations as
            the bra and ket sides of the wave function are no longer the adjoint
            of one another.
            Furthermore, in the orbital-adaptive formulation we need to include
            the time-dependence of the coefficients used in the orbital
            rotations.
            However, we can stack all the coefficients and/or the amplitudes
            after one another creating a long vector as each index in this
            vector can be updated individually as long as the coupling is
            handled in the derivatives.
            Thus, we are able to formulate the time-evolution of the
            parameters of the wave function by
            \begin{align}
                \dot{y} = f(y, t),
            \end{align}
            where $f(y, t)$ represents the right-hand side the time-dependent
            Schrödinger equation, the time-dependent Hartree-Fock equation, or
            the right-hand sides of the derivative of the coupled cluster
            Lagrangian.
            Discretizing time with a constant step $h = \Delta t$ such that $t_i
            = t_0 + i h$ for $i \in \brac{0, N_t}$, where $N_t$ is the number of
            time-steps and $t_0$ is the initial time-step.
            Note that we typically choose a time-step $h$ such that $N_t$ can be
            found by
            \begin{align}
                N_t = \floor{\frac{t_f - t_0}{h}} + 1,
            \end{align}
            where $t_f$ is the final time-step.
            We denote $y_0 \equiv y(t_0)$ and $y_i \equiv y(t_i)$, where $y_0$
            is the initial value of the problem.
            Typically we choose the initial value to be the ground state of the
            specific solver.
            This abstraction thus enables us to work with known ordinary
            differential equation solvers formulated in a familiar way.

        \subsection{Runge-Kutta}
            As a first approximation to solving the time-dependent equations,
            we've implemented the Runge-Kutta 4 algorithm.
            This algorithm is given by \cite{wiki:rk4}
            \begin{gather}
                y_{i + 1} = y_i + \frac{1}{6}\para{
                    k_1 + 2 k_2 + 2 k_3 + k_4
                }, \\
                t_{i + 1} = t_i + h,
            \end{gather}
            where the equations for the $k$'s are given by
            \begin{gather}
                k_1 = h f(y_i, t_i), \\
                k_2 = h f(y_i + k_1 / 2, t_i + h / 2), \\
                k_3 = h f(y_i + k_2 / 2, t_i + h / 2), \\
                k_4 = h f(y_i + k_3, t_i + h).
            \end{gather}
            Runge-Kutta 4 is a fourth-order method with a local numerical error
            of the order of $\mathcal{O}(h^5)$ and a total error of
            $\mathcal{O}(h^4)$.

        \subsection{Symplectic integrators}
            Looking back to \autoref{sec:time-evolution-operators} we note that
            for a time-dependent Hamiltonian in the Schrödinger picture, the
            time-evolution of a state $\ket{\psi(t)}$ is described by the
            time-evolution operator shown in \autoref{eq:td-evolution}, where
            the Hamiltonian in general does not commute at different time-steps.
            It is perhaps somewhat naïve to hope for a numerical method such as
            Runge-Kutta 4 to yield good results without some extra concern for
            the problem at hand.
            As discussed by \citeauthor{joshua-magnus} \cite{joshua-magnus} the
            solution to the time-dependent Schrödinger equation can be better
            approximated by the Magnus expansion \cite{magnus-expansion}.
            This leads to implicit differential equation integration methods
            that are \emph{symplectic} \cite{joshua-magnus}.
            These methods will in a much large degree conserve the energy as
            opposed to the explicit schemes such as Runge-Kutta 4.
            % TODO: Demonstrate this.
            We will also observe that the symplectic integrator that we have
            used, in a much larger degree preserves the unitarity requirement
            by, i.e., the bi-orthonormality, of the time-dependent coefficients
            in the orbital-adaptive time-dependent coupled-cluster method.
            % TODO: Demonstrate this as well.
            One of the first approximations to the Magnus expansion is the
            Crank-Nicolson algorithm \cite{ullrich2011time}.

%        \subsection{Crank-Nicolson}
%            Continuing the derivation from section 4.5 in
%            \citeauthor{ullrich2011time} \cite{ullrich2011time}, we look at how we
%            can numerically propagate in time a wave function with a time-dependent
%            Hamiltonian.
%            Luckily, due to the property of the time evolution operators that they
%            can be composed of several intermediate time steps, discretizing the
%            operators becomes quite natural.
%            That is, if we define $\tau_1 = t_0$, and $\tau_N = t$ with all
%            intermediate times given by
%            \begin{align}
%                \tau_{j + 1} = \tau_j + \Delta \tau,
%            \end{align}
%            where the time step $\Delta \tau$ is given by
%            \begin{align}
%                \Delta \tau = \frac{t - t_0}{N - 1},
%            \end{align}
%            and $N$ is the number of discrete time steps. This means that we can
%            write
%            \begin{align}
%                \ket{\Psi(\tau_{j + 1})}
%                = \hat{U}(\tau_{j + 1}, \tau_{j})\ket{\Psi(\tau_j)},
%            \end{align}
%            where the equality is maintained. For a time-dependent Hamiltonian
%            (which is what we will be working on), the time evolution operator is
%            given by \autoref{eq:td-evolution}, but this is intractable in a
%            numerical scheme, to say the least. We therefore approximate the
%            operator by
%            \begin{align}
%                \hat{U}(\tau_{j + 1}, \tau_{j})
%                \approx \exp\brac{
%                    -\frac{i}{\hslash}\hat{H}(\tau_{j} + \Delta\tau/2)\Delta\tau
%                }
%                \equiv
%                \exp\brac{
%                    -\frac{i}{\hslash}\hat{H}(\tau_{j + 1/2})\Delta\tau
%                },
%            \end{align}
%            that is, we evaluate the Hamiltonian at the midpoint between $\tau_{j +
%            1}$ and $\tau_{j}$. The choice of evaluating the time step at the
%            midpoint comes about in order to preserve the unitarity of the
%            approximated time evolution operator. That is,
%            \begin{align}
%                \ket{\Psi(\tau_j)}
%                = \hat{U}(\tau_j, \tau_{j + 1})\ket{\Psi(\tau_{j + 1})}
%                = \hat{U}(\tau_j, \tau_{j + 1})\hat{U}(\tau_{j + 1}, \tau_j)
%                \ket{\Psi(\tau_j)},
%            \end{align}
%            which means that
%            \begin{align}
%                \hat{U}(\tau_j, \tau_{j + 1})
%                \hat{U}(\tau_{j + 1}, \tau_j)
%                = \hat{U}^{\dagger}(\tau_{j + 1}, \tau_j)
%                \hat{U}(\tau_{j + 1}, \tau_j)
%                = \1.
%            \end{align}
%            Using the Crank-Nicolson algorithm, we can approximate the exponential
%            in the approximated time evolution operator to be
%            \begin{align}
%                \hat{U}(\tau_{j + 1}, \tau_j)
%                \approx \exp\brac{
%                    -\frac{i}{\hslash}\hat{H}(\tau_{j + 1/2})\Delta \tau
%                }
%                \approx
%                \frac{1 - i\hat{H}(\tau_{j + 1/2})\Delta\tau/(2\hslash)}
%                {1 + i\hat{H}(\tau_{j + 1/2})\Delta\tau/(2\hslash)},
%            \end{align}
%            which is correct to second order in $\Delta \tau$ and preserves
%            unitarity. Expressed as system of linear equations, we then have
%            \begin{align}
%                \para{
%                    1 + \frac{i}{2\hslash}\hat{H}(\tau_{j + 1/2})\Delta\tau
%                }\ket{\Psi(\tau_{j + 1})}
%                =
%                \para{
%                    1 - \frac{i}{2\hslash}\hat{H}(\tau_{j + 1/2})\Delta\tau
%                }\ket{\Psi(\tau_{j})}.
%            \end{align}
%            We can either solve this equation by inversion, or as a system of linear
%            equations. The latter will be more stable.
%
        \subsection{Gauss-Legendre}
            In an article on the stability of the time-dependent coupled-cluster
            singles-and-doubles on atomic systems subject to intense laser
            pulses, \citeauthor{pedersen2018symplectic}
            \cite{pedersen2018symplectic} used symplectic, implicit Runge-Kutta
            methods in an attempt to alleviate some of the stability issues they
            faced.
            As part of an ongoing publication we have been given the
            % TODO: Cite publication in progress.
            implementation of the Gauss integrator\footnote{%
                Note that the name ``Gauss integrator'' is a notoriously
                ambigious name as this can refer to a multitude of techniques
                and solvers due to the use of named polynomials such as
                Legendre, and Laguerre polynomials, etc.%
            } as used by \citeauthor{pedersen2018symplectic}
            \cite{pedersen2018symplectic} to include it in our libraries.
