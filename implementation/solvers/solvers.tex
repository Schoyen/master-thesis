\chapter{Solver implementations}
    In this chapter we'll discuss various implementation aspects of the \emph{ab
    initio} solvers discussed in \autoref{chap:hf} through \autoref{chap:ci} and
    \autoref{chap:cc}.

    \section{Configuration interaction}
        As the main goal of this thesis has been to implement coupled-cluster
        solvers, the configuration interaction solver has not been worked at to
        such a large degree.
        We have therefore implemented a ``naïve'' configuration interaction
        solver where we create the full Slater determinant space and store it in
        memory.
        From this we also create the full Hamiltonian matrix $\hamilmat$.
        Our implementation thus quickly absorb too much memory and therefore
        limits the number of particles and basis functions that can be explored.
        To improve on the current scheme, an implementation of the \emph{direct
        CI} methods \cite{helgaker-molecular} along with only storing non-zero
        elements in $\hamilmat$ will yield a more powerfull method supporting
        more particles and basis functions.

        \subsection{Constructing the Slater determinant basis}
            We represent the Slater determinants as NumPy-arrays \cite{numpy} of
            bit strings using unsigned integers.
            The default choice is to use \pyth{np.uint64}, i.e., 64-bit unsigned
            integers with room for 64 single-particle states, but other options
            such as 32-bit and 16-bit unsignd integers are available.
            If we have a system with $L > 64$ we add more integers in the array
            thus allowing for an integer mutiple of $64$ single-particle states at
            a time.
            Let $b$ be the number of bits in an integer, then the number of
            integers needed for a single Slater determinant $N_i$ is given by
            \begin{align}
                N_i = \left\lfloor\frac{L}{b}\right\rfloor
                + q,
            \end{align}
            where $q$ is either one or zero by
            \begin{align}
                q = \begin{cases}
                    1 & L \mod b > 0, \\
                    0 & L \mod b = 0,
                \end{cases}
            \end{align}
            where $L \mod b$ is the remainder of the integer division.
            The number of Slater determinants $N_s$ is given by a recursive
            function $N_s(S)$ depending on the order $S$ of the truncation,
            \begin{align}
                N_s(S) = \begin{cases}
                    1, & S = 0, \\
                    N_s(S - 1) \frac{(N - [S - 1])(M - [S - 1])}{S^2}, & S > 0.
                \end{cases}
            \end{align}
            where $N$ is the number of particles, $M = L - N$ is the number of
            virtual states, and we've denoted the order $S$ as an integer where
            $1$ represents singles, $2$ doubles, and so forth.
            This formula counts the number of ways $N$ particles can be
            distributed among $M$ positions moving $S$ particles at a time.
            For a given truncation level, e.g., singles-and-doubles (CISD), the
            number of Slater determinants is then
            \begin{align}
                N_s = N_s(2) + N_s(1) + N_s(0),
            \end{align}
            where $N_s(0) = 1$ counts the reference state.
            In \autoref{tab:num-slater-determinants} we demonstrate how the
            number of Slater determinants increase as a function of truncation
            for a fixed number of particles.
            The storage cost of the Hamiltonian matrix $\hamilmat$ is uncanny
            going from CIS to CISDTQ as the storage increases by $7$ orders of
            magnitude.
            \begin{table}
                \centering
                \caption{In this table we demonstrate how the number of
                Slater determinants $N_s$ increase as a function of truncation
                level for $N = 4$ and $L = 80$.
                We've also included the number of bytes needed to store the
                Slater determinants using \pyth{np.uint64}, i.e., 64-bit
                unsigned integers to represent the determinants, and the size of
                the Hamiltonian matrix in bytes where we assume 128-bit complex
                numbers as elements.
                The storage cost of the Hamiltonian matrix for CIS was
                $\SI{0.001}{\giga\byte}$, which does not show up in the
                designated one decimal point.}
                \renewcommand{\arraystretch}{1.3}
                \begin{tabular}{@{}lrrr@{}}
                    \toprule
                    Truncation & $N_s$ & Determinant storage $[\si{\byte}]$
                    & Hamiltonian storage $[\si{\giga\byte}]$ \\
                    \midrule
                    CIS & $305$ & $2440$ & $0.0$ \\
                    CISD & $17405$ & $139240$ & $4.5$ \\
                    CISDT & $298605$ & $2388840$ & $1328.7$ \\
                    CISDTQ & $1581580$ & $12652640$ & $37273.7$ \\
                    \bottomrule
                \end{tabular}
                \label{tab:num-slater-determinants}
            \end{table}

            In our code we construct the Slater determinant basis by creating
            the reference determinant where we set the $N$ first bits in the
            array of unsigned integers and then create $N_s - 1$ copies of this
            state.
            The setting of a single-particle state in a bit string is done using
            the binary OR command.
            An example of the setting of single-particle states represented as
            bits is shown in \autoref{alg:set-state-68}.
            \begin{algorithm}
                \inputpython{implementation/solvers/determinants.py}{0}{13}
                \caption{An example of how set the single-particle state $68$ in
                a binary state array \pyth{state} using \pyth{np.uint64}
                integers to represent determinants.}
                \label{alg:set-state-68}
            \end{algorithm}
            Other options are to use the binary XOR operation, but whichever one
            is chosen some care must be shown as bugs can arise if the
            single-particle state is already set.
            In the case of the OR operation this does not change the state, but
            the XOR operation will remove the state.
            For this reason we use the XOR operation in order to unset a bit,
            i.e., remove a single-particle state.

            The higher excited determinants are created by exciting the
            reference determinant in a recursive fashion.
            The excitation operator for a single Slater determinant is shown in
            \autoref{ag:excite-state}.
            \begin{algorithm}
                \inputpython{implementation/solvers/determinants.py}{16}{23}
                \caption{Function used to represent a series of excitation
                operators $\hat{X}^{a}_{i}$, neglecting the sign.}
                \label{alg:excite-state}
            \end{algorithm}
            This function excites all single-particle states in the array
            \pyth{o_remove} to the single-particle states in \pyth{v_insert}.
            Note that the Slater determinants are interpreted as being in
            canonical ordering and we ignore the sign handling when creating the
            basis of determinants.
            The signs are thus handled when computing matrix elements.
            To populate the \pyth{o_remove}- and \pyth{v_insert}-arrays, we have
            a function which recursively adds an occupied index into
            \pyth{o_remove} and then proceeds to add all the virtual indices in
            order into \pyth{v_insert}, before calling the excitation function
            defined in \autoref{alg:excite-state}.
            This function is shown in
            \begin{algorithm}
                \inputpython{implementation/solvers/determinants.py}{26}{56}
                \caption{Function creating all excited determinants of a given
                order \pyth{order}.}
                \label{alg:create-excited-states}
            \end{algorithm}

            The entire implementation of the configuration interaction method is
            uniquely defined by the basis of Slater determinants.
            This means that after a truncation order has been chosen and the
            basis of Slater determinants has been constructed, everything that
            follows will be solved in the same manner independently of the
            truncation level.

        \subsection{Constructing the Hamiltonian matrix}
            The arguably most effective first order optimization that can be
            performed for the configuration interaction method is to implement
            the Slater-Condon rules when evaluating matrix elements of operator
            strings, as opposed to brute force evaluation of the action of the
            second quantized operators on a determinant.
            When constructing the Hamiltonian matrix $\hamilmat$ with one- and
            two-body operators, we wish to evaluate the matrix elements
            \begin{align}
                \hamilten_{IJ}
                = \mel{\Phi_{I}}{\hamil}{\Phi_J}
                =
                \oneten^{p}_{q}
                \mel{\Phi_{I}}{\ccr{p}\can{q}}{\Phi_J}
                +
                \frac{1}{4}
                \twoten^{pq}_{rs}
                \mel{\Phi_{I}}{
                    \ccr{p}
                    \ccr{q}
                    \can{r}
                    \can{s}
                }{\Phi_J},
            \end{align}
            using the Slater-Condon rules defined in
            \autoref{lemma:slater-condon-one-body} and
            \autoref{lemma:slater-condon-two-body}.
            Given two Slater determinants $\ket{\slat_I}$ and $\ket{\slat_J}$
            which we represent as two occupation number states $\ket{\vfg{n}}$
            and $\ket{\vfg{m}}$, respectively, we need ways to evaluate the
            following:
            \begin{itemize}
                \item The sign given by
                    \begin{align}
                        (\Gamma_{-})^{\vfg{n}}_{p}
                        = \prod_{i = 1}^{p - 1}(-1)^{n_i},
                    \end{align}
                    as defined in \autoref{def:creation_1}.
                \item The Kronecker-Delta $\delta_{p \in \vfg{n}}$ checking if
                    the single-particle state $p$ is an occupied state in
                    $\ket{\vfg{n}}$.
                \item The difference $\abs{\vfg{n} - \vfg{m}}$, between the two
                    determinants $\ket{\vfg{n}}$ and $\ket{\vfg{m}}$.
                \item Convert the position of a set bit to an index $p$.
            \end{itemize}
            For the sign calculation we use the product for the phase
            $(\Gamma_{-})^{\vfg{n}}_i$ defined in \autoref{def:creation_1} by
            counting the number of set bits $k$ at positions below $i$, and
            computing $(-1)^k$.
            An implementation of this sign calculation is shown in
            \autoref{alg:gamma-phase}.
            \begin{algorithm}
                \inputpython{implementation/solvers/determinants.py}{59}{72}
                \caption{Function computing the sign of the action of a creation
                or annihilation operator for index \pyth{p} on a determinant
                \pyth{state}.
                This is the binary implementation of the phase defined in
                \autoref{def:creation_1}.}
                \label{alg:gamma-phase}
            \end{algorithm}
            The implementation of the Kronecker-Delta is shown in
            \autoref{alg:kronecker-delta}.
            \begin{algorithm}
                \inputpython{implementation/solvers/determinants.py}{75}{81}
                \caption{Implementation of the Kronecker-Delta $\delta_{p \in
                \vfg{n}}$.}
                \label{alg:kronecker-delta}
            \end{algorithm}
            To compute the difference between the two determinants we start by
            using the XOR operation to find the specific bits that are only set
            in either $\vfg{n}$ or $\vfg{m}$.

        \subsection{Diagonalization}
        \subsection{Time-evolution}
