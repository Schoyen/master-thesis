\chapter{Solver implementations}
    In this chapter we'll discuss various implementation aspects of the \emph{ab
    initio} solvers discussed in \autoref{chap:hf} through \autoref{chap:ci} and
    \autoref{chap:cc}.

    \section{Configuration interaction}
        As the main goal of this thesis has been to implement coupled-cluster
        solvers, the configuration interaction solver has not been worked at to
        such a large degree.
        We have therefore implemented a ``naïve'' configuration interaction
        solver where we create the full Slater determinant space and store it in
        memory.
        From this we also create the full Hamiltonian matrix $\hamilmat$.
        Our implementation thus quickly absorb too much memory and therefore
        limits the number of particles and basis functions that can be explored.
        To improve on the current scheme, an implementation of the \emph{direct
        CI} methods \cite{helgaker-molecular} along with only storing non-zero
        elements in $\hamilmat$ will yield a more powerfull method supporting
        more particles and basis functions.

        \subsection{Constructing the Slater determinant basis}
            We represent the Slater determinants as NumPy-arrays \cite{numpy} of
            bit strings using unsigned integers.
            The default choice is to use \pyth{np.uint64}, i.e., 64-bit unsigned
            integers with room for 64 single-particle states, but other options
            such as 32-bit and 16-bit unsignd integers are available.
            If we have a system with $L > 64$ we add more integers in the array
            thus allowing for an integer mutiple of $64$ single-particle states at
            a time.
            Let $b$ be the number of bits in an integer, then the number of
            integers needed for a single Slater determinant $N_i$ is given by
            \begin{align}
                N_i = \left\lfloor\frac{L}{b}\right\rfloor
                + q,
            \end{align}
            where $q$ is either one or zero by
            \begin{align}
                q = \begin{cases}
                    1 & L \mod b > 0, \\
                    0 & L \mod b = 0,
                \end{cases}
            \end{align}
            where $L \mod b$ is the remainder of the integer division.
            The number of Slater determinants $N_s$ is given by a recursive
            function $N_s(S)$ depending on the order $S$ of the truncation,
            \begin{align}
                N_s(S) = \begin{cases}
                    1, & S = 0, \\
                    N_s(S - 1) \frac{(N - [S - 1])(M - [S - 1])}{S^2}, & S > 0.
                \end{cases}
            \end{align}
            where $N$ is the number of particles, $M = L - N$ is the number of
            virtual states, and we've denoted the order $S$ as an integer where
            $1$ represents singles, $2$ doubles, and so forth.
            This formula counts the number of ways $N$ particles can be
            distributed among $M$ positions moving $S$ particles at a time.
            For a given truncation level, e.g., singles-and-doubles (CISD), the
            number of Slater determinants is then
            \begin{align}
                N_s = N_s(2) + N_s(1) + N_s(0),
            \end{align}
            where $N_s(0) = 1$ counts the reference state.
            In \autoref{tab:num-slater-determinants} we demonstrate how the
            number of Slater determinants increase as a function of truncation
            for a fixed number of particles.
            The storage cost of the Hamiltonian matrix $\hamilmat$ is uncanny
            going from CIS to CISDTQ as the storage increases by $7$ orders of
            magnitude.
            \begin{table}
                \centering
                \caption{In this table we demonstrate how the number of
                Slater determinants $N_s$ increase as a function of truncation
                level for $N = 4$ and $L = 80$.
                We've also included the number of bytes needed to store the
                Slater determinants using \pyth{np.uint64}, i.e., 64-bit
                unsigned integers to represent the determinants, and the size of
                the Hamiltonian matrix in bytes where we assume 128-bit complex
                numbers as elements.}
                \renewcommand{\arraystretch}{1.3}
                \begin{tabular}{@{}lrrr@{}}
                    \toprule
                    Truncation & $N_s$ & Determinant storage $[\si{\byte}]$
                    & Hamiltonian storage $[\si{\giga\byte}]$ \\
                    \midrule
                    CIS & $305$ & $2440$ & $0.001$ \\
                    CISD & $17405$ & $139240$ & $4.5$ \\
                    CISDT & $298605$ & $2388840$ & $1328.7$ \\
                    CISDTQ & $1581580$ & $12652640$ & $37273.7$ \\
                    \bottomrule
                \end{tabular}
                \label{tab:num-slater-determinants}
            \end{table}

            In our code we construct the Slater determinant basis by creating
            the reference determinant where we set the $N$ first bits in the
            array of unsigned integers and then create $N_s - 1$ copies of this
            state.
            The higher excited determinants are created by exciting the
            reference determinant in a recursive fashion.

        \subsection{Constructing the Hamiltonian}
        \subsection{Diagonalization}
