\chapter{Introduction}
    Quantum mechanics is a theory that describes the microscopic realm of
    nature.
    Given the wave function $\psi(x, t)$ we can in principle compute all there
    is to know about the underlying system described by the Hamiltonian
    $\hamil(x, t)$, where $x$ is a spatial coordinate and $t$ the time.
    If we know the wave function at a given initial time $t_0$ we can from the
    time-dependent Schrödinger equation,
    \begin{align}
        i\hslash\dpd{}{t}\psi(x, t) = \hamil(x, t)\psi(x, t),
    \end{align}
    compute all there is to know about the system for all future and earlier
    times.
    Analytical solutions to the time-dependent Schrödinger equation are truly
    rare, and often idealized.
    This means that we must resort to numerical methods for more complex
    systems.
    However, moving to the time-independent Schrödinger equation we can find the
    spectrum of wave functions by solving the eigenvalue equation,
    \begin{align}
        \hamil\psi_k(x) = \energy_k\psi_k(x),
    \end{align}
    where $\energy_k$ is the eigenenergy of the eigenstate $\psi_k(x)$.
    We can then build the time-dependent wave function from a linear combination
    of these stationary states.
    The inclusion of many particles -- or the so-called many-body problem -- is
    of great interest in many fields of physics and quantum chemistry, but this
    severely complicates the matter of solving the time-independent Schrödinger
    equation, and not to say, the time-dependent Schrödinger equation.
    As a result there exists a plethora of approximate methods, e.g.,
    Hartree-Fock \cite{hartree_1928, szabo1996modern}, density functional
    \cite{ullrich2011time}, variational Monte-Carlo \cite{hjorth2017advanced},
    configuration interaction \cite{helgaker-molecular}, and coupled-cluster
    theory \cite{coester1958421}, which lets us locate the ground state, i.e.,
    the eigenstate of the time-independent Schrödinger equation with the lowest
    eigenenergy, with various degrees of accuracy.
    Some of these methods allow for a description of higher excited states, and
    we often categorize a spectrum of the time-independent Schrödinger equation
    as a solved system.
    However, such a description fails to explain the dynamics of a system.

    Real-time methods let us simulate experiments and describe the physics as
    they occur.
    These methods provide a way to explore the evolution of particles in time as
    they respond to interactions from external stimuli \cite{joshua-magnus}.
    Real-time simulations can be described as a way of preparing a system in a
    specific configuration, i.e., choosing an initial state $\psi(x)$, before
    evolving the system in time using the time-dependent Schrödinger equation by
    including time-dependent interactions in the Hamiltonian.
    With the evolution of attosecond physics and the invention of several bright
    light sources, our numerical experiments more and more resemble experiments
    done in the laboratory \cite{joachain2012atoms}.
    Due to the interest being in the many-body realm -- as we study atoms and
    molecules -- we must use an appropriate many-body method.
    However, for interacting fermionic systems under the influence of strong,
    time-dependent, external pertubations, e.g., intense laser pulses, many of
    the widely used many-body methods like perturbation theory fail to describe
    the physics of the systems.
    This leads us to study more advanced \emph{ab initio} methods like
    coupled-cluster theory.
    Coupled-cluster theory allows for a systematic inclusion of many-body
    correlations.
    Compared to for example large scale diagonalization methods coupled-cluster
    theory resums to infinite order selected and important correlations,
    allowing us to study larger basis sets and a higher number of interacting
    particles.


    \section{Goals}
        Studies of time-dependent quantum mechanical systems present several
        challenges to traditional many-body methods, and the main goal of this
        thesis is thus to implement a stable and efficient approach to
        time-evolving systems.
        We have chosen to work with coupled-cluster theory, a widely used and
        popular method in several fields of physics and quantum chemistry
        \cite{coester1958421, shavitt2009many, Hagen_2014, gauss1995coupled,
        lohne, kvaal2012ab, kvaal2013variational, hjorth2017advanced,
        helgaker-molecular}.
        In particular we have implemented -- for the first time -- the
        orbital-adaptive time-dependent coupled-cluster method with doubles
        excitations (OATDCCD) \cite{kvaal2012ab}.
        We have developed a versatile framework which allows to study several
        types of quantum mechanical systems.
        In this thesis we focus on studies of atoms and molecules subject to
        intense laser pulses.
        To simulate a time-evolving system we need an initial condition to start
        the simulation.
        Typically we choose the ground state of the system to be our initial
        condition.
        We therefore need to implement ground state solutions to the many-body
        problem.
        As the OATDCCD-method is an extension of other time-dependent
        coupled-cluster methods we have also implemented the coupled-cluster
        doubles (CCD), and the coupled-cluster singles-and-doubles (CCSD) ground
        state and time-dependent solvers.
        Furthermore, the time-dependent Hartree-Fock (TDHF) and time-dependent
        configuration interaction (TDCI) methods have been implemented for
        comparison.

    \section{Our contribution}
        There already exists a plethora of many-body codes, but almost none
        concerning real-time solutions to the electronic many-body problem.
        To the author's knowledge there are no existing implementations of the
        orbital-adaptive time-dependent coupled-cluster method applicable to
        general many-body problems.
        Our main contribution is therefore an implementation of this novel
        solver along with time-dependent coupled-cluster solvers in the doubles
        and singles-and-doubles approximation.
        As part of this work we have implemented several Hartree-Fock methods.
        We have implemented general, restricted, and unrestricted Hartree-Fock
        methods \cite{szabo1996modern}.
        We have also implemented a time-dependent general Hartree-Fock method
        \cite{hochstuhl2014time}.
        We have implemented the time-dependent configuration interaction method
        to arbitrary truncation level.
        We have implented one-dimensional quantum systems with arbitrary
        potentials, two-dimensional quantum dots in single and double wells
        along with magnetic fields \cite{greg-winther}.
        We have implemented interfaces towards the libraries PySCF \cite{pyscf}
        and Psi4 \cite{psi4}.
        This opens for usage of different types of basis sets and systems.
        The time evolution of our systems is general, but we have only
        implemented a description of an external laser field in the dipole
        approximation.
        This should be extensible to other types of time evolution, e.g., a
        time-dependent potential.

    \section{Thesis structure}
        This thesis has been structured into four parts:
        \begin{enumerate}[I.]
            \item A theory part reviewing central parts of quantum mechanics
                and many-body quantum mechanics that we deem important for an
                understanding of what has been done in this thesis.
                The three last chapters of this part describe both the ground
                state and time-dependent theories for the Hartree-Fock, the
                configuration interaction, and the coupled-cluster many-body
                methods.
                We have tried to keep these chapters as general as possible
                without a discussion of numerical and computational
                implementations.
            \item In part two we discuss computational aspects, implementation
                of quantum systems and many-body solvers.
                We try to describe the most central parts of the implementation,
                but leave out the bulk of the implementations as the sheer
                size\footnote{%
                    We have implemented approximately $15000$ lines of code
                    according to \texttt{cloc}
                    (\url{http://cloc.sourceforge.net/}).
                }
                of the libraries would clutter this entire thesis.
            \item Part three contains the results and discussions from the work
                in this thesis.
                The first chapter demonstrates the validity of our code.
                In the second chapter we discuss the stability of the novel
                OATDCCD-method.
                In the third chapter we display the quality of the implemented
                code by studying systems explored in the literature.
                We demonstrate how the methods can be used to explore exotic
                effects such as spin-dependent laser fields and ionization.
        \end{enumerate}
        A note to the expert reader; a large part of this thesis has been
        dedicated to the author's desire at understanding much of the underlying
        theory and mathematics of the implemented methods.
        This has led to a rather extensive tome of work without necessarily
        highlighting the important discoveries done.
        The expert reader can therefore consider chapters 2 and 3 more as a
        review of known quantum mechanics and many-body theories.

    \section{Disclaimer}
        There is only so much that can be done in a year as a master's student,
        and indeed much of the code has been developed in collaboration with
        other students and researchers.
        Much of our work builds on the work done by
        \citeauthor{kristiansen2017time} \cite{kristiansen2017time}.
        \citeauthor{kristiansen2017time} has provided invaluable guidance and
        help as a supervisor.
        Both the work done by \citeauthor{greg-winther} \cite{greg-winther} and
        myself use the same many-body methods developed in collaboration, but
        our work has diverged in terms of focus and results.
        However, the collaboration has proved fruitful in the sense that we have
        arguably reached further in our work as a team than going our separate
        ways.
        The novelty and applicability of the libraries we have developed has
        spawned interest with researchers at the Hylleraas Center at the
        University of Oslo.
        This has led to several researchers using the code, and as a consequence
        they have provided us with valuable feedback.
        thus making the implementation more robust.
        Furthermore, we have received working implementations of the
        non-orthogonal coupled-cluster doubles (NOCCD) method \cite{rolf-nocc},
        the direct-inversion of the iterative subspace (DIIS) acceleration
        \cite{rolf-nocc}, and the Gauss-Legendre \cite{pedersen2018symplectic}
        integrator which we have integrated in our framework.
