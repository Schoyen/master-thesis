\chapter{Introduction}
    Quantum mechanics is a theory that describes the microscopical realm of
    nature.
    Given the wave function $\psi(x, t)$ we can in principle compute all there
    is to know about the underlying system described by the Hamiltonian
    $\hamil(t)$.
    If we know the wave function at a given initial time we can from the
    time-dependent Schrödinger equation,
    \begin{align}
        i\hslash\dpd{}{t}\psi(x, t) = \hamil(t)\psi(x, t),
    \end{align}
    compute all there is to know about the system for all future and earlier
    times.
    Analytical solutions to the time-dependent Schrödinger equation are truly
    rare, and often too small to be of much use.
    However, moving to the time-independent Schrödinger equation we can find the
    spectrum of wave functions by solving the eigenvalue equation
    \begin{align}
        \hamil\psi(x) = \energy\psi(x).
    \end{align}
    We can then build the time-dependent wave function from a linear combination
    of these stationary states.
    The inclusion of many particles -- or the so-called many-body problem -- are
    of great interest in quantum chemistry and nuclear physics, but this severly
    complicates the matter of solving the time-independent Schrödinger equation,
    and not to say, the time-dependent Schrödinger equation.
    As a result there exists a plethora of approximate methods, e.g.,
    Hartree-Fock, density functional, variational Monte-Carlo, configuration
    interaction, and coupled-cluster theory, which lets us locate the ground
    state, i.e., the eigenstate of the time-independent Schrödinger equation
    with the lowest eigenenergy, with various degrees of accuracy.
    % TODO: YOu are here


    % TODO: Something, something real-time.
    % TODO: Something, something perturbation theory.


    \section{Goals}
        The main goal of this thesis is to implement the orbital-adaptive
        time-dependent coupled-cluster method with doubles excitations (OATDCCD)
        \cite{kvaal2012ab}, and apply it to one- and two-dimensional quantum
        dots, and atoms and molecules.
        To simulate a time-evolving system we need an initial condition to start
        the simulation.
        We choose the ground state of the system to be our initial condition.
        We therefore need to implement ground state solutions to the many-body
        problem.

    \section{Our contribution}
        There already exists a plethora of many-body codes, but few exist that
        concern itself with real-time solutions to the electronic many-body
        problem.
        To the authors knowledge there are no existing implementations of the
        orbital-adaptive time-dependent coupled-cluster method applicable to
        general many-body problems.
        Our main contribution is therefore an implementation of this novel
        solver along with time-dependent coupled-cluster solvers in the doubles
        and singles-and-doubles approximation.
        We have also implemented the

    \section{Thesis structure}
        Would be nice!

    \section{Disclaimer}
        There is only so much that can be done in a year as a master's student,
        and indeed much of the code has been developed in collaboration with
        other students and researchers.
        Much of our work builds on the work done by Håkon Emil Kristiansen
        \cite{kristiansen2017time}, and Håkon has provided invaluable guidance
        and help as a supervisor in both validation and verification of the
        implemented methods.
        Both the work done by Sebastian Gregorius Winther-Larsen
        \cite{greg-winther} and myself use the same many-body methods developed
        in collaboration, but our work has diverged in terms of focus and
        results.
        % TODO: Should a list of implementations be included?
        However, the collaboration has proved fruitful in the sense that we have
        arguably reached further in our work as a team than going our separate
        ways.

        The novelty and applicability of the libraries we have developed has
        spawned interest with researchers at the Hylleraas Centre at the
        University of Oslo which has led to several researchers using the code.
        As a consequence they have provided us with much feedback on the code
        thus making the implementation more robust.
        Furthermore, we have recieved working implementations of the
        non-orthogonal coupled cluster doubles (NOCCD) method \cite{rolf-nocc},
        the direct-inversion of the iterative subspace (DIIS) acceleration
        \cite{rolf-nocc}, and the Gauss-Legendre \cite{pedersen2018symplectic}
        integrator to include in our framework.
