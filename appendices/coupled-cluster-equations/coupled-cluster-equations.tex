\chapter{Coupled cluster equations}
    In this appendix we will show the explicit equations used for ground state
    calculations in coupled cluster for different truncation levels.

    \section{Energy equations}
        \label{app:cc-energy-equations}
        In this section we demonstrate how we can derive the energy contribution
        to the coupled cluster energy from the contribution linear in the
        cluster operator in \autoref{eq:cc-energy-linear-contrib} and the squared
        cluster contribution in \autoref{eq:cc-energy-squared-contrib}.
        As stated in \autoref{subsec:cc-energy-equations}, we need only concern
        ourselves with the case where the cluster operator is given by
        \begin{align}
            \clust
            &= \clust_1 + \clust_2
            = \clustamp^{a}_{i}\brac{\ccr{a}\can{i}}
            + \frac{1}{4}\clustamp^{ab}_{ij}\brac{\ccr{a}\ccr{b}\can{j}\can{i}},
        \end{align}
        where we note that the cluster operators are normal-ordered by
        construction.
        Looking at the energy contribution linear in the cluster operator we get
        \begin{align}
            \bra{\slat}\brak{
                \hamil_N
                \clust
            }_c\ket{\slat}
            &=
            \bra{\slat}\brak{
                \hamil_N
                \clust_1
            }_c
            + \brak{
                \hamil_N
                \clust_2
            }_c\ket{\slat}
            \\
            &=
            \bra{\slat}\brak{
                \fock_N
                \clust_1
            }_c
            + \brak{
                \twohamil_N
                \clust_2
            }_c\ket{\slat},
        \end{align}
        where we have collected the non-zero contributions.
        As we are projecting onto the reference determinant, we are dependent on
        the operators being fully contracted.
        This means that all non-contracted operators will destroy the overlap
        and hence the doubles cluster operator cannot couple with the
        Fock-operator nor can a single singles cluster operator couple with the
        two-body-operator.
        Looking at each operator pair separately we get
        \begin{align}
            \bra{\slat}\brak{
                \fock_N\clust_1
            }_c\ket{\slat}
            &=
            \fockten^{p}_{q}\clustamp^{a}_{i}
            \bra{\slat}\brac{
                \ccr{p}\can{q}
            }\brac{
                \ccr{a}
                \can{i}
            }\ket{\slat}
            \\
            &=
            \fockten^{p}_{q}\clustamp^{a}_{i}
            \bra{\slat}
            \wick{
                \Big\{
                    \c2 {\ccr{p}}
                    \c1 {\can{q}}
                \Big\}
                \Big\{
                    \c1 {\ccr{a}}
                    \c2 {\can{i}}
                \Big\}
            }
            \ket{\slat}
            \\
            &=
            \fockten^{p}_{q}\clustamp^{a}_{i}
            \delta_{qa}\delta_{pi}
            = \fockten^{i}_{a}\clustamp^{a}_{i}.
        \end{align}
        The doubles operator on the two-body operator yields
        \begin{align}
            \bra{\slat}\brak{
                \twohamil_N \clust_2
            }_c
            \ket{\slat}
            &=
            \frac{1}{16}\twoten^{pq}_{rs}\clustamp^{ab}_{ij}
            \bra{\slat}
            \brac{
                \ccr{p}
                \ccr{q}
                \can{s}
                \can{r}
            }
            \brac{
                \ccr{a}
                \ccr{b}
                \can{j}
                \can{i}
            }
            \ket{\slat},
        \end{align}
        where the factor $1/16$ comes from the two factors $1/4$ of the doubles
        amplitudes and the two-body operator to avoid overcounting.
        For the sake of brevity we will restrict our attention to the operator
        strings when utilizing Wick's theorem.
        \begin{align}
            \brac{
                \ccr{p}
                \ccr{q}
                \can{s}
                \can{r}
            }
            \brac{
                \ccr{a}
                \ccr{b}
                \can{j}
                \can{i}
            }
            &=
            \wick{
                \Big\{
                    \c4 {\ccr{p}}
                    \c3 {\ccr{q}}
                    \c2 {\can{s}}
                    \c1 {\can{r}}
                \Big\}
                \Big\{
                    \c1 {\ccr{a}}
                    \c2 {\ccr{b}}
                    \c3 {\can{j}}
                    \c4 {\can{i}}
                \Big\}
            }
            \nonumber \\
            &\qquad
            +
            \wick{
                \Big\{
                    \c4 {\ccr{p}}
                    \c3 {\ccr{q}}
                    \c2 {\can{s}}
                    \c1 {\can{r}}
                \Big\}
                \Big\{
                    \c2 {\ccr{a}}
                    \c1 {\ccr{b}}
                    \c3 {\can{j}}
                    \c4 {\can{i}}
                \Big\}
            }
            \nonumber \\
            &\qquad
            +
            \wick{
                \Big\{
                    \c4 {\ccr{p}}
                    \c3 {\ccr{q}}
                    \c2 {\can{s}}
                    \c1 {\can{r}}
                \Big\}
                \Big\{
                    \c1 {\ccr{a}}
                    \c2 {\ccr{b}}
                    \c4 {\can{j}}
                    \c3 {\can{i}}
                \Big\}
            }
            \nonumber \\
            &\qquad
            +
            \wick{
                \Big\{
                    \c4 {\ccr{p}}
                    \c3 {\ccr{q}}
                    \c2 {\can{s}}
                    \c1 {\can{r}}
                \Big\}
                \Big\{
                    \c2 {\ccr{a}}
                    \c1 {\ccr{b}}
                    \c4 {\can{j}}
                    \c3 {\can{i}}
                \Big\}
            }
            \\
            &=
            \delta_{ra}
            \delta_{sb}
            \delta_{qj}
            \delta_{pi}
            -
            \delta_{rb}
            \delta_{sa}
            \delta_{qj}
            \delta_{pi}
            \nonumber \\
            &\qquad
            -
            \delta_{ra}
            \delta_{sb}
            \delta_{qi}
            \delta_{pj}
            +
            \delta_{rb}
            \delta_{sa}
            \delta_{qi}
            \delta_{pj}.
        \end{align}
        Inserting the Kronecker-delta functions back into the energy
        contribution and summing, we get
        \begin{align}
            \bra{\slat}\brak{
                \twohamil_N \clust_2
            }_c
            \ket{\slat}
            &=
            \frac{1}{16}\brak{
                \twoten^{ij}_{ab}
                - \twoten^{ij}_{ba}
                - \twoten^{ji}_{ab}
                + \twoten^{ji}_{ba}
            }\clustamp^{ab}_{ij}
            =
            \frac{1}{4}\twoten^{ij}_{ab}\clustamp^{ab}_{ij},
        \end{align}
        where we've used the antisymmetric properties of the two-body
        elements.

        Moving on to the squared cluster operator contribution we note that the
        only non-zero contribution to the energy can come from the singles
        cluster operator as this provides a doubly excited state from the
        reference state.
        This also means that we can only get a coupling with the two-body
        operator.
        We are thus left with
        \begin{align}
            \bra{\slat}\brak{
                \hamil_N \clust^2
            }_c
            \ket{\slat}
            &=
            \bra{\slat}\brak{
                \twohamil_N \clust_1^2
            }_c
            \ket{\slat}
            \\
            &=
            \frac{1}{4}\twoten^{pq}_{rs}\clustamp^{a}_{i}\clustamp^{b}_{j}
            \bra{\slat}
            \brac{
                \ccr{p}
                \ccr{q}
                \can{s}
                \can{r}
            }
            \brac{
                \ccr{a}
                \can{i}
            }
            \brac{
                \ccr{b}
                \can{j}
            }
            \ket{\slat}
            \\
            &=
            \twoten^{ij}_{ab}\clustamp^{a}_{i}\clustamp^{b}_{j},
        \end{align}
        where we note that the same contractions as in the previous term is
        performed for the squared cluster operators.
        Collecting all the contributions to the correlation energy we get
        \begin{align}
            \bra{\slat}\simhamil_N\ket{\slat}
            &=
            \bra{\slat}\brak{
                \hamil_N\clust
            }_{c}\ket{\slat}
            + \frac{1}{2!}
            \bra{\slat}\brak{
                \hamil_N\clust^2
            }_{c}\ket{\slat}
            \\
            &=
            \fockten^{i}_{a}\clustamp^{a}_{i}
            + \half \twoten^{ij}_{ab}\para{
                \clustamp^{ab}_{ij}
                + \clustamp^{a}_{i}\clustamp^{b}_{j}
            },
        \end{align}
        which is what we wanted to show.

    \section{Coupled cluster doubles}
        The energy for the doubles truncation is given by
        \begin{align}
            \energyccd
            &= \energyref
            + \braslat e^{-\clust_2}\HN e^{\clust_2}\ketslat
            = \energyref
            + \frac{1}{4}\clustamp^{cd}_{kl} u^{kl}_{cd}.
        \end{align}
        The doubles amplitude equations is given by\cite{shavitt2009many}
        \begin{align}
            0 &= g(f, u, \clustamp)
            \equiv \braslate{ab}{ij}e^{-\clust_2}\HN e^{\clust_2}\ketslat
            \\
            &=
            u^{ab}_{ij}
            + f^{b}_{c} \clustamp^{ac}_{ij} P(ab)
            - f^{k}_{j} \clustamp^{ab}_{ik} P(ij)
            + \half \clustamp^{cd}_{ij} u^{ab}_{cd}
            + \half \clustamp^{ab}_{kl} u^{kl}_{ij}
            \nonumber \\
            &\qquad
            + \clustamp^{ac}_{ik} u^{bk}_{jc} P(ab) P(ij)
            + \frac{1}{4} \clustamp^{cd}_{ij} \clustamp^{ab}_{kl} u^{kl}_{cd}
            + \clustamp^{ac}_{ik} \clustamp^{bd}_{jl} u^{kl}_{cd} P(ij)
            \nonumber \\
            &\qquad
            - \half \clustamp^{ab}_{lj} \clustamp^{dc}_{ik} u^{kl}_{cd} P(ij)
            - \half \clustamp^{ac}_{lk} \clustamp^{db}_{ij} u^{kl}_{cd} P(ab).
        \end{align}
        In order to reduce the number of FLOPS when contracting the tensors, we
        introduce so-called \emph{intermediates}\cite{hjorth2017advanced}. In
        practice, this consists of precomputing some of the terms by choosing
        which tensors to contract first. In the doubles approximation there are
        four sensible intermediates we can define\footnote{Note that we use the
        notation "$\gets$" to signify part of the expression, i.e., some of the
        terms contained in the function.}.
        \begin{gather}
            g(f, u, \clustamp)
            \gets
            \frac{1}{4} \clustamp^{cd}_{ij} \clustamp^{ab}_{kl} u^{kl}_{cd}
            + \half \clustamp^{cd}_{ij} u^{ab}_{cd}
            =
            \clustamp^{cd}_{ij}\para{
                \frac{1}{4} \clustamp^{ab}_{kl} u^{kl}_{cd}
                + \half u^{ab}_{cd}
            }
            = \clustamp^{cd}_{ij}W^{ab}_{cd},
            \\
            g(f, u, \clustamp)
            \gets
            \half \clustamp^{cd}_{jk} \clustamp^{ab}_{il} u^{kl}_{cd} P(ij)
            = \clustamp^{ab}_{il}\para{
                \half \clustamp^{cd}_{jk} u^{kl}_{cd}
            } P(ij)
            = \clustamp^{ab}_{il} W^{l}_{j} P(ij),
            \\
            g(f, u, \clustamp)
            \gets
            \half \clustamp^{ac}_{ij} \clustamp^{bd}_{kl} u^{kl}_{cd} P(ab)
            = \clustamp^{ac}_{ij}\para{
                \half \clustamp^{bd}_{kl} u^{kl}_{cd}
            } P(ab)
            = \clustamp^{ac}_{ij} W^{b}_{c} P(ab).
        \end{gather}
        The last intermediate requires a little work, as we have to insert an
        extra exchange operator, $P(ij)$, in one of the terms in order to group
        two terms into a single intermediate.
        \begin{align}
            g(f, u, \clustamp)
            &\gets
            \clustamp^{ac}_{ik} \clustamp^{bd}_{jl} u^{kl}_{cd} P(ab)
            + \clustamp^{ac}_{ik} u^{bk}_{jc} P(ab) P(ij)
            \\
            &= \clustamp^{ac}_{ik}\para{
                \half \clustamp^{bd}_{jl} u^{kl}_{cd}
                + u^{bk}_{jc}
            } P(ab) P(ij)
            \\
            &= \clustamp^{ac}_{ik} W^{bk}_{jc} P(ab) P(ij).
        \end{align}
        We summarize the expression for the intermediates in
        \autoref{tab:intermediates_ccd}, along with their computational
        complexity. The total right-hand side of the $\clustamp$-amplitudes in the
        doubles approximation using the intermediate calculations is thus
        \begin{align}
            g(f, u, \clustamp)
            &=
            f^{b}_{c} \clustamp^{ac}_{ij} P(ab)
            - f^{k}_{j} \clustamp^{ab}_{ik} P(ij)
            + \clustamp^{cd}_{ij} W^{ab}_{cd}
            + \clustamp^{ab}_{il} W^{l}_{j} P(ij)
            \nonumber \\
            &\qquad
            + \clustamp^{ac}_{ik} W^{bk}_{jc} P(ab) P(ij)
            - \clustamp^{ac}_{ij} W^{b}_{c} P(ab)
            + \half \clustamp^{ab}_{kl} u^{kl}_{ij}
            + u^{ab}_{ij}.
        \end{align}
        The most time-consuming contraction is now the term $\clustamp^{cd}_{ij}
        W^{ab}_{cd}$. This term uses $\mathcal{O}(m^4 n^2)$ FLOPS, which is a
        reduction from $\mathcal{O}(m^4 n^4)$ when computing $\clustamp^{cd}_{ij}
        \clustamp^{ab}_{kl} u^{kl}_{cd}$ directly.

        \begin{table}
            \centering
            \caption{In this table we summarize the intermediates used in
            the $\clustamp$-amplitudes of the coupled cluster doubles approximation. We
            also list the computational complexity in FLOPS needed to construct
            the intermediate. Recall that $n$ is the number of holes and $m$ the
            number of particles.}
            \begin{tabular}{cc}
                Intermediate & Complexity [FLOPS] \\
                \midrule
                $W^{ab}_{cd}
                = \frac{1}{4} \clustamp^{ab}_{kl} u^{kl}_{cd}
                + \half u^{ab}_{cd}$
                &
                $\mathcal{O}(m^4 n^2)$ \\
                \addlinespace
                $W^{l}_{j}
                = \half \clustamp^{cd}_{jk} u^{kl}_{cd}$
                &
                $\mathcal{O}(m^2 n^3)$
                \\
                \addlinespace
                $W^{b}_{c}
                = \half \clustamp^{bd}_{kl} u^{kl}_{cd}$
                &
                $\mathcal{O}(m^3 n^2)$
                \\
                \addlinespace
                $W^{bk}_{jc}
                = \half \clustamp^{bd}_{jl} u^{kl}_{cd}
                + u^{bk}_{jc}$
                &
                $\mathcal{O}(m^3n^3)$
            \end{tabular}
            \label{tab:intermediates_ccd}
        \end{table}
        Moving on to the $\clustl$-equations we have for the doubles
        approximation
        \begin{align}
            0 &=
        \end{align}

    \section{Coupled cluster singles doubles}
        The energy for the singles-doubles truncation is given by
        \begin{align}
            \energyccsd
            &= \energyref
            + \braslat e^{-\clust_1 - \clust_2} \HN e^{\clust_1 + \clust_2}
            \ketslat
            \\
            &=
            \energyref
            +
            f^{k}_{c} \clustamp^{c}_{k}
            - \half \clustamp^{c}_{l} \clustamp^{d}_{k} u^{kl}_{cd}
            + \frac{1}{4} \clustamp^{cd}_{kl} u^{kl}_{cd}.
        \end{align}
        The singles amplitude equations becomes
        \begin{align}
            0 & = \braslate{a}{i} e^{-\clust_1 - \clust_2} \HN
            e^{\clust_1 + \clust_2}\ketslat
            \\
            &=
            f^{a}_{i}
            - f^{k}_{c} \clustamp^{c}_{i} \clustamp^{a}_{k}
            + f^{k}_{c} \clustamp^{ac}_{ik}
            - f^{k}_{i} \clustamp^{a}_{k}
            + f^{a}_{c} \clustamp^{c}_{i}
            - \clustamp^{c}_{k} \clustamp^{d}_{i} \clustamp^{a}_{l} u^{kl}_{cd}
            - \clustamp^{c}_{k} \clustamp^{d}_{i} u^{ak}_{cd}
            \nonumber \\
            &\qquad
            + \clustamp^{c}_{k} \clustamp^{a}_{l} u^{kl}_{ic}
            + \clustamp^{c}_{k} \clustamp^{ad}_{il} u^{kl}_{cd}
            + \clustamp^{c}_{k} u^{ak}_{ic}
            - \frac{1}{2} \clustamp^{c}_{i} \clustamp^{ad}_{kl} u^{kl}_{cd}
            \nonumber \\
            &\qquad
            + \frac{1}{2} \clustamp^{a}_{l} \clustamp^{cd}_{ik} u^{kl}_{cd}
            + \frac{1}{2} \clustamp^{cd}_{ik} u^{ak}_{cd}
            - \frac{1}{2} \clustamp^{ac}_{kl} u^{kl}_{ic}.
        \end{align}
        % TODO: Introduce intermediates
        The doubles amplitude equations are
        \begin{align}
            0 & = \braslate{ab}{ij} e^{-\clust_1 - \clust_2} \HN
            e^{\clust_1 + \clust_2}\ketslat
            \\
            &=
            f^{k}_{c} \clustamp^{c}_{i} \clustamp^{ab}_{jk} P(ij)
            + f^{k}_{c} \clustamp^{a}_{k} \clustamp^{bc}_{ij} P(ab)
            + f^{k}_{i} \clustamp^{ab}_{jk} P(ij)
            - f^{a}_{c} \clustamp^{bc}_{ij} P(ab)
            \nonumber \\
            &\qquad
            + \clustamp^{c}_{k} \clustamp^{d}_{i} \clustamp^{ab}_{jl} u^{kl}_{cd} P(ij)
            + \clustamp^{c}_{k} \clustamp^{a}_{l} \clustamp^{bd}_{ij} u^{kl}_{cd} P(ab)
            - \clustamp^{c}_{k} \clustamp^{ad}_{ij} u^{bk}_{cd} P(ab)
            \nonumber \\
            &\qquad
            + \clustamp^{c}_{k} \clustamp^{ab}_{il} u^{kl}_{jc} P(ij)
            + \clustamp^{c}_{i} \clustamp^{d}_{j} \clustamp^{a}_{k} \clustamp^{b}_{l} u^{kl}_{cd}
            + \clustamp^{c}_{i} \clustamp^{d}_{j} \clustamp^{a}_{k} u^{bk}_{cd} P(ab)
            + \frac{1}{2} \clustamp^{c}_{i} \clustamp^{d}_{j} \clustamp^{ab}_{kl} u^{kl}_{cd} 1
            \nonumber \\
            &\qquad
            + \clustamp^{c}_{i} \clustamp^{d}_{j} u^{ab}_{cd}
            - \clustamp^{c}_{i} \clustamp^{a}_{k} \clustamp^{b}_{l} u^{kl}_{jc} P(ij)
            - \clustamp^{c}_{i} \clustamp^{a}_{k} \clustamp^{bd}_{jl} u^{kl}_{cd} P(ab) P(ij)
            \nonumber \\
            &\qquad
            - \clustamp^{c}_{i} \clustamp^{a}_{k} u^{bk}_{jc} P(ab) P(ij)
            - \clustamp^{c}_{i} \clustamp^{ad}_{jk} u^{bk}_{cd} P(ab) P(ij)
            - \frac{1}{2} \clustamp^{c}_{i} \clustamp^{ab}_{kl} u^{kl}_{jc} P(ij)
            \nonumber \\
            &\qquad
            - \clustamp^{c}_{i} u^{ab}_{jc} P(ij)
            + \frac{1}{2} \clustamp^{a}_{k} \clustamp^{b}_{l} \clustamp^{cd}_{ij} u^{kl}_{cd}
            + \clustamp^{a}_{k} \clustamp^{b}_{l} u^{kl}_{ij}
            + \frac{1}{2} \clustamp^{a}_{k} \clustamp^{cd}_{ij} u^{bk}_{cd} 1 P(ab)
            \nonumber \\
            &\qquad
            + \clustamp^{a}_{k} \clustamp^{bc}_{il} u^{kl}_{jc} P(ab) P(ij)
            + \clustamp^{a}_{k} u^{bk}_{ij} P(ab)
            + \frac{1}{4} \clustamp^{cd}_{ij} \clustamp^{ab}_{kl} u^{kl}_{cd} 1
            + \frac{1}{2} \clustamp^{cd}_{ij} u^{ab}_{cd}
            \nonumber \\
            &\qquad
            + \frac{1}{2} \clustamp^{cd}_{jk} \clustamp^{ab}_{il} u^{kl}_{cd} P(ij)
            + \clustamp^{ac}_{ik} \clustamp^{bd}_{jl} u^{kl}_{cd} P(ab)
            + \clustamp^{ac}_{ik} u^{bk}_{jc} P(ab) P(ij)
            \nonumber \\
            &\qquad
            - \frac{1}{2} \clustamp^{ac}_{ij} \clustamp^{bd}_{kl} u^{kl}_{cd} P(ab)
            + \frac{1}{2} \clustamp^{ab}_{kl} u^{kl}_{ij} 1
            + u^{ab}_{ij}.
        \end{align}
        % TODO: Introduce intermediates

    \section{Coupled cluster doubles triples}
        The energy of the doubles-triples truncation is given by
        \begin{align}
            \energyccdt
            &= \energyref
            + \braslat e^{-\clust_2 - \clust_3} \HN e^{\clust_2 + \clust_3}
            \ketslat
            \\
            &=
            E_0
            + \frac{1}{4} \clustamp^{de}_{lm} u^{lm}_{de}.
        \end{align}
        The doubles amplitude equations are
        \begin{align}
            0 & = \braslate{ab}{ij} e^{-\clust_2 - \clust_3} \HN
            e^{\clust_2 + \clust_3}\ketslat
            \\
            &=
            f^{l}_{d} \clustamp^{abd}_{ijl}
            + f^{l}_{i} \clustamp^{ab}_{jl} P(ij)
            - f^{a}_{d} \clustamp^{bd}_{ij} P(ab)
            + \frac{1}{4} \clustamp^{de}_{ij} \clustamp^{ab}_{lm} u^{lm}_{de}
            \nonumber \\
            &\qquad
            + \frac{1}{2} \clustamp^{de}_{ij} u^{ab}_{de}
            + \frac{1}{2} \clustamp^{de}_{jl} \clustamp^{ab}_{im} u^{lm}_{de} P(ij)
            + \clustamp^{ad}_{il} \clustamp^{be}_{jm} u^{lm}_{de} P(ab)
            \nonumber \\
            &\qquad
            + \clustamp^{ad}_{il} u^{bl}_{jd} P(ab) P(ij)
            - \frac{1}{2} \clustamp^{ad}_{ij} \clustamp^{be}_{lm} u^{lm}_{de} P(ab)
            + \frac{1}{2} \clustamp^{ab}_{lm} u^{lm}_{ij}
            \nonumber \\
            &\qquad
            + \frac{1}{2} \clustamp^{ade}_{ijl} u^{bl}_{de} P(ab)
            - \frac{1}{2} \clustamp^{abd}_{ilm} u^{lm}_{jd} P(ij)
            + u^{ab}_{ij}.
        \end{align}
        Now to the beast...

        The triples amplitude equations are
        \begin{align}
            0 & = \braslate{abc}{ijk} e^{-\clust_2 - \clust_3} \HN
            e^{\clust_2 + \clust_3}\ketslat
            \\
            &=
            f^{l}_{d} \clustamp^{ad}_{ij} \clustamp^{bc}_{kl} P(ab) P(ik)
            + f^{l}_{d} \clustamp^{ab}_{il} \clustamp^{cd}_{jk} P(ij)
            + f^{l}_{d} \clustamp^{ab}_{kl} \clustamp^{cd}_{ij}
            + f^{l}_{d} \clustamp^{ac}_{jl} \clustamp^{bd}_{ik} P(ab)
            \nonumber \\
            &\qquad
            - f^{l}_{i} \clustamp^{abc}_{jkl} P(ij)
            - f^{l}_{k} \clustamp^{abc}_{ijl}
            + f^{a}_{d} \clustamp^{bcd}_{ijk} P(ab)
            + f^{c}_{d} \clustamp^{abd}_{ijk}
            \nonumber \\
            &\qquad
            + \frac{1}{2} \clustamp^{de}_{il} \clustamp^{abc}_{jkm} u^{lm}_{de} P(ij)
            + \frac{1}{2} \clustamp^{de}_{ij} \clustamp^{ab}_{kl} u^{cl}_{de}
            - \frac{1}{2} \clustamp^{de}_{ij} \clustamp^{ac}_{kl} u^{bl}_{de} P(ab)
            \nonumber \\
            &\qquad
            + \frac{1}{4} \clustamp^{de}_{ij} \clustamp^{abc}_{klm} u^{lm}_{de}
            - \frac{1}{4} \clustamp^{de}_{ik} \clustamp^{abc}_{jlm} u^{lm}_{de} P(ij)
            + \frac{1}{2} \clustamp^{de}_{jk} \clustamp^{ab}_{il} u^{cl}_{de} P(ij)
            \nonumber \\
            &\qquad
            - \frac{1}{2} \clustamp^{de}_{jk} \clustamp^{ac}_{il} u^{bl}_{de} P(ab) P(ij)
            + \frac{1}{2} \clustamp^{de}_{kl} \clustamp^{abc}_{ijm} u^{lm}_{de}
            + \frac{1}{2} \clustamp^{ad}_{lm} \clustamp^{bce}_{ijk} u^{lm}_{de} P(ab)
            \nonumber \\
            &\qquad
            + \clustamp^{ad}_{il} \clustamp^{bc}_{jm} u^{lm}_{kd} P(ab)
            + \clustamp^{ad}_{il} \clustamp^{ce}_{jk} u^{bl}_{de} P(ab) P(ij)
            + \clustamp^{ad}_{il} \clustamp^{bce}_{jkm} u^{lm}_{de} P(ab) P(ij)
            \nonumber \\
            &\qquad
            - \clustamp^{ad}_{ij} \clustamp^{be}_{kl} u^{cl}_{de} P(ab)
            + \clustamp^{ad}_{ij} \clustamp^{ce}_{kl} u^{bl}_{de} P(ab)
            + \frac{1}{2} \clustamp^{ad}_{ij} \clustamp^{bce}_{klm} u^{lm}_{de} P(ab)
            \nonumber \\
            &\qquad
            - \clustamp^{ad}_{ij} u^{bc}_{kd} P(ab)
            + \clustamp^{ad}_{ik} \clustamp^{be}_{jl} u^{cl}_{de} P(ab) P(ij)
            + \frac{1}{2} \clustamp^{ad}_{ik} \clustamp^{bc}_{lm} u^{lm}_{jd} P(ab) P(ij)
            \nonumber \\
            &\qquad
            - \clustamp^{ad}_{ik} \clustamp^{ce}_{jl} u^{bl}_{de} P(ab) P(ij)
            - \frac{1}{2} \clustamp^{ad}_{ik} \clustamp^{bce}_{jlm} u^{lm}_{de} P(ab) P(ij)
            + \clustamp^{ad}_{ik} u^{bc}_{jd} P(ab) P(ij)
            \nonumber \\
            &\qquad
            + \clustamp^{ad}_{kl} \clustamp^{ce}_{ij} u^{bl}_{de} P(ab)
            + \clustamp^{ad}_{kl} \clustamp^{bce}_{ijm} u^{lm}_{de} P(ab)
            - \frac{1}{2} \clustamp^{ab}_{lm} \clustamp^{cd}_{ij} u^{lm}_{kd}
            \nonumber \\
            &\qquad
            + \frac{1}{2} \clustamp^{ab}_{lm} \clustamp^{cd}_{ik} u^{lm}_{jd} P(ij)
            + \frac{1}{4} \clustamp^{ab}_{lm} \clustamp^{cde}_{ijk} u^{lm}_{de}
            + \clustamp^{ab}_{il} \clustamp^{cd}_{jm} u^{lm}_{kd} P(ij)
            \nonumber \\
            &\qquad
            - \clustamp^{ab}_{il} \clustamp^{cd}_{km} u^{lm}_{jd} P(ij)
            + \frac{1}{2} \clustamp^{ab}_{il} \clustamp^{cde}_{jkm} u^{lm}_{de} P(ij)
            + \clustamp^{ab}_{il} u^{cl}_{jk} P(ij)
            \nonumber \\
            &\qquad
            + \clustamp^{ab}_{kl} \clustamp^{cd}_{im} u^{lm}_{jd} P(ij)
            + \frac{1}{2} \clustamp^{ab}_{kl} \clustamp^{cde}_{ijm} u^{lm}_{de}
            + \clustamp^{ab}_{kl} u^{cl}_{ij}
            \nonumber \\
            &\qquad
            + \frac{1}{2} \clustamp^{ac}_{lm} \clustamp^{bd}_{ij} u^{lm}_{kd} P(ab)
            - \frac{1}{4} \clustamp^{ac}_{lm} \clustamp^{bde}_{ijk} u^{lm}_{de} P(ab)
            - \clustamp^{ac}_{il} \clustamp^{bd}_{jm} u^{lm}_{kd} P(ab)
            \nonumber \\
            &\qquad
            + \clustamp^{ac}_{il} \clustamp^{bd}_{km} u^{lm}_{jd} P(ab) P(ij)
            - \frac{1}{2} \clustamp^{ac}_{il} \clustamp^{bde}_{jkm} u^{lm}_{de} P(ab) P(ij)
            - \clustamp^{ac}_{il} u^{bl}_{jk} P(ab) P(ij)
            \nonumber \\
            &\qquad
            - \clustamp^{ac}_{kl} \clustamp^{bd}_{im} u^{lm}_{jd} P(ab) P(ij)
            - \frac{1}{2} \clustamp^{ac}_{kl} \clustamp^{bde}_{ijm} u^{lm}_{de} P(ab)
            - \clustamp^{ac}_{kl} u^{bl}_{ij} P(ab)
            \nonumber \\
            &\qquad
            - \clustamp^{cd}_{ij} u^{ab}_{kd}
            + \clustamp^{cd}_{ik} u^{ab}_{jd} P(ij)
            - \frac{1}{2} \clustamp^{ce}_{lm} \clustamp^{abd}_{ijk} u^{lm}_{de}
            \nonumber \\
            &\qquad
            + \clustamp^{ce}_{im} \clustamp^{abd}_{jkl} u^{lm}_{de} P(ij)
            - \frac{1}{2} \clustamp^{ce}_{ij} \clustamp^{abd}_{klm} u^{lm}_{de}
            + \frac{1}{2} \clustamp^{ce}_{ik} \clustamp^{abd}_{jlm} u^{lm}_{de} P(ij)
            \nonumber \\
            &\qquad
            + \clustamp^{ce}_{km} \clustamp^{abd}_{ijl} u^{lm}_{de}
            + \frac{1}{2} \clustamp^{ade}_{ijk} u^{bc}_{de} P(ab)
            + \clustamp^{abd}_{ijl} u^{cl}_{kd}
            \nonumber \\
            &\qquad
            - \clustamp^{abd}_{ikl} u^{cl}_{jd} P(ij)
            + \frac{1}{2} \clustamp^{abc}_{ilm} u^{lm}_{jk} P(ij)
            + \frac{1}{2} \clustamp^{abc}_{klm} u^{lm}_{ij}
            \nonumber \\
            &\qquad
            - \clustamp^{acd}_{ijl} u^{bl}_{kd} P(ab)
            + \clustamp^{acd}_{ikl} u^{bl}_{jd} P(ab) P(ij)
            + \frac{1}{2} \clustamp^{cde}_{ijk} u^{ab}_{de}.
        \end{align}
