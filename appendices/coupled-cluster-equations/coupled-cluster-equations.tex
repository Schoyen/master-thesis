\chapter{Coupled-cluster equations}
    In this appendix we will show the explicit equations used in the
    coupled-cluster methods for different truncation levels.

    \section{Energy equations}
        \label{app:cc-energy-equations}
        In this section we derive the projected coupled-cluster correlation
        energy from the normal-ordered Hamiltonian.
        This consists of the linear contribution in
        \autoref{eq:cc-energy-linear-contrib} and the squared contribution in
        \autoref{eq:cc-energy-squared-contrib}.
        As stated in \autoref{subsec:cc-energy-equations}, we need only concern
        ourselves with the case where the cluster operator is given by
        \begin{align}
            \clust
            &= \clust_1 + \clust_2
            = \clustamp^{a}_{i}\normalord{\ccr{a}\can{i}}
            + \frac{1}{4}\clustamp^{ab}_{ij}
            \normalord{\ccr{a}\ccr{b}\can{j}\can{i}},
        \end{align}
        where we note that the cluster operators are normal-ordered by
        construction.
        Looking at the energy contribution linear in the cluster operator we get
        \begin{align}
            \mel{\slat}{
                \brak{
                    \hamil_N
                    \clust
                }_c
            }{\slat}
            &=
            \mel{\slat}{
                \brak{
                    \hamil_N
                    \clust_1
                }_c
            }{\slat}
            + \mel{\slat}{
                \brak{
                    \hamil_N
                    \clust_2
                }_c
            }{\slat}
            \\
            &=
            \mel{\slat}{
                \brak{
                    \fock_N
                    \clust_1
                }_c
            }{\slat}
            + \mel{\slat}{
                \brak{
                    \twohamil_N
                    \clust_2
                }_c
            }{\slat},
        \end{align}
        where we only keep the non-zero contributions in the last line.
        As we are projecting onto the reference determinant, we are dependent on
        the operators being fully contracted.
        This means that all non-contracted operators will destroy the overlap.
        Hence, the doubles cluster operator cannot couple with the Fock-operator
        nor can a single singles cluster operator couple with the
        two-body-operator.
        Looking at each operator pair separately we get
        \begin{align}
            \mel{\slat}{
                \brak{
                    \fock_N\clust_1
                }_c
            }{\slat}
            &=
            \fockten^{p}_{q}\clustamp^{a}_{i}
            \mel{\slat}{
                \normalord{
                    \ccr{p}\can{q}
                }\normalord{
                    \ccr{a}
                    \can{i}
                }
            }{\slat}
            =
            \fockten^{p}_{q}\clustamp^{a}_{i}
            \mel{\slat}{
                \wick{
                    \normalord{
                        \c2 {\ccr{p}}
                        \c1 {\can{q}}
                        \c1 {\ccr{a}}
                        \c2 {\can{i}}
                    }
                }
            }{\slat}
            \\
            &=
            \fockten^{p}_{q}\clustamp^{a}_{i}
            \delta_{qa}\delta_{pi}
            = \fockten^{i}_{a}\clustamp^{a}_{i}.
        \end{align}
        The doubles cluster operator on the two-body Hamiltonian yields
        \begin{align}
            \mel{\slat}{\brak{
                \twohamil_N \clust_2
            }_c}{\slat}
            &=
            \frac{1}{16}\twoten^{pq}_{rs}\clustamp^{ab}_{ij}
            \mel{\slat}{
                \normalord{
                    \ccr{p}
                    \ccr{q}
                    \can{s}
                    \can{r}
                }
                \normalord{
                    \ccr{a}
                    \ccr{b}
                    \can{j}
                    \can{i}
                }
            }{\slat}.
        \end{align}
        For the sake of brevity we will restrict our attention to the operator
        strings that are non-zero when utilizing Wick's theorem.
        \begin{align}
            \normalord{
                \ccr{p}
                \ccr{q}
                \can{s}
                \can{r}
            }
            \normalord{
                \ccr{a}
                \ccr{b}
                \can{j}
                \can{i}
            }
            &=
            \wick{
                \normalord{
                    \c4 {\ccr{p}}
                    \c3 {\ccr{q}}
                    \c2 {\can{s}}
                    \c1 {\can{r}}
                    \c1 {\ccr{a}}
                    \c2 {\ccr{b}}
                    \c3 {\can{j}}
                    \c4 {\can{i}}
                }
            }
            +
            \wick{
                \normalord{
                    \c4 {\ccr{p}}
                    \c3 {\ccr{q}}
                    \c2 {\can{s}}
                    \c1 {\can{r}}
                    \c2 {\ccr{a}}
                    \c1 {\ccr{b}}
                    \c3 {\can{j}}
                    \c4 {\can{i}}
                }
            }
            \nonumber \\
            &\qquad
            +
            \wick{
                \normalord{
                    \c4 {\ccr{p}}
                    \c3 {\ccr{q}}
                    \c2 {\can{s}}
                    \c1 {\can{r}}
                    \c1 {\ccr{a}}
                    \c2 {\ccr{b}}
                    \c4 {\can{j}}
                    \c3 {\can{i}}
                }
            }
            +
            \wick{
                \normalord{
                    \c4 {\ccr{p}}
                    \c3 {\ccr{q}}
                    \c2 {\can{s}}
                    \c1 {\can{r}}
                    \c2 {\ccr{a}}
                    \c1 {\ccr{b}}
                    \c4 {\can{j}}
                    \c3 {\can{i}}
                }
            }
            \\
            &=
            \delta_{ra}
            \delta_{sb}
            \delta_{qj}
            \delta_{pi}
            -
            \delta_{rb}
            \delta_{sa}
            \delta_{qj}
            \delta_{pi}
            \nonumber \\
            &\qquad
            -
            \delta_{ra}
            \delta_{sb}
            \delta_{qi}
            \delta_{pj}
            +
            \delta_{rb}
            \delta_{sa}
            \delta_{qi}
            \delta_{pj}.
        \end{align}
        Inserting the Kronecker-delta functions back into the energy
        contribution and summing, we get
        \begin{align}
            \mel{\slat}{
                \brak{
                    \twohamil_N \clust_2
                }_c
            }{\slat}
            &=
            \frac{1}{16}\brak{
                \twoten^{ij}_{ab}
                - \twoten^{ij}_{ba}
                - \twoten^{ji}_{ab}
                + \twoten^{ji}_{ba}
            }\clustamp^{ab}_{ij}
            =
            \frac{1}{4}\twoten^{ij}_{ab}\clustamp^{ab}_{ij},
        \end{align}
        where we've used the antisymmetric properties of the two-body
        elements to collect all four terms.

        Moving on to the squared cluster operator contribution we note that the
        only non-zero contribution to the energy can come from the singles
        cluster operator as this provides a doubly excited state from the
        reference state.
        This also means that we can only get a coupling with the two-body
        operator.
        We are thus left with
        \begin{align}
            \mel{\slat}{
                \brak{
                    \twohamil_N \clust_1^2
                }_c
            }{\slat}
            &=
            \frac{1}{4}\twoten^{pq}_{rs}\clustamp^{a}_{i}\clustamp^{b}_{j}
            \mel{\slat}{
                \normalord{
                    \ccr{p}
                    \ccr{q}
                    \can{s}
                    \can{r}
                }
                \normalord{
                    \ccr{a}
                    \can{i}
                }
                \normalord{
                    \ccr{b}
                    \can{j}
                }
            }{\slat}
            \\
            &=
            \twoten^{ij}_{ab}\clustamp^{a}_{i}\clustamp^{b}_{j},
        \end{align}
        where we note that the same contractions as in the previous term is
        performed for the squared cluster operators.
        Collecting all the contributions to the correlation energy we get
        \begin{align}
            \mel{\slat}{\simhamil_N}{\slat}
            &=
            \mel{\slat}{
                \brak{
                    \hamil_N\clust
                }_{c}
            }{\slat}
            + \frac{1}{2!}
            \mel{\slat}{
                \brak{
                    \hamil_N\clust^2
                }_{c}
            }{\slat}
            \\
            &=
            \fockten^{i}_{a}\clustamp^{a}_{i}
            + \half \twoten^{ij}_{ab}\para{
                \clustamp^{ab}_{ij}
                + \clustamp^{a}_{i}\clustamp^{b}_{j}
            },
        \end{align}
        which is what we wanted to show.

    \section{Coupled cluster amplitude equations}
        Using SymPy \cite{sympy} we are able to efficiently create amplitude
        equations by programmatically evaluating Wick's theorem.
        The labelling of the terms in the equations are inspired by the naming
        convention used in \citetitle{shavitt2009many} \cite{shavitt2009many},
        but with slight modifications.
        The first letter, either ``S'' or ``D'', denotes a singles or a doubles
        contribution respectively.
        The number is used to collect terms with a similar structure, e.g.,
        contractions between a singles amplitude, a doubles amplitude, and the
        two-body Hamiltonian will share the same number.
        To differentiate the different terms with the same type of contractions
        we tack on a second letter which is increased alphabetically.
        However, the number and the second letter do not have a deeper meaning
        as in the work by \citeauthor{shavitt2009many} \cite{shavitt2009many}.

        We denote the number of basis functions by $L$, the number of particles
        by $N$, and the number virtual states by $M = L - N$.
        Furthermore, we assume that the tensor contractions are performed as
        binary operations where the ordering of the contractions involving the
        lowest cost are performed.
        As an example, consider the contraction
        \begin{align}
            g^{ab}_{ij} = \clustamp^{ab}_{lj} \clustamp^{dc}_{ik}
            \twoten^{kl}_{cd}.
        \end{align}
        The naïve solution using explicit for-loops yields a $\mathcal{O}(M^4
        N^4)$-complexity.
        By first creating the intermediate contraction
        \begin{align}
            W^{l}_{i} = \clustamp^{dc}_{ik} \twoten^{kl}_{cd},
        \end{align}
        and then computing the total result from
        \begin{align}
            g^{ab}_{ij} = \clustamp^{ab}_{lj} W^{l}_{i},
        \end{align}
        we've reduced the complexity to $\mathcal{O}(M^2 N^3)$.
        This incurs a memory penalty, but the gain in reduction of the number of
        FLOPS far exceeds the cost.

        \begin{table}
            \centering
            \caption{Terms included in the CCD amplitudes.
            Note that the complexity is based on the tensor contraction being
            performed as a binary operation where the optimal ordering of the
            contraction is performed.}
            \renewcommand{\arraystretch}{1.3}
            \begin{tabular}{@{}lll@{}}
                \toprule
                Label & Term & Complexity \\
                \midrule
                D1 & $\twoten^{ab}_{ij}$ & $\mathcal{O}(M^2 N^2)$ \\
                \bottomrule
            \end{tabular}
            \label{tab:ccd-amplitude-terms}
        \end{table}

        The doubles amplitude equations is given by\cite{shavitt2009many}
        \begin{align}
            0 &= g(f, u, \clustamp)
            \equiv \braslate{ab}{ij}e^{-\clust_2}\HN e^{\clust_2}\ketslat
            \\
            &=
            u^{ab}_{ij}
            + f^{b}_{c} \clustamp^{ac}_{ij} P(ab)
            - f^{k}_{j} \clustamp^{ab}_{ik} P(ij)
            + \half \clustamp^{cd}_{ij} u^{ab}_{cd}
            + \half \clustamp^{ab}_{kl} u^{kl}_{ij}
            \nonumber \\
            &\qquad
            + \clustamp^{ac}_{ik} u^{bk}_{jc} P(ab) P(ij)
            + \frac{1}{4} \clustamp^{cd}_{ij} \clustamp^{ab}_{kl} u^{kl}_{cd}
            + \clustamp^{ac}_{ik} \clustamp^{bd}_{jl} u^{kl}_{cd} P(ij)
            \nonumber \\
            &\qquad
            - \half \clustamp^{ab}_{lj} \clustamp^{dc}_{ik} u^{kl}_{cd} P(ij)
            - \half \clustamp^{ac}_{lk} \clustamp^{db}_{ij} u^{kl}_{cd} P(ab).
        \end{align}
        In order to reduce the number of FLOPS when contracting the tensors, we
        introduce so-called \emph{intermediates}\cite{hjorth2017advanced}. In
        practice, this consists of precomputing some of the terms by choosing
        which tensors to contract first. In the doubles approximation there are
        four sensible intermediates we can define\footnote{Note that we use the
        notation "$\gets$" to signify part of the expression, i.e., some of the
        terms contained in the function.}.
        \begin{gather}
            g(f, u, \clustamp)
            \gets
            \frac{1}{4} \clustamp^{cd}_{ij} \clustamp^{ab}_{kl} u^{kl}_{cd}
            + \half \clustamp^{cd}_{ij} u^{ab}_{cd}
            =
            \clustamp^{cd}_{ij}\para{
                \frac{1}{4} \clustamp^{ab}_{kl} u^{kl}_{cd}
                + \half u^{ab}_{cd}
            }
            = \clustamp^{cd}_{ij}W^{ab}_{cd},
            \\
            g(f, u, \clustamp)
            \gets
            \half \clustamp^{cd}_{jk} \clustamp^{ab}_{il} u^{kl}_{cd} P(ij)
            = \clustamp^{ab}_{il}\para{
                \half \clustamp^{cd}_{jk} u^{kl}_{cd}
            } P(ij)
            = \clustamp^{ab}_{il} W^{l}_{j} P(ij),
            \\
            g(f, u, \clustamp)
            \gets
            \half \clustamp^{ac}_{ij} \clustamp^{bd}_{kl} u^{kl}_{cd} P(ab)
            = \clustamp^{ac}_{ij}\para{
                \half \clustamp^{bd}_{kl} u^{kl}_{cd}
            } P(ab)
            = \clustamp^{ac}_{ij} W^{b}_{c} P(ab).
        \end{gather}
        The last intermediate requires a little work, as we have to insert an
        extra exchange operator, $P(ij)$, in one of the terms in order to group
        two terms into a single intermediate.
        \begin{align}
            g(f, u, \clustamp)
            &\gets
            \clustamp^{ac}_{ik} \clustamp^{bd}_{jl} u^{kl}_{cd} P(ab)
            + \clustamp^{ac}_{ik} u^{bk}_{jc} P(ab) P(ij)
            \\
            &= \clustamp^{ac}_{ik}\para{
                \half \clustamp^{bd}_{jl} u^{kl}_{cd}
                + u^{bk}_{jc}
            } P(ab) P(ij)
            \\
            &= \clustamp^{ac}_{ik} W^{bk}_{jc} P(ab) P(ij).
        \end{align}
        We summarize the expression for the intermediates in
        \autoref{tab:intermediates_ccd}, along with their computational
        complexity. The total right-hand side of the $\clustamp$-amplitudes in the
        doubles approximation using the intermediate calculations is thus
        \begin{align}
            g(f, u, \clustamp)
            &=
            f^{b}_{c} \clustamp^{ac}_{ij} P(ab)
            - f^{k}_{j} \clustamp^{ab}_{ik} P(ij)
            + \clustamp^{cd}_{ij} W^{ab}_{cd}
            + \clustamp^{ab}_{il} W^{l}_{j} P(ij)
            \nonumber \\
            &\qquad
            + \clustamp^{ac}_{ik} W^{bk}_{jc} P(ab) P(ij)
            - \clustamp^{ac}_{ij} W^{b}_{c} P(ab)
            + \half \clustamp^{ab}_{kl} u^{kl}_{ij}
            + u^{ab}_{ij}.
        \end{align}
        The most time-consuming contraction is now the term $\clustamp^{cd}_{ij}
        W^{ab}_{cd}$. This term uses $\mathcal{O}(m^4 n^2)$ FLOPS, which is a
        reduction from $\mathcal{O}(m^4 n^4)$ when computing $\clustamp^{cd}_{ij}
        \clustamp^{ab}_{kl} u^{kl}_{cd}$ directly.

        \begin{table}
            \centering
            \caption{In this table we summarize the intermediates used in
            the $\clustamp$-amplitudes of the coupled cluster doubles approximation. We
            also list the computational complexity in FLOPS needed to construct
            the intermediate. Recall that $n$ is the number of holes and $m$ the
            number of particles.}
            \renewcommand{\arraystretch}{1.3}
            \begin{tabular}{@{}ll@{}}
                \toprule
                Intermediate & Complexity [FLOPS] \\
                \midrule
                $W^{ab}_{cd}
                = \frac{1}{4} \clustamp^{ab}_{kl} u^{kl}_{cd}
                + \half u^{ab}_{cd}$
                &
                $\mathcal{O}(m^4 n^2)$ \\
                \addlinespace
                $W^{l}_{j}
                = \half \clustamp^{cd}_{jk} u^{kl}_{cd}$
                &
                $\mathcal{O}(m^2 n^3)$
                \\
                \addlinespace
                $W^{b}_{c}
                = \half \clustamp^{bd}_{kl} u^{kl}_{cd}$
                &
                $\mathcal{O}(m^3 n^2)$
                \\
                \addlinespace
                $W^{bk}_{jc}
                = \half \clustamp^{bd}_{jl} u^{kl}_{cd}
                + u^{bk}_{jc}$
                &
                $\mathcal{O}(m^3n^3)$
                \\
                \bottomrule
            \end{tabular}
            \label{tab:intermediates_ccd}
        \end{table}
        Moving on to the $\clustl$-equations we have for the doubles
        approximation
        \begin{align}
            0 &=
        \end{align}

    \section{CCSD amplitude equations}
        The energy for the singles-doubles truncation is given by
        \begin{align}
            \energyccsd
            &= \energyref
            + \braslat e^{-\clust_1 - \clust_2} \HN e^{\clust_1 + \clust_2}
            \ketslat
            \\
            &=
            \energyref
            +
            f^{k}_{c} \clustamp^{c}_{k}
            - \half \clustamp^{c}_{l} \clustamp^{d}_{k} u^{kl}_{cd}
            + \frac{1}{4} \clustamp^{cd}_{kl} u^{kl}_{cd}.
        \end{align}
        The singles amplitude equations becomes
        \begin{align}
            0 & = \braslate{a}{i} e^{-\clust_1 - \clust_2} \HN
            e^{\clust_1 + \clust_2}\ketslat
            \\
            &=
            f^{a}_{i}
            - f^{k}_{c} \clustamp^{c}_{i} \clustamp^{a}_{k}
            + f^{k}_{c} \clustamp^{ac}_{ik}
            - f^{k}_{i} \clustamp^{a}_{k}
            + f^{a}_{c} \clustamp^{c}_{i}
            - \clustamp^{c}_{k} \clustamp^{d}_{i} \clustamp^{a}_{l} u^{kl}_{cd}
            - \clustamp^{c}_{k} \clustamp^{d}_{i} u^{ak}_{cd}
            \nonumber \\
            &\qquad
            + \clustamp^{c}_{k} \clustamp^{a}_{l} u^{kl}_{ic}
            + \clustamp^{c}_{k} \clustamp^{ad}_{il} u^{kl}_{cd}
            + \clustamp^{c}_{k} u^{ak}_{ic}
            - \frac{1}{2} \clustamp^{c}_{i} \clustamp^{ad}_{kl} u^{kl}_{cd}
            \nonumber \\
            &\qquad
            + \frac{1}{2} \clustamp^{a}_{l} \clustamp^{cd}_{ik} u^{kl}_{cd}
            + \frac{1}{2} \clustamp^{cd}_{ik} u^{ak}_{cd}
            - \frac{1}{2} \clustamp^{ac}_{kl} u^{kl}_{ic}.
        \end{align}
        % TODO: Introduce intermediates
        The doubles amplitude equations are
        \begin{align}
            0 & = \braslate{ab}{ij} e^{-\clust_1 - \clust_2} \HN
            e^{\clust_1 + \clust_2}\ketslat
            \\
            &=
            f^{k}_{c} \clustamp^{c}_{i} \clustamp^{ab}_{jk} P(ij)
            + f^{k}_{c} \clustamp^{a}_{k} \clustamp^{bc}_{ij} P(ab)
            + f^{k}_{i} \clustamp^{ab}_{jk} P(ij)
            - f^{a}_{c} \clustamp^{bc}_{ij} P(ab)
            \nonumber \\
            &\qquad
            + \clustamp^{c}_{k} \clustamp^{d}_{i} \clustamp^{ab}_{jl} u^{kl}_{cd} P(ij)
            + \clustamp^{c}_{k} \clustamp^{a}_{l} \clustamp^{bd}_{ij} u^{kl}_{cd} P(ab)
            - \clustamp^{c}_{k} \clustamp^{ad}_{ij} u^{bk}_{cd} P(ab)
            \nonumber \\
            &\qquad
            + \clustamp^{c}_{k} \clustamp^{ab}_{il} u^{kl}_{jc} P(ij)
            + \clustamp^{c}_{i} \clustamp^{d}_{j} \clustamp^{a}_{k} \clustamp^{b}_{l} u^{kl}_{cd}
            + \clustamp^{c}_{i} \clustamp^{d}_{j} \clustamp^{a}_{k} u^{bk}_{cd} P(ab)
            + \frac{1}{2} \clustamp^{c}_{i} \clustamp^{d}_{j} \clustamp^{ab}_{kl} u^{kl}_{cd} 1
            \nonumber \\
            &\qquad
            + \clustamp^{c}_{i} \clustamp^{d}_{j} u^{ab}_{cd}
            - \clustamp^{c}_{i} \clustamp^{a}_{k} \clustamp^{b}_{l} u^{kl}_{jc} P(ij)
            - \clustamp^{c}_{i} \clustamp^{a}_{k} \clustamp^{bd}_{jl} u^{kl}_{cd} P(ab) P(ij)
            \nonumber \\
            &\qquad
            - \clustamp^{c}_{i} \clustamp^{a}_{k} u^{bk}_{jc} P(ab) P(ij)
            - \clustamp^{c}_{i} \clustamp^{ad}_{jk} u^{bk}_{cd} P(ab) P(ij)
            - \frac{1}{2} \clustamp^{c}_{i} \clustamp^{ab}_{kl} u^{kl}_{jc} P(ij)
            \nonumber \\
            &\qquad
            - \clustamp^{c}_{i} u^{ab}_{jc} P(ij)
            + \frac{1}{2} \clustamp^{a}_{k} \clustamp^{b}_{l} \clustamp^{cd}_{ij} u^{kl}_{cd}
            + \clustamp^{a}_{k} \clustamp^{b}_{l} u^{kl}_{ij}
            + \frac{1}{2} \clustamp^{a}_{k} \clustamp^{cd}_{ij} u^{bk}_{cd} 1 P(ab)
            \nonumber \\
            &\qquad
            + \clustamp^{a}_{k} \clustamp^{bc}_{il} u^{kl}_{jc} P(ab) P(ij)
            + \clustamp^{a}_{k} u^{bk}_{ij} P(ab)
            + \frac{1}{4} \clustamp^{cd}_{ij} \clustamp^{ab}_{kl} u^{kl}_{cd} 1
            + \frac{1}{2} \clustamp^{cd}_{ij} u^{ab}_{cd}
            \nonumber \\
            &\qquad
            + \frac{1}{2} \clustamp^{cd}_{jk} \clustamp^{ab}_{il} u^{kl}_{cd} P(ij)
            + \clustamp^{ac}_{ik} \clustamp^{bd}_{jl} u^{kl}_{cd} P(ab)
            + \clustamp^{ac}_{ik} u^{bk}_{jc} P(ab) P(ij)
            \nonumber \\
            &\qquad
            - \frac{1}{2} \clustamp^{ac}_{ij} \clustamp^{bd}_{kl} u^{kl}_{cd} P(ab)
            + \frac{1}{2} \clustamp^{ab}_{kl} u^{kl}_{ij} 1
            + u^{ab}_{ij}.
        \end{align}
        % TODO: Introduce intermediates
