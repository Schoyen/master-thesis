\chapter{Coupled-cluster equations}
    In this appendix we will show the explicit equations used in the
    coupled-cluster methods for different truncation levels.

    \section{Energy equations}
        \label{app:cc-energy-equations}
        In this section we derive the projected coupled-cluster correlation
        energy from the normal-ordered Hamiltonian.
        This consists of the linear contribution in
        \autoref{eq:cc-energy-linear-contrib} and the squared contribution in
        \autoref{eq:cc-energy-squared-contrib}.
        As stated in \autoref{subsec:cc-energy-equations}, we need only concern
        ourselves with the case where the cluster operator is given by
        \begin{align}
            \clust
            &= \clust_1 + \clust_2
            = \clustamp^{a}_{i}\normalord{\ccr{a}\can{i}}
            + \frac{1}{4}\clustamp^{ab}_{ij}
            \normalord{\ccr{a}\ccr{b}\can{j}\can{i}},
        \end{align}
        where we note that the cluster operators are normal-ordered by
        construction.
        Looking at the energy contribution linear in the cluster operator we get
        \begin{align}
            \mel{\slat}{
                \brak{
                    \hamil_N
                    \clust
                }_c
            }{\slat}
            &=
            \mel{\slat}{
                \brak{
                    \hamil_N
                    \clust_1
                }_c
            }{\slat}
            + \mel{\slat}{
                \brak{
                    \hamil_N
                    \clust_2
                }_c
            }{\slat}
            \\
            &=
            \mel{\slat}{
                \brak{
                    \fock_N
                    \clust_1
                }_c
            }{\slat}
            + \mel{\slat}{
                \brak{
                    \twohamil_N
                    \clust_2
                }_c
            }{\slat},
        \end{align}
        where we only keep the non-zero contributions in the last line.
        As we are projecting onto the reference determinant, we are dependent on
        the operators being fully contracted.
        This means that all non-contracted operators will destroy the overlap.
        Hence, the doubles cluster operator cannot couple with the Fock-operator
        nor can a single singles cluster operator couple with the
        two-body-operator.
        Looking at each operator pair separately we get
        \begin{align}
            \mel{\slat}{
                \brak{
                    \fock_N\clust_1
                }_c
            }{\slat}
            &=
            \fockten^{p}_{q}\clustamp^{a}_{i}
            \mel{\slat}{
                \normalord{
                    \ccr{p}\can{q}
                }\normalord{
                    \ccr{a}
                    \can{i}
                }
            }{\slat}
            =
            \fockten^{p}_{q}\clustamp^{a}_{i}
            \mel{\slat}{
                \wick{
                    \normalord{
                        \c2 {\ccr{p}}
                        \c1 {\can{q}}
                        \c1 {\ccr{a}}
                        \c2 {\can{i}}
                    }
                }
            }{\slat}
            \\
            &=
            \fockten^{p}_{q}\clustamp^{a}_{i}
            \delta_{qa}\delta_{pi}
            = \fockten^{i}_{a}\clustamp^{a}_{i}.
        \end{align}
        The doubles cluster operator on the two-body Hamiltonian yields
        \begin{align}
            \mel{\slat}{\brak{
                \twohamil_N \clust_2
            }_c}{\slat}
            &=
            \frac{1}{16}\twoten^{pq}_{rs}\clustamp^{ab}_{ij}
            \mel{\slat}{
                \normalord{
                    \ccr{p}
                    \ccr{q}
                    \can{s}
                    \can{r}
                }
                \normalord{
                    \ccr{a}
                    \ccr{b}
                    \can{j}
                    \can{i}
                }
            }{\slat}.
        \end{align}
        For the sake of brevity we will restrict our attention to the operator
        strings that are non-zero when utilizing Wick's theorem.
        \begin{align}
            \normalord{
                \ccr{p}
                \ccr{q}
                \can{s}
                \can{r}
            }
            \normalord{
                \ccr{a}
                \ccr{b}
                \can{j}
                \can{i}
            }
            &=
            \wick{
                \normalord{
                    \c4 {\ccr{p}}
                    \c3 {\ccr{q}}
                    \c2 {\can{s}}
                    \c1 {\can{r}}
                    \c1 {\ccr{a}}
                    \c2 {\ccr{b}}
                    \c3 {\can{j}}
                    \c4 {\can{i}}
                }
            }
            +
            \wick{
                \normalord{
                    \c4 {\ccr{p}}
                    \c3 {\ccr{q}}
                    \c2 {\can{s}}
                    \c1 {\can{r}}
                    \c2 {\ccr{a}}
                    \c1 {\ccr{b}}
                    \c3 {\can{j}}
                    \c4 {\can{i}}
                }
            }
            \nonumber \\
            &\qquad
            +
            \wick{
                \normalord{
                    \c4 {\ccr{p}}
                    \c3 {\ccr{q}}
                    \c2 {\can{s}}
                    \c1 {\can{r}}
                    \c1 {\ccr{a}}
                    \c2 {\ccr{b}}
                    \c4 {\can{j}}
                    \c3 {\can{i}}
                }
            }
            +
            \wick{
                \normalord{
                    \c4 {\ccr{p}}
                    \c3 {\ccr{q}}
                    \c2 {\can{s}}
                    \c1 {\can{r}}
                    \c2 {\ccr{a}}
                    \c1 {\ccr{b}}
                    \c4 {\can{j}}
                    \c3 {\can{i}}
                }
            }
            \\
            &=
            \delta_{ra}
            \delta_{sb}
            \delta_{qj}
            \delta_{pi}
            -
            \delta_{rb}
            \delta_{sa}
            \delta_{qj}
            \delta_{pi}
            \nonumber \\
            &\qquad
            -
            \delta_{ra}
            \delta_{sb}
            \delta_{qi}
            \delta_{pj}
            +
            \delta_{rb}
            \delta_{sa}
            \delta_{qi}
            \delta_{pj}.
        \end{align}
        Inserting the Kronecker-delta functions back into the energy
        contribution and summing, we get
        \begin{align}
            \mel{\slat}{
                \brak{
                    \twohamil_N \clust_2
                }_c
            }{\slat}
            &=
            \frac{1}{16}\brak{
                \twoten^{ij}_{ab}
                - \twoten^{ij}_{ba}
                - \twoten^{ji}_{ab}
                + \twoten^{ji}_{ba}
            }\clustamp^{ab}_{ij}
            =
            \frac{1}{4}\twoten^{ij}_{ab}\clustamp^{ab}_{ij},
        \end{align}
        where we've used the antisymmetric properties of the two-body
        elements to collect all four terms.

        Moving on to the squared cluster operator contribution we note that the
        only non-zero contribution to the energy can come from the singles
        cluster operator as this provides a doubly excited state from the
        reference state.
        This also means that we can only get a coupling with the two-body
        operator.
        We are thus left with
        \begin{align}
            \mel{\slat}{
                \brak{
                    \twohamil_N \clust_1^2
                }_c
            }{\slat}
            &=
            \frac{1}{4}\twoten^{pq}_{rs}\clustamp^{a}_{i}\clustamp^{b}_{j}
            \mel{\slat}{
                \normalord{
                    \ccr{p}
                    \ccr{q}
                    \can{s}
                    \can{r}
                }
                \normalord{
                    \ccr{a}
                    \can{i}
                }
                \normalord{
                    \ccr{b}
                    \can{j}
                }
            }{\slat}
            \\
            &=
            \twoten^{ij}_{ab}\clustamp^{a}_{i}\clustamp^{b}_{j},
        \end{align}
        where we note that the same contractions as in the previous term is
        performed for the squared cluster operators.
        Collecting all the contributions to the correlation energy we get
        \begin{align}
            \mel{\slat}{\simhamil_N}{\slat}
            &=
            \mel{\slat}{
                \brak{
                    \hamil_N\clust
                }_{c}
            }{\slat}
            + \frac{1}{2!}
            \mel{\slat}{
                \brak{
                    \hamil_N\clust^2
                }_{c}
            }{\slat}
            \\
            &=
            \fockten^{i}_{a}\clustamp^{a}_{i}
            + \half \twoten^{ij}_{ab}\para{
                \clustamp^{ab}_{ij}
                + \clustamp^{a}_{i}\clustamp^{b}_{j}
            },
        \end{align}
        which is what we wanted to show.

    \section{Coupled cluster $\clustamp$-amplitude equations}
        \label{app:cc-tau-amplitudes}
        We use the amplitude expressions from the book
        \citetitle{shavitt2009many} \cite{shavitt2009many} for the
        $\clustamp$-amplitudes.
        The task at hand is to evaluate
        \begin{align}
            \Omega_{\mu}(\vfg{\clustamp})
            = \mel*{\slat_{\mu}}{
                \exponential(-\clust)
                \hamil_N
                \exponential(\clust)
            }{\slat},
        \end{align}
        for the CCD approximation with $\clust = \clust_2$ and CCSD with $\clust
        = \clust_1 + \clust_2$.
        We denote the number of basis functions by $L$, the number of particles
        by $N$, and the number virtual states by $M = L - N$.
        We assume that $N < L / 2$ so that it is better to replace a contraction
        along a virtual index with an occupied index.
        Furthermore, we assume that the tensor contractions are performed as
        binary operations where the ordering of the contractions involving the
        lowest cost are performed.
        The permutation operator $P(ab)$ is defined by the action
        \begin{align}
            \fockten^{b}_{c} \clustamp^{ac}_{ij} P(ab)
            =
            \fockten^{b}_{c} \clustamp^{ac}_{ij}
            -
            \fockten^{a}_{c} \clustamp^{bc}_{ij},
        \end{align}
        that is, it subtracts the same term, but with two indices replaced.

        The CCD $\clustamp$-amplitudes are shown in
        \autoref{tab:ccd-tau-amplitude-terms}.
        For the CCSD $\clustamp$-amplitudes we have the doubles amplitudes from
        CCD in \autoref{tab:ccd-tau-amplitude-terms} along with the new doubles
        terms in \autoref{tab:ccsd-tau-2-amplitude-terms}.
        Note that term D8a in \autoref{tab:ccsd-tau-2-amplitude-terms} is
        different by a sign from the one in \citetitle{shavitt2009many} as the
        latter contains a typo.
        In \autoref{tab:ccsd-tau-1-amplitude-terms} we show the singles
        amplitude contributions for CCSD.

        \begin{table}
            \centering
            \caption{Terms and intermediates included in the CCD
            $\clustamp$-amplitudes.}
            \renewcommand{\arraystretch}{1.5}
            \begin{tabular}{@{}llll@{}}
                \toprule
                Label & Intermediate & Term & Complexity \\
                \midrule
                D1 & & $\twoten^{ab}_{ij}$ & $\mathcal{O}(M^2 N^2)$ \\
                D2a & & $\fockten^{b}_{c} \clustamp^{ac}_{ij} P(ab)$
                & $\mathcal{O}(M^3 N^2)$ \\
                D2b & & $-\fockten^{k}_{j} \clustamp^{ab}_{ik} P(ij)$
                & $\mathcal{O}(M^2 N^3)$ \\
                D2c & & $\half \clustamp^{cd}_{ij} \twoten^{ab}_{cd}$
                & $\mathcal{O}(M^4 N^2)$ \\
                D2d & & $\half \clustamp^{ab}_{kl} \twoten^{kl}_{ij}$
                & $\mathcal{O}(M^2 N^4)$ \\
                D2e & & $\clustamp^{ac}_{ik} \twoten^{kb}_{cj} P(ab) P(ij)$
                & $\mathcal{O}(M^3 N^3)$ \\
                D3a
                & $W^{kl}_{ij} = \frac{1}{4} \clustamp^{cd}_{ij}
                \twoten^{kl}_{cd}$
                & $\clustamp^{ab}_{kl} W^{kl}_{ij}$
                & $\mathcal{O}(M^2 N^4)$ \\
                D3b
                & $W^{bk}_{jc} = \clustamp^{bd}_{jl} \twoten^{kl}_{cd}$
                & $\clustamp^{ac}_{ik} W^{bk}_{jc} P(ij)$
                & $\mathcal{O}(M^3 N^3)$ \\
                D3c
                & $W^{l}_{i} = \half \clustamp^{dc}_{ik} \twoten^{kl}_{cd}$
                & $-\clustamp^{ab}_{lj} W^{l}_{i} P(ij)$
                & $\mathcal{O}(M^2 N^3)$ \\
                D3d
                & $W^{a}_{d} = \half \clustamp^{ac}_{lk} \twoten^{kl}_{cd}$
                & $-\clustamp^{db}_{ij} W^{a}_{d} P(ab)$
                & $\mathcal{O}(M^3 N^2)$ \\
                \bottomrule
            \end{tabular}
            \label{tab:ccd-tau-amplitude-terms}
        \end{table}

        \begin{center}
            \renewcommand{\arraystretch}{1.5}
            \begin{longtable}{@{}llll@{}}
                \caption{Terms and intermediates included in the CCSD
                $\clustamp_1$-amplitudes.  Empty lines continue from the line
                above.}
                \label{tab:ccsd-tau-1-amplitude-terms} \\
                \toprule

                Label & Intermediate & Term & Complexity \\
                \midrule

                \endfirsthead
                \caption{(continued)} \\
                \toprule

                Label & Intermediate & Term & Complexity \\
                \midrule

                \endhead

                \bottomrule

                \endfoot

                S1
                &
                & $\fockten^{a}_{i}$
                & $\mathcal{O}(M N)$ \\

                S2a
                &
                & $\fockten^{k}_{c} \clustamp^{ac}_{ik}$
                & $\mathcal{O}(M^2 N^2)$ \\

                S2b
                &
                & $\half \twoten^{ak}_{cd} \clustamp^{cd}_{ik}$
                & $\mathcal{O}(M^3 N^2)$ \\

                S2c
                &
                & $-\half \twoten^{kl}_{ic} \clustamp^{ac}_{kl}$
                & $\mathcal{O}(M^2 N^3)$ \\

                S3a
                &
                & $\fockten^{a}_{c} \clustamp^{c}_{i}$
                & $\mathcal{O}(M^2 N)$ \\

                S3b
                &
                & $-\fockten^{k}_{i} \clustamp^{a}_{k}$
                & $\mathcal{O}(M N^2)$ \\

                S3c
                &
                & $\twoten^{ak}_{ic} \clustamp^{c}_{k}$
                & $\mathcal{O}(M^2 N^2)$ \\

                S4a
                & $W^{kl}_{di} = -\half\twoten^{kl}_{cd} \clustamp^{c}_{i}$
                & $\clustamp^{ad}_{kl} W^{kl}_{di}$
                & $\mathcal{O}(M^2 N^3)$ \\

                S4b
                & $W^{k}_{i} = -\half \twoten^{kl}_{cd} \clustamp^{cd}_{il}$
                & $\clustamp^{a}_{k} W^{k}_{i}$
                & $\mathcal{O}(M^2 N^3)$ \\

                S4c
                & $W^{l}_{d} = \twoten^{kl}_{cd} \clustamp^{c}_{k}$
                & $W^{l}_{d} \clustamp^{da}_{li}$
                & $\mathcal{O}(M^2 N^2)$ \\

                S5a
                & $W^{k}_{i} = -\fockten^{k}_{c} \clustamp^{c}_{i}$
                & $\clustamp^{a}_{k} W^{k}_{i}$
                & $\mathcal{O}(M N^2)$ \\

                S5b
                & $W^{ak}_{di} = \twoten^{ak}_{cd} \clustamp^{c}_{i}$
                & $W^{ak}_{di} \clustamp^{d}_{k}$
                & $\mathcal{O}(M^3 N^2)$ \\

                S5c
                & $W^{k}_{i} = -\twoten^{kl}_{ic} \clustamp^{c}_{l}$
                & $\clustamp^{a}_{k} W^{k}_{i}$
                & $\mathcal{O}(M N^3)$ \\

                S6
                & $W^{k}_{c} = -\twoten^{kl}_{cd} \clustamp^{d}_{l}$ \\
                & $W^{k}_{i} = W^{k}_{c} \clustamp^{c}_{i}$
                & $\clustamp^{a}_{k} W^{k}_{i}$
                & $\mathcal{O}(M^2 N^2)$ \\
            \end{longtable}
        \end{center}

        \begin{center}
            \renewcommand{\arraystretch}{1.5}
            \begin{longtable}{@{}llll@{}}
                \caption{New terms included in the CCSD $\clustamp_2$-amplitudes.
                These terms should be added along with the ones from CCD in
                \autoref{tab:ccd-tau-amplitude-terms}.
                Empty lines continue from the line above.}
                \label{tab:ccsd-tau-2-amplitude-terms} \\
                \toprule

                Label & Intermediate & Term & Complexity \\
                \midrule

                \endfirsthead
                \caption{(continued)} \\
                \toprule

                Label & Intermediate & Term & Complexity \\
                \midrule

                \endhead

                \bottomrule

                \endfoot

                D4a
                &
                & $\twoten^{ab}_{cj} \clustamp^{c}_{i} P(ij)$
                & $\mathcal{O}(M^3 N^2)$ \\

                D4b
                &
                & $-\twoten^{kb}_{ij} \clustamp^{a}_{k} P(ab)$
                & $\mathcal{O}(M^2 N^3)$ \\

                D5a
                & $W^{k}_{i} = \fockten^{k}_{c} \clustamp^{c}_{i}$
                & $-t^{ab}_{kj} W^{k}_{i} P(ij)$
                & $\mathcal{O}(M^2 N^3)$ \\

                D5b
                & $W^{a}_{c} = \clustamp^{a}_{k} \fockten^{k}_{c}$
                & $- W^{a}_{c} \clustamp^{cb}_{ij} P(ab)$
                & $\mathcal{O}(M^3 N^2)$ \\

                D5c
                & $W^{ak}_{di} = \twoten^{ak}_{cd} \clustamp^{c}_{i}$
                & $W^{ak}_{di} \clustamp^{db}_{kj} P(ab) P(ij)$
                & $\mathcal{O}(M^3 N^3)$ \\

                D5d
                & $W^{al}_{ic} = \clustamp^{a}_{k} \twoten^{kl}_{ic}$
                & $-W^{al}_{ic} \clustamp^{cb}_{lj} P(ab) P(ij)$
                & $\mathcal{O}(M^3 N^3)$ \\

                D5e
                & $W^{kb}_{ij} = \twoten^{kb}_{cd} \clustamp^{cd}_{ij}$
                & $-\half \clustamp^{a}_{k} W^{kb}_{ij} P(ab)$
                & $\mathcal{O}(M^3 N^3)$ \\

                D5f
                & $W^{kl}_{ji} = \twoten^{kl}_{cj} \clustamp^{c}_{i}$
                & $\half \clustamp^{ab}_{kl} W^{kl}_{ji} P(ij)$
                & $\mathcal{O}(M^2 N^4)$ \\

                D5g
                & $W^{a}_{d} = \twoten^{ka}_{cd} \clustamp^{c}_{k}$
                & $W^{a}_{d} \clustamp^{db}_{ij} P(ab)$
                & $\mathcal{O}(M^3 N^2)$ \\

                D5h
                & $W^{l}_{i} = \twoten^{kl}_{ci} \clustamp^{c}_{k}$
                & $-\clustamp^{ab}_{lj} W^{l}_{i} P(ij)$
                & $\mathcal{O}(M^2 N^3)$ \\

                D6a
                & $W^{ab}_{di} = \twoten^{ab}_{cd} \clustamp^{c}_{i}$
                & $W^{ab}_{di} \clustamp^{d}_{j}$
                & $\mathcal{O}(M^4 N)$ \\

                D6b
                & $W^{bk}_{ij} = \clustamp^{b}_{l} \twoten^{kl}_{ij}$
                & $\clustamp^{a}_{k} W^{bk}_{ij}$
                & $\mathcal{O}(M^2 N^3)$ \\

                D6c
                & $W^{kb}_{ji} = -\twoten^{kb}_{cj} \clustamp^{c}_{i}$
                & $\clustamp^{a}_{k} W^{kb}_{ji} P(ab) P(ij)$
                & $\mathcal{O}(M^2 N^3)$ \\

                D7a
                & $W^{kl}_{di} = \half \twoten^{kl}_{cd} \clustamp^{c}_{i}$
                \\
                & $W^{kl}_{ij} = W^{kl}_{di} \clustamp^{d}_{j}$
                & $\clustamp^{ab}_{kl} W^{kl}_{ij}$
                & $\mathcal{O}(M^2 N^4)$ \\

                D7b
                & $W^{kl}_{ij} = \half \twoten^{kl}_{cd} \clustamp^{cd}_{ij}$ \\
                & $W^{bk}_{ij} = \clustamp^{b}_{l} W^{kl}_{ij}$
                & $\clustamp^{a}_{k} W^{bk}_{ij}$
                & $\mathcal{O}(M^2 N^4)$ \\

                D7c
                & $W^{al}_{cd} = - \clustamp^{a}_{k} \twoten^{kl}_{cd}$ \\
                & $W^{al}_{di} = W^{al}_{cd} \clustamp^{c}_{i}$
                & $W^{al}_{di} \clustamp^{db}_{lj} P(ij) P(ab)$
                & $\mathcal{O}(M^3 N^3)$ \\

                D7d
                & $W^{l}_{d} = -\twoten^{kl}_{cd} \clustamp^{c}_{k}$ \\
                & $W^{l}_{i} = W^{l}_{d} \clustamp^{d}_{i}$
                & $\clustamp^{ab}_{lj} W^{l}_{i} P(ij)$
                & $\mathcal{O}(M^2 N^3)$ \\

                D7e
                & $W^{l}_{d} = -\twoten^{kl}_{cd} \clustamp^{c}_{k}$ \\
                & $W^{a}_{d} = \clustamp^{a}_{l} W^{l}_{d}$
                & $W^{a}_{d} \clustamp^{db}_{ij} P(ab)$
                & $\mathcal{O}(M^3 N^2)$ \\

                D8a
                & $W^{kb}_{di} = -\twoten^{kb}_{cd} \clustamp^{c}_{i}$ \\
                & $W^{kb}_{ij} = W^{kb}_{di} \clustamp^{d}_{j}$
                & $\clustamp^{a}_{k} W^{kb}_{ij} P(ab)$
                & $\mathcal{O}(M^3 N^2)$
                \\

                D8b
                & $W^{kl}_{ji} = \twoten^{kl}_{cj} \clustamp^{c}_{i}$ \\
                & $W^{bk}_{ji} = \clustamp^{b}_{l} W^{kl}_{ji}$
                & $\clustamp^{a}_{k} W^{bk}_{ji} P(ij)$
                & $\mathcal{O}(M^2 N^3)$
                \\

                D9
                & $W^{kl}_{di} = \twoten^{kl}_{cd} \clustamp^{c}_{i}$ \\
                & $W^{kl}_{ij} = W^{kl}_{di} \clustamp^{d}_{j}$ \\
                & $W^{bk}_{ij} = \clustamp^{b}_{l} W^{kl}_{ij}$
                & $\clustamp^{a}_{k} W^{bk}_{ij}$
                & $\mathcal{O}(M^2 N^3)$
                \\

            \end{longtable}
        \end{center}

    \section{Coupled cluster $\clustlamp$-amplitude equations}
        \label{app:cc-lambda-amplitudes}
        Using SymPy \cite{sympy} we are able to efficiently create amplitude
        equations by programmatically evaluating Wick's theorem.
        The labelling of the terms in the equations are inspired by the naming
        convention used in \citetitle{shavitt2009many} \cite{shavitt2009many},
        but with slight modifications.
        The first letter, either ``S'' or ``D'', denotes a singles or a doubles
        contribution respectively.
        The number is used to collect terms with a similar structure, e.g.,
        contractions between a singles amplitude, a doubles amplitude, and the
        two-body Hamiltonian will share the same number.
        To differentiate the different terms with the same type of contractions
        we tack on a second letter which is increased alphabetically.
        However, the number and the second letter do not have a deeper meaning
        as in the work by \citeauthor{shavitt2009many} \cite{shavitt2009many}.

        \begin{center}
            \renewcommand{\arraystretch}{1.5}
            \begin{longtable}{@{}llll@{}}
                \caption{Terms included in the CCD $\clustlamp_2$-amplitudes.
                Empty lines continue from the line above.}
                \label{tab:ccd-lambda-amplitude-terms} \\
                \toprule

                Label & Intermediate & Term & Complexity \\
                \midrule

                \endfirsthead
                \caption{(continued)} \\
                \toprule

                Label & Intermediate & Term & Complexity \\
                \midrule

                \endhead

                \bottomrule

                \endfoot

                D1
                &
                & $\twoten^{ij}_{ab}$
                & $\mathcal{O}(M^2 N^2)$
                \\

                D2a
                &
                & $\half \clustlamp^{kl}_{ab} \twoten^{ij}_{kl}$
                & $\mathcal{O}(M^2 N^4)$
                \\

                D2b
                &
                & $\half \clustlamp^{ij}_{dc} \twoten^{dc}_{ab}$
                & $\mathcal{O}(M^4 N^2)$
                \\

                D2c
                &
                & $-\fockten^{c}_{a} \clustlamp^{ij}_{bc} P(ab)$
                & $\mathcal{O}(M^3 N^2)$
                \\

                D2d
                &
                & $\fockten^{i}_{k} \clustlamp^{jk}_{ab} P(ij)$
                & $\mathcal{O}(M^2 N^3)$
                \\

                D2e
                &
                & $\clustlamp^{jk}_{bc} \twoten^{ic}_{ak} P(ab) P(ij)$
                & $\mathcal{O}(M^3 N^3)$
                \\

                D3a
                & $W^{c}_{a} = \half \clustamp^{dc}_{kl} u^{kl}_{ad}$
                & $-\clustlamp^{ij}_{bc} W^{c}_{a} P(ab)$
                & $\mathcal{O}(M^3 N^2)$
                \\

                D3b
                & $W^{ij}_{kl} = \frac{1}{4} \clustlamp^{ij}_{dc} \clustamp^{dc}_{kl}$
                & $W^{ij}_{kl} \twoten^{kl}_{ab}$
                & $\mathcal{O}(M^2 N^4)$
                \\

                D3c
                & $W^{i}_{k} = \half\clustamp^{dc}_{kl} \twoten^{il}_{dc}$
                & $\clustlamp^{jk}_{ab} W^{i}_{k} P(ij)$
                & $\mathcal{O}(M^2 N^3)$
                \\

                D3d
                & $W^{jd}_{bl} = \clustlamp^{jk}_{bc} \clustamp^{dc}_{kl}$
                & $-W^{jd}_{bl} \twoten^{il}_{ad} P(ab) P(ij)$
                & $\mathcal{O}(M^3 N^3)$
                \\

                D3e
                & $W^{j}_{l} = \half\clustlamp^{jk}_{dc} \clustamp^{dc}_{kl}$
                & $W^{j}_{l} \twoten^{il}_{ab} P(ij)$
                & $\mathcal{O}(M^2 N^3)$
                \\

                D3f
                & $W^{ij}_{kl} = \frac{1}{4}\clustamp^{dc}_{kl} \twoten^{ij}_{dc}$
                & $\clustlamp^{kl}_{ab} W^{ij}_{kl}$
                & $\mathcal{O}(M^2 N^4)$
                \\

                D3g
                & $W^{d}_{b} = \half \clustlamp^{kl}_{bc} \clustamp^{dc}_{kl}$
                & $-W^{d}_{b} \twoten^{ij}_{ad} P(ab)$
                & $\mathcal{O}(M^3 N^2)$
                \\

            \end{longtable}
        \end{center}

        \begin{center}
            \renewcommand{\arraystretch}{1.5}
            \begin{longtable}{@{}llll@{}}
                \caption{New terms included in the CCSD
                $\clustlamp_2$-amplitudes.
                These terms should be added along with the ones from CCD in
                \autoref{tab:ccd-lambda-amplitude-terms}.
                Empty lines continue from the line above.}
                \label{tab:ccsd-lambda-2-amplitude-terms} \\
                \toprule

                Label & Intermediate & Term & Complexity \\
                \midrule

                \endfirsthead
                \caption{(continued)} \\
                \toprule

                Label & Intermediate & Term & Complexity \\
                \midrule

                \endhead

                \bottomrule

                \endfoot

            \end{longtable}
        \end{center}

    \section{Untruncated $Q$-space equations}
        \label{app:untruncated-q-space}
        Let $\brac{\chi_{\alpha}}$ be an initial biorthonormal single-particle
        basis with the corresponding dual states such that
        \begin{gather}
            \ket*{\phi_p} = C_{\alpha p} \ket*{\chi_{\alpha}}, \\
            \bra*{\tilde{\phi}_p} = \tilde{C}_{p \alpha}
            \bra*{\tilde{\chi}_{\alpha}},
        \end{gather}
        are the time-evolved biorthonormal orbitals in the OATDCC-method.
        It is important to note that we are now in a position where we can
        truncate the number of basis states $\brac{\phi_p}$.
        That is, if we let $p \in \brac{1, \dots, L}$ we can choose a $K \leq L$
        such that $p \in \brac{1, \dots, K}$ and thus lower the number of
        orbitals that we need to evolve in time.
        The time-dependency is kept in the coefficients and we have that
        \begin{align}
            \braket*{\tilde{\phi}_p}{\phi_q}
            = \tilde{C}_{p\alpha} C_{q\beta}
            \braket*{\tilde{\chi}_{\alpha}}{\chi_{\beta}}
            = \tilde{C}_{p\alpha} C_{q\alpha}
            = \delta_{pq},
        \end{align}
        at equal times.
        Now, if we do not truncate the basis of time-evolved orbitals
        $\brac{\phi_p}$, i.e., $K = L$, we have the inverse transformation
        \begin{gather}
            \ket*{\chi_{\beta}}
            = \delta_{\alpha \beta} \ket*{\chi_{\alpha}}
            = \tilde{C}_{p \beta} C_{\alpha p} \ket*{\chi_{\alpha}}
            = \tilde{C}_{p \beta} \ket*{\phi_p}, \\
            \bra*{\tilde{\chi}_{\beta}}
            = \delta_{\beta \alpha} \bra*{\tilde{\chi}_{\alpha}}
            = \tilde{C}_{p \alpha} C_{\beta p} \bra*{\tilde{\chi}_{\alpha}}
            = C_{\beta p} \bra*{\phi_p}.
        \end{gather}
        In order to find equations for the coefficients using \autoref{eq:ket-q}
        and \autoref{eq:bra-q} we left-project the former equation with
        $\bra*{\tilde{\chi}_{\alpha}}$ and right-project the latter equation with
        $\ket*{{\chi}_{\alpha}}$.
        Looking at the one-body Hamiltonian term from \autoref{eq:ket-q} we have
        \begin{align}
            \densityten^{q}_{p}\mel*{\tilde{\chi}_{\alpha}}{
                \hat{Q}\onehamil
            }{\phi_q}
            &=
            \densityten^{q}_{p}\mel*{\tilde{\chi}_{\alpha}}{
                \onehamil
            }{\phi_q}
            - \densityten^{q}_{p}\oneten^{r}_{q}
            \braket*{\tilde{\chi}_{\alpha}}{\phi_r}
            \\
            &=
            \densityten^{q}_{p}\oneten^{r}_{q}
            C_{\alpha r}
            - \densityten^{q}_{p} \oneten^{r}_{q}
            C_{\beta r} \delta_{\alpha \beta}
            = 0,
        \end{align}
        if the basis set over the time-evolved orbitals is untruncated.
        This exact same cancellation will occur for the two-body Hamiltonian
        term in \autoref{eq:ket-q} as well as both the one- and two-body
        Hamiltonian terms in \autoref{eq:bra-q}.
