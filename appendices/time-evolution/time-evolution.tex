\chapter{Time evolution of the coupled cluster wavefunction}
    We compute the time evolution of any wavefunction from an initial state at
    time $t_0$ to a later time $t$ by
    \begin{align}
        P(t_0 \to t)
        \equiv |\braket{\psi(t)}{\psi(t_0)}|^2.
    \end{align}
    That is, we compute the squared overlap between the initial state
    $\ket{\psi(t_0)}$ and the final state $\ket{\psi(t)}$. In the case of
    coupled cluster and the use of the bivariational principle some care must be
    taken as to how the squared overlap should be computed. We get
    \begin{align}
        P(t_0 \to t)
        &\equiv |\braket{\tilde{\Psi}(t)}{\Psi(t_0)}|^2
        = \braket{\tilde{\Psi}(t)}{\Psi(t_0)}
        \braket{\tilde{\Psi}(t_0)}{\Psi(t)}.
    \end{align}
    Choosing $t_0 = 0$ as the ground state we can compute the overlap of the
    ground state to all later states $t$. For time-independent spin-orbitals we
    only evolve the amplitudes in time. We thus have to find an expression for
    the two inner-products below.
    \begin{align}
        \braket{\tilde{\Psi}(t)}{\Psi(0)}
        &=
        \bra{\tilde{\Phi}}\brak{1 + \Lambda(t)}e^{-T(t)}
        e^{T}\ket{\Phi},
        \\
        \braket{\tilde{\Psi}(0)}{\Psi(t)}
        &=
        \bra{\tilde{\Phi}}\brak{1 + \Lambda}e^{-T}
        e^{T(t)}\ket{\Phi}.
    \end{align}
    Note that $T(t) \neq T$ and $\Lambda(t) \neq \Lambda$.  We split up the
    equations on $\Lambda$ and expand the exponentials. As $T$ provides a net
    excitation of at least $1$ and $\Lambda$ a net relaxation of at least
    $1$\footnote{Note that this applies to the time-dependent versions of these
    operators as well as it is only the amplitudes that are time-dependent and
    not the creation nor the annihilation operators.}, only terms with a
    combination of $\Lambda$ and $T$ will survive. This yields
    \begin{align}
        \braket{\tilde{\Psi}(t)}{\Psi(0)}
        &=
        \bra{\tilde{\Psi}(t)}e^{-T(t)}e^T\ket{\Psi(0)}
        +
        \bra{\tilde{\Psi}(t)}\Lambda(t)e^{-T(t)}e^T\ket{\Psi(0)}
        \\
        &=
        1
        + \sum_{n = 0}^{\infty}\sum_{m = 0}^{\infty}
        \frac{1}{n!m!}
        \bra{\tilde{\Psi}(t)}\Lambda(t)[-T(t)]^nT^m\ket{\Psi(0)}.
    \end{align}
    This equation looks the same with the time-dependent state switched to the
    ket-side except that only the rightmost $T$ is now time-dependent.

    \section{Time evolution of the CCSD wavefunction}
        Restricting ourselves to singles and doubles approximation we will get
        that the $T$ operator can yield a net excitation of 1 and 2, whereas
        $\Lambda$ yields a net relaxation of 1 and 2. This truncates the
        infinite sums to $n, m \in \brac{0, 1, 2}$.
