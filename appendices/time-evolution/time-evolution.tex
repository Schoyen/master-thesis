\chapter{Coupled cluster autocorrelation}
    \label{app:cc-autocorrelation}
    We compute the autocorrelation of any wavefunction from an initial state at
    time $t_0$ to a later time $t$ by
    \begin{align}
        P(t_0 \to t)
        \equiv \abs{\braket{\psi(t)}{\psi(t_0)}}^2.
    \end{align}
    That is, we compute the squared overlap between the initial state
    $\ket{\psi(t_0)}$ and the final state $\ket{\psi(t)}$.
    % TODO: This should be in the theory section.
    In the bivariational formulation of Hilbert space, where the left- and
    right-hand states of the coupled cluster wave function differ, some care
    must be taken as to how the autocorrelation should be computed.
    \begin{align}
        P(t_0 \to t)
        &\equiv \abs{\braket{\tilde{\Psi}(t)}{\Psi(t_0)}}^2
        = \braket{\tilde{\Psi}(t)}{\Psi(t_0)}
        \braket{\tilde{\Psi}(t_0)}{\Psi(t)},
    \end{align}
    where we note that the two latter terms no loner a complex conjugates of one
    another.
    This is a consequence of treating the two different Hilbert spaces
    indepedently.
    For clarity we set $t_0 = 0$ and define $\ket{\Psi(0)}$ as the coupled
    cluster ground state, and $\bra{\tilde{\Psi}(0)}$ as the left-hand ground
    state.
    % TODO: Refer to the autocorrelation for the orbital-adaptive method.
    Restricting ourselves to the case of time-independent spin-orbitals, we need
    to evolve the $\clustlamp$- and $\clustamp$-amplitudes in time.
    We thus have to find an expression for the inner product of
    \begin{gather}
        \braket{\tilde{\Psi}(t)}{\Psi(0)}
        =
        \mel{\slat}{
            \brak{\1 + \clustl(t)}
            \exp[-\clust(t)]
            \exp[\clust]
        }{\slat},
    \end{gather}
    and where the second inner product can be found by replacing the
    time-dependence of the amplitudes to the other state.
    Note that $\clust(t) \neq \clust$ and $\clustl(t) \neq \clustl$.
    We split up the $\brak{\1 + \clustl}$-term, and expand the exponentials.
    The term with $\1$ will only give back the overlap between the reference
    states.
    As $\clust$ provides a net excitation of at least $1$ and $\clustl$ a net
    relaxation of at least $1$,\footnote{
        Note that this applies to the time-dependent versions
        of these operators as well. It is only the amplitudes that are
        time-dependent and not the creation nor the annihilation operators.
    } only terms with a combination of $\clustl$ and $\clust$ will survive.
    \begin{align}
        \braket{\tilde{\Psi}(t)}{\Psi(0)}
        &=
        1
        +
        \mel{\slat}{
            \clustl(t)
            \exp[-\clust(t)]
            \exp[\clust]
        }{\slat}
        \\
        &=
        1
        + \sum_{n = 0}^{\infty}\sum_{m = 0}^{\infty}
        \frac{(-1)^{n}}{n!m!}
        \mel{\slat}{
            \clustl(t)
            \clust^n(t)
            \clust^m
        }{\slat},
        \label{eq:cc-autocorrelation}
    \end{align}
    where the alternating sign comes from the left-hand side, time-dependent,
    cluster operator.
    To find explicit expressions for the autocorrelation, we need to look at
    specific truncation levels for the cluster operators.

    \section{Doubles autocorrelation}
        In the doubles approximation $\clust$ and $\clustl$ yield a net
        excitation and relaxation of 2, respectively.
        This means that $n, m \in \brac{0, 1}$ as any higher exponentials will
        leave the reference excited after the action of $\clustl$.
        Furthermore, for $n = m = 0$, $\clustl$ will annihilate the reference as
        it acts as a relaxation operator on the Fermi vacuum.
        We also have for $n = m = 1$ the reference will be left doubly excited
        thus annihilating the overlap.
        This leaves us with
        \begin{gather}
            \braket{\tilde{\Psi}(t)}{\Psi(0)}
            = 1
            + \mel{\slat}{
                \clustl(t)\brak{
                    -\clust(t) + \clust
                }
            }{\slat},
        \end{gather}
        Using SymPy \cite{sympy} to compute Wick's theorem and only keeping
        fully contracted terms, we get
        \begin{gather}
            \braket{\tilde{\Psi}(t)}{\Psi(0)}
            = 1
            + \frac{1}{4}\clustamp^{ab}_{ij} \clustlamp^{ij}_{ab}(t)
            - \frac{1}{4}\clustlamp^{ij}_{ab}(t) \clustamp^{ab}_{ij}(t).
        \end{gather}
        The bivariational conjugate of this equation consists of removing the
        time-dependence from the $\clustlamp$-amplitudes and switching the
        time-dependence in the $\clustamp$-amplitudes.

    \section{Singles-and-doubles autocorrelation}
        Restricting ourselves to the singles and doubles approximation we have
        that the $\clust$ operator can yield a net excitation of 1 and 2,
        whereas $\clustl$ can give a net relaxation of 1 and 2.
        This truncates the infinite sums in \autoref{eq:cc-autocorrelation} to
        $n, m \in \brac{0, 1, 2}$.
        Note however that for $n = m = 0$, $\clustl$ will annihilate the Fermi
        vacuum.
        We are then left with
        \begin{align}
            \braket{\tilde{\Psi}(t)}{\Psi(0)}
            &= 1
            + \mel{\slat}{
                \clustl(t)\brak{
                    - \clust(t) + \clust - \clust(t)\clust
                    + \half \clust(t)^2 + \half \clust^2
                }
            }{\slat}.
        \end{align}
        Using SymPy \cite{sympy} to compute this expression using Wick's theorem
        and only keeping the fully contracted terms, we find
        \begin{align}
            \braket{\tilde{\Psi}(t)}{\Psi(0)}
            &=
            1
            + \clustlamp^{i}_{a}(t) \brak{
                \clustamp^{a}_{i} - \clustamp^{a}_{i}(t)
            }
            + \clustlamp^{ij}_{ab}(t) \biggl[
                \frac{1}{4}\clustamp^{ab}_{ij}
                - \half \clustamp^{a}_{j} \clustamp^{b}_{i}
                \nonumber \\
                &\qquad
                - \clustamp^{a}_{i}(t) \clustamp^{b}_{j}
                - \half \clustamp^{a}_{j}(t) \clustamp^{b}_{i}(t)
                - \frac{1}{4} \clustamp^{ab}_{ij}(t)
            \biggr]
        \end{align}
        and the bivariational conjugate of this equation by switching the
        time-dependence of the amplitudes.
