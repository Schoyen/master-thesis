\chapter{Overlap between time-evolved coupled cluster wavefunctions}
    We compute the overlap of any wavefunction from an initial state at time
    $t_0$ to a later time $t$ by
    \begin{align}
        P(t_0 \to t)
        \equiv |\braket{\psi(t)}{\psi(t_0)}|^2.
    \end{align}
    That is, we compute the squared overlap between the initial state
    $\ket{\psi(t_0)}$ and the final state $\ket{\psi(t)}$. In the case of
    coupled cluster and the use of the bivariational principle some care must be
    taken as to how the squared overlap should be computed. We get
    \begin{align}
        P(t_0 \to t)
        &\equiv |\braket{\tilde{\Psi}(t)}{\Psi(t_0)}|^2
        = \braket{\tilde{\Psi}(t)}{\Psi(t_0)}
        \braket{\tilde{\Psi}(t_0)}{\Psi(t)}.
    \end{align}
    This is a consequence of treating the two different Hilbert spaces
    indepedently.  Choosing $t_0 = 0$ as the ground state we can compute the
    overlap of the ground state to all later states $t$. For time-independent
    spin-orbitals we only evolve the amplitudes in time. We thus have to find an
    expression for the two inner-products below.
    \begin{align}
        \braket{\tilde{\Psi}(t)}{\Psi(0)}
        &=
        \bra{\tilde{\Phi}}\brak{1 + \clustl(t)}e^{-\clust(t)}
        e^{\clust}\ket{\Phi},
        \\
        \braket{\tilde{\Psi}(0)}{\Psi(t)}
        &=
        \bra{\tilde{\Phi}}\brak{1 + \clustl}e^{-\clust}
        e^{\clust(t)}\ket{\Phi}.
    \end{align}
    Note that $\clust(t) \neq \clust$ and $\clustl(t) \neq \clustl$.  We split
    up the equations on $\clustl$ and expand the exponentials. As $\clust$
    provides a net excitation of at least $1$ and $\clustl$ a net relaxation of
    at least $1$\footnote{Note that this applies to the time-dependent versions
    of these operators as well. It is only the amplitudes that are
    time-dependent and not the creation nor the annihilation operators.}, only
    terms with a combination of $\clustl$ and $\clust$ will survive. This yields
    \begin{align}
        \braket{\tilde{\Psi}(t)}{\Psi(0)}
        &=
        \bra{\tilde{\Phi}}e^{-\clust(t)}e^{\clust}\ket{\Phi}
        +
        \bra{\tilde{\Phi}}\clustl(t)e^{-\clust(t)}e^{\clust}\ket{\Phi}
        \\
        &=
        1
        + \sum_{n = 0}^{\infty}\sum_{m = 0}^{\infty}
        \frac{1}{n!m!}
        \bra{\tilde{\Phi}}\clustl(t)[-\clust(t)]^n \clust^m\ket{\Phi}.
    \end{align}
    The conjugate of this equation is then
    \begin{align}
        \braket{\tilde{\Psi}(0)}{\Psi(t)}
        &=
        1 + \sum_{n = 0}^{\infty}\sum_{m = 0}^{\infty}
        \frac{1}{n!m!}
        \bra{\tilde{\Phi}}\clustl \brak{-\clust^n} \clust(t)^m\ket{\Phi}.
    \end{align}

    \section{Time-dependent overlap for the coupled cluster doubles
    wavefunction}
        In the doubles approximation $\clust$ and $\clustl$ yield a net
        excitation and relaxation of 2, respectively. This means that $n, m \in
        \brac{0, 1}$ as any higher exponentials will leave the reference
        excited. Furthermore, for $n = m = 0$ $\clustl$ will annihilate the
        reference. We also have for $n = m = 1$ that the reference will be left
        doubly excited thus annihilating the overlap. We then get
        \begin{align}
            \braket{\tilde{\Psi}(t)}{\Psi(0)}
            &= 1 + \brabislat\clustl(t)\brak{
                -\clust(t) + \clust
            }\ketslat,
            \\
            \braket{\tilde{\Psi}(0)}{\Psi(t)}
            &= 1 + \brabislat\clustl\brak{
                -\clust + \clust(t)
            }\ketslat.
        \end{align}
        Using SymPy \cite{sympy} to construct the tensor contractions we are
        left with
        \begin{align}
            \braket{\tilde{\Psi}(t)}{\Psi(0)}
            &= 1 + \frac{1}{4}\clustamp^{ab}_{ij} \clustlamp(t)^{ij}_{ab}
            - \frac{1}{4}\clustlamp(t)^{ij}_{ab}
            \clustamp(t)^{ab}_{ij},
            \\
            \braket{\tilde{\Psi}(0)}{\Psi(t)}
            &= 1 - \frac{1}{4}\clustlamp^{ij}_{ab} \clustamp^{ab}_{ij}
            + \frac{1}{4}\clustlamp^{ij}_{ab}
            \clustamp(t)^{ab}_{ij}.
        \end{align}

    \section{Time-dependent overlap for the coupled cluster singles-and-doubles
    wavefuntion}
        Restricting ourselves to the singles and doubles approximation we will
        get that the $\clust$ operator can yield a net excitation of 1 and 2, whereas
        $\clustl$ yields a net relaxation of 1 and 2. This truncates the
        infinite sums to $n, m \in \brac{0, 1, 2}$. Note however that for $n = m
        = 0$, $\clustl$ will annihilate the vacuum. We are then left with
        \begin{align}
            \braket{\tilde{\Psi}(t)}{\Psi(0)}
            &= 1
            + \bra{\tilde{\Phi}}\clustl(t)\brak{
                - \clust(t) + \clust - \clust(t)\clust
                + \half \clust(t)^2 + \half \clust^2
            }\ket{\Phi},
            \\
            \braket{\tilde{\Psi}(0)}{\Psi(t)}
            &= 1
            + \bra{\tilde{\Phi}}\clustl\brak{
                - \clust + \clust(t) - \clust\clust(t)
                + \half \clust^2 + \half \clust(t)^2
            }\ket{\Phi}.
        \end{align}
        We again utilize SymPy \cite{sympy} to get explicit tensor contractions.
        This yields
        \begin{align}
            \braket{\tilde{\Psi}(t)}{\Psi(0)}
            &=
            1
            + \clustlamp(t)^{i}_{a} \brak{
                \clustamp^{a}_{i} - \clustamp(t)^{a}_{i}
            }
            \nonumber \\
            &\qquad
            + \clustlamp(t)^{ij}_{ab} \brak{
                \frac{1}{4}\clustamp^{ab}_{ij}
                - \half \clustamp^{a}_{j} \clustamp^{b}_{i}
                - \clustamp(t)^{a}_{i} \clustamp^{b}_{j}
                - \half \clustamp(t)^{a}_{j} \clustamp(t)^{b}_{i}
                - \frac{1}{4} \clustamp(t)^{ab}_{ij}
            },
            \\
            \braket{\tilde{\Psi}(0)}{\Psi(t)}
            &=
            1
            + \clustlamp^{i}_{a} \brak{
                \clustamp(t)^{a}_{i}
                - \clustamp^{a}_{i}
            }
            \nonumber \\
            &\qquad
            + \clustlamp^{ij}_{ab} \brak{
                \frac{1}{4}\clustamp(t)^{ab}_{ij}
                - \half \clustamp^{a}_{j} \clustamp^{b}_{i}
                - \clustamp(t)^{a}_{i}\clustamp^{b}_{j}
                - \half \clustamp(t)^{a}_{j} \clustamp(t)^{b}_{i}
                - \frac{1}{4} \clustamp^{ab}_{ij}
            }.
        \end{align}
