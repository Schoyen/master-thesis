\chapter{Reformulating the amplitude equations}
    \label{app:ccsd-dot-products}
    We will in this appendix show how to formulate the tensor contractions
    occuring in the the coupled cluster equations from Gauss et
    al.\cite{gauss1995coupled} as matrix products. The reason we wish to do this
    is to be able to perform these contractions as dot products (or matrix
    products) as there exists highly optimized code performing these operations,
    e.g., BLAS\footnote{BLAS can be found here:
    \url{http://www.netlib.org/blas/}}.

    To be able to treat tensors of rank $> 2$ as matrices we have to create
    \emph{compound indices} by stacking the dimensions after one another. For
    instance, by looking at the tensor $g \in \mathbb{C}^{I \times J \times K
    \times L}$ where we denote a single element by $g_{ijkl}$. Here $g$ denotes
    a tensor of rank 4. By creating compound indices $\tilde{I} = IJ$ and
    $\tilde{K} = KL$ we can create a new tensor $\tilde{g} =
    \mathbb{C}^{\tilde{I} \times \tilde{K}}$ of rank 2 (represented as a
    matrix). Using the indices $\tilde{i} = iJ + j$ and $\tilde{k} = kL + l$ we
    now construct $\tilde{g}$ in such a way that $\tilde{g}_{\tilde{i}\tilde{k}}
    = g_{ijkl}$.
