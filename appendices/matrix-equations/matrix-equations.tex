\chapter{Reformulating the amplitude equations as matrix products}
    \label{app:ccsd-dot-products}
    We will in this appendix show how to formulate the tensor contractions
    occuring in the coupled cluster equations as matrix products. The reason we
    wish to do this is to be able to perform these contractions as dot products
    (or matrix products) as there exists highly optimized code performing these
    operations, e.g., BLAS\footnote{BLAS can be found here:
    \url{http://www.netlib.org/blas/}}.

    To be able to treat tensors of rank $> 2$ as matrices we have to create
    \emph{compound indices} by stacking the dimensions after one another. For
    instance, by looking at the tensor $g \in \mathbb{C}^{I \times J \times K
    \times L}$, where we denote a single element by $g_{ijkl}$. Here $g$ is a
    tensor of rank 4. By creating compound indices $\tilde{I} = IJ$ and
    $\tilde{K} = KL$ we can create a new tensor $\tilde{g} =
    \mathbb{C}^{\tilde{I} \times \tilde{K}}$ of rank 2 (represented as a
    matrix). Using the indices $\tilde{i} = iJ + j$ and $\tilde{k} = kL + l$ we
    now construct $\tilde{g}$ in such a way that $\tilde{g}_{\tilde{i}\tilde{k}}
    = g_{ijkl}$.

    It is also possible to create compound indices of more than two indices. For
    instance; choosing $\tilde{J} = JKL$ and setting $\tilde{j} = jKL + kL + l$
    we can construct $\bar{g} = \mathbb{C}^{I\times \tilde{J}}$ where
    $\bar{g}_{i\tilde{j}} = g_{ijkl}$.

    For the sake of brevity and clarity we will in the following avoid renaming
    the compound indices and their sizes, but we will instead indicate with a
    comma where we construct new indices.
    % TODO: Explain the notation with paranthesis for reshapes and the use of
    % tilde when building a new u-matrix.
    % TODO: Add examples of the notation with twiddle u, reshapes and compound
    % indices.
    % TODO: Explain the index ordering with upper indices as the most
    % significant and lower indices as the least significant.
    % TODO: Explain the cost with reshapes and how this is potentially faster
    % than the price of cache misses with a slanted view on a matrix.

    \section{Reformulating the CCD equations}

    \section{Reformulating the CCSD equations}
        We use the expressions for the CCSD equations derived by Gauss et
        al.\cite{gauss1995coupled}. We start with the \emph{effective double
        excitation amplitudes} found at the bottom of table 3 in their article.
        Note that we rename $\tilde{\tau} \to \xi$ thus reserving the twiddle
        for intermediate calculations.

        \begin{gather}
            \tau^{ab}_{ij} = t^{ab}_{ij}
            + \frac{1}{2}P(ij)P(ab)t^{a}_{i} t^{b}_{j}
            \\
            \implies \tau_{ab, ij} = t_{ab, ij}
            + \frac{1}{2}P(ij)P(ab)\left(t_{a, i} t_{b, j}\right)_{ab, ij},
            \\
            \xi^{ab}_{ij} = t^{ab}_{ij}
            + \frac{1}{4}P(ij)P(ab)t^{a}_{i}t^{b}_{j}
            \\
            \implies
            \xi_{ab, ij} = t_{ab, ij}
            + \frac{1}{4}P(ij)P(ab)\left(t_{a, i} t_{b, j}\right)_{ab, ij}.
        \end{gather}

        Next we look at the one-body intermediates found at the top of table 3
        in the article by Gauss et al.\cite{gauss1995coupled}. We use the
        notation
        \begin{align}
            u^{am}_{ef} \equiv \bra{am}\ket{ef},
        \end{align}
        that is, we treat the matrix elements $u$ as the antisymmetric matrix
        elements of the two-body operator.
        \begin{gather}
            F^{a}_{e} = f^{a}_{e}
            - \frac{1}{2}f^{m}_{e}t^{a}_{m}
            + t^{f}_{m}u^{am}_{ef}
            - \frac{1}{2}\xi^{af}_{mn}u^{mn}_{ef}
            \\
            \implies
            F_{a, e} = f_{a, e}
            - \frac{1}{2}t_{a, m}f_{m, e}
            + \left(t_{fm}\tilde{u}_{fm, ae}\right)_{a, e}
            - \frac{1}{2}\xi_{a, fmn}\tilde{u}_{fmn, e},
            \\
            F^{m}_{i} = f^{m}_{i} + \frac{1}{2}f^{m}_{e}t^{e}_{i}
            + t^{e}_{n}u^{mn}_{ie}
            + \frac{1}{2}\xi^{ef}_{in}u^{mn}_{ef}
            \\
            \implies
            F_{m, i} = f_{m, i}
            + \frac{1}{2}f_{m, e} t_{e, i}
            + \left(t_{en}\tilde{u}_{en, mi}\right)_{m, i}
            + \frac{1}{2}\tilde{u}_{m, nef}\tilde{\xi}_{nef, i},
            \\
            F^{m}_{e} = f^{m}_{e} + t^{f}_{n}u^{mn}_{ef}
            \\
            \implies
            F_{m, e} = f_{m, e}
            + \left(t_{fn}\tilde{u}_{fn, me}\right)_{m, e}.
        \end{gather}

        % TODO: Add an overview of new views on top of the tensors that is
        % needed.

        We now move on to the two-body intermediates found just below the
        one-body intermediates in table 3 in the article by Gauss et
        al.\cite{gauss1995coupled}. To avoid storing two matrices with $M^4$
        elements we will not create the intermediate $W^{ab}_{ef}$ but rather
        compute the products in place in the amplitude equations by splitting up
        the products and do them one-by-one (this will shown in due time).
        % TODO: Add link to the explicit amplitude equations where this occurs.
        We will therefore still preserve the asymptotical scaling
        $\mathcal{O}(M^4N^2)$ but add a constant term at the price of saving
        memory.
