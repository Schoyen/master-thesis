\chapter{Quantum Mechanics}
    In this appendix we'll add proofs and derivations of expressions arising
    from the theory section on quantum mechanics and many-body theory.

    \section{Deriving the reference energy}
        \label{sec:deriving-the-reference-energy}
        For the orthormal basis $\brac{\ket{\phi_p}}_{p = 1}^{L}$ with the $N$
        first states occupied in the reference state $\ket{\refslat}$, we
        compute the reference energy from \autoref{eq:reference-energy}.
        We use Wick's theorem with normal ordering relative to the Fermi vacuum,
        that is, we treat the reference state as our vacuum state.
        Starting with the expectation value of the one-body operator, we get
        \begin{align}
            \bra{\refslat}\onehamil\ket{\refslat}
            &=
            \oneten^{p}_{q}
            \bra{\refslat}\ccr{p}\can{q}\ket{\refslat}
            =
            \oneten^{p}_{q}
            \bra{\refslat}\para{
                \wick{
                    \c {\ccr{p}}
                    \c {\can{q}}
                }
                + \brac{
                    \ccr{p}\can{q}
                }
            }
            \ket{\refslat}
            \\
            &=
            \oneten^{p}_{q}
            \delta_{p \in o}
            \delta_{pq}
            \braket{\refslat}{\refslat}
            +
            0
            = \oneten^{i}_{i},
        \end{align}
        where we use $\delta_{p \in o}$ to denote that the general index
        $p$ must be contained in the set of occupied indices $o = \brac{1,
        \dots, N}$.
        This notation is similar to the one used by
        \citeauthor{crawford2000introduction} in
        \citetitle{crawford2000introduction}\cite{crawford2000introduction}.
        To see why the latter term becomes zero, we have to consider the four
        combinations of indices that can be summed over in the operators.
        From \autoref{sec:fermi-vacuum} we know that only terms with both
        occupied or both virtual indices will contribute to the expectation
        value.
        As the operators are normal-ordered relative to the Fermi vacuum they
        will destroy the reference state.
        The expectation value of the two-body term is given by
        \begin{align}
            \bra{\refslat}\twohamil\ket{\refslat}
            &=
            \frac{1}{4}u^{pq}_{rs}
            \bra{\refslat}\ccr{p}\ccr{q}\can{s}\can{r}\ket{\refslat}.
        \end{align}
        For brevity, we will only write out the operator strings.
        \begin{align}
            \ccr{p}\ccr{q}\can{s}\can{r}
            &=
            \wick{
                \c1 {\ccr{p}}
                \c2 {\ccr{q}}
                \c1 {\can{s}}
                \c2 {\can{r}}
            }
            +
            \wick{
                \c2 {\ccr{p}}
                \c1 {\ccr{q}}
                \c1 {\can{s}}
                \c2 {\can{r}}
            }
            + \brac{
                \wick{
                    \c {\ccr{p}}
                    {\ccr{q}}
                    \c {\can{s}}
                    {\can{r}}
                }
            }
            + \brac{
                \wick{
                    \c {\ccr{p}}
                    {\ccr{q}}
                    {\can{s}}
                    \c {\can{r}}
                }
            }
            \nonumber \\
            &\qquad
            + \brac{
                \wick{
                    {\ccr{p}}
                    \c {\ccr{q}}
                    \c {\can{s}}
                    {\can{r}}
                }
            }
            + \brac{
                \wick{
                    {\ccr{p}}
                    \c {\ccr{q}}
                    {\can{s}}
                    \c {\can{r}}
                }
            }
            + \brac{
                {\ccr{p}}
                {\ccr{q}}
                {\can{s}}
                {\can{r}}
            }
            \\
            &=
            -\delta_{p \in o}
            \delta_{q \in o}
            \delta_{ps}
            \delta_{qr}
            + \delta_{p \in o}
            \delta_{q \in o}
            \delta_{pr}
            \delta_{qs}
            - \delta_{p \in o}
            \delta_{ps}
            \brac{
                \ccr{q}
                \can{r}
            }
            \nonumber \\
            &\qquad
            + \delta_{p \in o}
            \delta_{pr}
            \brac{
                \ccr{q}
                \can{s}
            }
            + \delta_{q \in o}
            \delta_{qs}
            \brac{
                \ccr{p}
                \can{r}
            }
            - \delta_{q \in o}
            \delta_{qr}
            \brac{
                \ccr{p}
                \can{s}
            }
            \nonumber \\
            &\qquad
            + \brac{
                {\ccr{p}}
                {\ccr{q}}
                {\can{s}}
                {\can{r}}
            }.
        \end{align}
        For the same reason as with the expectation value of the one-body
        operator, all terms with a normal-ordered pair of creation and
        annihilation operators will be zero.
        This means that we are left with the two fully contracted terms and the
        normal-ordered product with two pairs of creation and annihilation
        operators.
        We note that there exists a combination of indices which will \emph{not}
        destroy the reference even though the operators are normal-ordered.
        This occurs when both creation operators act on virtual states and the
        two annihilation operators act on occupied states, viz.
        \begin{align}
            \brac{
                {\ccr{a}}
                {\ccr{b}}
                {\can{j}}
                {\can{i}}
            }
            \ket{\refslat}
            = A\ket{\slat^{ab}_{ij}},
        \end{align}
        where $A$ is some phase factor.
        In other words, this combination of operators will leave the reference
        state excited, but due to the orthonormality of the basis states, the
        overlap between the reference state and the excited state will be zero.
        We are thus left with
        \begin{align}
            \bra{\refslat}\twohamil\ket{\refslat}
            &=
            \frac{1}{4}\twoten^{pq}_{rs}\brac{
                -\delta_{p \in o}
                \delta_{q \in o}
                \delta_{ps}
                \delta_{qr}
                + \delta_{p \in o}
                \delta_{q \in o}
                \delta_{pr}
                \delta_{qs}
            }\braket{\refslat}{\refslat}
            \\
            &=
            -\frac{1}{4} \twoten^{ij}_{ji}
            + \frac{1}{4} \twoten^{ij}_{ij}
            = \half \twoten^{ij}_{ij},
        \end{align}
        where we've used the anti-symmetry of the two-body tensor to collect the
        two remaining terms, that is, $\twoten^{ij}_{ij} =-\twoten^{ij}_{ji}$.
        In total we are left with the reference energy of the electronic
        Hamiltonian
        \begin{align}
            \energyref
            &=
            \bra{\refslat}\hamil\ket{\refslat}
            =
            \bra{\refslat}\onehamil\ket{\refslat}
            + \bra{\refslat}\twohamil\ket{\refslat}
            =
            \oneten^{i}_{i}
            + \half \twoten^{ij}_{ij},
        \end{align}
        which is what we wanted to show.
