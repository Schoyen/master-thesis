\chapter{Conclusions and perspective}
    % Summarize achievements and goals
    Our initial goal of this thesis was to implement time-dependent
    coupled-cluster doubles (CCD/TDCCD), singles-and-doubles (CCSD/TDCCSD), and
    orbital-adaptive time-dependent coupled-cluster doubles (NOCCD/OATDCCD)
    methods.
    These were to be tested on one- and two-dimensional single and double
    quantum dots.
    However, due to the novelty of the OATDCCD method this thesis diverged to a
    new set of goals.
    We therefore switched focus to the stability of real-time coupled-cluster
    methods -- which in itself is a rather new research area -- with
    collaborators at the Hylleraas Center.
    This will lead to a forthcoming publication \cite{oa-stability}.
    The focus is on atoms and molecules as these systems are more applicable and
    common among quantum chemists than quantum dots.
    Also, due to the close collaboration with \citeauthor{greg-winther} this
    proved a natural divergence on the systems to be explored on our separate
    theses.
    To fully challenge and explore the coupled-cluster methods we've implemented
    a set of Hartree-Fock (HF/RHF/UHF/TDHF) solvers and configuration
    interaction (CISD\dots/TDCISD\dots) methods which allow us to improve and
    compare results from the coupled-cluster solvers.

    In this thesis we've described the implementation of the TDHF, TDCI and
    TDCC/OATDCCD methods along with descriptions on how to represent the quantum
    systems we are exploring.
    We've demonstrated that our methods work as expected by comparing with known
    results from the scientific literature.
    We've also compared the various truncation levels of the
    coupled-cluster\footnote{%
        Up to and including the CCSD truncation level.
    }
    and the configuration interaction methods, and demonstrated how the
    Hartree-Fock approach can be used to improve these results by providing an
    optimal single reference state.
    We then moved on to demonstrate that orbital-adaptive time-dependent
    coupled-cluster doubles is more stable than time-dependent coupled-cluster
    singles-and-doubles subject to very intense fields.
    We demonstrated that the inclusion of explicit orbital rotations -- as
    conjectured by \citeauthor{pedersen2018symplectic}
    \cite{pedersen2018symplectic} -- made the method more robust for the Helium
    and Beryllium systems subject to very intense laser field than with static
    orbitals in the regular time-dependent coupled-cluster method.

    Having verified that our implemented methods worked as expected we went on
    to apply the methodology to four different topics.
    \begin{enumerate}
        \item We explored the \ch{LiH}-molecule subject to a laser field
            polarized along the $x$- and $z$-direction and measured absorption
            energies using TDHF, TDCCSD, and OATDCCD.
            This was a study done by \citeauthor{nest} \cite{nest}, and we were
            able to reproduce their results.
            We also used CISD to compute all dipole-allowed transition energies
            from the ground state to higher excited states.
            This allows us to compare the coupled-cluster results with
            transitions from higher lying levels than reported by
            \citeauthor{nest} \cite{nest}.
        \item We then set out to demonstrate that the coupled-cluster
            formulation lets us explore systems that are much larger than what
            is possible using full configuration interaction theory.
            We chose to explore the absorption energies of \ch{Ne} and \ch{Ar},
            that is, atoms with $N = 10$ and $N = 18$ particles respectively.
            We used two different basis sets and thereby also demonstrated that
            in order to get good results we require a diffuse correlation
            consistent basis set (aug-cc-pVDZ).
            The results for \ch{Ne} were compared with existing works.
            We were unable to find comparable results for \ch{Ar}, but we report
            the first dipole allowed transition in the aug-cc-pVDZ basis to be
            $\Delta\energy = \SI{0.4677}{\hartree}$.
        \item Having demonstrated the applicability of the implemented methods,
            we set out to demonstrate its versatility as well.
            We modelled a spin-dependent laser field as done by
            \citeauthor{isborn} \cite{isborn}.
            We were able to achieve comparable results using OATDCCD, but we
            also showed how the general TDHF-method failed to reproduce the
            results from TDUHF reported by \citeauthor{isborn} \cite{isborn} for
            larger molecules.
        \item Finally we explored ionization in a one-dimensional Beryllium
            atom.
            We demonstrated how the choice of basis limits the amount of
            ionization.
            We noted that if one is to consider ionization in a
            three-dimensional system more flexible basis sets should be
            utilized.
    \end{enumerate}
    All in all we have demonstrated that our implemented libraries are robust
    and that they can be applied to a wide range of physical and chemical
    systems.
    Furthermore, this conclusion should be seen in conjunction with the results
    from the work of \citeauthor{greg-winther} \cite{greg-winther} as we've used
    the same framework applied to different systems.
    We've limited the discussion of the quantum dots in this thesis, but these
    are explored in detail by \citeauthor{greg-winther}.
    Also, as part of these theses there is an ongoing publication where we apply
    these methods to more exotic systems of quantum dots \cite{td-quantum-dots}.


    \section{Future prospects}
        \label{sec:future-work}
        Even though we have demonstrated a wide range of applications, there is
        still a vast amount of studies that can be done in lieu of this work.
        The future prospects from this thesis can be divided into two parts: the
        continued development of the methods and solvers, and the application of
        the methods to unexplored systems.
        We will discuss these two prospect categories separately.

        \subsection{Development of the libraries}
            In this thesis we have explored a significant amount of methodology
            and techniques used in real-time electronic many-body theories, but
            we have in no way exhausted the space of possibilities.
            Furthermore, much of our work is written in such a way that we wish
            to inspire continued development by other students and researchers,
            and we will hopefully be able to publish a software specification.
            But as such, we will here list some topics we deem interesting to
            explore and include in the implemented methods.
            \begin{itemize}
                \item CuPy \cite{cupy} is a Python library resembling NumPy
                    \cite{numpy} in both use and content, however CuPy is an
                    implementation of arrays and linear algebra on a graphics
                    processing unit (GPU).
                    Our libraries were originally set up to allow for the usage
                    of CuPy, but due to a few shortcomings we've not had the
                    time to fully integrate this into our systems.
                    This should absolutely be explored
                    further as this will almost surely increase the speed of the
                    methods.
                \item There exists excellent differential equation libraries
                    such as SciPy \cite{scipy} and diffeqpy \cite{julia-diff}
                    which would remove the need for self-implemented solvers as
                    these can potentially lead to bugs and are most likely not
                    properly optimized.
                    However, do note that for the same reasons as discussed in
                    \autoref{subsec:symplectic}, we want our integrators to be
                    symplectic, and SciPy does not have any implemented
                    symplectic solvers.
                    Diffeqpy on the other hand do contain quite a few solvers
                    which are symplectic.
                    A proper interface towards either or both of these libraries
                    should be implemented and a thorough study of the effects of
                    various solvers can be explored.
                \item To properly study ionization we should investigate
                    grid-based basis sets \cite{takeshi, miyagi_and_madsen,
                    hochstuhl2014time}.
                    This allows for a more flexible description away from
                    equilibrium.
                    In particular the discrete-variable-representation as used
                    by \citeauthor{miyagi_and_madsen} \cite{miyagi_and_madsen}
                    is of interest.
                \item The inclusion of triples amplitudes in TDCCSDT and OATDCCD
                    is of interest.
                    In the latter case this would include a substantial
                    increase in complexity in the orbital equations.
                    However, the OATDCCDT-method has -- to the authors knowledge
                    -- \emph{never} been implemented.
                \item One of the main obstacles -- in the authors opinion -- for
                    an effective workflow with the implemented libraries are: the
                    reimplementation of methods used to sample various
                    quantities, e.g., the dipole moment, and the occasional
                    manual optimization to ensure convergence.
                    We've stressed a philosophy where all methods should behave
                    in a similar way, that is, they compute the ground state,
                    and they open up for propagation.
                    All quantities should have a similar interface when it comes
                    to measurement, but occasionally time and lack of creativity
                    have hindered this philosophy leading to small discrepancies
                    in the interfaces between the different methods.
                    This is frustrating as each sampling script must be tailored
                    to fit each method.

                    The second obstacle with manual optimization should be
                    improved by using adaptive techniques allowing for automatic
                    improvement of convergence thresholds.
                    For example, using a gradient descent method to improve upon
                    a converged state.
                    Directly related to this is also an attempt at implementing
                    the Kalman filter as an improvement to the alpha filter
                    discussed in \autoref{sec:convergence}.
            \end{itemize}

        \subsection{Application to physical and chemical systems}
            We have in this thesis and the thesis by \citeauthor{greg-winther}
            \cite{greg-winther} demonstrated a wide range of applications of the
            libraries we have developed.
            One of the main difficulties in writing this thesis has been the
            downscaling of the amount of results to present.
            There are a multitude of systems that we have not had the time to
            include between the four covers that our theses consist of.
            Along with our work there is also two manuscripts in preparation.
            The first manuscript concerns itself with the stability of
            time-dependent coupled-cluster methods \cite{oa-stability} and
            serves as a follow up to the work by
            \citeauthor{pedersen2018symplectic} \cite{pedersen2018symplectic}.
            The second manuscript is an extensive study on the time evolution of
            quantum dots \cite{td-quantum-dots} and seeks to apply the developed
            framework to studies on various quantum dot systems.
            Therefore, it is not surprising that much of this work will be
            presented in later studies as part of \citeauthor{greg-winther}'s
            and my own PhD-studies at the Centre for Computing in Science
            Education.

            \begin{itemize}
                \item Interesting systems to include are: three-dimensional
                    quantum dots, magnetic fields with spin-coupling, and
                    exploration, and relativistic systems.
                \item Large systems of atoms and molecules are often restricted
                    to Hartree-Fock simulations.
                    Real-time simulations of any such system could potentially
                    lead to new results.
                \item In the work of \citeauthor{greg-winther}
                    \cite{greg-winther} the inclusion of an orbital-dependent
                    magnetic field is done.
                    By including a spin-dependent magnetic field this could
                    potentially lead to interesting physics.
                    In \autoref{sec:isborn} we saw some examples of a
                    spin-dependent laser field leading to ``exotic''
                    transitions.
                \item Studies of nuclear systems and nuclear reactions such as
                    fission and fusion are immensely interesting.
                    The exploration of such phenomena using real-time
                    coupled-cluster methods holds great promise for applying
                    \emph{ab initio} approaches to these challenging topics.
                \item In \autoref{sec:noble-gasses} we demonstrated the
                    importance of basis sets when it comes to the quality of the
                    results achieved from dynamical systems.
                    A proper study of this has been left out of this thesis, but
                    is of fundamental interest.
                \item We have in this thesis limited our attention to laser
                    fields described in the dipole approximation.
                    The inclusion of higher-order multipoles could potentially
                    let us move into the high-energy regime of Röntgen and the
                    like.
                    This would also require relativistic considerations as well.
            \end{itemize}
