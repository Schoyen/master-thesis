\chapter{Conclusions and perspective}
    % Summarize achievements and goals
    Our initial goal of this thesis was to implement time-dependent
    coupled-cluster doubles, singles-and-doubles, and orbital-adaptive
    time-dependent coupled-cluster doubles methods.
    These were to be tested on one- and two-dimensional single and double
    quantum dots.
    However, due to the novelty of the  orbital-adaptive time-dependent
    coupled-cluster doubles method this thesis diverged to a new set of goals.
    We therefore switched focus to the stability of real-time coupled-cluster
    methods -- which in itself is a rather new research area -- with researchers
    at the Hylleraas Centre.
    This lead to a larger focus on atoms and molecules as these systems are more
    known and explored for quantum chemists than quantum dots.
    Furthermore, to fully challenge and explore the coupled-cluster method we've
    implemented a set of Hartree-Fock solvers and configuration interaction
    methods which lets us improve and compare results from the coupled-cluster
    solvers.

    In this thesis we've described the implementation of all of these methods
    along with descriptions on how to represent the quantum systems we are
    exploring.
    We've demonstrated that our methods work as expected by comparing with known
    results from litterature.
    We've also compared the various truncation levels of the coupled-cluster
    and configuration interaction methods and demonstrated how Hartree-Fock can
    be used to improve on these results.
    We then moved on to demonstrate that orbital-adaptive time-dependent
    coupled-cluster doubles is more stable than time-dependent coupled-cluster
    singles-and-doubles for very intense fields.
    We concluded that the inclusion of explicit orbital rotations -- as
    conjectured by \citeauthor{pedersen2018symplectic}
    \cite{pedersen2018symplectic} -- made the method more robust for the Helium
    and Beryllium systems subject to very intense laser field than with static
    orbitals in the regular time-dependent coupled-cluster method.
    Further studies of this phenomenon is discussed in \citetitle{oa-stability}
    \cite{oa-stability} of which this author is a part of.

    Having verified that our implemented methods worked as expected we went on
    to apply the methodology to four different topics.
    \begin{enumerate}
        \item We explored the \ch{LiH}-molecule subject to a laser field
            polarized along the $x$- and $z$-direction and measured absorption
            energies using TDHF, TDCCSD, and OATDCCD.
            This was a study done by \citeauthor{nest} \cite{nest}, and we were
            able to get comparable results.
            We also used CISD to compute all dipole-allowed transition energies
            from the ground state to a higher excited state which let us compare
            higher energy transitions than reported by \citeauthor{nest}.
        \item We then set out to demonstrate that the coupled-cluster
            formulation lets us explore systems that are much larger than what
            is possible using full configuration interaction.
            We chose to explore the absorption energies of \ch{Ne} and \ch{Ar},
            that is, atoms with $N = 10$ and $N = 18$ particles respectively.
            We used two different basis sets and thereby also demonstrated that
            in order to get good results we require a diffuse correlation
            consisted basis set.
            The results for Neon was compared with litterature, whereas the
            Argon results are just presented as we were unable to find any known
            results for these absorption energies.
        \item Having demonstrated the applicability of the implemented methods,
            we set out to demonstrate its versatility as well.
            We modelled a spin-dependent laser field as done by
            \citeauthor{isborn} \cite{isborn}.
            We were able to achieve comparable results using OATDCCD, but we
            also showed how the general TDHF-method failed to reproduce the
            results from TDUHF reported by \citeauthor{isborn} for the larger
            molecules.
        \item Finally we explored ionization in a one-dimensional Beryllium
            atom.
            We demonstrated how the choice of basis limits the amount of
            ionization and discussed that in order to achieve ``proper''
            ionization we require continuum states either from a change grid
            representation of the underlying atomic orbital basis or through
            some other means.
    \end{enumerate}
    All in all we have demonstrated that our implemented libraries are robust
    and that they can be applied to a wide range of physical and chemical
    systems.
    Furthermore, this conclusion should be seen in conjunction with the results
    from the work of \citeauthor{greg-winther} \cite{greg-winther} as we've used
    the same framework applied to different systems.
    We've limited the discussion of the quantum dots in this thesis, but these
    are explored in greater detail by \citeauthor{greg-winther}.
    Also, as part of these theses there is an ongoing publication where we apply
    these methods to more exotic systems of quantum dots \cite{td-quantum-dots}.


    \section{Future prospects}
        \label{sec:future-work}
        Even though we have demonstrated a wide range of applications, there is
        still a vast range of studies that can be done in lieu of this work.
        In fact, much of the challenge in this thesis has been the narrowing of
        focus in order to get sensible results.
        The future prospects from this thesis can be divided into two parts: the
        continued development of the methods and solvers, and the application of
        the methods to unexplored systems.
        We will discuss these two prospect categories separately.

        \subsection{Development of the libraries}
            In this thesis we have explored a significant amount of methodology
            and techniques used in real-time electronic many-body theories, but
            we have in no way exhausted the space of possibilities.
            Furthermore, much of our work is written in such a way that we wish
            to inspire continued development by other students and researchers,
            and we will hopefully be able to publish a software specification.
            But as such, we will here list some topics we deem interesting to
            explore and include in the implemented methods.
            \begin{itemize}
                \item GPU.
                \item Differential equations.
                \item Grid-solvers.
                \item OATDCCDT.
                \item TDCCSDT.
                \item Three-dimensional quantum dots.
                \item Magnetic fields with spin-coupling.
                \item Relativistic considerations.
                \item Optimzation using C++ or Julia.
                \item Implement the Kalman filter.
            \end{itemize}

        \subsection{Application to physical and chemical systems}
            We have in this thesis and the thesis by \citeauthor{greg-winther}
            \cite{greg-winther} demonstrated a wide range of applications of the
            libraries we have developed.
            One of the main difficulties in writing this thesis has been the
            downscaling of the amount of results to present.
            There are a multitude of systems that we have not had the time to
            include between the four covers that our theses consist of.
            Along with our work there is also two manuscripts in preparation.
            The first manuscript concerns itself with the stability of
            time-dependent coupled-cluster methods \cite{oa-stability} and
            serves as a response to the work by
            \citeauthor{pedersen2018symplectic} \cite{pedersen2018symplectic}.
            The second manuscript is an extensive study on the time-evolution of
            quantum dots \cite{td-quantum-dots} and seeks to apply the developed
            framework to studies on various quantum dot systems.
            Therefore, it is not surprising that much of this work will be
            presented in later studies as part of \citeauthor{greg-winther}'s
            and my PhD-studies at the Centre for Computing in Science Education.

            \begin{enumerate}
                \item Large atoms.
                \item Large molecules.
                \item Atoms, molecules and quantum-dots subject to magnetic
                    fields. Spin-orbit coupling.
                \item Real-time fission and fusion simulations.
            \end{enumerate}
