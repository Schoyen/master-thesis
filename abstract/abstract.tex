Solutions of the time-dependent Schrödinger equation are central to the
understanding of the interaction between particles and external probes.
The increasing availability of intense laser fields in experiments has spawned
an interest in the study of the dynamics of many-body systems interacting with
lasers.
However, the complexity of the many-body problem quickly becomes a significant
roadblock in the exploration of larger atoms and molecules thus limiting the
size of the systems that can be explored.
Real-time \textit{ab initio} electronic structure theory provides promising
methods for investigating the dynamics of matter-field interactions and we've
implemented several many-body methods which we use to analyze atoms and
molecules subject to intense laser fields.

We implement three different \emph{ab initio} real-time methods: Hartree-Fock,
configuration interaction, and coupled-cluster which we apply to systems of
atoms and molecules.
A thorough theory section outlines the foundation of our work.
We demonstrate the strengths of the implemented methods and highlight the
applicability of the orbital-adaptive time-dependent coupled-cluster method with
doubles excitations by showcasing how this method is stable where the more
conventional time-dependent coupled-cluster method with singles-and-doubles
excitations fail.
We include -- to our knowledge -- unseen results on dipole transition energies
for the Argon atom in an aug-cc-pVDZ basis, a system of $N = 18$ particles.
% TODO: List these energies
Finally, we end this thesis demonstrating the versatility of our method by
exhibiting simulations of exotic systems with spin-dependent fields an
ionization of one-dimensional Beryllium.
