\documentclass{beamer}
\usefonttheme{professionalfonts}
\usefonttheme{serif}
\usepackage{fontspec}

\usepackage{packages-presentation}


\pgfplotsset{compat=newest}

\usetikzlibrary{external}
\usetikzlibrary{arrows}
\usetikzlibrary{pgfplots.colorbrewer}
\usepgfplotslibrary{groupplots}
\usepgfplotslibrary{colorbrewer}
\usepgfplotslibrary{polar}

\tikzexternalize[prefix=figures/, mode=list and make]

\addbibresource{references.bib}

\author{Øyvind Sigmundson Schøyen}
\title{Real-time quantum many-body dynamics}
\institute{University of Oslo}
\date{\today}

\begin{document}

\begin{frame}
    \titlepage
\end{frame}
\begin{frame}
    \frametitle{Overview}
    \tableofcontents
\end{frame}

\section{Topic}

\begin{frame}
    \frametitle{Topic}
    Real-time simulations of ``large'' quantum mechanical systems.
\end{frame}

\begin{frame}
    \frametitle{Quantum mechanical system}
    Described by the wave function $\Psi(\vfg{x}, t)$ and the Hamiltonian
    $\hamil(\vfg{x}, t)$.
    The wave function is a solution to the time-dependent Schrödinger equation
    \begin{align}
        i\hslash \partial_t\Psi(\vfg{x}, t)
        = \hamil(\vfg{x}, t)\Psi(\vfg{x}, t),
    \end{align}
    with the initial condition $\Psi(\vfg{x})$ chosen as the ground state found
    from the time-independent Schrödinger equation
    \begin{align}
        \hamil(\vfg{x})\Psi(\vfg{x}) = \energy\Psi(\vfg{x}).
    \end{align}
\end{frame}

\begin{frame}
    \frametitle{Real-time simulations}
    \begin{enumerate}
        \item Model of a physical system executing at the same rate as ``wall
            clock'' time.
            Not realistic for us.
        \item Resembling an experiment.
    \end{enumerate}
\end{frame}

\begin{frame}
    \frametitle{Large systems}
    In our case, anything with more than $2$ particles.
    The wave function $\Psi(\vfg{x}, t)$ describes $N$ particles with $\vfg{x} =
    (x_1, \dots x_N)$ a set of $N$ coordinates with spin and position.
    We denote
    \begin{align}
        x_i \equiv (\vfg{r}_i, m_s),
    \end{align}
    where $\vfg{r}_i$ is the spatial position $i$ and $m_s$ the spin quantum
    number.
\end{frame}

\begin{frame}
    \frametitle{The challenge}
    \begin{enumerate}
        \item How do we represent a quantum mechanical system?
        \item How do we solve the time-independent Schrödinger equation?
        \item How do we solve the time-dependent Schrödinger equation?
    \end{enumerate}
\end{frame}

\section{Implementation}

\begin{frame}
    \frametitle{Quantum systems}
    Described by the molecular Hamiltonian in the Born-Oppenheimer approximation
    using a dipole laser field, viz.
    % TODO: Cite Hamiltonian, Born-Oppenheimer, and dipole laser.
    \begin{align}
        \hamil(\vfg{x}, t)
        = \hat{t}(\vfg{x})
        + \hat{v}(\vfg{x})
        + \hat{f}(\vfg{x}, t)
        + \hat{u}(\vfg{x})
        + \hat{h}_0.
    \end{align}
    We use atomic units with $\hslash = m_e = e = (4\pi\epsilon_0)^{-1} = 1$.
    \begin{itemize}
        \item Kinetic energy term
            \begin{align}
                \hat{t}(\vfg{x})
                = \sum_{i}-\half\nabla^{2}_i.
            \end{align}
        \item Nuclear-nuclear repulsion energy term
            \begin{align}
                \hat{h}_0
                = \sum_{i < j}\frac{Z_i Z_j}{\abs{\vfg{R}_I - \vfg{R}_J}}.
            \end{align}
    \end{itemize}
\end{frame}

\begin{frame}
    \begin{itemize}
        \item Nuclear potential energy term
            \begin{align}
                \hat{v}(\vfg{x})
                = - \sum_{i, j}\frac{Z_i}{\abs{\vfg{R}_i - \positionvec_i}}.
            \end{align}
        \item Dipole Laser field term
            \begin{align}
                \hat{f}(\vfg{x}, t)
                = \sum_{i} \hat{\vfg{d}}(\vfg{r}_i)\cdot\vfg{E}(t).
            \end{align}
        \item Coulomb interaction
            \begin{align}
                \hat{u}(\vfg{x})
                = \frac{1}{\abs{\positionvec_i - \positionvec_j}}.
            \end{align}
    \end{itemize}
\end{frame}

\begin{frame}
    \frametitle{Slater determinants}
    The ground state solution to the non-interacting, time-independent system
    for fermions is given by the reference Slater determinant
    \begin{align}
        \slat(\vfg{x})
        = \frac{1}{\sqrt{N}}
        \begin{vmatrix}
            \phi_1(x_1) & \dots & \phi_N(x_1) \\
            \vdots & \ddots & \vdots \\
            \phi_1(x_N) & \dots & \phi_N(x_N)
        \end{vmatrix}.
    \end{align}
    Here $\brac{\phi_i}$ will be a basis of $L$ single-particle states.
\end{frame}

\begin{frame}
    \frametitle{Second quantization}
    Represented in a basis of $L$ single-particle states $\brac{\chi_{\alpha}}$.
    \begin{itemize}
        \item Overlap elements
            \begin{align}
                \overlapten^{\alpha}_{\beta}
                \equiv \braket*{\chi_{\alpha}}{\chi_{\beta}}.
            \end{align}
        \item One-body Hamiltonian elements
            \begin{align}
                \oneten^{\alpha}_{\beta}
                \equiv \mel*{\chi_{\alpha}}{\onehamil}{\chi_{\beta}}
                = \mel*{\chi_{\alpha}}{\hat{t}}{\chi_{\beta}}
                + \mel*{\chi_{\alpha}}{\hat{v}}{\chi_{\beta}}.
            \end{align}
        \item Two-body Coulomb interaction elements
            \begin{align}
                \twotensym^{\alpha\beta}_{\gamma\delta}
                \equiv
                \mel*{\chi_{\alpha}\chi_{\beta}}{
                    \twohamil
                }{\chi_{\gamma}\chi_{\delta}}.
            \end{align}
        \item Dipole moment elements
            \begin{align}
                \vfg{d}^{\alpha}_{\beta}
                \equiv
                \mel*{\chi_{\alpha}}{\hat{\vfg{d}}}{\chi_{\beta}}.
            \end{align}
    \end{itemize}
\end{frame}

\begin{frame}
    \frametitle{Methods}
    \begin{enumerate}
        \item Hartree-Fock wave function \footcite{szabo1996modern}
            \begin{align}
                \ket*{\Psi} \equiv \ket*{\slat}.
            \end{align}
        \item Configuration interaction wave function
            \footcite{helgaker-molecular}
            \begin{align}
                \ket*{\Psi} \equiv \hat{C}_{I}\ket*{\slat}.
            \end{align}
        \item Coupled-cluster wave function \footcite{coester1958421}
            \begin{align}
                \ket*{\Psi} \equiv \exponential(\clust)\ket*{\slat}.
            \end{align}
    \end{enumerate}
\end{frame}

\begin{frame}
    \frametitle{Hartree-Fock}
    % TODO: Discuss that $C$ only describes a single Slater determinant!!!
    Ground state solution
    \begin{gather}
        \fock\ket*{\phi_i} = \epsilon_i\ket*{\phi_i}
        \implies
        \fockmat\vfg{C}
        = \overlapmat\vfg{C}\vfg{\epsilon}.
    \end{gather}
    Time-evolution
    \begin{gather}
        i\hslash\partial_t\ket*{\phi_i(t)}
        = \fock(t)\ket*{\phi_i(t)}
        \implies
        \dot{\vfg{C}} = -i\vfg{F}(t)\vfg{C}(t).
    \end{gather}
\end{frame}

\begin{frame}
    \frametitle{Configuration interaction}
    % TODO: Explain the difference between the coefficient matrices in HF and
    % CI.
    Ground state solution (more precisely, spectrum)
    \begin{align}
        \hamilmat\vfg{C} = \vfg{\energy}\vfg{C}.
    \end{align}
    Time-evolution
    \begin{align}
        \dot{\vfg{c}}(t) = -\hamilmat(t)\vfg{c}(t).
    \end{align}
\end{frame}

\begin{frame}
    \frametitle{Coupled-cluster}
\end{frame}

\begin{frame}
    \frametitle{Orbital-adaptive coupled-cluster}
\end{frame}

\section{Results and discussion}

\begin{frame}
    \frametitle{Results}
\end{frame}

\section{Summary remarks}

\begin{frame}
    \frametitle{Summary remarks}
\end{frame}

\begin{frame}
    Hello \footcite{pedersen2018symplectic}
\end{frame}

\end{document}
