\documentclass{beamer}
\usefonttheme{professionalfonts}
\usefonttheme{serif}
\usepackage{fontspec}

\usepackage{packages-presentation}


\pgfplotsset{compat=newest}

\usetikzlibrary{external}
\usetikzlibrary{arrows}
\usetikzlibrary{pgfplots.colorbrewer}
\usepgfplotslibrary{groupplots}
\usepgfplotslibrary{colorbrewer}
\usepgfplotslibrary{polar}

\tikzexternalize[prefix=figures/, mode=list and make]

\addbibresource{references.bib}

\author{Øyvind Sigmundson Schøyen}
\title{Real-time quantum many-body dynamics}
\institute{University of Oslo}
\date{\today}

\begin{document}

\begin{frame}
    \titlepage
\end{frame}
\begin{frame}
    \frametitle{Overview}
    \tableofcontents
\end{frame}

\section{Topic}

\begin{frame}
    \frametitle{Topic}
    Real-time simulations of ``large'' quantum mechanical systems.
\end{frame}

\begin{frame}
    \frametitle{Quantum mechanical system}
    Described by the wave function $\Psi(\vfg{x}, t)$ and the Hamiltonian
    $\hamil(\vfg{x}, t)$.
    Satisfies the time-dependent Schrödinger equation
    \begin{align}
        i\hslash \partial_t\Psi(\vfg{x}, t)
        = \hamil(\vfg{x}, t)\Psi(\vfg{x}, t).
    \end{align}
\end{frame}

\begin{frame}
    \frametitle{Real-time simulations}
    \begin{enumerate}
        \item Model of a physical system executing at the same rate as ``wall
            clock'' time.
            Not realistic for us.
        \item Resembling an experiment.
    \end{enumerate}
\end{frame}

\begin{frame}
    \frametitle{Large systems}
    In our case, anything with more than $2$ particles.
    Wave function $\Psi(\vfg{x}, t)$ describes $N$ particles.
\end{frame}

\begin{frame}
    \frametitle{The challenge}
    \begin{enumerate}
        \item How do we represent a quantum mechanical system?
        \item How do we solve the Schrödinger equation?
    \end{enumerate}
\end{frame}

\section{Implementation}

\begin{frame}
    \frametitle{Quantum systems}
    Represented in a basis of $L$ single-particle states $\brac{\chi_{\alpha}}$.
    \begin{itemize}
        \item One-body Hamiltonian elements
            \begin{align}
                \oneten^{\alpha}_{\beta}
                \equiv \mel*{\chi_{\alpha}}{\onehamil}{\chi_{\beta}}.
                %= \int\dd x \chi^{*}_{\alpha}(x)\onehamil(x)\chi_{\beta}(x).
            \end{align}
        \item Two-body Coulomb interaction elements
            \begin{align}
                \twotensym^{\alpha\beta}_{\gamma\delta}
                \equiv
                \mel*{\chi_{\alpha}\chi_{\beta}}{
                    \twohamil
                }{\chi_{\gamma}\chi_{\delta}}.
            \end{align}
        \item Dipole moment elements
            \begin{align}
                \vfg{d}^{\alpha}_{\beta}
                \equiv
                \mel*{\chi_{\alpha}}{\hat{\vfg{d}}}{\chi_{\beta}}.
            \end{align}
    \end{itemize}
\end{frame}

\begin{frame}
    \frametitle{Methods}
    \begin{enumerate}
        \item Hartree-Fock wave function
            \begin{align}
                \ket*{\Psi} \equiv \ket*{\slat}.
            \end{align}
        \item Configuration interaction wave function
            \begin{align}
                \ket*{\Psi} \equiv \hat{C}_{I}\ket*{\slat}.
            \end{align}
        \item Coupled-cluster wave function
            \begin{align}
                \ket*{\Psi} \equiv \exponential(\clust)\ket*{\slat}.
            \end{align}
    \end{enumerate}
\end{frame}

\begin{frame}
\end{frame}

\section{Results and discussion}

\begin{frame}
    \frametitle{Results}
\end{frame}

\section{Summary remarks}

\begin{frame}
    \frametitle{Summary remarks}
\end{frame}

\begin{frame}
    Hello \footcite{pedersen2018symplectic}
\end{frame}

\end{document}
