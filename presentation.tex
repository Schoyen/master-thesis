\documentclass{beamer}
\setbeamertemplate{navigation symbols}{}
\usetheme{boxes}
\usefonttheme{professionalfonts}
\usefonttheme{serif}
\usepackage{fontspec}

\usepackage{packages-presentation}


\pgfplotsset{compat=newest}

\usetikzlibrary{external}
\usetikzlibrary{arrows}
\usetikzlibrary{pgfplots.colorbrewer}
\usepgfplotslibrary{groupplots}
\usepgfplotslibrary{colorbrewer}
\usepgfplotslibrary{polar}

\tikzexternalize[prefix=figures/, mode=list and make]

\addbibresource{references.bib}

\author{Øyvind Sigmundson Schøyen}
\title{Real-time quantum many-body dynamics}
\institute{University of Oslo}
\date{\today}

\begin{document}

\begin{frame}
    \titlepage
\end{frame}

\begin{frame}
    \frametitle{Overview}
    \tableofcontents
\end{frame}

\section{Topic}

\begin{frame}
    \frametitle{Topic}
    Real-time simulations of large quantum mechanical systems.
\end{frame}

\begin{frame}
    \frametitle{Quantum mechanical system}
    \begin{itemize}
        \item Described by the wave function $\Psi(\vfg{x}, t)$ and the Hamiltonian
            $\hamil(\vfg{x}, t)$.
        \item The wave function is a solution to the time-dependent Schrödinger
            equation
            \begin{align}
                i\hslash \partial_t\Psi(\vfg{x}, t)
                = \hamil(\vfg{x}, t)\Psi(\vfg{x}, t).
            \end{align}
        \item Initial condition $\Psi(\vfg{x})$ chosen as the ground state found
            from the time-independent Schrödinger equation
            \begin{align}
                \hamil(\vfg{x})\Psi(\vfg{x}) = \energy\Psi(\vfg{x}).
            \end{align}
    \end{itemize}
\end{frame}

\begin{frame}
    \frametitle{Real-time simulations}
    \begin{itemize}
        \item Model of a physical system executing at the same rate as ``wall
            clock'' time.
        \item Resembling an experiment.
    \end{itemize}
\end{frame}

\begin{frame}
    \frametitle{Large systems}
    In our case, anything with more than $2$ particles.
\end{frame}

%\begin{frame}
%    \frametitle{The challenge}
%    \begin{enumerate}
%        \item How do we represent a quantum mechanical system?
%        \item How do we solve the time-independent Schrödinger equation?
%        \item How do we solve the time-dependent Schrödinger equation?
%    \end{enumerate}
%\end{frame}

%\section{Many-body intermezzo}
%
%\begin{frame}
%    \frametitle{Many-body intermezzo}
%    \begin{itemize}
%        \item Single-particle states $\brac{\phi_i}$.
%            These states are solutions to the one-body Hamiltonian.
%        \item Many-body states $\brac{\Psi_K}$.
%            Solutions to the many-body Hamiltonian including interactions.
%    \end{itemize}
%    Many-body states are built from single-particle states.
%\end{frame}
%
%
%\begin{frame}
%    \frametitle{Slater determinants}
%    The ground state solution to the non-interacting, time-independent system
%    for fermions is given by the reference Slater determinant
%    \begin{align}
%        \slat(\vfg{x})
%        = \frac{1}{\sqrt{N}}
%        \begin{vmatrix}
%            \phi_1(x_1) & \dots & \phi_N(x_1) \\
%            \vdots & \ddots & \vdots \\
%            \phi_1(x_N) & \dots & \phi_N(x_N)
%        \end{vmatrix}.
%    \end{align}
%    Here $\brac{\phi_i}$ will be a basis of $L$ single-particle states.
%\end{frame}
%
%\begin{frame}
%    \frametitle{Second quantization}
%    Represented in a basis of $L$ single-particle states $\brac{\chi_{\alpha}}$.
%    \begin{itemize}
%        \item Overlap elements
%            \begin{align}
%                \overlapten^{\alpha}_{\beta}
%                \equiv \braket*{\chi_{\alpha}}{\chi_{\beta}}.
%            \end{align}
%        \item One-body Hamiltonian elements
%            \begin{align}
%                \oneten^{\alpha}_{\beta}
%                \equiv \mel*{\chi_{\alpha}}{\onehamil}{\chi_{\beta}}
%                = \mel*{\chi_{\alpha}}{\hat{t}}{\chi_{\beta}}
%                + \mel*{\chi_{\alpha}}{\hat{v}}{\chi_{\beta}}.
%            \end{align}
%        \item Two-body Coulomb interaction elements
%            \begin{align}
%                \twotensym^{\alpha\beta}_{\gamma\delta}
%                \equiv
%                \mel*{\chi_{\alpha}\chi_{\beta}}{
%                    \twohamil
%                }{\chi_{\gamma}\chi_{\delta}}.
%            \end{align}
%        \item Dipole moment elements
%            \begin{align}
%                \vfg{d}^{\alpha}_{\beta}
%                \equiv
%                \mel*{\chi_{\alpha}}{\hat{\vfg{d}}}{\chi_{\beta}}.
%            \end{align}
%    \end{itemize}
%\end{frame}

\section{Implementation}

\begin{frame}[plain, c]
    \begin{center}
        \usebeamerfont*{frametitle}
        \usebeamercolor[fg]{frametitle}
        \Huge Implementation
    \end{center}
\end{frame}

\begin{frame}
    \frametitle{Atoms and molecules}
    \begin{itemize}
        \item Molecular Hamiltonian.
        \item Born-Oppenheimer approximation.
        \item Coulomb interaction.
        \item Semi-classical dipole laser field.
    \end{itemize}
    Matrix elements from PySCF.\footcite{pyscf}
\end{frame}

%\begin{frame}
%    \frametitle{Quantum systems}
%    \begin{itemize}
%        \item One-dimensional quantum dots.
%        \item Atoms and molecules.
%    \end{itemize}
%\end{frame}

%\begin{frame}
%    \frametitle{One-dimensional quantum dots}
%\end{frame}

%\begin{frame}
%    \frametitle{Atoms and molecules}
%    Described by the molecular Hamiltonian in the Born-Oppenheimer approximation
%    using a dipole laser field, viz.
%    % TODO: Cite Hamiltonian, Born-Oppenheimer, and dipole laser.
%    We use PySCF\footcite{pyscf} to get matrix elements.
%    \begin{align}
%        \hamil(\vfg{x}, t)
%        = \hat{t}(\vfg{x})
%        + \hat{v}(\vfg{x})
%        + \hat{f}(\vfg{x}, t)
%        + \hat{u}(\vfg{x})
%        + \hat{h}_0.
%    \end{align}
%    We use atomic units with $\hslash = m_e = e = (4\pi\epsilon_0)^{-1} = 1$.
%    \begin{itemize}
%        \item Kinetic energy term
%            \begin{align}
%                \hat{t}(\vfg{x})
%                = \sum_{i}-\half\nabla^{2}_i.
%            \end{align}
%        \item Nuclear-nuclear repulsion energy term
%            \begin{align}
%                \hat{h}_0
%                = \sum_{i < j}\frac{Z_i Z_j}{\abs{\vfg{R}_I - \vfg{R}_J}}.
%            \end{align}
%    \end{itemize}
%\end{frame}
%
%\begin{frame}
%    \begin{itemize}
%        \item Nuclear potential energy term
%            \begin{align}
%                \hat{v}(\vfg{x})
%                = - \sum_{i, j}\frac{Z_i}{\abs{\vfg{R}_i - \positionvec_i}}.
%            \end{align}
%        \item Dipole Laser field term
%            \begin{align}
%                \hat{f}(\vfg{x}, t)
%                = \sum_{i} \hat{\vfg{d}}(\vfg{r}_i)\cdot\vfg{E}(t).
%            \end{align}
%        \item Coulomb interaction
%            \begin{align}
%                \hat{u}(\vfg{x})
%                = \frac{1}{\abs{\positionvec_i - \positionvec_j}}.
%            \end{align}
%    \end{itemize}
%\end{frame}

\begin{frame}
    \frametitle{Methods}
    \begin{itemize}
        \item Hartree-Fock theory.
        \item Configuration interaction.
        \item Coupled-cluster theory.
    \end{itemize}
\end{frame}

\begin{frame}
    \frametitle{Hartree-Fock theory}
    Ansatz:
    \begin{align}
        \ket*{\Psi} \equiv \ket*{\slat}
        = \ket*{\phi_1 \dots \phi_N},
        \quad
        \braket*{\phi_i}{\phi_j} = \delta_{ij}.
    \end{align}
    Molecular orbitals:
    \begin{align}
        \ket*{\phi_i} = C_{\alpha i}\ket*{\chi_{\alpha}}.
    \end{align}
    \begin{itemize}
        \item Mean-field approximation.
        \item Optimal Slater determinant for single-reference problems.
        \item Time-evolution of spin-orbitals.
        \item Computationally inexpensive method.
        \item No correlations.
    \end{itemize}
    Excellent reference state for \emph{post Hartree-Fock} methods.
\end{frame}

%\begin{frame}
%    \frametitle{Hartree-Fock}
%    Ansatz:
%    \begin{align}
%        \ket*{\Psi} \equiv \ket*{\slat}
%        = \ket*{\phi_1 \dots \phi_N},
%        \quad
%        \braket*{\phi_i}{\phi_j} = \delta_{ij}.
%    \end{align}
%    Molecular orbitals:
%    \begin{align}
%        \ket*{\phi_i} = C_{\alpha i}\ket*{\chi_{\alpha}}.
%    \end{align}
%    Ground state solution:
%    \begin{gather}
%        \fockmat\vfg{C} = \overlapmat\vfg{C}\vfg{\epsilon}.
%    \end{gather}
%    Important basis for all \emph{post Hartree-Fock methods}.
%\end{frame}

%\begin{frame}
%    Hartree-Fock Lagrangian:
%    \begin{align}
%        L[\slat, \slat^{*}, \vfg{\lambda}]
%        = \mel*{\slat}{\hamil}{\slat}
%        - \lambda_{ji}(\braket*{\phi_i}{\phi_j} - \delta_{ij})
%    \end{align}
%    Ground state solution:
%    \begin{gather}
%        \fock\ket*{\phi_i}
%        = (\onehamil + \hat{J} - \hat{K})\ket*{\phi_i}
%        = \epsilon_i\ket*{\phi_i}.
%    \end{gather}
%    In a basis of atomic orbitals:
%    \begin{gather}
%        \fockmat\vfg{C} = \overlapmat\vfg{C}\vfg{\epsilon}.
%    \end{gather}
%\end{frame}

%\begin{frame}
%    Time-dependent Lagrangian:
%    \begin{align}
%        L[\slat, \slat^{*}, \vfg{\lambda}]
%        &=
%        \mel*{\slat(t)}{
%            (i\hslash\partial_t - \hamil(t))
%        }{\slat(t)}
%        \nonumber \\
%        &\qquad
%        - \lambda_{ji}
%        (\braket*{\phi_i(t)}{\phi_j(t)} - \delta_{ij})
%    \end{align}
%    Time-dependent variational principle:
%    \begin{align}
%        S[\slat, \slat^{*}]
%        = \int \dd t
%        L[\slat, \slat^{*}, \vfg{\lambda}]
%    \end{align}
%    The stationary conditions yield the equations of motion:
%    \begin{gather}
%        i\hslash\partial_t\ket*{\phi_i(t)}
%        = \fock(t)\ket*{\phi_i(t)}.
%    \end{gather}
%    In a basis of atomic orbitals:
%    \begin{gather}
%        \dot{\vfg{C}} = -i\vfg{F}(t)\vfg{C}(t).
%    \end{gather}
%\end{frame}


\begin{frame}
    \frametitle{Configuration interaction}
    Ansatz:
    \begin{align}
        \ket*{\Psi} \equiv C_0\ket*{\slat}
        + C^{a}_{i}\ket*{\slat^{a}_{i}}
        + \frac{1}{4}C^{ab}_{ij}\ket*{\slat^{ab}_{ij}}
        + \dots.
    \end{align}
    \begin{itemize}
        \item The ``natural'' approach.
        \item Exact within the given single-particle basis.
        \item Includes the full spectrum of the Hamiltonian.
        \item Time-evolution of determinant coefficients.
        \item Exponential computational complexity.
        \item Truncated configuration interaction is not size-consistent.
    \end{itemize}
\end{frame}

%\begin{frame}
%    \frametitle{Configuration interaction}
%    Ansatz:\footcite{helgaker-molecular}
%    \begin{align}
%        \ket*{\Psi} \equiv C_0\ket*{\slat}
%        + C^{a}_{i}\ket*{\slat^{a}_{i}}
%        + \frac{1}{4}C^{ab}_{ij}\ket*{\slat^{ab}_{ij}}
%        + \dots.
%    \end{align}
%    Exact within the truncated single-particle space.
%    % TODO: Explain the difference between the coefficient matrices in HF and
%    % CI.
%    % TODO: Explain that CI is ``exact''.
%\end{frame}
%
%\begin{frame}
%    Ground state solution:
%    \begin{align}
%        \hamilmat\vfg{C} = \vfg{\energy}\vfg{C},
%    \end{align}
%    where we assume orthonormal Slater determinants.
%    Here $\vfg{C}$ includes \emph{all many-body states}.
%\end{frame}
%
%\begin{frame}
%    Time-evolution of a single many-body state:
%    \begin{align}
%        \dot{\vfg{c}}(t) = -\hamilmat(t)\vfg{c}(t).
%    \end{align}
%    % TODO: Comment on the time-dependence being in the coefficients.
%    % TODO: Note (in the Stability section) that CI allows zero overlap with the
%    % reference state.
%\end{frame}
%
%\begin{frame}
%    Exponential complexity.
%
%    Truncated configuration interaction is not \emph{size consistent}.
%\end{frame}


\begin{frame}
    \frametitle{Coupled-cluster theory}
    Ansatz:
    \begin{gather}
        \ket*{\Psi} \equiv \exponential(\clust)\ket*{\slat},
        \quad
        \bra*{\tilde{\Psi}} \equiv
        \bra*{\slat}(\1 + \clustl)\exponential(\clust).
    \end{gather}
    Cluster operators:
    \begin{gather}
        \clust \equiv \clustamp_0
        + \clustamp^{a}_{i}\ccr{a}\can{i}
        + \frac{1}{4}\clustamp^{ab}_{ij}\ccr{a}\ccr{b}\can{j}\can{i}
        + \dots
        = \clustamp_{\mu}\hat{X}_{\mu},
        \\
        \clustl \equiv
        \clustlamp^{i}_{a}\ccr{i}\can{a}
        + \frac{1}{4}\clustlamp^{ij}_{ab}\ccr{i}\ccr{j}\can{b}\can{a}
        + \dots
        = \clustlamp_{\mu}\hat{X}^{\dagger}_{\mu}.
    \end{gather}
    \begin{itemize}
        \item Size-consistent.
        \item Correlations beyond Hartree-Fock.
        \item Polynomial scaling.
        \item Described by two sets of amplitudes $\vfg{\clustamp}$ and
            $\vfg{\clustlamp}$.
        \item Time-evolution of amplitudes.
    \end{itemize}
\end{frame}

\begin{frame}
    \frametitle{Orbital-adaptive coupled-cluster theory}
    Opens up for orbital rotations.
    \begin{gather}
        \ket*{\phi_p} = C_{\alpha p}\ket*{\chi_{\alpha}}, \\
        \bra*{\tilde{\phi}_{p}} = \tilde{C}_{p \alpha} \bra*{\chi_{\alpha}}.
    \end{gather}
    \begin{itemize}
        \item More flexible description than ``regular'' coupled-cluster theory
            with static orbitals.
        \item Same scaling.
        \item Time-evolution of both amplitudes and coefficients.
    \end{itemize}
\end{frame}

%
%\begin{frame}
%    \frametitle{Coupled-cluster theory}
%    Lagrangian ansatz:\footcite{helgaker1989}
%    \begin{gather}
%        \ket*{\Psi} \equiv \exponential(\clust)\ket*{\slat},
%        \\
%        \bra*{\tilde{\Psi}} \equiv
%        \bra*{\slat}(\1 + \clustl)\exponential(\clust).
%    \end{gather}
%    This is a \emph{bi-variational}\footcite{arponen1983311} formulation.
%    Cluster operators:
%    \begin{gather}
%        \clust \equiv \clustamp_0
%        + \clustamp^{a}_{i}\ccr{a}\can{i}
%        + \frac{1}{4}\clustamp^{ab}_{ij}\ccr{a}\ccr{b}\can{j}\can{i}
%        + \dots
%        = \clustamp_{\mu}\hat{X}_{\mu},
%        \\
%        \clustl \equiv
%        \clustlamp^{i}_{a}\ccr{i}\can{a}
%        + \frac{1}{4}\clustlamp^{ij}_{ab}\ccr{i}\ccr{j}\can{b}\can{a}
%        + \dots
%        = \clustlamp_{\mu}\hat{X}^{\dagger}_{\mu}.
%    \end{gather}
%\end{frame}
%
%\begin{frame}
%    Coupled-cluster Lagrangian:
%    \begin{align}
%        L(\vfg{\clustamp}, \vfg{\clustlamp})
%        = \mel*{\tilde{\Psi}}{\hamil}{\Psi}.
%    \end{align}
%    Ground state solutions:
%    \begin{gather}
%        \dpd{L}{\clustlamp_{\mu}}
%        = \mel*{\slat_{\mu}}{
%            \exponential(-\clust)
%            \hamil
%            \exponential(\clust)
%        }{\slat}
%        = 0, \\
%        \dpd{L}{\clustamp_{\mu}}
%        = \mel*{\slat}{
%            (\1 + \clustl)
%            \exponential(-\clust)
%            \com{\hamil}{\hat{X}_{\mu}}
%            \exponential(\clust)
%        }{\slat}
%        = 0.
%    \end{gather}
%\end{frame}
%
%\begin{frame}
%    Time-dependent bi-variational principle:
%    \begin{align}
%        S[\Psi, \tilde{\Psi}]
%        =
%        \int \dd t
%        \mel*{\tilde{\Psi}(t)}{
%            (i\hslash\partial_t - \hamil(t))
%        }{\Psi(t)}.
%    \end{align}
%    Time dependence in amplitudes and Hamiltonian.
%    Equations of motion:
%    \begin{gather}
%        i\hslash\partial_t\clustamp_{\mu}
%        = \mel*{\slat_{\mu}}{
%            \exponential(-\clust)
%            \hamil
%            \exponential(\clust)
%        }{\slat},
%        \\
%        -i\hslash\partial_t\clustlamp_{\mu}
%        = \mel*{\slat}{
%            (\1 + \clustl)
%            \exponential(-\clust)
%            \com{\hamil}{\hat{X}_{\mu}}
%            \exponential(\clust)
%        }{\slat}.
%    \end{gather}
%\end{frame}
%
%\begin{frame}
%    \frametitle{Orbital-adaptive coupled-cluster}
%\end{frame}

\section{Results and discussion}

\begin{frame}[plain, c]
    \begin{center}
        \usebeamerfont*{frametitle}
        \usebeamercolor[fg]{frametitle}
        \Huge Results and discussion
    \end{center}
\end{frame}

\begin{frame}
    \frametitle{Validation}
    \begin{itemize}
        \item Two one-dimensional harmonic oscillators explored by
            \citeauthor{zanghellini_2004}.\footcite{zanghellini_2004}
        \item Hydrogen molecule explored by
            \citeauthor{li_2005}.\footcite{li_2005}
    \end{itemize}
\end{frame}

\begin{frame}
    \begin{figure}
        \centering
        \begin{tikzpicture}
            \begin{groupplot}[
                    group style={
                        group size=1 by 2,
                        vertical sep=30pt,
                        xlabels at=edge bottom,
                    },
                    width=10cm,
                    height=4cm,
                    xlabel={$x$ $[\si{\bohr}]$},
                ]
                \nextgroupplot[
                        grid=major,
                        ylabel={$\densityten(x, 0)$},
                        restrict x to domain=-6:6,
                        enlarge x limits=false,
                        ymin=0,
                        ymax=0.4,
                        enlarge y limits=false,
                        title={Ground state particle density},
                    ]
                    \addplot+ [
                        mark=none,
                        thick,
                    ]
                    table
                    {results/benchmarks/zanghellini/dat/rho_tdcisd_real.dat};
                    \addlegendentry{TDCISD}

                    \addplot+ [
                        mark=none,
                        thick,
                        dashed,
                    ]
                    table
                    {results/benchmarks/zanghellini/dat/rho_tdhf_real.dat};
                    \addlegendentry{TDHF}
                \nextgroupplot[
                        ymode=log,
                        grid=major,
                        ylabel={$\abs{\Delta\densityten(x, 0)}$},
                        restrict x to domain=-6:6,
                        enlarge x limits=false,
                    ]
                    \addplot+ [
                        mark=none,
                        thick,
                    ]
                    table
                    {results/benchmarks/zanghellini/dat/rho_diff_tdccsd_tdcisd.dat};
            \end{groupplot}
        \end{tikzpicture}
        \label{fig:one-body-density-zanghellini}
    \end{figure}
\end{frame}

\begin{frame}
    \begin{figure}
        \centering
        \begin{tikzpicture}
            \begin{groupplot}[
                    group style={
                        group size=1 by 2,
                        vertical sep=30pt,
                        xlabels at=edge bottom,
                    },
                    width=10cm,
                    height=4cm,
                    xlabel={$\Omega t / (2\pi)$},
                ]
                \nextgroupplot[
                        grid=major,
                        ylabel={$\abs{\braket*{\Psi(t)}{\Psi(0)}}^2$},
                        enlarge x limits=false,
                        title={Time-dependent overlap},
                    ]
                    \addplot+ [
                        mark=none,
                        thick,
                    ]
                    table
                    {results/benchmarks/zanghellini/dat/overlap_tdcisd_real.dat};
                    \addlegendentry{TDCISD}

                    \addplot+ [
                        mark=none,
                        thick,
                        dashed,
                    ]
                    table
                    {results/benchmarks/zanghellini/dat/overlap_tdhf_real.dat};
                    \addlegendentry{TDHF}
                \nextgroupplot[
                        ymode=log,
                        grid=major,
                        ylabel={$\abs{\Delta\braket*{\Psi(t)}{\Psi(0)}}^2$},
                        restrict x to domain=-6:6,
                        enlarge x limits=false,
                    ]
                    \addplot+ [
                        mark=none,
                        thick,
                    ]
                    table
                    {results/benchmarks/zanghellini/dat/overlap_diff_tdccsd_tdcisd.dat};
            \end{groupplot}
        \end{tikzpicture}
        \label{fig:overlap-zanghellini}
    \end{figure}
\end{frame}

\begin{frame}
    \frametitle{Stability}
    \citeauthor{pedersen2018symplectic}
    \footcite{pedersen2018symplectic} reports how TDCCSD breaks down under
    extreme conditions.
    In an ongoing article \footcite{oa-stability} we demonstrate that OATDCCD is
    more stable during these simulations.
\end{frame}

\begin{frame}
    \begin{figure}
        \centering
        \begin{tikzpicture}
            \begin{groupplot}[
                group style={
                    group size=1 by 2,
                    vertical sep=30pt,
                    xlabels at=edge bottom,
                },
                width=10cm,
                height=4cm,
                xlabel={$t$ $[\si{\hslash/\hartree}]$},
            ]
                \nextgroupplot[
                    grid=major,
                    ylabel={$\abs{\braket*{\slat}{\Psi(t)}}^2$},
                    enlarge x limits=false,
                    title={Helium-simulation},
                ]
                    \addplot+[
                        mark=none,
                        thick,
                    ]
                    table
                    {results/stability/stability-runs/dat/he_tdfci_phase_real.dat};
                    \addlegendentry{TDFCI}

                    \addplot+[
                        mark=none,
                        thick,
                    ]
                    table
                    {results/stability/stability-runs/dat/he_oatdccd_phase_real.dat};
                    \addlegendentry{OATDCCD}

                    \addplot+[
                        mark=none,
                        ultra thick,
                        dashed,
                    ]
                    table
                    {results/stability/stability-runs/dat/he_tdccsd_phase_real.dat};
                    \addlegendentry{TDCCSD}

                \nextgroupplot[
                    grid=major,
                    ylabel={Magnitude},
                    enlarge x limits=false,
                ]

                    \addplot+[
                        mark=none,
                        thick,
                    ]
                    table
                    {results/stability/stability-runs/dat/he_oatdccd_norm_t2_real.dat};

                    \addplot+[
                        mark=none,
                        thick,
                    ]
                    table
                    {results/stability/stability-runs/dat/he_oatdccd_norm_l2_real.dat};

                    \addplot+[
                        mark=none,
                        thick,
                    ]
                    table
                    {results/stability/stability-runs/dat/he_tdccsd_norm_t1_real.dat};

                    \addplot+[
                        mark=none,
                        thick,
                    ]
                    table
                    {results/stability/stability-runs/dat/he_tdccsd_norm_t2_real.dat};

                    \addplot+[
                        mark=none,
                        thick,
                    ]
                    table
                    {results/stability/stability-runs/dat/he_tdccsd_norm_l1_real.dat};

                    \addplot+[
                        mark=none,
                        thick,
                    ]
                    table
                    {results/stability/stability-runs/dat/he_tdccsd_norm_l2_real.dat};

            \end{groupplot}
        \end{tikzpicture}
        \label{fig:he-stability}
    \end{figure}

\end{frame}

\begin{frame}
    \begin{figure}
        \centering
        \begin{tikzpicture}
            \begin{groupplot}[
                group style={
                    group size=1 by 2,
                    vertical sep=30pt,
                    xlabels at=edge bottom,
                },
                width=10cm,
                height=4cm,
                xlabel={$t$ $[\si{\hslash/\hartree}]$},
            ]
                \nextgroupplot[
                    grid=major,
                    ylabel={$\abs{\braket*{\slat}{\Psi(t)}}^2$},
                    enlarge x limits=false,
                    title={Beryllium-simulation},
                ]
                    \addplot+[
                        mark=none,
                        thick,
                    ]
                    table
                    {results/stability/stability-runs/dat/be_tdfci_phase_real.dat};
                    \addlegendentry{TDFCI}

                    \addplot+[
                        mark=none,
                        thick,
                    ]
                    table
                    {results/stability/stability-runs/dat/be_oatdccd_phase_real.dat};
                    \addlegendentry{OATDCCD}

                    \addplot+[
                        mark=none,
                        ultra thick,
                        dashed,
                    ]
                    table
                    {results/stability/stability-runs/dat/be_tdccsd_phase_real.dat};
                    \addlegendentry{TDCCSD}

                \nextgroupplot[
                    grid=major,
                    ylabel={Magnitude},
                    enlarge x limits=false,
                ]

                    \addplot+[
                        mark=none,
                        thick,
                    ]
                    table
                    {results/stability/stability-runs/dat/be_oatdccd_norm_t2_real.dat};

                    \addplot+[
                        mark=none,
                        thick,
                    ]
                    table
                    {results/stability/stability-runs/dat/be_oatdccd_norm_l2_real.dat};

                    \addplot+[
                        mark=none,
                        thick,
                    ]
                    table
                    {results/stability/stability-runs/dat/be_tdccsd_norm_t1_real.dat};

                    \addplot+[
                        mark=none,
                        thick,
                    ]
                    table
                    {results/stability/stability-runs/dat/be_tdccsd_norm_t2_real.dat};

                    \addplot+[
                        mark=none,
                        thick,
                    ]
                    table
                    {results/stability/stability-runs/dat/be_tdccsd_norm_l1_real.dat};

                    \addplot+[
                        mark=none,
                        thick,
                    ]
                    table
                    {results/stability/stability-runs/dat/be_tdccsd_norm_l2_real.dat};
            \end{groupplot}
        \end{tikzpicture}
        \label{fig:be-stability}
    \end{figure}
\end{frame}

\begin{frame}
    \frametitle{Applications}
    \begin{itemize}
        \item Study of \ch{LiH}-molecule by \citeauthor{nest}.\footcite{nest}
        \item Study of \ch{Ne} and \ch{Ar} atoms.
        \item Spin-dependent laser field by
            \citeauthor{isborn}.\footcite{isborn}
        \item Ionization of one-dimensional \ch{Be}-atom by
            \citeauthor{miyagi_and_madsen}.\footcite{miyagi_and_madsen}
    \end{itemize}
\end{frame}

\begin{frame}
    \begin{figure}
        \centering
        \begin{tikzpicture}
            \begin{axis}[
                width=10cm,
                height=6cm,
                title={Absorption spectra for Neon},
                xlabel={Excitation energy $[\si{\hartree}]$},
                grid=major,
                ylabel={Intensity},
                enlarge x limits=false,
                xmax=6,
            ]
                \addplot+[
                    mark=none,
                    thick,
                ]
                table
                {results/atoms-and-molecules/time-evolution-noble-gasses/dat/fft_ne_ccpvdz_oatdccd_new_real.dat};
                \addlegendentry{cc-pVDZ}
                \addplot+[
                    mark=none,
                    thick,
                ]
                table
                {results/atoms-and-molecules/time-evolution-noble-gasses/dat/fft_ne_aug-ccpvdz_oatdccd_new_real.dat};
                \addlegendentry{aug-cc-pVDZ}
            \end{axis}
        \end{tikzpicture}
        \label{fig:dipole-ne}
    \end{figure}
\end{frame}

\begin{frame}
    \begin{figure}
        \centering
        \begin{tikzpicture}
            \begin{axis}[
                width=10cm,
                height=6cm,
                title={Absorption spectra for Argon},
                xlabel={Excitation energy $[\si{\hartree}]$},
                grid=major,
                ylabel={Intensity},
                enlarge x limits=false,
                xmax=3,
            ]
                \addplot+[
                    mark=none,
                    thick,
                ]
                table
                {results/atoms-and-molecules/time-evolution-noble-gasses/dat/fft_ar_ccpvdz_oatdccd_new_real.dat};
                \addlegendentry{cc-pVDZ}
                \addplot+[
                    mark=none,
                    thick,
                ]
                table
                {results/atoms-and-molecules/time-evolution-noble-gasses/dat/fft_ar_aug-ccpvdz_oatdccd_new_real.dat};
                \addlegendentry{aug-cc-pVDZ}
            \end{axis}
        \end{tikzpicture}
        \label{fig:dipole-ar}
    \end{figure}
\end{frame}

\section{Summary remarks}

\begin{frame}
    \frametitle{Summary remarks}
    \begin{itemize}
        \item We have implemented a highly flexibly framework for quantum
            many-body dynamics.
        \item We have demonstrated how the orbital-adaptive formulation of
            the coupled-cluster method is more suitable for time-evolution than
            the static formulation.
    \end{itemize}
\end{frame}

\end{document}
