This thesis would not have come about were it not for my two excellent
supervisors: Håkon and Morten.
I would like to thank Håkon for allowing me to continue on his work, and for
including me in his research.
Your open-mindedness towards new suggestions on any topic is truly mind-boggling
and inspiring for someone as stubborn as myself.
I am indebted to your counsel and friendship throughout this work.
I would also like to thank Morten for providing me with such an interesting
topic, and for all the guidance you have given me throughout the work with this
thesis.
Your insight, fascination, and knowledge on all things related to many-body
physics and computations are awe-inspiring.
Not to mention your passion for teaching and the well-being of students at the
computational physics group.
I look forward to continuing to collaborate with both of you.

Working with Sebastian on this topic have made the last couple of years a truly
valuable experience.
I believe that we together have achieved so much more than what would have been
possible if we went our separate ways.
Your excellent humor and work ethic makes the long days of work immensely more
enjoyable.
So many hours have been spent at the old computational physics group and most
of them were with my long-time office partners Alocias, Magnus, and Vilde.
Thank you all for an exciting and memorable time.
A huge thanks to all the people at computational physics for making my time here
so pleasant.
I would also like to thank the guys and gals at OSI Elvepadling for bringing me
along on adventures to the rivers all over Norway.

To my brother, Vemund, for being a truly wonderful guy!
To my parents, Helle and Sigmund, for always being there and always rooting for
Vemund and me.
To Jenny, for your love, support, and patience throughout these last months.
Your presence always make the days a little brighter, and I feel truly lucky to
have you in my life.
I cannot wait to explore the future with you.



Finally, to the reader of this text, wherein there are no statements of topics
or derivations being ``easy'', ``simple'', or ``trivial''.
If a topic takes you several years of advanced physics study to understand, then
it is \emph{emphatically not} easy \cite{nontrivial-manifesto}.
