This thesis would not have come about where it not for my two excellent
supervisors: Håkon Emil Kristiansen and Morten Hjorth-Jensen.
I would like to thank Håkon for a genuine interest in my contributions to a
topic which he quite clearly finds fascinating.
Your open-mindedness towards new suggestions on any topic is truly inspiring!
Also, it is mind-boggling for someone as stubborn as myself!
I would also like to thank Morten for providing me with such an interesting topic,
and for always being available when it is needed!
Your insight, fascination, and knowledge on all things related to physics and
computations is awe-inspiring!

Working with Sebastian Gregorius Winther-Larsen on this topic has made the work
so much more fun!
I have truly enjoyed these last two years, both as partners in crime on our
theses, but perhaps mostly due to all the shenanigans!
Our trip to Texas and to the UK were a blast, and I hope that we get to continue
all our ramblings in the next four years as PhD candidates together.

So many hours have been spent at the old Computational Physics group, and most
of them were with my long-time office partners Alocias, Magnus, and Vilde!

Takk til Ministry
Takk til mor og far
Takk til Vemund
Takk til Jenny
Takk til OSI Elvepadling

This thesis marks the end of 21 years of schooling, and I am happy to say that
it has been a blast!
Even though my first day in school at the age of 6 ended with me asking how soon
I could quit.

To the reader of this text, wherein there are no statements of topics or
derivations being ``easy'', ``simple'', or ``trivial''.
If a topic takes you several years of advanced physics study to understand, then
it is \emph{emphatically not} easy \cite{nontrivial-manifesto}.
