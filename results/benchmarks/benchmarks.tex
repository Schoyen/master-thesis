\chapter{Benchmarking against litterature}
    In this chapter we use our developed code to reproduce results from relevant
    articles to verify that our implementation works as expected.

    \section{Single quantum dots}
        In this section we'll discuss results for quantum dots subject to a
        harmonic oscillator potential in one and two dimensions.
        These systems are rather well studied and therefore provides ample
        results which can be used as verification.

        \subsection{The one-dimensional harmonic oscillator}
            An excellent starting point is the one-dimensional harmonic
            oscillator as discussed in \autoref{subsec:one-dim-ho}.
            A study done by \citeauthor{zanghellini_2004}
            \cite{zanghellini_2004} explores the multi-configuration
            time-dependent Hartree-Fock method on the one-dimensional harmonic
            oscillator system for two particles compared to a semi-analytic
            result\footnote{Semi-analytic as the time-evolution is solved
            numerically.}.
            In this study the Hamiltonian is given by
            \begin{align}
                \hamil
                &=
                \sum_{i = 1}^{N}
                \para{
                    \frac{\momentum^2_i}{2m}
                    + \half m \omega^2 \position^2_i
                    + \mathcal{E}_0 \sin(\Omega t)\position_i
                }
                \nonumber \\
                &\qquad
                + \sum_{i, j = 1}^{N}
                \frac{1}{4\epsilon_0\pi}
                \frac{e^2}{\sqrt{(\position_i - \position_j)^2 + a^2}}.
            \end{align}
            However, \citeauthor{zanghellini_2004} uses the more convenient set
            of Hartree atomic units where $m = e = \hslash = k_e =
            \para{4\epsilon_0\pi}^{-1} = 1$.
            The term $a$ is a smoothing parameter included to remove the
            singularity when we integrate over all space and the two positions
            $x_i$ and $x_j$ overlap. \cite{suq, zanghellini_2004}
            In the simulation done by \citeauthor{zanghellini_2004} they set $a
            = 0.25$, $\omega = 0.25$, $\mathcal{E}_0 = 1$ and $\Omega = 8\omega$
            on a grid where $x \in [-10, 10]$.
            The laser pulse is turned on for the entirety of the simulation.

            In \autoref{fig:one-body-density-zanghellini} we plot the ground
            state one-body particle densities using Hartree-Fock and
            configuration interaction with singles and doubles, i.e., full
            configuration interaction for two particles.
            In the lower plot in \autoref{fig:one-body-density-zanghellini} we
            demonstrate that the coupled cluster method with singles and double
            reproduces the exact solution as achieved from configuration
            interaction to a high precision.
            Furthermore, in \autoref{fig:overlap-zanghellini}, we show the
            time-dependent overlap for two one-dimensional quantum dots subject
            to the same dipole laser as in the study done by
            \citeauthor{zanghellini_2004} \cite{zanghellini_2004}.
            We also show how the coupled cluster method with singles and doubles
            reproduces the exact solution to a large degree.

            It is worth re-iterating that we have not included a comparison
            using the orbital-adaptive time-dependent coupled cluster method due
            to the frustrating fact that the orbital rotations induced by this
            method makes the process of computing the time-dependent overlap a
            challenging task and has therefore been treated as an out-of-scope
            task for this thesis.

            \begin{figure}
                \centering
                \begin{tikzpicture}
                    \begin{groupplot}[
                            group style={
                                group size=1 by 2,
                                vertical sep=30pt,
                                xlabels at=edge bottom,
                            },
                            width=11cm,
                            height=6cm,
                            xlabel={$x$ $[\text{a.u.}]$},
                        ]
                        \nextgroupplot[
                                grid=major,
                                ylabel={$\densityten(x, 0)$},
                                restrict x to domain=-6:6,
                                enlarge x limits=false,
                                ymin=0,
                                ymax=0.4,
                                enlarge y limits=false,
                            ]
                            \addplot+ [
                                mark=none,
                                thick,
                            ]
                            table
                            {results/benchmarks/zanghellini/dat/rho_tdcisd_real.dat};
                            \addlegendentry{TDCISD}

                            \addplot+ [
                                mark=none,
                                thick,
                                dashed,
                            ]
                            table
                            {results/benchmarks/zanghellini/dat/rho_tdhf_real.dat};
                            \addlegendentry{TDHF}
                        \nextgroupplot[
                                ymode=log,
                                grid=major,
                                ylabel={$\abs{\Delta\densityten(x, 0)}$},
                                restrict x to domain=-6:6,
                                enlarge x limits=false,
                            ]
                            \addplot+ [
                                mark=none,
                                thick,
                            ]
                            table
                            {results/benchmarks/zanghellini/dat/rho_diff_tdccsd_tdcisd.dat};
                    \end{groupplot}
                \end{tikzpicture}
                \caption{In the top figure we have reproduced Figure 1. in the
                study done by \citeauthor{zanghellini_2004}
                \cite{zanghellini_2004}, that is, we have plotted the one-body
                particle density two one-dimensional quantum dots in an harmonic
                oscillator trap using full-configuration interaction and a
                general Hartree-Fock solver.
                In the lower figure we have plotted the absolute difference
                between the one-body particle densities from the
                full-configuration interaction and coupled cluster with singles
                and doubles methods.}
                \label{fig:one-body-density-zanghellini}
             \end{figure}

            \begin{figure}
                \centering
                \begin{tikzpicture}
                    \begin{groupplot}[
                            group style={
                                group size=1 by 2,
                                vertical sep=30pt,
                                xlabels at=edge bottom,
                            },
                            width=11cm,
                            height=6cm,
                            xlabel={$\omega t / (2\pi)$},
                        ]
                        \nextgroupplot[
                                grid=major,
                                ylabel={$\abs{\braket*{\Psi(t)}{\Psi(0)}}^2$},
                                enlarge x limits=false,
                            ]
                            \addplot+ [
                                mark=none,
                                thick,
                            ]
                            table
                            {results/benchmarks/zanghellini/dat/overlap_tdcisd_real.dat};
                            \addlegendentry{TDCISD}

                            \addplot+ [
                                mark=none,
                                thick,
                                dashed,
                            ]
                            table
                            {results/benchmarks/zanghellini/dat/overlap_tdhf_real.dat};
                            \addlegendentry{TDHF}
                        \nextgroupplot[
                                ymode=log,
                                grid=major,
                                ylabel={$\Delta\abs{\braket*{\Psi(t)}{\Psi(0)}}^2$},
                                restrict x to domain=-6:6,
                                enlarge x limits=false,
                            ]
                            \addplot+ [
                                mark=none,
                                thick,
                            ]
                            table
                            {results/benchmarks/zanghellini/dat/overlap_diff_tdccsd_tdcisd.dat};
                    \end{groupplot}
                \end{tikzpicture}
                \caption{In the top figure we have plotted the time-dependent
                overlap between the initial ground state $\ket{\Psi(0)}$ and the
                state $\ket{\Psi(t)}$ at a later time.
                We compare the time-dependent general Hartree-Fock method and
                the time-dependent full configuration interaction method.
                The figure is a reproduction of Figure 2. in the study done by
                \citeauthor{zanghellini_2004} \cite{zanghellini_2004}.
                In the lower figure we show the absolute difference in the
                overlap between the time-dependent full configuration
                interaction method the time-dependent coupled cluster method
                with singles and doubles.}
                \label{fig:overlap-zanghellini}
             \end{figure}


            \cite{skattum2013time} \cite{kristiansen2017time}
    \section{Double-well quantum dots}

    \section{Atoms and molecules}
        In this section we'll look at how our methods perform when applied to
        atomic and molecular systems.
        Building atomic and molecular systems is a large field in itself and we
        will not make our own basis sets as there are already excellent
        libraries available.
        We'll mainly use two libraries throughout this thesis for creating
        atomic systems and that is PySCF \cite{pyscf} and Psi4 \cite{psi4}.

        \subsection{Two-electron atoms}
            In a study on the Hydrogen molecule subject to an intense laser
            field done by \citeauthor{li_2005} \cite{li_2005} a comparison of
            the time-dependent Hartree-Fock method and the time-dependent full
            configuration interaction\footnote{%
                Note that \citeauthor{li_2005} calls the time-dependent
                full-configuration interaction method for the time-dependent
                Schrödinger equation.%
            } is performed.
            This provides us with an ample opportunity to repeat the
            experiments in order for us to verify our methods.
            Particularly, we are in a position to test the orbital-adaptive
            time-dependent coupled cluster method as the dipole moment requires
            us to do a basis transformation to the new orbital basis and compute
            the one-body density matrix.
            That is, we compute
            \begin{align}
                \tilde{\vfg{d}}^{p}_{q}(t)
                = \mel*{\tilde{\phi}_p(t)}{\hat{\vfg{d}}}{\phi_q(t)}
                = \tilde{C}_{p\alpha}(t)
                \mel*{\chi_{\alpha}}{\hat{\vfg{d}}}{\chi_{\beta}}
                C_{\beta q}(t),
            \end{align}
            as the basis transformed dipole matrix elements.
            Having compute the one-body density matrix $\densityten^{q}_{p}(t)$
            we can find the induced dipole-moment along the $z$-axis by picking
            out the $z$-component of the dipole elements, that is,
            \begin{align}
                \expv{z(t)}
                &= \densityten^{q}_{p}(t) \tilde{z}^{p}_{q}(t),
            \end{align}
            where $\tilde{z}^{p}_{q}(t)$ is the $z$-component of
            $\tilde{\vfg{d}}^{p}_{q}(t)$.
            The Hydrogen molecule is expressed in $14$ Gaussian type orbitals,
            the $6-311++G(d, p)$ basis set, with an equilibrium geometry of $R_e
            = \SI{0.7354}{\angstrom} \approx \SI{1.3897}{\bohr}$.
            We center the molecule around the origin with each atom centered at
            $\pm R_e/2$.
            We then make the basis set doubly occupied by including spin.
            The basis is run through a general Hartree-Fock solver which finds
            the Hartree-Fock molecular orbitals and changes to this basis.
            The laser field used by \citeauthor{li_2005} is
            \begin{align}
                \hat{\vfg{d}} \cdot \vfg{f}(t)
                = \hat{\vfg{d}} \cdot \vfg{E}(t)\sin(\omega t),
            \end{align}
            where $\hat{\vfg{d}}$ is the dipole operator in three dimensions and
            the envelope $\vfg{E}(t) = E(t) \cdot \vfg{\epsilon}$ with
            $\vfg{\epsilon}$ as the polarization vector.
            We let the envelope cycle as a function of time by
            \begin{align}
                E(t) = \begin{cases}
                    (\omega t / 2\pi) E_m, & t\omega \in [0, 2\pi], \\
                    E_m, & t\omega \in [2\pi, 4\pi], \\
                    [3 - \omega t / (2\pi)], & t\omega \in [4\pi, 6\pi], \\
                    0, & t\omega \neq [0, 6\pi],
                \end{cases}
            \end{align}
            where we set $E_m = \SI{0.07}{\text{a.u.}}$ and $\omega =
            \SI{0.1}{\text{a.u.}}$.
            In our programs the dipole moment is defined with a positive sign,
            which means that we need to introduce a negative sign in the
            envelope or the polarization vector to include the negative charge
            of the electrons.
            We have chosen to set the polarization vector along the negative
            $z$-direction.
            A plot of the induced dipole moment in the $z$-direction where we
            simulate for a total of $t_f = \SI{225}{\text{a.u.}}$ with the laser
            turned on from start until $2\pi/\omega \approx
            \SI{62.8}{\text{a.u.}}$ is shown in \autoref{fig:dipole-moment-li}.
            In the lower figure we show the absolute error in the induced dipole
            moment as calculated by the time-dependent coupled cluster singles
            and doubles method and the orbital-adaptive time-dependent coupled
            cluster doubles method compared with the exact full configuration
            interaction solution.

            It is interesting to note in the top figure in
            \autoref{fig:dipole-moment-li} that the time-dependent Hartree-Fock
            method performs surprisingly well.
            There are small discrepancies, but at the intensity we use for our
            laser the mean-field approximation performs excellently.
            However, we expect that as the intensity increases, this method will
            prove far more inferior to the correlated methods.
            % TODO: Back this up.

            \begin{figure}
                \centering
                \begin{tikzpicture}
                    \begin{groupplot}[
                            group style={
                                group size=1 by 2,
                                vertical sep=30pt,
                                xlabels at=edge bottom,
                            },
                            width=11cm,
                            height=6cm,
                            xlabel={$t$ $[\text{a.u.}]$},
                        ]
                        \nextgroupplot[
                                grid=major,
                                ylabel={$\expv{z(t)}$},
                                enlarge x limits=false,
                                enlarge y limits=false,
                            ]
                            \addplot+ [
                                mark=none,
                                thick,
                            ]
                            table
                            {results/benchmarks/li/dat/dipole_z_tdcisd_real.dat};
                            \addlegendentry{TDCISD}

                            \addplot+ [
                                mark=none,
                                thick,
                            ]
                            table
                            {results/benchmarks/li/dat/dipole_z_tdhf_real.dat};
                            \addlegendentry{TDHF}

                        \nextgroupplot[
                                ymode=log,
                                grid=major,
                                ylabel={$\abs{\Delta\expv{z(t)}}$},
                                enlarge x limits=false,
                                enlarge y limits=false,
                            ]
                            \addplot+ [
                                mark=none,
                                thick,
                            ]
                            table
                            {results/benchmarks/li/dat/dipole_z_diff_tdccsd_tdcisd.dat};
                            \addlegendentry{TDCCSD}

                            \addplot+ [
                                mark=none,
                                thick,
                            ]
                            table
                            {results/benchmarks/li/dat/dipole_z_diff_oatdccd_tdcisd.dat};
                            \addlegendentry{OATDCCD}
                    \end{groupplot}
                \end{tikzpicture}
                \caption{In the top figure we have plotted the induced dipole
                moment by radiating a Hydrogen molecule with a dipole laser.
                The figure reproduces Figure. 4 by \citeauthor{li_2005}
                \cite{li_2005} using both the time-dependent general
                Hartree-Fock solver and the time-dependent full
                configuration-interation solver.}
                \label{fig:dipole-moment-li}
            \end{figure}
