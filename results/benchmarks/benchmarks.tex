\chapter{Benchmarking against litterature}
    In this chapter we use our developed code to reproduce results from relevant
    articles to verify that our implementation works as expected.
    Furthermore, this chapter demonstrates the versatility our methods by the
    application of many different types of systems.

    \section{The one-dimensional harmonic oscillator}
        An excellent starting point is the one-dimensional harmonic
        oscillator described using the discretized one-dimensional quantum dot
        system as discussed in \autoref{sec:one-dim-qd}.
        A study done by \citeauthor{zanghellini_2004}
        \cite{zanghellini_2004, skattum2013time, kristiansen2017time}
        explores the multi-configuration time-dependent Hartree-Fock method
        on the one-dimensional harmonic oscillator system for two particles
        compared to a semi-analytic result.\footnote{%
            Semi-analytic as the time-evolution is solved numerically.
        }
        In this study the external potential is given by
        \begin{align}
            \hat{v} = \half \omega^2\hat{x}^2,
        \end{align}
        where atomic units are used.
        \citeauthor{zanghellini_2004} uses a monochromatic, dipole, laser field
        described by
        \begin{align}
            E(t) = E_0\sin(\Omega t)\hat{x},
        \end{align}
        which is active for the entire simulation.
        The Coulomb interaction is described by the shielded Coulomb interaction
        from \autoref{eq:shielded-coulomb} with $\alpha = 1$ and the screening
        parameter set to $a = 0.25$.
        In the simulation done by \citeauthor{zanghellini_2004} the parameters
        are chosen to be $\omega = 0.25$, $E_0 = 1$ and $\Omega =
        8\omega$.
        The simulation is run on a one-dimensional grid where $x \in [-10, 10]$.
        We use a basis of $L = 20$ spin-orbitals, that is, the $L / 2 = 10$
        first one-dimensional harmonic oscillator orbitals found by
        diagonalizing the one-body Hamiltonian.

        In \autoref{fig:one-body-density-zanghellini} we plot the ground
        state one-body particle densities using Hartree-Fock and CISD, i.e.,
        full configuration interaction for two particles.
        In the lower plot in \autoref{fig:one-body-density-zanghellini} we
        demonstrate that CCSD reproduces the exact solution as achieved from
        configuration interaction to a high precision.
        We've listed the ground state energies in \autoref{tab:gs-zanghellini}
        for the different solver methods.
        We compute the relative error by
        \begin{align}
            e_{rel} = \frac{\abs{E - E_{ref}}}{\abs{E_{ref}}},
        \end{align}
        where $E_{ref} = \SI{0.8247}{\hartree}$ as the exact value from
        \citeauthor{zanghellini_2004} \cite{zanghellini_2004}.
        Note that the relative error is computed for the rounded values.

        \begin{table}
            \centering
            \caption{In this table we list the ground state energies computed
            for the one-dimensional harmonic oscillator.
            The values are rounded to four decimal places.}
            \renewcommand{\arraystretch}{1.3}
            \begin{tabular}{@{}lll@{}}
                \toprule
                Method & Energy $[\si{\hartree}]$
                & Relative error \\
                \midrule
                HF     & $1.1798$ & $0.4306$ \\
                CID    & $1.0516$ & $0.2751$ \\
                CCD    & $1.0516$ & $0.2751$ \\
                CID-HF & $0.8384$ & $0.0166$ \\
                CCD-HF & $0.8384$ & $0.0166$ \\
                CISD   & $0.8253$ & $0.0007$ \\
                CCSD   & $0.8253$ & $0.0007$ \\
                \bottomrule
            \end{tabular}
            \label{tab:gs-zanghellini}
        \end{table}

        \begin{figure}
            \centering
            \begin{tikzpicture}
                \begin{groupplot}[
                        group style={
                            group size=1 by 2,
                            vertical sep=30pt,
                            xlabels at=edge bottom,
                        },
                        width=11cm,
                        height=6cm,
                        xlabel={$x$ $[\si{\bohr}]$},
                    ]
                    \nextgroupplot[
                            grid=major,
                            ylabel={$\densityten(x, 0)$},
                            restrict x to domain=-6:6,
                            enlarge x limits=false,
                            ymin=0,
                            ymax=0.4,
                            enlarge y limits=false,
                            title={Ground state particle density},
                        ]
                        \addplot+ [
                            mark=none,
                            thick,
                        ]
                        table
                        {results/benchmarks/zanghellini/dat/rho_tdcisd_real.dat};
                        \addlegendentry{TDCISD}

                        \addplot+ [
                            mark=none,
                            thick,
                            dashed,
                        ]
                        table
                        {results/benchmarks/zanghellini/dat/rho_tdhf_real.dat};
                        \addlegendentry{TDHF}
                    \nextgroupplot[
                            ymode=log,
                            grid=major,
                            ylabel={$\abs{\Delta\densityten(x, 0)}$},
                            restrict x to domain=-6:6,
                            enlarge x limits=false,
                        ]
                        \addplot+ [
                            mark=none,
                            thick,
                        ]
                        table
                        {results/benchmarks/zanghellini/dat/rho_diff_tdccsd_tdcisd.dat};
                \end{groupplot}
            \end{tikzpicture}
            \caption{In the top figure we have reproduced Figure 1. in the
            study done by \citeauthor{zanghellini_2004}
            \cite{zanghellini_2004}, that is, we have plotted the one-body
            particle density two one-dimensional quantum dots in an harmonic
            oscillator trap using full-configuration interaction and a
            general Hartree-Fock solver.
            In the lower figure we have plotted the absolute difference
            between the one-body particle densities from the
            full-configuration interaction and coupled cluster with singles
            and doubles methods.}
            \label{fig:one-body-density-zanghellini}
        \end{figure}

        In \autoref{fig:overlap-zanghellini} we show the
        time-dependent overlap for two one-dimensional quantum dots subject
        to the same dipole laser as in the study done by
        \citeauthor{zanghellini_2004} \cite{zanghellini_2004}.
        We also show how CCSD reproduces the exact CISD solution to a large
        degree.

        \begin{figure}
            \centering
            \begin{tikzpicture}
                \begin{groupplot}[
                        group style={
                            group size=1 by 2,
                            vertical sep=30pt,
                            xlabels at=edge bottom,
                        },
                        width=11cm,
                        height=6cm,
                        xlabel={$\omega t / (2\pi)$},
                    ]
                    \nextgroupplot[
                            grid=major,
                            ylabel={$\abs{\braket*{\Psi(t)}{\Psi(0)}}^2$},
                            enlarge x limits=false,
                            title={Time-dependent overlap},
                        ]
                        \addplot+ [
                            mark=none,
                            thick,
                        ]
                        table
                        {results/benchmarks/zanghellini/dat/overlap_tdcisd_real.dat};
                        \addlegendentry{TDCISD}

                        \addplot+ [
                            mark=none,
                            thick,
                            dashed,
                        ]
                        table
                        {results/benchmarks/zanghellini/dat/overlap_tdhf_real.dat};
                        \addlegendentry{TDHF}
                    \nextgroupplot[
                            ymode=log,
                            grid=major,
                            ylabel={$\Delta\abs{\braket*{\Psi(t)}{\Psi(0)}}^2$},
                            restrict x to domain=-6:6,
                            enlarge x limits=false,
                        ]
                        \addplot+ [
                            mark=none,
                            thick,
                        ]
                        table
                        {results/benchmarks/zanghellini/dat/overlap_diff_tdccsd_tdcisd.dat};
                \end{groupplot}
            \end{tikzpicture}
            \caption{In the top figure we have plotted the time-dependent
            overlap between the initial ground state $\ket{\Psi(0)}$ and the
            state $\ket{\Psi(t)}$ at a later time.
            We compare the time-dependent general Hartree-Fock method and
            the time-dependent full configuration interaction method.
            The figure is a reproduction of Figure 2. in the study done by
            \citeauthor{zanghellini_2004} \cite{zanghellini_2004}.
            In the lower figure we show the absolute difference in the
            overlap between the time-dependent full configuration
            interaction method the time-dependent coupled cluster method
            with singles and doubles.}
            \label{fig:overlap-zanghellini}
        \end{figure}

        Before moving on to more exotic systems, we will look shortly at the
        performance of CCD and CID applied to the same one-dimensional harmonic
        oscillator system.
        The reason for doing this is to demonstrate the importance of the
        singles excitation operator.
        Looking at \autoref{fig:one-body-density-doubles-zanghellini} we can see
        how the doubles approximations CID and CCD with the harmonic oscillator
        basis are quite far from the full configuration-interaction benchmark.
        With a Hartree-Fock basis CID and CCD are much better at reproducing the
        full configuration-interaction result.
        This demonstrates the effectiveness of Brillouin's theorem by moving to
        the optimal single-reference determinant found in Hartree-Fock.
        However, this does not completely remove the need for a singles
        operator.

        \begin{figure}
            \centering
            \begin{tikzpicture}
                \begin{groupplot}[
                        group style={
                            group size=1 by 2,
                            vertical sep=30pt,
                            xlabels at=edge bottom,
                        },
                        width=11cm,
                        height=6cm,
                        xlabel={$x$ $[\si{\bohr}]$},
                    ]
                    \nextgroupplot[
                            grid=major,
                            ylabel={$\densityten(x, 0)$},
                            restrict x to domain=-6:6,
                            enlarge x limits=false,
                            ymin=0,
                            ymax=0.5,
                            enlarge y limits=false,
                            title={Ground state particle density},
                        ]

                        \addplot+ [
                            mark=none,
                            thick,
                        ]
                        table
                        {results/benchmarks/zanghellini/dat/rho_tdcid_real.dat};
                        \addlegendentry{TDCID}

                        \addplot+ [
                            mark=none,
                            thick,
                        ]
                        table
                        {results/benchmarks/zanghellini/dat/rho_tdcid_hf_real.dat};
                        \addlegendentry{TDCID-HF}

                        \addplot+ [
                            mark=none,
                            thick,
                            dashed,
                        ]
                        table
                        {results/benchmarks/zanghellini/dat/rho_tdcisd_real.dat};
                        \addlegendentry{TDCISD}

                    \nextgroupplot[
                            ymode=log,
                            grid=major,
                            ylabel={$\abs{\Delta\densityten(x, 0)}$},
                            restrict x to domain=-6:6,
                            enlarge x limits=false,
                        ]
                        \addplot+ [
                            mark=none,
                            thick,
                        ]
                        table
                        {results/benchmarks/zanghellini/dat/rho_diff_tdccd_tdcid.dat};
                        \addlegendentry{HO}

                        \addplot+ [
                            mark=none,
                            thick,
                        ]
                        table
                        {results/benchmarks/zanghellini/dat/rho_diff_tdccd_tdcid_hf.dat};
                        \addlegendentry{HF}
                \end{groupplot}
            \end{tikzpicture}
            \caption{In this figure we try to reproduce
            \autoref{fig:one-body-density-zanghellini} using plain CID and CID with
            Hartree-Fock basis.
            The lower figure shows the absolute difference between CCD and CID
            with harmonic oscillator (HO) basis and Hartree-Fock (HF) basis.}
            \label{fig:one-body-density-doubles-zanghellini}
        \end{figure}

        In \autoref{fig:overlap-doubles-zanghellini} we show CID and CCD with
        and without a Hartree-Fock basis is unable to give a good reconstruction
        of the exact solution from full configuration interaction.
        The lack of singles excitations become glaringly apparent once we start
        the time-evolution of the system.
        This can be seen as an effect of the laser field being described by a
        singles operator.
        It is therefore likely that a singles-excitation operator best describes
        this interaction and the doubles approximation will have problems
        properly depicting the dynamics.

        \begin{figure}
            \centering
            \begin{tikzpicture}
                \begin{groupplot}[
                        group style={
                            group size=1 by 2,
                            vertical sep=30pt,
                            xlabels at=edge bottom,
                        },
                        width=11cm,
                        height=6cm,
                        xlabel={$\omega t / (2\pi)$},
                    ]
                    \nextgroupplot[
                            grid=major,
                            ylabel={$\abs{\braket*{\Psi(t)}{\Psi(0)}}^2$},
                            enlarge x limits=false,
                            title={Time-dependent overlap},
                        ]

                        \addplot+ [
                            mark=none,
                            thick,
                        ]
                        table
                        {results/benchmarks/zanghellini/dat/overlap_tdcid_real.dat};
                        \addlegendentry{TDCID}

                        \addplot+ [
                            mark=none,
                            thick,
                        ]
                        table
                        {results/benchmarks/zanghellini/dat/overlap_tdcid_hf_real.dat};
                        \addlegendentry{TDCID-HF}

                        \addplot+ [
                            mark=none,
                            thick,
                            dashed,
                        ]
                        table
                        {results/benchmarks/zanghellini/dat/overlap_tdcisd_real.dat};
                        \addlegendentry{TDCISD}

                    \nextgroupplot[
                            ymode=log,
                            grid=major,
                            ylabel={$\Delta\abs{\braket*{\Psi(t)}{\Psi(0)}}^2$},
                            restrict x to domain=-6:6,
                            enlarge x limits=false,
                        ]
                        \addplot+ [
                            mark=none,
                            thick,
                        ]
                        table
                        {results/benchmarks/zanghellini/dat/overlap_diff_tdccd_tdcid.dat};
                        \addlegendentry{HO}

                        \addplot+ [
                            mark=none,
                            thick,
                        ]
                        table
                        {results/benchmarks/zanghellini/dat/overlap_diff_tdccd_tdcid_hf.dat};
                        \addlegendentry{HF}
                \end{groupplot}
            \end{tikzpicture}
            \caption{Here we reproduce \autoref{fig:overlap-zanghellini} with
            the doubles approximations CID and CCD using both the original
            harmonic oscillator (HO) basis, and the Hartree-Fock (HF) basis.
            In the lower figure we demonstrate the absolute error between CID
            and CCD using both basis sets.}
            \label{fig:overlap-doubles-zanghellini}
        \end{figure}



    \section{Two-electron molecule}
        In a study on the optical response of molecules subject to an
        intense laser fields done by \citeauthor{li_2005} \cite{li_2005} a
        comparison of the time-dependent Hartree-Fock method and the
        time-dependent full configuration interaction\footnote{%
            Note that \citeauthor{li_2005} denotes the time-dependent
            full-configuration interaction method by: ``the time-dependent
            Schrödinger equation''.
        } is performed.
        This provides us with an ample opportunity to repeat the
        experiments in order for us to verify our methods.
        We will therefore reproduce parts of the study done by
        \citeauthor{li_2005} using TDCISD, TDHF, TDCCSD, and OATDCCD.

        We'll look at the smallest molecule \ch{H2}, that is, the Hydrogen
        molecule with two electrons.
        This means that TDCISD, TDCCSD, and OATDCCD should produce exact
        results for the single-particle basis that is used.
        %That is, we compute
        %\begin{align}
        %    \tilde{\vfg{d}}^{p}_{q}(t)
        %    = \mel*{\tilde{\phi}_p(t)}{\hat{\vfg{d}}}{\phi_q(t)}
        %    = \tilde{C}_{p\alpha}(t)
        %    \mel*{\chi_{\alpha}}{\hat{\vfg{d}}}{\chi_{\beta}}
        %    C_{\beta q}(t),
        %\end{align}
        %as the basis transformed dipole matrix elements.
        %Having computed the one-body density matrix $\densityten^{q}_{p}(t)$
        %from OATDCCD, we can find the induced dipole-moment along the
        %$z$-axis by picking out the $z$-component of the dipole elements,
        %that is,
        %\begin{align}
        %    \expval{z(t)}
        %    &= \densityten^{q}_{p}(t) \tilde{z}^{p}_{q}(t),
        %\end{align}
        %where $\tilde{z}^{p}_{q}(t)$ is the $z$-component of
        %$\tilde{\vfg{d}}^{p}_{q}(t)$.
        We express \ch{H2} in $14$ Gaussian type orbitals, that is, the
        $6-311++G(d, p)$ basis set, with an equilibrium geometry of $R_e =
        \SI{0.7354}{\angstrom} \approx \SI{1.3897}{\bohr}$.
        We center the molecule around the origin with each atom located at
        $\pm R_e/2$ in the $z$-direction zero for the $x$- and
        $y$-direction.
        We then make the basis set doubly occupied by including spin.
        The basis is run through a general Hartree-Fock solver which finds
        the Hartree-Fock molecular orbitals and changes to this basis.
        The laser pulse used by \citeauthor{li_2005} is
        \begin{align}
            \hat{\vfg{d}} \cdot \vfg{f}(t)
            = \hat{\vfg{d}} \cdot \vfg{E}(t)\sin(\omega t),
            \label{eq:li-field}
        \end{align}
        where $\hat{\vfg{d}}$ is the dipole operator in three dimensions and
        the envelope $\vfg{E}(t) = E(t)\vfg{\epsilon}$ with $\vfg{\epsilon}$
        as the polarization vector.
        We let the envelope cycle as a function of time by
        \begin{align}
            E(t) = \begin{cases}
                (\omega t / 2\pi) E_m, & \omega t \in [0, 2\pi], \\
                E_m, & \omega t \in [2\pi, 4\pi], \\
                [3 - \omega t / (2\pi)] E_m, & \omega t \in [4\pi, 6\pi], \\
                0, & \omega t \neq [0, 6\pi],
            \end{cases}
            \label{eq:li-laser}
        \end{align}
        where we set $E_m = \SI{0.07}{\hartree}$ and $\omega =
        \SI{0.1}{\hartree/\hslash}$.
        This envelope is slightly unconventional as compared to the more
        common envelope functions discussed in \autoref{subsec:envelope}.
        A plot of the laser field during the simulation is shown in
        \autoref{fig:li-laser}.
        \begin{figure}
            \centering
            \begin{tikzpicture}
                \begin{axis}[
                    width=11cm,
                    height=6cm,
                    xlabel={$t$ $[\text{a.u.}]$},
                    ylabel={$E(t)$ $[\text{a.u.}]$},
                    grid=major,
                    enlarge x limits=false,
                    enlarge y limits=false,
                ]
                    \addplot+[
                        mark=none,
                        thick,
                    ]
                    table
                    {results/benchmarks/li/dat/li_laser.dat};
                \end{axis}
            \end{tikzpicture}
            \caption{In this figure we've plotted the laser field from
            \autoref{eq:li-field} with the envelope described in
            \autoref{eq:li-laser} as done in the study by
            \citeauthor{li_2005} \cite{li_2005}.}
            \label{fig:li-laser}
        \end{figure}
        In our programs the dipole moment is defined with a positive sign,
        which means that we need to introduce a negative sign in the
        envelope or the polarization vector to include the negative charge
        of the electrons.
        We have chosen to set the polarization vector along the negative
        $z$-direction.
        A plot of the instantaneous dipole moment of \ch{H2} in the
        $z$-direction where we simulate for a total of $t_f =
        \SI{225}{\hslash/\hartree}$ with the laser turned on from start
        until $2\pi/\omega \approx \SI{62.8}{\hslash/\hartree}$ is shown in
        \autoref{fig:dipole-moment-li}.
        A by-eye comparison of the top figure in
        \autoref{fig:dipole-moment-li} with figure 4 in \citetitle{li_2005}
        \cite{li_2005} shows that we are in perfect agreement with the
        results of \citeauthor{li_2005}.
        In the lower figure we show the absolute error in the induced dipole
        moment as calculated by the TDCCSD and OATDCCD methods compared with
        the exact full configuration interaction solution.
        The convergence threshold of the Gauss-Legendre integrator was set
        to $\num{1e-6}$ which is larger than the absolute error between the
        methods.
        We therefore conclude that TDCCSD and OATDCCD reproduce the exact
        results to a satisfying degree.

        It is interesting to note in the top figure in
        \autoref{fig:dipole-moment-li} that the time-dependent Hartree-Fock
        method performs surprisingly well.
        There are small discrepancies, but at the intensity we use for our
        laser the mean-field approximation performs excellently.
        However, we expect that as the intensity increases, this method will
        prove inferior to the correlated methods.
        % TODO: Back this up.

        \begin{figure}
            \centering
            \begin{tikzpicture}
                \begin{groupplot}[
                        group style={
                            group size=1 by 2,
                            vertical sep=30pt,
                            xlabels at=edge bottom,
                        },
                        width=11cm,
                        height=6cm,
                        xlabel={$t$ $[\text{a.u.}]$},
                    ]
                    \nextgroupplot[
                            grid=major,
                            ylabel={$\expval{z(t)}$},
                            enlarge x limits=false,
                            enlarge y limits=false,
                        ]
                        \addplot+ [
                            mark=none,
                            thick,
                        ]
                        table
                        {results/benchmarks/li/dat/dipole_z_tdcisd_real.dat};
                        \addlegendentry{TDCISD}

                        \addplot+ [
                            mark=none,
                            thick,
                        ]
                        table
                        {results/benchmarks/li/dat/dipole_z_tdhf_real.dat};
                        \addlegendentry{TDHF}

                    \nextgroupplot[
                            ymode=log,
                            grid=major,
                            ylabel={$\abs{\Delta\expval{z(t)}}$},
                            enlarge x limits=false,
                            enlarge y limits=false,
                            legend pos=south east,
                        ]
                        \addplot+ [
                            mark=none,
                            thick,
                        ]
                        table
                        {results/benchmarks/li/dat/dipole_z_diff_tdccsd_tdcisd.dat};
                        \addlegendentry{TDCCSD}

                        \addplot+ [
                            mark=none,
                            thick,
                        ]
                        table
                        {results/benchmarks/li/dat/dipole_z_diff_oatdccd_tdcisd.dat};
                        \addlegendentry{OATDCCD}
                \end{groupplot}
            \end{tikzpicture}
            \caption{In the top figure we have plotted the instantaneous
            dipole by radiating a \ch{H2} molecule with a dipole laser.
            The lower figure shows the absolute error of the dipole using
            the TDCCSD and OATDCCD methods compared to the TDCISD solver.}
            \label{fig:dipole-moment-li}
        \end{figure}

    \section{Ionization of one-dimensional atoms}
        The process of modelling ionization of electrons in atoms and
        molecules is a tricky subject, particularly in our regime.
        That is because we are forcing our systems to be bound by simulating a
        restricted space and requiring that all particles be contained inside
        the simulation box.
        As a consequence, ionization has to be inferred in some way or another.
        e.g., an absorbing potential with a time-dependent atomic orbital basis
        on a grid \cite{kosloff1986363, miyagi_and_madsen, takeshi}.
        However, in our formalism we use static atomic orbitals which we
        force to zero at the boundaries of the simulation box.
        This makes the use of an absorbing potential useless and ionization
        must be inferred in some other way.

        In a study done by \citeauthor{miyagi_and_madsen}
        \cite{miyagi_and_madsen} they explored the dynamics of one-dimensional
        atoms subject to a dipole laser.
        This study provides us with an excellent benchmark to observe a form
        of ionization and see if this is reproducible in our formalism.
        Note that \citeauthor{miyagi_and_madsen} used a
        discrete-variable-representation basis (DVR) which is why they include
        an absorbing potential.
        Our solution differs from this as we use the static one-dimensional
        quantum dot basis with a one-dimensional atom potential.
        In a position basis with Hartree atomic units, the one-body
        Hamiltonian is given by \cite{miyagi_and_madsen}
        \begin{align}
            \oneten(x, t)
            &= -\half \dod[2]{}{x}
            - \frac{Z}{\sqrt{x^2 + 1}}
            + xF(t)
            - iW(x),
        \end{align}
        where $Z = N_e$, that is, the number of electrons in the system,
        $W(x)$ is the absorbing potential, which we'll set to zero.
        The second term gives rise to the naming of these systems as
        one-dimensional atoms as this term serves as the electron-nuclear
        interaction in the atomic Hamiltonian, but with the three-dimensional
        position replaced with $x$.
        One of the interesting aspects of the one-dimensional atoms is that many
        open-shell systems such as \ch{C} gets closed shells in the
        one-dimensional case.
        We will however, limit our study to the one-dimensional Beryllium
        system with $N = 4$ electrons in a closed shell.

        The laser pulse $F(t)$ is given by \cite{miyagi_and_madsen}
        \begin{align}
            F(t)
            &= -\dod[]{A(t)}{t}
            = -\dod[]{}{t}\brak{
                \frac{F_0}{\omega}
                \sin^2\para{\frac{\pi t}{T}}
                \sin(\omega t)
            }
            \\
            &=
            -\sin\para{\frac{\pi t}{T}}\brak{
                \omega \sin\para{\frac{\pi t}{T}}
                \cos(\omega t)
                + \frac{2\pi}{T}
                \cos\para{\frac{\pi t}{T}}
                \sin(\omega t)
            }.
            \label{eq:miyagi-laser}
        \end{align}
        where this field is active for $t \in [0, T]$.
        A plot of the laser pulse during the simulation is shown in
        \autoref{fig:miyagi-laser}.
        The Coulomb interaction is given by \autoref{eq:shield-coulomb} with a
        shielding parameter of $a = 1$ and $\alpha = 1$.
        To compare with the study by \citeauthor{miyagi_and_madsen} we will run
        for $T = \SI{331}{\hslash/\hartree}$ with $\Delta t =
        \SI{1e-2}{\hslash/\hartree}$ which corresponds to $n_t = 33100$
        time-steps.
        This computation is rather involved and we will therefore limit
        ourselves to the time-dependent Hartree-Fock method and the
        orbital-adaptive time-dependent coupled cluster method as the
        full-configuration interaction method becomes quite expensive.
        In fact, for $n = 4$ and $l = 40$ we have to create a Hamiltonian
        matrix with $91390^2$ elements, each of size $\SI{16}{\text{B}}$ as
        we use complex numbers.
        To solve this system we require a much smarter implementation of the
        configuration interaction method which is out of scope for this
        thesis.

        \begin{figure}
            \centering
            \begin{tikzpicture}
                \begin{axis}[
                        width=11cm,
                        height=6cm,
                        xlabel={$t$ $[\text{a.u.}]$},
                        grid=major,
                        ylabel={$F(t)$},
                        enlarge x limits=false,
                        enlarge y limits=false,
                    ]
                    \addplot+ [
                        mark=none,
                        thick,
                    ]
                    table
                    {results/benchmarks/miyagi/dat/miyagi_laser.dat};
                \end{axis}
            \end{tikzpicture}
            \caption{In this figure we have plotted the laser pulse from
            \autoref{eq:miyagi-laser} for $T = \SI{331}{\text{a.u.}}$,
            $\omega = \SI{0.057}{\text{a.u.}}$, and $F_0 =
            \SI{0.0755}{\text{a.u.}}$ \cite{miyagi_and_madsen}.}
            \label{fig:miyagi-laser}
        \end{figure}


        In order for us to observe actual ionization we need to have a basis
        containing scattered states, that is, states that have an eigenenergy
        above the potential well such that we get a non-zero portion of the wave
        function outside the potential well.
        If the basis set is too small all states will be bound in the atomic
        potential and it is not possible to make a linear combination where
        some parts of the total wave function is outside the potential well
        more than a small tunneling effect.
        In \autoref{fig:spf-potential-miyagi} we see the highest lying
        orbitals we've used in the one-dimensional Beryllium system.
        The plot demonstrates that we need a large basis in order to capture
        scattered states.

        \begin{figure}
            \centering
            \begin{tikzpicture}
                \begin{axis}[
                    width=11cm,
                    height=8cm,
                    xlabel={$x$ $[\text{a.u.}]$},
                    ylabel={$E$ $[\text{a.u.}]$},
                    xmin=-100,
                    xmax=100,
                    ymin=-0.25,
                    legend pos=south west,
                    grid=major,
                ]
                    \addplot+[
                        mark=none,
                        dashed,
                    ]
                    table
                    {results/benchmarks/miyagi/dat/miyagi_potential.dat};

                    \addplot+[
                        mark=none,
                        thick,
                    ]
                    table
                    {results/benchmarks/miyagi/dat/spf_l=10_sq.dat};
                    \addlegendentry{$\abs{\psi_{10}(x)}^2$};

                    \addplot+[
                        mark=none,
                        thick,
                    ]
                    table
                    {results/benchmarks/miyagi/dat/spf_l=15_sq.dat};
                    \addlegendentry{$\abs{\psi_{15}(x)}^2$};

                    \addplot+[
                        mark=none,
                        thick,
                    ]
                    table
                    {results/benchmarks/miyagi/dat/spf_l=20_sq.dat};
                    \addlegendentry{$\abs{\psi_{20}(x)}^2$};
                \end{axis}
            \end{tikzpicture}
            \caption{In this figure we've plotted some of the high lying
            single-particle functions squared and scaled by their
            eigenenergy inside the one-dimensional Beryllium potential.
            We've only included a select few single-particle functions to
            avoid the figure being too cluttered.}
            \label{fig:spf-potential-miyagi}
        \end{figure}

        We can see this from \autoref{fig:one-particle-density-miyagi}
        where a basis of $l = 20$ spin-orbitals has been used.
        This basis is not able to capture much of the scattered states and
        looking at the figure for $t = T$ we see that the system is
        relatively confined still.
        We can also see this from \autoref{fig:spf-potential-miyagi} as $l =
        20$ means the highest single-particle function used is
        $\psi_{10}(x)$ which is clearly confined in the potential well.
        However, looking at \autoref{fig:one-particle-density-miyagi-l=40}
        we see the same Beryllium system with a basis set of $l = 40$
        spin-orbitals.
        It is apparent that the system undergoes a larger degree of
        ionization with more of the one-particle density further away from
        the central potential.
        This is also see in \autoref{fig:spf-potential-miyagi} as $l = 40$
        includes the state $\psi_{20}(x)$ which is more dispersed.

        \begin{figure}
            \centering
            \begin{tikzpicture}
                \begin{groupplot}[
                        group style={
                            group size=1 by 3,
                            vertical sep=30pt,
                            xlabels at=edge bottom,
                            xticklabels at=edge bottom,
                            ylabels at=edge left,
                            vertical sep=10pt,
                        },
                        width=11cm,
                        height=6cm,
                        xlabel={$x$ $[\text{a.u.}]$},
                        ylabel={$\rho(x, t)$ $[\text{a.u.}]$},
                        xmin=-100,
                        xmax=100,
                        ymin=1e-4,
                    ]
                    \nextgroupplot[
                            grid=major,
                            ymode=log,
                        ]
                        \addplot+ [
                            mark=none,
                            thick,
                        ]
                        table
                        {results/benchmarks/miyagi/dat/rho_tdhf_start_real.dat};
                        \addlegendentry{TDHF}

                        \addplot+ [
                            mark=none,
                            thick,
                        ]
                        table
                        {results/benchmarks/miyagi/dat/rho_oatdccd_start_real.dat};
                        \addlegendentry{OATDCCD}

                        \node[anchor=north west] at (rel axis cs:0,1)
                        {$t = 0$};
                    \nextgroupplot[
                            grid=major,
                            ymode=log,
                        ]
                        \addplot+ [
                            mark=none,
                            thick,
                        ]
                        table
                        {results/benchmarks/miyagi/dat/rho_tdhf_half_real.dat};
                        \addlegendentry{TDHF}

                        \addplot+ [
                            mark=none,
                            thick,
                        ]
                        table
                        {results/benchmarks/miyagi/dat/rho_oatdccd_half_real.dat};
                        \addlegendentry{OATDCCD}

                        \node[anchor=north west] at (rel axis cs:0,1)
                        {$t = T/2$};
                    \nextgroupplot[
                            grid=major,
                            ymode=log,
                        ]
                        \addplot+ [
                            mark=none,
                            thick,
                        ]
                        table
                        {results/benchmarks/miyagi/dat/rho_tdhf_end_real.dat};
                        \addlegendentry{TDHF}

                        \addplot+ [
                            mark=none,
                            thick,
                        ]
                        table
                        {results/benchmarks/miyagi/dat/rho_oatdccd_end_real.dat};
                        \addlegendentry{OATDCCD}

                        \node[anchor=north west] at (rel axis cs:0,1)
                        {$t = T$};
                \end{groupplot}
            \end{tikzpicture}
            \caption{In these figures we have included the one-particle
            density for the initial ground state, after half the simulation
            has been run and for the final state of the one-dimensional
            Beryllium system.
            For these figures we have $l = 20$ spin-orbitals, that is, $l /
            2 = 10$ orbitals.
            The $y$-axis has been truncated at $y = \num{1e-4}$ to avoid
            including noise in the figures.}
            \label{fig:one-particle-density-miyagi}
        \end{figure}


        \begin{figure}
            \centering
            \begin{tikzpicture}
                \begin{groupplot}[
                        group style={
                            group size=1 by 3,
                            vertical sep=30pt,
                            xlabels at=edge bottom,
                            xticklabels at=edge bottom,
                            ylabels at=edge left,
                            vertical sep=10pt,
                        },
                        width=11cm,
                        height=6cm,
                        xlabel={$x$ $[\text{a.u.}]$},
                        ylabel={$\rho(x, t)$ $[\text{a.u.}]$},
                        xmin=-100,
                        xmax=100,
                        ymin=1e-4,
                    ]
                    \nextgroupplot[
                            grid=major,
                            ymode=log,
                        ]
                        \addplot+ [
                            mark=none,
                            thick,
                        ]
                        table
                        {results/benchmarks/miyagi/dat/rho_tdhf_start_real_l=40.dat};
                        \addlegendentry{TDHF}

                        \addplot+ [
                            mark=none,
                            thick,
                        ]
                        table
                        {results/benchmarks/miyagi/dat/rho_oatdccd_start_real_l=40.dat};
                        \addlegendentry{OATDCCD}

                        \node[anchor=north west] at (rel axis cs:0,1)
                        {$t = 0$};
                    \nextgroupplot[
                            grid=major,
                            ymode=log,
                        ]
                        \addplot+ [
                            mark=none,
                            thick,
                        ]
                        table
                        {results/benchmarks/miyagi/dat/rho_tdhf_half_real_l=40.dat};
                        \addlegendentry{TDHF}

                        \addplot+ [
                            mark=none,
                            thick,
                        ]
                        table
                        {results/benchmarks/miyagi/dat/rho_oatdccd_half_real_l=40.dat};
                        \addlegendentry{OATDCCD}

                        \node[anchor=north west] at (rel axis cs:0,1)
                        {$t = T/2$};
                    \nextgroupplot[
                            grid=major,
                            ymode=log,
                        ]
                        \addplot+ [
                            mark=none,
                            thick,
                        ]
                        table
                        {results/benchmarks/miyagi/dat/rho_tdhf_end_real_l=40.dat};
                        \addlegendentry{TDHF}

                        \addplot+ [
                            mark=none,
                            thick,
                        ]
                        table
                        {results/benchmarks/miyagi/dat/rho_oatdccd_end_real_l=40.dat};
                        \addlegendentry{OATDCCD}

                        \node[anchor=north west] at (rel axis cs:0,1)
                        {$t = T$};
                \end{groupplot}
            \end{tikzpicture}
            \caption{These figures show the same situation as in
            \autoref{fig:one-particle-density-miyagi} but for $l = 40$
            spin-orbitals.
            We have truncated the $y$-axis to $y = \num{1e-4}$ as lower
            values include more noise and provides no more insight.}
            \label{fig:one-particle-density-miyagi-l=40}
        \end{figure}

    \section{Electronic spectra of molecules}
        In an article by \citeauthor{nest} \cite{nest} the
        multiconfiguration time-dependent Hartree-Fock method was tested on
        a system of lithium hydride and methane subject to spatially uniform
        laser field on the form\footnote{%
            Note that \citeauthor{nest} \cite{nest} does not explicitly add
            a formula for the laser, but based on their description we infer
            the form of the laser.
        }
        \begin{align}
            \vfg{E}(t)
            = -\vfg{\epsilon} E(t)
            \sin[2](\pi t / T)
            = -\vfg{\epsilon} E_0 \sin(\omega t)
            \sin[2](\pi t / T),
            \label{eq:nest-laser}
        \end{align}
        where $\vfg{\epsilon}$ is the polarization vector, $\omega$ is the
        frequency corresponding to a photon energy of $E_h =
        \SI{5.44}{\electronvolt}$.
        The laser field is active for $t \in \brak{0, T}$ where $T =
        \SI{1}{\femto\second} \approx \SI{41.3}{\planckbar / \hartree}$.
        The intensity of the laser field is chosen to be $I =
        \SI{3.5e12}{\watt/\cm^2}$.
        The simulation is run for $\SI{100}{\femto\second}$.
        A plot of the laser field in the first $t = \SI{50}{\planckbar /
        \hartree}$ of the simulation is shown in \autoref{fig:nest-laser}.
        \begin{figure}
            \centering
            \begin{tikzpicture}
                \begin{axis}[
                        width=11cm,
                        height=6cm,
                        xlabel={$t$ $[\si{\planckbar / \hartree}]$},
                        grid=major,
                        ylabel={
                            $E(t)$ $[
                                \si{\hartree / (\elementarycharge \bohr)}
                            ]$
                        },
                        enlarge x limits=false,
                    ]
                    \addplot+ [
                        mark=none,
                        thick,
                    ]
                    table
                    {results/benchmarks/nest/dat/nest_laser.dat};
                \end{axis}
            \end{tikzpicture}
            \caption{In this figure we have plotted the laser pulse from
            \autoref{eq:nest-laser} for $t = \SI{50}{\text{a.u.}}$, with
            parameters as defined in the text \cite{nest}.}
            \label{fig:nest-laser}
        \end{figure}
        We repeat this study for the lithium hydride molecule.
        We use a bond-length of $\SI{3.08}{\bohr}$ in the $z$-direction
        between the lithium and the hydrogen atoms.
        We use the basis 6-31G*.
        Finally, we run two simulations by setting the polarization vector
        in $z$ and $x$ direction and measure the dipole moment in the same
        direction as the polarization of the laser field.
        A word of caution, when reproducing the results in this article some
        care must be taken when converting to atomic units for the
        frequency, time and the intensity of the laser as it is not clear
        how much one should round the numbers.
        We have used the Python library \citetitle{pint} \cite{pint} to
        convert from SI units to atomic units as they stand in the text
        without any rounding.
        These conversions are summarized in \autoref{tab:nest-conversions}.
        \begin{table}
            \centering
            \caption{In this table we show the conversions used to move from
            the SI units given in the article by \citeauthor{nest}
            \cite{nest} to the atomic units used in the simulation.}
            \renewcommand{\arraystretch}{1.3}
            \begin{tabular}{@{}lll@{}}
                \toprule
                Quantity & SI units & Atomic units \\
                \midrule
                $\omega$ & $\SI{5.44}{\electronvolt}$ &
                \SI{1.999163e-1}{\text{a.u.}}\\
                $F(t)$ & $\SI{3.5e12}{\watt/\cm^{2}}$ &
                \SI{9.973085e-5}{\text{a.u.}} \\
                $t$ & $\SI{1}{\femto\second}$ &
                \SI{4.134137e+1}{\text{a.u.}}
                \\
                \bottomrule
            \end{tabular}
            \label{tab:nest-conversions}
        \end{table}
        % TODO: Consider rounding these numbers.

        In \autoref{fig:fft_dipole_nest} we have included the Fourier
        transform of the sampled dipole moment in $x$- and $z$-direction in
        time, respectively.
        These plots show the emission and absorption spectra for the
        lithium hydride molecule.
        \begin{figure}
            \centering
            \begin{tikzpicture}
                \begin{groupplot}[
                    group style={
                        group size=1 by 2,
                        vertical sep=60pt,
                        xlabels at=edge bottom,
                    },
                    width=11cm,
                    height=6cm,
                    xlabel={Excitation energy $[\text{a.u.}]$},
                ]
                    \nextgroupplot[
                        title={$FT\brak{\expval{x(t)}}$},
                        grid=major,
                        ylabel={Intensity},
                        enlarge x limits=false,
                        %ymax=0.5e-4,
                    ]
                        \addplot+[
                            mark=none,
                            thick,
                        ]
                        table
                        {results/benchmarks/nest/dat/fft_dipole_x_oatdccd_real.dat};
                        \addlegendentry{OATDCCD}

                        \addplot+[
                            mark=none,
                            thick,
                        ]
                        table
                        {results/benchmarks/nest/dat/fft_dipole_x_tdhf_real.dat};
                        \addlegendentry{TDHF}

                        \addplot+[
                            mark=none,
                            thick,
                        ]
                        table
                        {results/benchmarks/nest/dat/fft_dipole_x_tdccsd_real.dat};
                        \addlegendentry{TDCCSD}
                    \nextgroupplot[
                        title={$FT\brak{\expval{z(t)}}$},
                        ylabel={Intensity},
                        grid=major,
                        enlarge x limits=false,
                        %ymax=0.5e-4,
                    ]
                        \addplot+[
                            mark=none,
                            thick,
                        ]
                        table
                        {results/benchmarks/nest/dat/fft_dipole_z_oatdccd_real.dat};
                        \addlegendentry{OATDCCD}

                        \addplot+[
                            mark=none,
                            thick,
                        ]
                        table
                        {results/benchmarks/nest/dat/fft_dipole_z_tdhf_real.dat};
                        \addlegendentry{TDHF}

                        \addplot+[
                            mark=none,
                            thick,
                        ]
                        table
                        {results/benchmarks/nest/dat/fft_dipole_z_tdccsd_real.dat};
                        \addlegendentry{TDCCSD}
                \end{groupplot}
            \end{tikzpicture}
            \caption{In these figures we show the Fourier transform of the
            dipole moment in $x$-and $y$-direction using time-dependent
            Hartree-Fock, time-dependent coupled cluster with singles and
            doubles, orbital-adaptive time-dependent coupled cluster with
            doubles excitations.
            The top figure shows the results in $x$-direction and the lower
            in $z$-direction.
            Note that we have removed the zero-frequency component from
            these spectra.}
            \label{fig:fft_dipole_nest}
        \end{figure}
        % TODO: Discuss sampling of the dipole moment and taking the Fourier
        % transform in order to find the energy spectra.
        % TODO: Add plot of energy conservation after laser pulse.
        % TODO: Add results, remember to remove the zero-frequency
        % component.
        % TODO: The laser pulse used is wrong! Fix figures and results!

        % TODO: ADD NEW RESULTS!
