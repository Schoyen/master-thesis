\chapter{Benchmarking against litterature}
    In this chapter we use our developed code to reproduce results from relevant
    articles to verify that our implementation works as expected.

    \section{Single quantum dots}
        In this section we'll discuss results for quantum dots subject to a
        harmonic oscillator potential in one and two dimensions.
        These systems are rather well studied and therefore provides ample
        results which can be used as verification.

        \subsection{The one-dimensional harmonic oscillator}
            An excellent starting point is the one-dimensional harmonic
            oscillator as discussed in \autoref{subsec:one-dim-ho}.
            A study done by \citeauthor{zanghellini_2004}
            \cite{zanghellini_2004, skattum2013time, kristiansen2017time}
            explores the multi-configuration time-dependent Hartree-Fock method
            on the one-dimensional harmonic oscillator system for two particles
            compared to a semi-analytic result\footnote{Semi-analytic as the
            time-evolution is solved numerically.}.
            In this study the Hamiltonian is given by
            \begin{align}
                \hamil
                &=
                \sum_{i = 1}^{N}
                \para{
                    \frac{\momentum^2_i}{2m}
                    + \half m \omega^2 \position^2_i
                    + \mathcal{E}_0 \sin(\Omega t)\position_i
                }
                \nonumber \\
                &\qquad
                + \sum_{i, j = 1}^{N}
                \frac{1}{4\epsilon_0\pi}
                \frac{e^2}{\sqrt{(\position_i - \position_j)^2 + a^2}}.
            \end{align}
            However, \citeauthor{zanghellini_2004} uses the more convenient set
            of Hartree atomic units where $m = e = \hslash = k_e =
            \para{4\epsilon_0\pi}^{-1} = 1$.
            The term $a$ is a smoothing parameter included to remove the
            singularity when we integrate over all space and the two positions
            $x_i$ and $x_j$ overlap. \cite{suq, zanghellini_2004}
            In the simulation done by \citeauthor{zanghellini_2004} they set $a
            = 0.25$, $\omega = 0.25$, $\mathcal{E}_0 = 1$ and $\Omega = 8\omega$
            on a grid where $x \in [-10, 10]$.
            The laser pulse is turned on for the entirety of the simulation.

            In \autoref{fig:one-body-density-zanghellini} we plot the ground
            state one-body particle densities using Hartree-Fock and
            configuration interaction with singles and doubles, i.e., full
            configuration interaction for two particles.
            In the lower plot in \autoref{fig:one-body-density-zanghellini} we
            demonstrate that the coupled cluster method with singles and double
            reproduces the exact solution as achieved from configuration
            interaction to a high precision.
            Furthermore, in \autoref{fig:overlap-zanghellini}, we show the
            time-dependent overlap for two one-dimensional quantum dots subject
            to the same dipole laser as in the study done by
            \citeauthor{zanghellini_2004} \cite{zanghellini_2004}.
            We also show how the coupled cluster method with singles and doubles
            reproduces the exact solution to a large degree.

            It is worth re-iterating that we have not included a comparison
            using the orbital-adaptive time-dependent coupled cluster method due
            to the frustrating fact that the orbital rotations induced by this
            method makes the process of computing the time-dependent overlap a
            challenging task and has therefore been treated as an out-of-scope
            task for this thesis.

            \begin{figure}
                \centering
                \begin{tikzpicture}
                    \begin{groupplot}[
                            group style={
                                group size=1 by 2,
                                vertical sep=30pt,
                                xlabels at=edge bottom,
                            },
                            width=11cm,
                            height=6cm,
                            xlabel={$x$ $[\text{a.u.}]$},
                        ]
                        \nextgroupplot[
                                grid=major,
                                ylabel={$\densityten(x, 0)$},
                                restrict x to domain=-6:6,
                                enlarge x limits=false,
                                ymin=0,
                                ymax=0.4,
                                enlarge y limits=false,
                            ]
                            \addplot+ [
                                mark=none,
                                thick,
                            ]
                            table
                            {results/benchmarks/zanghellini/dat/rho_tdcisd_real.dat};
                            \addlegendentry{TDCISD}

                            \addplot+ [
                                mark=none,
                                thick,
                                dashed,
                            ]
                            table
                            {results/benchmarks/zanghellini/dat/rho_tdhf_real.dat};
                            \addlegendentry{TDHF}
                        \nextgroupplot[
                                ymode=log,
                                grid=major,
                                ylabel={$\abs{\Delta\densityten(x, 0)}$},
                                restrict x to domain=-6:6,
                                enlarge x limits=false,
                            ]
                            \addplot+ [
                                mark=none,
                                thick,
                            ]
                            table
                            {results/benchmarks/zanghellini/dat/rho_diff_tdccsd_tdcisd.dat};
                    \end{groupplot}
                \end{tikzpicture}
                \caption{In the top figure we have reproduced Figure 1. in the
                study done by \citeauthor{zanghellini_2004}
                \cite{zanghellini_2004}, that is, we have plotted the one-body
                particle density two one-dimensional quantum dots in an harmonic
                oscillator trap using full-configuration interaction and a
                general Hartree-Fock solver.
                In the lower figure we have plotted the absolute difference
                between the one-body particle densities from the
                full-configuration interaction and coupled cluster with singles
                and doubles methods.}
                \label{fig:one-body-density-zanghellini}
             \end{figure}

            \begin{figure}
                \centering
                \begin{tikzpicture}
                    \begin{groupplot}[
                            group style={
                                group size=1 by 2,
                                vertical sep=30pt,
                                xlabels at=edge bottom,
                            },
                            width=11cm,
                            height=6cm,
                            xlabel={$\omega t / (2\pi)$},
                        ]
                        \nextgroupplot[
                                grid=major,
                                ylabel={$\abs{\braket*{\Psi(t)}{\Psi(0)}}^2$},
                                enlarge x limits=false,
                            ]
                            \addplot+ [
                                mark=none,
                                thick,
                            ]
                            table
                            {results/benchmarks/zanghellini/dat/overlap_tdcisd_real.dat};
                            \addlegendentry{TDCISD}

                            \addplot+ [
                                mark=none,
                                thick,
                                dashed,
                            ]
                            table
                            {results/benchmarks/zanghellini/dat/overlap_tdhf_real.dat};
                            \addlegendentry{TDHF}
                        \nextgroupplot[
                                ymode=log,
                                grid=major,
                                ylabel={$\Delta\abs{\braket*{\Psi(t)}{\Psi(0)}}^2$},
                                restrict x to domain=-6:6,
                                enlarge x limits=false,
                            ]
                            \addplot+ [
                                mark=none,
                                thick,
                            ]
                            table
                            {results/benchmarks/zanghellini/dat/overlap_diff_tdccsd_tdcisd.dat};
                    \end{groupplot}
                \end{tikzpicture}
                \caption{In the top figure we have plotted the time-dependent
                overlap between the initial ground state $\ket{\Psi(0)}$ and the
                state $\ket{\Psi(t)}$ at a later time.
                We compare the time-dependent general Hartree-Fock method and
                the time-dependent full configuration interaction method.
                The figure is a reproduction of Figure 2. in the study done by
                \citeauthor{zanghellini_2004} \cite{zanghellini_2004}.
                In the lower figure we show the absolute difference in the
                overlap between the time-dependent full configuration
                interaction method the time-dependent coupled cluster method
                with singles and doubles.}
                \label{fig:overlap-zanghellini}
             \end{figure}

    \section{Double-well quantum dots}

    \section{Atoms and molecules}
        In this section we'll look at how our methods perform when applied to
        atomic and molecular systems.
        Building atomic and molecular systems is a large field in itself and we
        will not make our own basis sets as there are already excellent
        libraries available.
        We'll mainly use two libraries throughout this thesis for creating
        atomic systems and that is PySCF \cite{pyscf} and Psi4 \cite{psi4}.

        \subsection{Two-electron atoms}
            In a study on the Hydrogen molecule subject to an intense laser
            field done by \citeauthor{li_2005} \cite{li_2005} a comparison of
            the time-dependent Hartree-Fock method and the time-dependent full
            configuration interaction\footnote{%
                Note that \citeauthor{li_2005} calls the time-dependent
                full-configuration interaction method for the time-dependent
                Schrödinger equation.%
            } is performed.
            This provides us with an ample opportunity to repeat the
            experiments in order for us to verify our methods.
            Particularly, we are in a position to test the orbital-adaptive
            time-dependent coupled cluster method as the dipole moment requires
            us to do a basis transformation to the new orbital basis and compute
            the one-body density matrix.
            That is, we compute
            \begin{align}
                \tilde{\vfg{d}}^{p}_{q}(t)
                = \mel*{\tilde{\phi}_p(t)}{\hat{\vfg{d}}}{\phi_q(t)}
                = \tilde{C}_{p\alpha}(t)
                \mel*{\chi_{\alpha}}{\hat{\vfg{d}}}{\chi_{\beta}}
                C_{\beta q}(t),
            \end{align}
            as the basis transformed dipole matrix elements.
            Having compute the one-body density matrix $\densityten^{q}_{p}(t)$
            we can find the induced dipole-moment along the $z$-axis by picking
            out the $z$-component of the dipole elements, that is,
            \begin{align}
                \expv{z(t)}
                &= \densityten^{q}_{p}(t) \tilde{z}^{p}_{q}(t),
            \end{align}
            where $\tilde{z}^{p}_{q}(t)$ is the $z$-component of
            $\tilde{\vfg{d}}^{p}_{q}(t)$.
            The Hydrogen molecule is expressed in $14$ Gaussian type orbitals,
            the $6-311++G(d, p)$ basis set, with an equilibrium geometry of $R_e
            = \SI{0.7354}{\angstrom} \approx \SI{1.3897}{\bohr}$.
            We center the molecule around the origin with each atom centered at
            $\pm R_e/2$.
            We then make the basis set doubly occupied by including spin.
            The basis is run through a general Hartree-Fock solver which finds
            the Hartree-Fock molecular orbitals and changes to this basis.
            The laser field used by \citeauthor{li_2005} is
            \begin{align}
                \hat{\vfg{d}} \cdot \vfg{f}(t)
                = \hat{\vfg{d}} \cdot \vfg{E}(t)\sin(\omega t),
            \end{align}
            where $\hat{\vfg{d}}$ is the dipole operator in three dimensions and
            the envelope $\vfg{E}(t) = E(t) \cdot \vfg{\epsilon}$ with
            $\vfg{\epsilon}$ as the polarization vector.
            We let the envelope cycle as a function of time by
            \begin{align}
                E(t) = \begin{cases}
                    (\omega t / 2\pi) E_m, & t\omega \in [0, 2\pi], \\
                    E_m, & t\omega \in [2\pi, 4\pi], \\
                    [3 - \omega t / (2\pi)], & t\omega \in [4\pi, 6\pi], \\
                    0, & t\omega \neq [0, 6\pi],
                \end{cases}
            \end{align}
            where we set $E_m = \SI{0.07}{\text{a.u.}}$ and $\omega =
            \SI{0.1}{\text{a.u.}}$.
            In our programs the dipole moment is defined with a positive sign,
            which means that we need to introduce a negative sign in the
            envelope or the polarization vector to include the negative charge
            of the electrons.
            We have chosen to set the polarization vector along the negative
            $z$-direction.
            A plot of the induced dipole moment in the $z$-direction where we
            simulate for a total of $t_f = \SI{225}{\text{a.u.}}$ with the laser
            turned on from start until $2\pi/\omega \approx
            \SI{62.8}{\text{a.u.}}$ is shown in \autoref{fig:dipole-moment-li}.
            In the lower figure we show the absolute error in the induced dipole
            moment as calculated by the time-dependent coupled cluster singles
            and doubles method and the orbital-adaptive time-dependent coupled
            cluster doubles method compared with the exact full configuration
            interaction solution.

            It is interesting to note in the top figure in
            \autoref{fig:dipole-moment-li} that the time-dependent Hartree-Fock
            method performs surprisingly well.
            There are small discrepancies, but at the intensity we use for our
            laser the mean-field approximation performs excellently.
            However, we expect that as the intensity increases, this method will
            prove far more inferior to the correlated methods.
            % TODO: Back this up.

            \begin{figure}
                \centering
                \begin{tikzpicture}
                    \begin{groupplot}[
                            group style={
                                group size=1 by 2,
                                vertical sep=30pt,
                                xlabels at=edge bottom,
                            },
                            width=11cm,
                            height=6cm,
                            xlabel={$t$ $[\text{a.u.}]$},
                        ]
                        \nextgroupplot[
                                grid=major,
                                ylabel={$\expv{z(t)}$},
                                enlarge x limits=false,
                                enlarge y limits=false,
                            ]
                            \addplot+ [
                                mark=none,
                                thick,
                            ]
                            table
                            {results/benchmarks/li/dat/dipole_z_tdcisd_real.dat};
                            \addlegendentry{TDCISD}

                            \addplot+ [
                                mark=none,
                                thick,
                            ]
                            table
                            {results/benchmarks/li/dat/dipole_z_tdhf_real.dat};
                            \addlegendentry{TDHF}

                        \nextgroupplot[
                                ymode=log,
                                grid=major,
                                ylabel={$\abs{\Delta\expv{z(t)}}$},
                                enlarge x limits=false,
                                enlarge y limits=false,
                            ]
                            \addplot+ [
                                mark=none,
                                thick,
                            ]
                            table
                            {results/benchmarks/li/dat/dipole_z_diff_tdccsd_tdcisd.dat};
                            \addlegendentry{TDCCSD}

                            \addplot+ [
                                mark=none,
                                thick,
                            ]
                            table
                            {results/benchmarks/li/dat/dipole_z_diff_oatdccd_tdcisd.dat};
                            \addlegendentry{OATDCCD}
                    \end{groupplot}
                \end{tikzpicture}
                \caption{In the top figure we have plotted the induced dipole
                moment by radiating a Hydrogen molecule with a dipole laser.
                The figure reproduces Figure. 4 by \citeauthor{li_2005}
                \cite{li_2005} using both the time-dependent general
                Hartree-Fock solver and the time-dependent full
                configuration-interation solver.}
                \label{fig:dipole-moment-li}
            \end{figure}

        \subsection{Ionization of one-dimensional atoms}
            The process of modelling ionization of electrons is atoms and
            molecules is a tricky subject.
            This is because we are forcing our systems to be bound by simulating
            a restricted space and requiring that all particles are contained
            inside the simulation box.
            As a consequence, ionziation has to be inferred in some way or
            another, e.g., an absorbing potential \cite{kosloff1986363,
            miyagi_and_madsen}.
            However, in our formalizm we use static atomic orbitals which we
            force to zero at the boundaries of the simulation box.
            This makes the use of an absorbing potential useless and ionization
            must be inferred in some other way.

            In a study done by \citeauthor{miyagi_and_madsen}
            \cite{miyagi_and_madsen} they explored the dynamics of
            one-dimensional atoms subject to a dipole laser.
            This study provides us with an excellent benchmark to observe a form
            of ionization and see if this is reproducible in our formalism.
            Note that \citeauthor{miyagi_and_madsen} used a
            discrete-variable-representation basis (DVR) which is why they
            include an absorbing potential.
            Our solution differs from this as we use the static one-dimensional
            quantum dot basis with a one-dimensional atom potential.
            In a position basis with Hartree atomic units, the one-body
            Hamiltonian is given by \cite{miyagi_and_madsen}
            \begin{align}
                \oneten(x, t)
                &= -\half \dod[2]{}{x}
                - \frac{Z}{\sqrt{x^2 + 1}}
                + xF(t)
                - iW(x),
            \end{align}
            where $Z = N_e$, that is, the number of electrons in the system,
            $W(x)$ is the absorbing potential, which we'll set to zero.
            The second term gives rise to naming these systems as
            one-dimensional atoms as this term serves as the electron-nuclear
            interaction in the atomic Hamiltonian, but with the
            three-dimensional position replaced with $x$.
            One of the interesting aspects of the one-dimensional atoms is that
            many open-shell systems such as Carbon gets closed shells in the
            one-dimensional case.
            We will however, limit our study to the one-dimensional Beryllium
            system with $n = 4$ electrons in a closed shell.

            The laser pulse $F(t)$ is given by \cite{miyagi_and_madsen}
            \begin{align}
                F(t)
                &= -\dod[]{A(t)}{t}
                = -\dod[]{}{t}\brak{
                    \frac{F_0}{\omega}
                    \sin^2\para{\frac{\pi t}{T}}
                    \sin(\omega t)
                }
                \\
                &=
                -\sin\para{\frac{\pi t}{T}}\brak{
                    \omega \sin\para{\frac{\pi t}{T}}
                    \cos(\omega t)
                    + \frac{2\pi}{T}
                    \cos\para{\frac{\pi t}{T}}
                    \sin(\omega t)
                }.
                \label{eq:miyagi-laser}
            \end{align}
            where this field is only turned on for $t \in [0, T]$.
            A plot of the laser pulse is shown in \autoref{fig:miyagi-laser}.
            The two-body Hamiltonian in coordinate representation is given by
            \begin{align}
                \twoten(x_i, x_j) = \frac{1}{\sqrt{(x_1 + x_2)^2 + 1}},
            \end{align}
            which corresponds to a shielding parameter of $a = 1$ and $\alpha =
            1$.
            % TODO: Refer to the shielded Coulomb model in the one-dimensional
            % quantum dot.
            To compare with the study by \citeauthor{miyagi_and_madsen} we will
            run for $T = \SI{331}{\text{a.u.}}$ with $\Delta t = \num{1e-2}$
            which corresponds to $n_t = 33100$ time-steps.
            This computation is rather involved and we will therefore limit
            ourselves to the time-dependent Hartree-Fock method and the
            orbital-adaptive time-dependent coupled cluster method as the
            full-configuration interaction method becomes quite expensive.
            In fact, for $n = 4$ and $l = 40$ we have to create a Hamiltonian
            matrix with $91390^2$ elements, each of size $\SI{16}{\text{B}}$ as
            we use complex numbers.
            To solve this system we require a much smarter implementation of the
            configuration interaction method which is out of scope for this
            thesis.

            \begin{figure}
                \centering
                \begin{tikzpicture}
                    \begin{axis}[
                            width=11cm,
                            height=6cm,
                            xlabel={$t$ $[\text{a.u.}]$},
                            grid=major,
                            ylabel={$F(t)$},
                            enlarge x limits=false,
                            enlarge y limits=false,
                        ]
                        \addplot+ [
                            mark=none,
                            thick,
                        ]
                        table
                        {results/benchmarks/miyagi/dat/miyagi_laser.dat};
                    \end{axis}
                \end{tikzpicture}
                \caption{In this figure we have plotted the laser pulse from
                \autoref{eq:miyagi-laser} for $T = \SI{331}{\text{a.u.}}$,
                $\omega = \SI{0.057}{\text{a.u.}}$, and $F_0 =
                \SI{0.0755}{\text{a.u.}}$ \cite{miyagi_and_madsen}.}
                \label{fig:miyagi-laser}
            \end{figure}


            In order for us to observe ionization we need to have a basis
            containing scattered states, that is, states that have an eigenenegy
            above the potential well, such that we get a non-zero portion of the
            wave function outside the potential well.
            % TODO: This probably needs to be rephrased
            If the basis set is too small all states will be bound in the atomic
            potential and it is not possible to make a linear combination where
            some parts of the total wave function is outside the potential well
            more than a small tunneling effect.
            % TODO: Include this study
            In \autoref{fig:spf-potential-miyagi} we see the highest lying
            orbitals we've used in the one-dimensional Beryllium system.
            The plot demonstrates that we need a large basis in order to capture
            scattered states.

            \begin{figure}
                \centering
                \begin{tikzpicture}
                    \begin{axis}[
                        width=11cm,
                        height=8cm,
                        xlabel={$x$ $[\text{a.u.}]$},
                        ylabel={$E$ $[\text{a.u.}]$},
                        xmin=-100,
                        xmax=100,
                        ymin=-0.25,
                        legend pos=south west,
                    ]
                        \addplot+[
                            mark=none,
                            dashed,
                        ]
                        table
                        {results/benchmarks/miyagi/dat/miyagi_potential.dat};

                        \addplot+[
                            mark=none,
                            thick,
                        ]
                        table
                        {results/benchmarks/miyagi/dat/spf_l=10_sq.dat};
                        \addlegendentry{$\abs{\psi_{10}(x)}^2$};

                        \addplot+[
                            mark=none,
                            thick,
                        ]
                        table
                        {results/benchmarks/miyagi/dat/spf_l=15_sq.dat};
                        \addlegendentry{$\abs{\psi_{15}(x)}^2$};

                        \addplot+[
                            mark=none,
                            thick,
                        ]
                        table
                        {results/benchmarks/miyagi/dat/spf_l=20_sq.dat};
                        \addlegendentry{$\abs{\psi_{20}(x)}^2$};
                    \end{axis}
                \end{tikzpicture}
                \caption{In this figure we've plotted some of the high lying
                single-particle functions absolute squared and scaled by their
                eigenenergy inside the one-dimensional Beryllium potential.
                We've only included a select few single-particle functions to
                avoid the figure being too cluttered.}
                \label{fig:spf-potential-miyagi}
            \end{figure}

            We can see this from \autoref{fig:one-particle-density-miyagi}
            where a basis of $l = 20$ spin-orbitals has been used.
            This basis is not able to capture much of the scattered states and
            looking at the figure for $t = T$ we see that the system is
            relatively confined still.
            We can also see this from \autoref{fig:spf-potential-miyagi} as $l =
            20$ means the highest single-particle function used is
            $\psi_{10}(x)$ which is clearly confined in the potential well.
            However, looking at \autoref{fig:one-particle-density-miyagi-l=40}
            we see the same Beryllium system with a basis set of $l = 40$
            spin-orbitals.
            It is apparent that the system undergoes a larger degree of
            ionization with more of the one-particle density further away from
            the central potential.
            This is also see in \autoref{fig:spf-potential-miyagi} as $l = 40$
            includes the state $\psi_{20}(x)$ which is more dispersed.

            \begin{figure}
                \centering
                \begin{tikzpicture}
                    \begin{groupplot}[
                            group style={
                                group size=1 by 3,
                                vertical sep=30pt,
                                xlabels at=edge bottom,
                                xticklabels at=edge bottom,
                                ylabels at=edge left,
                                vertical sep=10pt,
                            },
                            width=11cm,
                            height=6cm,
                            xlabel={$x$ $[\text{a.u.}]$},
                            ylabel={$\rho(x, t)$ $[\text{a.u.}]$},
                            xmin=-100,
                            xmax=100,
                            ymin=1e-4,
                        ]
                        \nextgroupplot[
                                grid=major,
                                ymode=log,
                            ]
                            \addplot+ [
                                mark=none,
                                thick,
                            ]
                            table
                            {results/benchmarks/miyagi/dat/rho_tdhf_start_real.dat};
                            \addlegendentry{TDHF}

                            \addplot+ [
                                mark=none,
                                thick,
                            ]
                            table
                            {results/benchmarks/miyagi/dat/rho_oatdccd_start_real.dat};
                            \addlegendentry{OATDCCD}

                            \node[anchor=north west] at (rel axis cs:0,1)
                            {$t = 0$};
                        \nextgroupplot[
                                grid=major,
                                ymode=log,
                            ]
                            \addplot+ [
                                mark=none,
                                thick,
                            ]
                            table
                            {results/benchmarks/miyagi/dat/rho_tdhf_half_real.dat};
                            \addlegendentry{TDHF}

                            \addplot+ [
                                mark=none,
                                thick,
                            ]
                            table
                            {results/benchmarks/miyagi/dat/rho_oatdccd_half_real.dat};
                            \addlegendentry{OATDCCD}

                            \node[anchor=north west] at (rel axis cs:0,1)
                            {$t = T/2$};
                        \nextgroupplot[
                                grid=major,
                                ymode=log,
                            ]
                            \addplot+ [
                                mark=none,
                                thick,
                            ]
                            table
                            {results/benchmarks/miyagi/dat/rho_tdhf_end_real.dat};
                            \addlegendentry{TDHF}

                            \addplot+ [
                                mark=none,
                                thick,
                            ]
                            table
                            {results/benchmarks/miyagi/dat/rho_oatdccd_end_real.dat};
                            \addlegendentry{OATDCCD}

                            \node[anchor=north west] at (rel axis cs:0,1)
                            {$t = T$};
                    \end{groupplot}
                \end{tikzpicture}
                \caption{In these figures we have included the one-particle
                density for the initial ground state, after half the simulation
                has been run and for the final state of the one-dimensional
                Beryllium system.
                For these figures we have $l = 20$ spin-orbitals, that is, $l /
                2 = 10$ orbitals.
                The $y$-axis has been truncated at $y = \num{1e-4}$ to avoid
                including noise in the figures.}
                \label{fig:one-particle-density-miyagi}
            \end{figure}


            \begin{figure}
                \centering
                \begin{tikzpicture}
                    \begin{groupplot}[
                            group style={
                                group size=1 by 3,
                                vertical sep=30pt,
                                xlabels at=edge bottom,
                                xticklabels at=edge bottom,
                                ylabels at=edge left,
                                vertical sep=10pt,
                            },
                            width=11cm,
                            height=6cm,
                            xlabel={$x$ $[\text{a.u.}]$},
                            ylabel={$\rho(x, t)$ $[\text{a.u.}]$},
                            xmin=-100,
                            xmax=100,
                            ymin=1e-4,
                        ]
                        \nextgroupplot[
                                grid=major,
                                ymode=log,
                            ]
                            \addplot+ [
                                mark=none,
                                thick,
                            ]
                            table
                            {results/benchmarks/miyagi/dat/rho_tdhf_start_real_l=40.dat};
                            \addlegendentry{TDHF}

                            \addplot+ [
                                mark=none,
                                thick,
                            ]
                            table
                            {results/benchmarks/miyagi/dat/rho_oatdccd_start_real_l=40.dat};
                            \addlegendentry{OATDCCD}

                            \node[anchor=north west] at (rel axis cs:0,1)
                            {$t = 0$};
                        \nextgroupplot[
                                grid=major,
                                ymode=log,
                            ]
                            \addplot+ [
                                mark=none,
                                thick,
                            ]
                            table
                            {results/benchmarks/miyagi/dat/rho_tdhf_half_real_l=40.dat};
                            \addlegendentry{TDHF}

                            \addplot+ [
                                mark=none,
                                thick,
                            ]
                            table
                            {results/benchmarks/miyagi/dat/rho_oatdccd_half_real_l=40.dat};
                            \addlegendentry{OATDCCD}

                            \node[anchor=north west] at (rel axis cs:0,1)
                            {$t = T/2$};
                        \nextgroupplot[
                                grid=major,
                                ymode=log,
                            ]
                            \addplot+ [
                                mark=none,
                                thick,
                            ]
                            table
                            {results/benchmarks/miyagi/dat/rho_tdhf_end_real_l=40.dat};
                            \addlegendentry{TDHF}

                            \addplot+ [
                                mark=none,
                                thick,
                            ]
                            table
                            {results/benchmarks/miyagi/dat/rho_oatdccd_end_real_l=40.dat};
                            \addlegendentry{OATDCCD}

                            \node[anchor=north west] at (rel axis cs:0,1)
                            {$t = T$};
                    \end{groupplot}
                \end{tikzpicture}
                \caption{These figures show the same situation as in
                \autoref{fig:one-particle-density-miyagi} but for $l = 40$
                spin-orbitals.
                We have truncated the $y$-axis to $y = \num{1e-4}$ as lower
                values include more noise and provides no more insight.}
                \label{fig:one-particle-density-miyagi-l=40}
            \end{figure}
