\chapter{Benchmarking against litterature}
    In this chapter we use our developed code to reproduce results from relevant
    articles to verify that our implementation works as expected.

    \section{Single quantum dots}
        In this section we'll discuss results for quantum dots subject to a
        harmonic oscillator potential in one and two dimensions.
        These systems are rather well studied and therefore provides ample
        results which can be used as verification.

        \subsection{The one-dimensional harmonic oscillator}
            An excellent starting point is the one-dimensional harmonic
            oscillator as discussed in \autoref{subsec:one-dim-ho}.
            A study done by \citeauthor{zanghellini_2004}
            \cite{zanghellini_2004} explores the multi-configuration
            time-dependent Hartree-Fock method on the one-dimensional harmonic
            oscillator system for two particles compared to a semi-analytic
            result\footnote{Semi-analytic as the time-evolution is solved
            numerically.}.
            In this study the Hamiltonian is given by
            \begin{align}
                \hamil
                &=
                \sum_{i = 1}^{N}
                \para{
                    \frac{\momentum^2_i}{2m}
                    + \half m \omega^2 \position^2_i
                    + \mathcal{E}_0 \sin(\Omega t)\position_i
                }
                \nonumber \\
                &\qquad
                + \sum_{i, j = 1}^{N}
                \frac{1}{4\epsilon_0\pi}
                \frac{e^2}{\sqrt{(\position_i - \position_j)^2 + a^2}}.
            \end{align}
            However, \citeauthor{zanghellini_2004} uses the more convenient set
            of Hartree atomic units where $m = e = \hslash = k_e =
            \para{4\epsilon_0\pi}^{-1} = 1$.
            The term $a$ is a smoothing parameter included to remove the
            singularity when we integrate over all space and the two positions
            $x_i$ and $x_j$ overlap. \cite{suq, zanghellini_2004}
            In the simulation done by \citeauthor{zanghellini_2004} they set $a
            = 0.25$, $\omega = 0.25$, $\mathcal{E}_0 = 1$ and $\Omega = 8\omega$
            on a grid where $x \in [-10, 10]$.
            The laser pulse is turned on for the entirety of the simulation.


            \cite{skattum2013time} \cite{kristiansen2017time}
    \section{Double-well quantum dots}
