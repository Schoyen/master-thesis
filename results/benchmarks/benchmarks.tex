\chapter{Benchmarking against litterature}
    In this chapter we use our developed code to reproduce results from relevant
    articles to verify that our implementation works as expected.

    \section{Single quantum dots}
        In this section we'll discuss results for quantum dots subject to a
        harmonic oscillator potential in one and two dimensions.
        These systems are rather well studied and therefore provides ample
        results which can be used as verification.

        \subsection{The one-dimensional harmonic oscillator}
            An excellent starting point is the one-dimensional harmonic
            oscillator as discussed in \autoref{subsec:one-dim-ho}.
            A study done by \citeauthor{zanghellini_2004}
            \cite{zanghellini_2004} explores the multi-configuration
            time-dependent Hartree-Fock method on the one-dimensional harmonic
            oscillator system for two particles compared to a semi-analytic
            result\footnote{Semi-analytic as the time-evolution is solved
            numerically.}.
            In this study the Hamiltonian is given by
            \begin{align}
                \hamil
                &=
                \sum_{i = 1}^{N}
                \para{
                    \frac{\momentum^2_i}{2m}
                    + \half m \omega^2 \position^2_i
                    + \mathcal{E}_0 \sin(\Omega t)\position_i
                }
                \nonumber \\
                &\qquad
                + \sum_{i, j = 1}^{N}
                \frac{1}{4\epsilon_0\pi}
                \frac{e^2}{\sqrt{(\position_i - \position_j)^2 + a^2}}.
            \end{align}
            However, \citeauthor{zanghellini_2004} uses the more convenient set
            of Hartree atomic units where $m = e = \hslash = k_e =
            \para{4\epsilon_0\pi}^{-1} = 1$.
            The term $a$ is a smoothing parameter included to remove the
            singularity when we integrate over all space and the two positions
            $x_i$ and $x_j$ overlap. \cite{suq, zanghellini_2004}
            In the simulation done by \citeauthor{zanghellini_2004} they set $a
            = 0.25$, $\omega = 0.25$, $\mathcal{E}_0 = 1$ and $\Omega = 8\omega$
            on a grid where $x \in [-10, 10]$.
            The laser pulse is turned on for the entirety of the simulation.

            In \autoref{fig:one-body-density-zanghellini} we plot the ground
            state one-body particle densities using Hartree-Fock and
            configuration interaction with singles and doubles, i.e., full
            configuration interaction for two particles.
            In the lower plot in \autoref{fig:one-body-density-zanghellini} we
            demonstrate that the coupled cluster method with singles and double
            reproduces the exact solution as achieved from configuration
            interaction to a high precision.
            Furthermore, in \autoref{fig:overlap-zanghellini}, we show the
            time-dependent overlap for two one-dimensional quantum dots subject
            to the same dipole laser as in the study done by
            \citeauthor{zanghellini_2004} \cite{zanghellini_2004}.
            We also show how the coupled cluster method with singles and doubles
            reproduces the exact solution to a large degree.

            \begin{figure}
                \centering
                \begin{tikzpicture}
                    \begin{groupplot}[
                            group style={
                                group size=1 by 2,
                                vertical sep=30pt,
                                xlabels at=edge bottom,
                            },
                            width=11cm,
                            height=6cm,
                            xlabel={$x$ $[\text{a.u.}]$},
                        ]
                        \nextgroupplot[
                                grid=major,
                                ylabel={$\densityten(x, 0)$},
                                restrict x to domain=-6:6,
                                enlarge x limits=false,
                                ymin=0,
                                ymax=0.4,
                                enlarge y limits=false,
                            ]
                            \addplot+ [
                                mark=none,
                                thick,
                            ]
                            table
                            {results/benchmarks/zanghellini/dat/rho_tdcisd_real.dat};
                            \addlegendentry{TDCISD}

                            \addplot+ [
                                mark=none,
                                thick,
                                dashed,
                            ]
                            table
                            {results/benchmarks/zanghellini/dat/rho_tdhf_real.dat};
                            \addlegendentry{TDHF}
                        \nextgroupplot[
                                ymode=log,
                                grid=major,
                                ylabel={$\abs{\Delta\densityten(x, 0)}$},
                                restrict x to domain=-6:6,
                                enlarge x limits=false,
                            ]
                            \addplot+ [
                                mark=none,
                                thick,
                            ]
                            table
                            {results/benchmarks/zanghellini/dat/rho_diff_tdccsd_tdcisd.dat};
                    \end{groupplot}
                \end{tikzpicture}
                \caption{In the top figure we have reproduced Figure 1. in the
                study done by \citeauthor{zanghellini_2004}
                \cite{zanghellini_2004}, that is, we have plotted the one-body
                particle density two one-dimensional quantum dots in an harmonic
                oscillator trap using full-configuration interaction and a
                general Hartree-Fock solver.
                In the lower figure we have plotted the absolute difference
                between the one-body particle densities from the
                full-configuration interaction and coupled cluster with singles
                and doubles methods.}
                \label{fig:one-body-density-zanghellini}
             \end{figure}

            \begin{figure}
                \centering
                \begin{tikzpicture}
                    \begin{groupplot}[
                            group style={
                                group size=1 by 2,
                                vertical sep=30pt,
                                xlabels at=edge bottom,
                            },
                            width=11cm,
                            height=6cm,
                            xlabel={$\omega t / (2\pi)$},
                        ]
                        \nextgroupplot[
                                grid=major,
                                ylabel={$\abs{\braket*{\Psi(t)}{\Psi(0)}}^2$},
                                enlarge x limits=false,
                            ]
                            \addplot+ [
                                mark=none,
                                thick,
                            ]
                            table
                            {results/benchmarks/zanghellini/dat/overlap_tdcisd_real.dat};
                            \addlegendentry{TDCISD}

                            \addplot+ [
                                mark=none,
                                thick,
                                dashed,
                            ]
                            table
                            {results/benchmarks/zanghellini/dat/overlap_tdhf_real.dat};
                            \addlegendentry{TDHF}
                        \nextgroupplot[
                                ymode=log,
                                grid=major,
                                ylabel={$\Delta\abs{\braket*{\Psi(t)}{\Psi(0)}}^2$},
                                restrict x to domain=-6:6,
                                enlarge x limits=false,
                            ]
                            \addplot+ [
                                mark=none,
                                thick,
                            ]
                            table
                            {results/benchmarks/zanghellini/dat/overlap_diff_tdccsd_tdcisd.dat};
                    \end{groupplot}
                \end{tikzpicture}
                \caption{In the top figure we have plotted the time-dependent
                overlap between the initial ground state $\ket{\Psi(0)}$ and the
                state $\ket{\Psi(t)}$ at a later time.
                We compare the time-dependent general Hartree-Fock method and
                the time-dependent full configuration interaction method.
                The figure is a reproduction of Figure 2. in the study done by
                \citeauthor{zanghellini_2004} \cite{zanghellini_2004}.
                In the lower figure we show the absolute difference in the
                overlap between the time-dependent full configuration
                interaction method the time-dependent coupled cluster method
                with singles and doubles.}
                \label{fig:overlap-zanghellini}
             \end{figure}


            \cite{skattum2013time} \cite{kristiansen2017time}
    \section{Double-well quantum dots}
