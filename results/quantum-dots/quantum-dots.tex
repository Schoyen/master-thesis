\chapter{Quantum dots}
    In this chapter we'll demonstrate results for the one- and
    two-dimensional quantum dots subject to various potentials.
    We'll explore both the time-independent and the time-dependent situations.

    \section{Ground state energies}
        In \autoref{tab:gs-tdho} we list a select few ground state energies for
        the two-dimensional harmonic osccilator to demonstrate that we are able
        to reproduce energies found in the literature.
        A few things to note is that we chose the number of spin-orbitals to be
        $L = 90$ as this provided reasonable precision and few convergence
        problems while still being fast.
        Initially we run a restricted Hartree-Fock calculation, and transform to
        this basis.
        We then use this molecular orbital basis for the CCD, CCSD, and NOCCD
        methods.

        However, as a slight surprise it turns that CCSD and NOCCD had most
        problems with convergence once the number of particles increased.
        We have marked the numbers which required manual intervening after the
        initial run had finished by a $*$.
        We'll discuss what needed to be done in order to have these values
        converge.

        For $N = 12$ with $\omega = 0.28$, and $N = 12$ with $\omega = 0.5$
        using CCSD we had to increase the convergence threshold from
        $\num{1e-4}$ to $\num{1e-3}$ for the residuals of the amplitudes to
        avoid diverging results.
        The singles amplitudes converged to a satisfying result, but the doubles
        amplitudes were unable to go as low as $\num{1e-4}$.

        For $N = 12$ with $\omega = 0.28$ and $\omega = 0.5$ using NOCCD we had
        to increase the convergence threshold of the amplitudes and the orbital
        rotations from $\num{1e-3}$ to $\num{5e-3}$ to avoid diverging results.
        For $N = 20$ with $\omega = 0.5$ and $\omega = 1.0$ using NOCCD required
        that we set the tolerance to $\num{1e-2}$.
        A similar type of behaviour occured for NOCCD as for CCSD where we
        seemingly move towards a converging result, but at some point we make a
        rotation which makes the doubles amplitudes start oscillating around a
        value close to convergence.
        % TODO: Should we include quantitative results for this?
        % We can potentially add a graph of the residuals.

        \begin{table}
            \centering
            \caption{In this table we demonstrate a select few ground state
            energies for $N$ two-dimensional harmonic oscillators.
            We have used a fixed basis size of $L = 90$ spin-orbitals which
            corresponds to $9$ closed shells.
            The CCD, CCSD, and NOCCD methods use the (spin-doubled) RHF-state as
            reference state.
            These results match well with the values found by \citeauthor{fei}
            \cite{fei}, and \citeauthor{lohne} \cite{lohne}.}
            \renewcommand{\arraystretch}{1.3}
            \begin{tabular}{@{}llllll@{}}
                \toprule
                \input{results/quantum-dots/two-dim-quantum-dots/ground-state/dat/gs_tab.csv}
            \end{tabular}
            \label{tab:gs-tdho}
        \end{table}

        The behaviour where the singles amplitudes, or singles rotations,
        destroy the convergence is seemingly a little strange as the CCD method
        had no problems with convergence for the listed results.
        % TODO: Is it possible make an analysis of this?

    \section{Wigner crystallization}
        In a seminal paper by \citeauthor{wigner-crystal}
        \cite{wigner-crystal} the concept of crystallization of the electrons in
        an electronic system is discussed.
        If the potential energy -- here potential energy includes the
        correlation energy from the Coulomb interaction -- exceeds the kinetic
        energy, then the quantum dots will crystalize
        \cite{Cavaliere_2009, zeng-wigner, constantine-wigner, Mikhailov2002,
        akman1999277, hogberget2013quantum}.
        As we've reduced the Hamiltonian of our two-dimensional harmonic
        oscillator system to a dimensionless quantity where $\omega$ is the only
        adjustable parameter, viz. \autoref{eq:tdho-hamiltonian}, we need to
        lower the value of the potential well in order to achieve
        crystallization.
        When the system gets dispersed enough, i.e., $\omega \to 0$, then the
        particles will move into localized spread out shells.

        However, low frequency systems are notoriously difficult to solve using
        coupled cluster methods as they require a substantial amount of basis
        functions to accurately represent the dispersed system.
        Our rather general coupled cluster solvers requires the full two-body
        elements in memory in order to run a simulation.
        This provides a limiting factor on the number of basis functions that
        are tractable to work with.
        As a consequence, we will not demonstrate Wigner crystallization for
        larger systems than $6$ particles.
        A plot of the particle densities for $N = 2$ and $N = 6$ are shown in
        \autoref{fig:wigner-tdho-2-6}.
        \begin{figure}
            \centering
            \begin{tabular}{cc}
                $N = 2$ & $N = 6$
                \\
                \begin{tikzpicture}
                    \pgfplotsset{small}
                    \begin{polaraxis}[
                            %colorbar,
                            colormap/viridis,
                            view={0}{90},
                            xtick={0, 90, 180, 270},
                            xticklabels={
                                $0$,
                                $\pi/2$,
                                $\pi$,
                                $3\pi/2$,
                            },
                            axis on top,
                            title={$\omega = \SI{0.28}{\text{a.u.}}$},
                            title style={
                                rotate=90,
                                at={(axis description cs:-0.2, 0.5)},
                            },
                        ]
                        \addplot3[
                            surf,
                            data cs=polarrad,
                            mesh/rows=101,
                        ]
                        table
                        {results/quantum-dots/two-dim-quantum-dots/wigner/dat/ccsd_n=2_l=132_omega=0.28_rho_real.dat};
                    \end{polaraxis}
                \end{tikzpicture}
                &
                \begin{tikzpicture}
                    \pgfplotsset{small}
                    \begin{polaraxis}[
                            %colorbar,
                            colormap/viridis,
                            view={0}{90},
                            xtick={0, 90, 180, 270},
                            xticklabels={
                                $0$,
                                $\pi/2$,
                                $\pi$,
                                $3\pi/2$,
                            },
                            axis on top,
                        ]
                        \addplot3[
                            surf,
                            data cs=polarrad,
                            mesh/rows=101,
                        ]
                        table
                        {results/quantum-dots/two-dim-quantum-dots/wigner/dat/ccsd_n=6_l=90_omega=0.28_rho_real.dat};
                    \end{polaraxis}
                \end{tikzpicture}
                \\
                \begin{tikzpicture}
                    \pgfplotsset{small}
                    \begin{polaraxis}[
                            %colorbar,
                            colormap/viridis,
                            view={0}{90},
                            xtick={0, 90, 180, 270},
                            xticklabels={
                                $0$,
                                $\pi/2$,
                                $\pi$,
                                $3\pi/2$,
                            },
                            axis on top,
                            title={$\omega = \SI{0.10}{\text{a.u.}}$},
                            title style={
                                rotate=90,
                                at={(axis description cs:-0.2, 0.5)},
                            },
                        ]
                        \addplot3[
                            surf,
                            data cs=polarrad,
                            mesh/rows=101,
                        ]
                        table
                        {results/quantum-dots/two-dim-quantum-dots/wigner/dat/ccsd_n=2_l=132_omega=0.1_rho_real.dat};
                    \end{polaraxis}
                \end{tikzpicture}
                &
                \begin{tikzpicture}
                    \pgfplotsset{small}
                    \begin{polaraxis}[
                            %colorbar,
                            colormap/viridis,
                            view={0}{90},
                            xtick={0, 90, 180, 270},
                            xticklabels={
                                $0$,
                                $\pi/2$,
                                $\pi$,
                                $3\pi/2$,
                            },
                            axis on top,
                        ]
                        \addplot3[
                            surf,
                            data cs=polarrad,
                            mesh/rows=201,
                        ]
                        table
                        {results/quantum-dots/two-dim-quantum-dots/wigner/dat/oaccd_n=6_l=72_omega=0.1_rho_real.dat};
                    \end{polaraxis}
                \end{tikzpicture}
                \\
                \begin{tikzpicture}
                    \pgfplotsset{small}
                    \begin{polaraxis}[
                            %colorbar,
                            colormap/viridis,
                            view={0}{90},
                            xtick={0, 90, 180, 270},
                            xticklabels={
                                $0$,
                                $\pi/2$,
                                $\pi$,
                                $3\pi/2$,
                            },
                            axis on top,
                            title={$\omega = \SI{0.01}{\text{a.u.}}$},
                            title style={
                                rotate=90,
                                at={(axis description cs:-0.2, 0.5)},
                            },
                        ]
                        \addplot3[
                            surf,
                            data cs=polarrad,
                            mesh/rows=201,
                        ]
                        table
                        {results/quantum-dots/two-dim-quantum-dots/wigner/dat/ccsd_n=2_l=132_omega=0.01_rho_real.dat};
                    \end{polaraxis}
                \end{tikzpicture}
                &
                \begin{tikzpicture}
                    \pgfplotsset{small}
                    \begin{polaraxis}[
                            %colorbar,
                            colormap/viridis,
                            view={0}{90},
                            xtick={0, 90, 180, 270},
                            xticklabels={
                                $0$,
                                $\pi/2$,
                                $\pi$,
                                $3\pi/2$,
                            },
                            axis on top,
                        ]
                        \addplot3[
                            surf,
                            data cs=polarrad,
                            mesh/rows=201,
                        ]
                        table
                        {results/quantum-dots/two-dim-quantum-dots/wigner/dat/oaccd_n=6_l=30_omega=0.01_rho_real.dat};
                    \end{polaraxis}
                \end{tikzpicture}
            \end{tabular}
            \caption{In this figure we see how the particle densities moves into
            shell structures -- resembling waves in a pond -- when the frequency
            is lowered.
            A brighter color denotes a larger number and a higher particle
            density.
            This is the system moving into a Wigner crystal.}
            \label{fig:wigner-tdho-2-6}
        \end{figure}
        In \autoref{fig:wigner-tdho-2-6} we are able to observe how particle
        density gets ``pushed'' away from the center of the well when we lower
        the frequency.
        This is the crystallization effect where the particles move into
        localized shell structures.

        In order to produce the figures in
        \autoref{fig:wigner-tdho-2-6} we have used the
        restricted Hartree-Fock method to produce a reference state before using
        either CCSD or NOCCD to compute the ground state particle densities.
        For all three systems with $N = 2$ and the system with $N = 6$ and
        $\omega = \SI{0.28}{\text{a.u.}}$ we have used CCSD with $L = 132$ (this
        corresponds to $12$ full shells) spin-orbitals to produce the particle
        densities.
        The two last systems were created with NOCCD after the initial
        restricted Hartree-Fock run.
        Strangely enough NOCCD suffers convergence issues for large basis sets
        and we were therefore forced to lower the number of spin-orbitals.
        For $\omega = \SI{0.1}{\text{a.u.}}$ we used $L = 72$ spin-orbitals ($9$
        shells) whereas for $\omega = \SI{0.01}{\text{a.u.}}$ only converged for
        $L = 30$ spin-orbitals ($5$ shells).
        % TODO: Discuss this more in depth?

    \section{The Harmonic potential theorem}
        \label{sec:hpt}
        A remarkable result for quantum dots trapped in parabolic quantum wells
        is that the system behaves as a single large harmonic oscillator
        independent of the number of particles \cite{kohn, brey}.
        This means that we are unable to see a ``many-body effect'' when the
        system of quantum dots are trapped in an harmonic oscillator potential
        well as all inter-particle interactions are not observable, and the
        system behaves as a single particle.
        A way to observe this phenomenom known as the harmonic potential theorem
        is by observing the one-body particle density in time and by looking at
        the Fourier transform of the dipole moment.
        For the former we should observe the system moving as a single stiff
        object and the latter results in a single frequency corresponding to the
        oscillator trap frequency.
        % TODO: Expand on this section.
