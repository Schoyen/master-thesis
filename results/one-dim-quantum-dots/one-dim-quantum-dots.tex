\section{One-dimensional quantum dots}
    As discussed in \autoref{sec:one-dim-ho} our implementation of the
    one-dimensional quantum dot is done on a grid, which means that we can study
    more or less any one-dimensional potentials we can think of.
    However, some care must be taken as the matrix elements of the one- and
    two-body Hamiltonian will also be solved on a grid thus potenially leading
    to significant numerical errors.

    \subsection{Eigenstates}
        As discussed in \autoref{subsec:discretizing-the-odqd} our discrete
        solution to the Schrödinger equation for the one-dimensional quantum dot
        allows us to explore arbitrary time-independent potentials, but at the
        cost of numerical errors.

    \subsection{Rigid motion}
        In \autoref{sec:hpt} we discuss the harmonic potential theorem where a
        system of interacting quantum dots in a harmonic potential well will be
        indistinguishable from a single quantum dot in the same potential in
        terms of its emission and absorption spectrum.
        That is, we will only observe a single frequency in the emission and
        absorption spectrum no matter the number of particles in the well.
