\chapter{Stability of coupled-cluster methods}
    \label{chap:stability}
    In this chapter we'll explore some aspects of the stability of the
    implemented coupled-cluster methods.
    We also seek to answer the question of why we should even consider the
    conceptually more complicated orbital-adaptive time-dependent
    coupled-cluster methods as opposed to the known time-dependent
    coupled-cluster methods with static orbitals.

    \section{Why bother with orbital rotations?}
        In \autoref{chap:validation} we demonstrated how both TDCCSD and OATDCCD
        provided excellent agreement with exact solution for the $2$-particle
        systems.
        An important question to answer is then why we should even bother
        implementing OATDCCD, when TDCCSD seems to suffice.
        Especially so, as we in our implementation do not truncate the
        single-particle basis for OATDCCD.
        This is one of its strengths as an approximation to the
        multi-configuration time-dependent Hartree-Fock family of methods.
        Removing this strength from OATDCCD seems to leave us with a method that
        performs just as well as the TDCCSD method.
        As part of the original motivation for this thesis, we wished to study
        the OATDCCD-method as an academic study of a rather novel method.
        However, as pointed out by \citeauthor{pedersen2018symplectic}
        \cite{pedersen2018symplectic}, the TDCCSD-method is \emph{unstable} when
        simulating a system subject to a strong external field.
        This is even the case for $2$-particle systems where TDCCSD is formally
        exact within the finite single-particle basis.
        \citeauthor{pedersen2018symplectic} conjectured that the inclusion of
        orbital rotations might alleviate the stability issues experienced by
        TDCCSD, and we will here verify that this is indeed the case.

        We will use exactly the same fields, atoms, and parameters as
        \citeauthor{pedersen2018symplectic}.
        The dipole laser pulse is given by
        \begin{align}
            \vfg{E}(t)
            = \vfg{E} \cosine(\omega t) G(t),
        \end{align}
        where the envelope function $G(t)$ is given by
        \begin{align}
            G(t)
            = \sine^2(\pi t / t_d).
        \end{align}
        Here $t_d$ is the duration of the laser pulse and the envelope is active
        for $t \in [0, t_d]$.
        Note that atomic units are assumed throughout this study.
        We let the negative charge of the electrons occur in the polarization
        vector.
        The atoms we look at are \ch{He} (Helium) and \ch{Be} (Beryllium) in the
        cc-pVDZ-basis, and we'll use TDFCI (TDCISDTQ) simulations as the ground
        truth.
        We'll run the simulations using both TDCCSD and OATDCCD with the
        restricted Hartree-Fock reference determinant from PySCF as our
        reference state.
        We set $t_d = \SI{5}{\hslash/\hartree}$ and run the simulation for $t_f
        = \SI{10}{\hslash/\hartree}$.
        We've chosen a rather crude time step of $\Delta t =
        \SI{1e-2}{\hslash/\hartree}$, but for the Gauss-integrator with $s = 3$
        and $\epsilon = \num{1e-6}$ this should be acceptable.
        The frequency for \ch{He} is set to $\omega_{\ch{He}} =
        \SI{2.8735643}{\hartree/\hslash}$, and for \ch{Be} we have
        $\omega_{\ch{Be}} = \SI{0.2068175}{\hartree/\hslash}$ in accordance with
        \citeauthor{pedersen2018symplectic} \cite{pedersen2018symplectic}.
        As we merely wish to demonstrate that OATDCCD is stable where TDCCSD is
        not, we only look at the most intense case with an electric field
        strength of $E_{\ch{He}} = \SI{100}{\hartree/(\elementarycharge \bohr)}$
        for \ch{He} and $E_{\ch{Be}} =
        \SI{1}{\hartree/(\elementarycharge\bohr)}$ for \ch{Be}.
        %A plot of these laser pulses for the simulation time listed above is
        %shown in \autoref{fig:stability-lasers}.

        %\begin{figure}
        %    \centering
        %    \begin{tikzpicture}
        %        \begin{groupplot}[
        %                group style={
        %                    group size=1 by 2,
        %                    vertical sep=30pt,
        %                    xlabels at=edge bottom,
        %                },
        %                width=11cm,
        %                height=6cm,
        %                xlabel={$t$ $[\si{\hslash/\hartree}]$},
        %            ]
        %            \nextgroupplot[
        %                    grid=major,
        %                    ylabel={
        %                        $E$ $[\si{\hartree/(\elementarycharge\bohr)}]$
        %                    },
        %                    enlarge x limits=false,
        %                    title={Laser pulses},
        %                ]
        %                \addplot+[
        %                    mark=none,
        %                    thick,
        %                ]
        %                table
        %                {results/stability/stability-runs/dat/laser_he.dat};
        %            \nextgroupplot[
        %                    grid=major,
        %                    ylabel={
        %                        $E$ $[\si{\hartree/(\elementarycharge\bohr)}]$
        %                    },
        %                    enlarge x limits=false,
        %                ]
        %                \addplot+[
        %                    mark=none,
        %                    thick,
        %                ]
        %                table
        %                {results/stability/stability-runs/dat/laser_be.dat};
        %        \end{groupplot}
        %    \end{tikzpicture}
        %    \caption{In these figures we plot the laser pulses used to study the
        %    stability of the OATDCCD- and TDCCSD-methods.
        %    The upper figure is the laser used for the simulation of \ch{He}
        %    whereas the lower pulse is for the simulation of \ch{Be}.}
        %    \label{fig:stability-lasers}
        %\end{figure}

        As there are no expressions for the autocorrelation of the
        OATDCCD-method, we are unable to compare with the autocorrelation as
        done in the article by \citeauthor{pedersen2018symplectic}.
        Therefore, as part of the stability analysis we will sample the
        Frobenius norm of the cluster amplitudes, and the time-dependent overlap
        with the reference determinant as discussed in \autoref{sec:cc-phase}.
        We'll compare this with the weight of the reference determinant in the
        TDFCI simulations, i.e., $\abs{c_0(t)}^2$, the zeroth compononent in the
        TDFCI coefficient vector.

        In \autoref{fig:he-stability} we see the results from the
        \ch{He}-simulation.
        We see that the dashed line with the results from the TDCCSD-method for
        the time-dependent overlap and the time-dependent energy stops after a
        short time, at $t \approx \SI{0.87}{\hslash/\hartree}$.
        This was where the Gauss-integrator crashed due to a non-converging
        fix-point iteration.
        We see that this point more or less exactly matches the point where the
        overlap with reference state becomes close to zero as conjectured.
        Furthermore, by looking at the lowermost palette we see a perfect
        correlation with the norm of the TDCCSD-amplitudes sky-rocketing when
        the overlap goes to zero.
        We can push this simulation through by lowering the time step, but we've
        left this out of the study as we wish to demonstrate that OATDCCD is
        stable for the given time step.
        A too low time step would result in an extremely impractical simulation
        where the time to reach any reasonable length scales would take orders
        of magnitude more simulation time.
        \begin{figure}
            \centering
            \begin{tikzpicture}
                \begin{groupplot}[
                    group style={
                        group size=1 by 3,
                        vertical sep=30pt,
                        xlabels at=edge bottom,
                    },
                    width=11cm,
                    height=6cm,
                    xlabel={$t$ $[\si{\hslash/\hartree}]$},
                ]
                    \nextgroupplot[
                        grid=major,
                        ylabel={$\abs{\braket*{\slat}{\Psi(t)}}^2$},
                        enlarge x limits=false,
                        title={Helium-simulation},
                    ]
                        \addplot+[
                            mark=none,
                            thick,
                        ]
                        table
                        {results/stability/stability-runs/dat/he_tdfci_phase_real.dat};
                        \addlegendentry{TDFCI}

                        \addplot+[
                            mark=none,
                            thick,
                        ]
                        table
                        {results/stability/stability-runs/dat/he_oatdccd_phase_real.dat};
                        \addlegendentry{OATDCCD}

                        \addplot+[
                            mark=none,
                            ultra thick,
                            dashed,
                        ]
                        table
                        {results/stability/stability-runs/dat/he_tdccsd_phase_real.dat};
                        \addlegendentry{TDCCSD}

                    \nextgroupplot[
                        grid=major,
                        ylabel={$\Re(\energy)$ $[\si{\hartree}]$},
                        enlarge x limits=false,
                    ]
                        \addplot+[
                            mark=none,
                            thick,
                        ]
                        table
                        {results/stability/stability-runs/dat/he_tdfci_energy_real.dat};
                        \addlegendentry{TDFCI}

                        \addplot+[
                            mark=none,
                            thick,
                        ]
                        table
                        {results/stability/stability-runs/dat/he_oatdccd_energy_real.dat};
                        \addlegendentry{OATDCCD}

                        \addplot+[
                            mark=none,
                            ultra thick,
                            dashed,
                        ]
                        table
                        {results/stability/stability-runs/dat/he_tdccsd_energy_real.dat};
                        \addlegendentry{TDCCSD}

                    \nextgroupplot[
                        grid=major,
                        ylabel={Magnitude},
                        enlarge x limits=false,
                    ]

                        \addplot+[
                            mark=none,
                            thick,
                        ]
                        table
                        {results/stability/stability-runs/dat/he_oatdccd_norm_t2_real.dat};
                        \addlegendentry{$\norm{\tilde{\clustamp}_2}$}

                        \addplot+[
                            mark=none,
                            thick,
                        ]
                        table
                        {results/stability/stability-runs/dat/he_oatdccd_norm_l2_real.dat};
                        \addlegendentry{$\norm{\tilde{\clustlamp}_2}$}

                        \addplot+[
                            mark=none,
                            thick,
                        ]
                        table
                        {results/stability/stability-runs/dat/he_tdccsd_norm_t1_real.dat};
                        \addlegendentry{$\norm{\clustamp_1}$}

                        \addplot+[
                            mark=none,
                            thick,
                        ]
                        table
                        {results/stability/stability-runs/dat/he_tdccsd_norm_t2_real.dat};
                        \addlegendentry{$\norm{\clustamp_2}$}

                        \addplot+[
                            mark=none,
                            thick,
                        ]
                        table
                        {results/stability/stability-runs/dat/he_tdccsd_norm_l1_real.dat};
                        \addlegendentry{$\norm{\clustlamp_1}$}

                        \addplot+[
                            mark=none,
                            thick,
                        ]
                        table
                        {results/stability/stability-runs/dat/he_tdccsd_norm_l2_real.dat};
                        \addlegendentry{$\norm{\clustlamp_2}$}

                \end{groupplot}
            \end{tikzpicture}
            \caption{In these figures we've plotted the overlap with the
            reference state in time, and the real part of the time-dependent
            energy of the three solvers TDFCI (TDCISD), TDCCSD, and OATDCCD for
            the \ch{He}-system.
            In the lowermost palette we've plotted the magnitude of the cluster
            amplitudes and the Lagrange multipliers for OATDCCD (marked with a
            tilde) and TDCCSD.}
            \label{fig:he-stability}
        \end{figure}

        In \autoref{fig:be-stability} the results from the \ch{Be}-simulation is
        shown.
        Again the same behaviour as for the \ch{He}-simulation is exhibited;
        once the overlap with the reference determinant becomes small, the
        amplitudes sky-rocket and the TDCCSD-method crashes.
        \begin{figure}
            \centering
            \begin{tikzpicture}
                \begin{groupplot}[
                    group style={
                        group size=1 by 3,
                        vertical sep=30pt,
                        xlabels at=edge bottom,
                    },
                    width=11cm,
                    height=6cm,
                    xlabel={$t$ $[\si{\hslash/\hartree}]$},
                ]
                    \nextgroupplot[
                        grid=major,
                        ylabel={$\abs{\braket*{\slat}{\Psi(t)}}^2$},
                        enlarge x limits=false,
                        title={Beryllium-simulation},
                    ]
                        \addplot+[
                            mark=none,
                            thick,
                        ]
                        table
                        {results/stability/stability-runs/dat/be_tdfci_phase_real.dat};
                        \addlegendentry{TDFCI}

                        \addplot+[
                            mark=none,
                            thick,
                        ]
                        table
                        {results/stability/stability-runs/dat/be_oatdccd_phase_real.dat};
                        \addlegendentry{OATDCCD}

                        \addplot+[
                            mark=none,
                            ultra thick,
                            dashed,
                        ]
                        table
                        {results/stability/stability-runs/dat/be_tdccsd_phase_real.dat};
                        \addlegendentry{TDCCSD}

                    \nextgroupplot[
                        grid=major,
                        ylabel={$\Re(\energy)$ $[\si{\hartree}]$},
                        enlarge x limits=false,
                    ]
                        \addplot+[
                            mark=none,
                            thick,
                        ]
                        table
                        {results/stability/stability-runs/dat/be_tdfci_energy_real.dat};
                        \addlegendentry{TDFCI}

                        \addplot+[
                            mark=none,
                            thick,
                        ]
                        table
                        {results/stability/stability-runs/dat/be_oatdccd_energy_real.dat};
                        \addlegendentry{OATDCCD}

                        \addplot+[
                            mark=none,
                            ultra thick,
                            dashed,
                        ]
                        table
                        {results/stability/stability-runs/dat/be_tdccsd_energy_real.dat};
                        \addlegendentry{TDCCSD}

                    \nextgroupplot[
                        grid=major,
                        ylabel={Magnitude},
                        enlarge x limits=false,
                    ]

                        \addplot+[
                            mark=none,
                            thick,
                        ]
                        table
                        {results/stability/stability-runs/dat/be_oatdccd_norm_t2_real.dat};
                        \addlegendentry{$\norm{\tilde{\clustamp}_2}$}

                        \addplot+[
                            mark=none,
                            thick,
                        ]
                        table
                        {results/stability/stability-runs/dat/be_oatdccd_norm_l2_real.dat};
                        \addlegendentry{$\norm{\tilde{\clustlamp}_2}$}

                        \addplot+[
                            mark=none,
                            thick,
                        ]
                        table
                        {results/stability/stability-runs/dat/be_tdccsd_norm_t1_real.dat};
                        \addlegendentry{$\norm{\clustamp_1}$}

                        \addplot+[
                            mark=none,
                            thick,
                        ]
                        table
                        {results/stability/stability-runs/dat/be_tdccsd_norm_t2_real.dat};
                        \addlegendentry{$\norm{\clustamp_2}$}

                        \addplot+[
                            mark=none,
                            thick,
                        ]
                        table
                        {results/stability/stability-runs/dat/be_tdccsd_norm_l1_real.dat};
                        \addlegendentry{$\norm{\clustlamp_1}$}

                        \addplot+[
                            mark=none,
                            thick,
                        ]
                        table
                        {results/stability/stability-runs/dat/be_tdccsd_norm_l2_real.dat};
                        \addlegendentry{$\norm{\clustlamp_2}$}
                \end{groupplot}
            \end{tikzpicture}
            \caption{In these figures we've plotted the overlap with the
            reference state in time, and the real part of the time-dependent
            energy of the three solvers TDFCI (TDCISD), TDCCSD, and OATDCCD for
            the \ch{Be}-system.
            The lower palette exhibits the magnitude of the cluster amplitudes
            and the Lagrange multipliers for OATDCCD (marked with a tilde) and
            TDCCSD.}
            \label{fig:be-stability}
        \end{figure}
        In both simulations we see that the TDFCI solutions drops down to a low
        overlap with the reference state, but this is not a problem for the
        TDFCI-methods as they can always remove the weigth of the reference
        state.
        This does however demonstrate that the initial Hartree-Fock reference
        determinant is poor choice for the intense fields we explore in these
        simulations.
        It is more interesting to see how the OATDCCD-method seems completely
        unaffected by the choice of a poor reference state as it is able to
        rotate to a new set of orbitals thereby creating a much better reference
        state.
        We also see his behavior in the norm of the amplitudes of the
        OATDCCD-method.
        The more optimal reference determinant seems to remove the stress from
        the amplitudes and OATDCCD does not seem to exhibit the same stability
        issues in the simulations we've demonstrated in
        \autoref{fig:he-stability} and \autoref{fig:be-stability}.

        We do not however state that the OATDCCD-method is infinitely stable as
        there are several situations which we believe would break the
        simulation, e.g., a near multireference state, oscillations between two
        (almost) equally good reference states.
        In short, if there does not exist a single good reference state, we
        believe the OATDCCD-method will suffer the same fate as for the
        TDCCSD-method.


    \section{Summing up the stability analysis}
        We have now demonstrated why the OATDCCD-method is an important method
        for dynamics of many-body systems as it provides a more stable solver
        than the TDCCSD-method.
        A further study of this phenomenon will not included in this thesis as
        the author is participating in an ongoing work on the stability of the
        OATDCCD-method titled \citetitle{oa-stability} \cite{oa-stability}.
        In the following we will not look at stability issues in more detail,
        but rather focus on the application of the methods to various systems.




\clearpage
