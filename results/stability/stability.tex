\chapter{Stability}
    \label{chap:stability}
    In this chapter we'll explore some aspects of the stability of our developed
    methods.
    We also seek to answer the question of why we should even consider the more
    complicated orbital-adaptive time-dependent coupled-cluster methods as
    opposed to the known time-dependent coupled-cluster methods with static
    orbitals.

    \section{Why bother with orbital rotations?}
        In \autoref{chap:validation} we demonstrated how both TDCCSD and OATDCCD
        provided excellent agreement with exact solution for the $2$-particle
        systems.
        An important question to answer is then why we should even bother
        implementing OATDCCD, when TDCCSD seems to do the trick.
        Especially in our implementation where we do not truncate the
        single-particle basis further for OATDCCD, thus removing one of its
        strengths as compared to the static TDCCSD solver.
        As part of the original motivation for this thesis, we wished to study
        the OATDCCD-method as an academic study of a rather novel method.
        However, as pointed out by \citeauthor{pedersen2018symplectic}
        \cite{pedersen2018symplectic}, the TDCCSD-method is \emph{unstable} in
        the limit of very strong laser pulses.
        This is even the case for $2$-particle systems where TDCCSD is formally
        exact within the finite single-particle basis.
        \citeauthor{pedersen2018symplectic} conjectured that the inclusion of
        orbital rotations might be a solution to the stability issues
        experienced by TDCCSD, and we will here verify that this is indeed the
        case.

        We will use exactly the same fields, atoms, and parameters as
        \citeauthor{pedersen2018symplectic}.
        The dipole laser pulse is given by
        \begin{align}
            \vfg{E}(t)
            = \vfg{E} \sine(\omega t) G(t),
        \end{align}
        where the envelope function $G(t)$ is given by
        \begin{align}
            G(t)
            = \sine^2(\pi t / t_d),
        \end{align}
        where $t_d$ is the duration of the laser pulse and the envelope is
        active for $t \in [0, t_d]$.
        We let the negative charge of the electrons occur in the polarization
        vector.
        The atoms we look at are \ch{He} and \ch{Be}, and we'll use TDFCI
        simulations as the ground truth.
        We'll run the simulations using both TDCCSD and OATDCCD with the
        restricted Hartree-Fock reference determinant from PySCF.
        As has been stated several times is the lack of the time-dependent
        overlap for the OATDCCD-method.
        Therefore, as part of the stability analysis we will sample the
        Frobenius norm of the cluster amplitudes, and the time-dependent overlap
        with the reference determinant as discussed in \autoref{sec:cc-phase}.
        We'll compare this with the weight of the reference determinant in the
        TDFCI simulations, i.e., $\abs{c_0(t)}^2$, the zeroth compononent in the
        TDFCI coefficient vector.


    \section{Summing up the stability analysis}
        We have now demonstrated why the OATDCCD-method is an important method
        for dynamics of many-body systems as it provides a more stable solver
        than the TDCCSD-method.
        A further study of this phenomenon will not included in this thesis as
        the author is participating in an ongoing work on the stability of the
        OATDCCD-method with \citeauthor{oa-stability} \cite{oa-stability}.
        Also, in the following results we do not include any stability analysis,
        but merely compare with known litterature where we are able.
