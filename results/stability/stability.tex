\chapter{Stability}
    \label{chap:stability}
    In this chapter we'll explore some aspects of the stability of our developed
    methods.
    We also seek to answer the question of why we should even consider the more
    complicated orbital-adaptive time-dependent coupled-cluster methods as
    opposed to the known time-dependent coupled-cluster methods with static
    orbitals.

    \section{Why bother with orbital rotations?}
        In \autoref{chap:validation} we demonstrated how both TDCCSD and OATDCCD
        provided excellent agreement with exact solution for the $2$-particle
        systems.
        An important question to answer is then why we should even bother
        implementing OATDCCD, when TDCCSD seems to do the trick.
        Especially in our implementation where we do not truncate the
        single-particle basis further for OATDCCD, thus removing one of its
        strengths as compared to the static TDCCSD solver.
        As part of the original motivation for this thesis, we wished to study
        the OATDCCD-method as an academic study of a rather novel method.
        However, as pointed out by \citeauthor{pedersen2018symplectic}
        \cite{pedersen2018symplectic}, the TDCCSD-method is \emph{unstable} in
        the limit of very strong laser pulses.
        This is even the case for $2$-particle systems where TDCCSD is formally
        exact within the finite single-particle basis.
        \citeauthor{pedersen2018symplectic} conjectured that the inclusion of
        orbital rotations might be a solution to the stability issues
        experienced by TDCCSD, and we will here verify that this is indeed the
        case.

        We will use exactly the same fields, atoms, and parameters as
        \citeauthor{pedersen2018symplectic}.
        The dipole laser pulse is given by
        \begin{align}
            \vfg{E}(t)
            = \vfg{E} \cosine(\omega t) G(t),
        \end{align}
        where the envelope function $G(t)$ is given by
        \begin{align}
            G(t)
            = \sine^2(\pi t / t_d),
        \end{align}
        where $t_d$ is the duration of the laser pulse and the envelope is
        active for $t \in [0, t_d]$.
        Note that atomic units are assume throughout this study.
        We let the negative charge of the electrons occur in the polarization
        vector.
        The atoms we look at are \ch{He} and \ch{Be} in the cc-pVDZ-basis, and
        we'll use TDFCI simulations as the ground truth.
        We'll run the simulations using both TDCCSD and OATDCCD with the
        restricted Hartree-Fock reference determinant from PySCF.
        We set $t_d = \num{5}$ and run the simulation for $t_f = \num{10}$.
        We've chosen a rather coarse time step of $\Delta t = \num{1e-2}$, but
        for the Gauss-integrator this should be acceptable.
        A lower time step will make the simulations so long that they lose much
        of their practical usage.
        The frequency for \ch{He} is set to $\omega_{\ch{He}} =
        \num{2.8735643}$, and for \ch{Be} we have $\omega_{\ch{Be}} =
        \num{0.2068175}$ in accordance with \citeauthor{pedersen2018symplectic}
        \cite{pedersen2018symplectic}.
        As we merely wish to demonstrate that OATDCCD is stable where TDCCSD is
        not, we only look at the most intense case with an electric field
        strength of $E_{\ch{He}} = \num{100}$ for \ch{He} and $E_{\ch{Be}} =
        \num{1}$ for \ch{Be}.
        A plot of these laser pulses for the simulation time listed above is
        shown in \autoref{fig:stability-lasers}.

        \begin{figure}
            \centering
            \begin{tikzpicture}
                \begin{groupplot}[
                        group style={
                            group size=1 by 2,
                            vertical sep=30pt,
                            xlabels at=edge bottom,
                        },
                        width=11cm,
                        height=6cm,
                        xlabel={$t$ $[\si{\hslash/\hartree}]$},
                    ]
                    \nextgroupplot[
                            grid=major,
                            ylabel={
                                $E$ $[\si{\hartree/(\elementarycharge\bohr)}]$
                            },
                            enlarge x limits=false,
                            title={Laser pulses},
                        ]
                        \addplot+[
                            mark=none,
                            thick,
                        ]
                        table
                        {results/stability/stability-runs/dat/laser_he.dat};
                    \nextgroupplot[
                            grid=major,
                            ylabel={
                                $E$ $[\si{\hartree/(\elementarycharge\bohr)}]$
                            },
                            enlarge x limits=false,
                        ]
                        \addplot+[
                            mark=none,
                            thick,
                        ]
                        table
                        {results/stability/stability-runs/dat/laser_be.dat};
                \end{groupplot}
            \end{tikzpicture}
            \caption{In these figures we plot the laser pulses used to study the
            stability of the OATDCCD- and TDCCSD-methods.
            The upper figure is the laser used for the simulation of \ch{He}
            whereas the lower pulse is for the simulation of \ch{Be}.}
            \label{fig:stability-lasers}
        \end{figure}

        As there are no expressions for the autocorrelation of the
        OATDCCD-method, we are unable to compare with the autocorrelation as
        done in the article by \citeauthor{pedersen2018symplectic}.
        Therefore, as part of the stability analysis we will sample the
        Frobenius norm of the cluster amplitudes, and the time-dependent overlap
        with the reference determinant as discussed in \autoref{sec:cc-phase}.
        We'll compare this with the weight of the reference determinant in the
        TDFCI simulations, i.e., $\abs{c_0(t)}^2$, the zeroth compononent in the
        TDFCI coefficient vector.

        In \autoref{fig:he-stability} we see the results from the
        \ch{He}-simulation.
        We see that the dashed line with the results from the TDCCSD-method
        stops after a short time, at approximately $t = \num{1}$.
        This was where the Gauss-integrator crashed due to a non-converging
        fixed point iteration.
        We see that this point more or less exactly matches the point where the
        overlap with reference state becomes close to zero as conjectured.
        We can push this simulation through by lowering the time step, but we've
        left this out of the study as we wish to demonstrate that OATDCCD is
        stable for the given time step.
        And, as mentioned, a too low time step is not practical as the
        simulation times would grow very long.

        In \autoref{fig:be-stability} the results from the \ch{Be}-simulation is
        shown.
        Again the same behaviour as for the \ch{He}-simulation is exhibited;
        once the overlap with the reference determinant becomes small, the
        TDCCSD-method crashes.

        \begin{figure}
            \centering
            \begin{tikzpicture}
                \begin{groupplot}[
                    group style={
                        group size=1 by 2,
                        vertical sep=30pt,
                        xlabels at=edge bottom,
                    },
                    width=11cm,
                    height=6cm,
                    xlabel={$t$ $[\si{\hslash/\hartree}]$},
                ]
                    \nextgroupplot[
                        grid=major,
                        ylabel={$\abs{\braket*{\slat}{\Psi(t)}}^2$},
                        enlarge x limits=false,
                        title={\ch{He}-simulation},
                    ]
                        \addplot+[
                            mark=none,
                            thick,
                        ]
                        table
                        {results/stability/stability-runs/dat/he_tdfci_phase_real.dat};
                        \addlegendentry{TDFCI}

                        \addplot+[
                            mark=none,
                            thick,
                        ]
                        table
                        {results/stability/stability-runs/dat/he_oatdccd_phase_real.dat};
                        \addlegendentry{OATDCCD}

                        \addplot+[
                            mark=none,
                            ultra thick,
                            dashed,
                        ]
                        table
                        {results/stability/stability-runs/dat/he_tdccsd_phase_real.dat};
                        \addlegendentry{TDCCSD}

                    \nextgroupplot[
                        grid=major,
                        ylabel={$\Re(\energy)$ $[\si{\hartree}]$},
                        enlarge x limits=false,
                    ]
                        \addplot+[
                            mark=none,
                            thick,
                        ]
                        table
                        {results/stability/stability-runs/dat/he_tdfci_energy_real.dat};
                        \addlegendentry{TDFCI}

                        \addplot+[
                            mark=none,
                            thick,
                        ]
                        table
                        {results/stability/stability-runs/dat/he_oatdccd_energy_real.dat};
                        \addlegendentry{OATDCCD}

                        \addplot+[
                            mark=none,
                            ultra thick,
                            dashed,
                        ]
                        table
                        {results/stability/stability-runs/dat/he_tdccsd_energy_real.dat};
                        \addlegendentry{TDCCSD}
                \end{groupplot}
            \end{tikzpicture}
            \caption{In these figures we've plotted the overlap with the
            reference state in time, and the real part of the time-dependent
            energy of the three solvers TDFCI (TDCISD), TDCCSD, and OATDCCD for
            the \ch{He}-system.}
            \label{fig:he-stability}
        \end{figure}

        \begin{figure}
            \centering
            \begin{tikzpicture}
                \begin{groupplot}[
                    group style={
                        group size=1 by 2,
                        vertical sep=30pt,
                        xlabels at=edge bottom,
                    },
                    width=11cm,
                    height=6cm,
                    xlabel={$t$ $[\si{\hslash/\hartree}]$},
                ]
                    \nextgroupplot[
                        grid=major,
                        ylabel={$\abs{\braket*{\slat}{\Psi(t)}}^2$},
                        enlarge x limits=false,
                        title={\ch{Be}-simulation},
                    ]
                        \addplot+[
                            mark=none,
                            thick,
                        ]
                        table
                        {results/stability/stability-runs/dat/be_tdfci_phase_real.dat};
                        \addlegendentry{TDFCI}

                        \addplot+[
                            mark=none,
                            thick,
                        ]
                        table
                        {results/stability/stability-runs/dat/be_oatdccd_phase_real.dat};
                        \addlegendentry{OATDCCD}

                        \addplot+[
                            mark=none,
                            ultra thick,
                            dashed,
                        ]
                        table
                        {results/stability/stability-runs/dat/be_tdccsd_phase_real.dat};
                        \addlegendentry{TDCCSD}

                    \nextgroupplot[
                        grid=major,
                        ylabel={$\Re(\energy)$ $[\si{\hartree}]$},
                        enlarge x limits=false,
                    ]
                        \addplot+[
                            mark=none,
                            thick,
                        ]
                        table
                        {results/stability/stability-runs/dat/be_tdfci_energy_real.dat};
                        \addlegendentry{TDFCI}

                        \addplot+[
                            mark=none,
                            thick,
                        ]
                        table
                        {results/stability/stability-runs/dat/be_oatdccd_energy_real.dat};
                        \addlegendentry{OATDCCD}

                        \addplot+[
                            mark=none,
                            ultra thick,
                            dashed,
                        ]
                        table
                        {results/stability/stability-runs/dat/be_tdccsd_energy_real.dat};
                        \addlegendentry{TDCCSD}
                \end{groupplot}
            \end{tikzpicture}
            \caption{In these figures we've plotted the overlap with the
            reference state in time, and the real part of the time-dependent
            energy of the three solvers TDFCI (TDCISD), TDCCSD, and OATDCCD for
            the \ch{Be}-system.}
            \label{fig:be-stability}
        \end{figure}

        % TODO: Discuss the overlap with the reference determinants


    \section{Summing up the stability analysis}
        We have now demonstrated why the OATDCCD-method is an important method
        for dynamics of many-body systems as it provides a more stable solver
        than the TDCCSD-method.
        A further study of this phenomenon will not included in this thesis as
        the author is participating in an ongoing work on the stability of the
        OATDCCD-method with \citeauthor{oa-stability} \cite{oa-stability}.
        Also, in the following results we do not include any stability analysis,
        but merely compare with known litterature where we are able.
