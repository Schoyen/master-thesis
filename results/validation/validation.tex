\chapter{Validation}
    \label{chap:validation}
    \epigraph{Testing shows the presence, not the absence of bugs.}
    {--- Edsger Wybe Dijkstra}

    As the quote by E. W. Dijkstra so succintly puts, testing will only reveal
    the presence of bugs and there are most likely bugs present in the developed
    code.
    However, continuous testing over time will help make sure that new additions
    and evolution of existing code will not break at a later time.
    Our codes have contribution from several authors and we have therefore
    implemented a wide range of unit tests where these make sense, integration
    tests running longer simulations and checking the results against known
    results.
    In fact, both the studies in this chapter are implemented as tests in a more
    brief version to make sure that changes to the code should still reproduce
    what we present in this chapter.
    To ensure continuous testing -- and deployment of the documentation --
    we use Travis CI.\footnote{%
        See: \url{https://travis-ci.com/}
    }
    This means that all changes to the git-repositories run through the
    test-suites automatically and checks that known values are reproduced.

    The focus in this thesis is on time-evolution of quantum mechanical systems,
    and as such we will mainly look at time-dependent phenomenon with little
    focus on ground state solutions.
    However, note that several known ground state values, in particular for one-
    and two-dimensional quantum dots, are implemented as tests.
    In this chapter we wish to demonstrate that our implementations work as
    expected by reproducing known results from the literature.
    We will in this section limit our attention to small systems.

    \section{The one-dimensional harmonic oscillator}
        An excellent starting point for validation is the one-dimensional
        harmonic oscillator using the discretized one-dimensional quantum dot
        system as discussed in \autoref{sec:one-dim-qd}.
        A study done by \citeauthor{zanghellini_2004} \cite{zanghellini_2004}
        explores the multi-configuration time-dependent Hartree-Fock method on
        the one-dimensional harmonic oscillator system for two particles
        compared to a semi-analytic result.\footnote{%
            Semi-analytic as the time-evolution is solved numerically.
        }
        In this study the confining potential is given by the harmonic
        oscillator potential,
        \begin{align}
            \hat{v} = \half \omega^2\hat{x}^2,
        \end{align}
        where atomic units are used.
        \citeauthor{zanghellini_2004} uses a monochromatic, dipole, laser field
        described by
        \begin{align}
            E(t) = E_0\sin(\Omega t)\hat{x},
        \end{align}
        which is active for the entire simulation.
        The Coulomb interaction is described by the shielded Coulomb interaction
        from \autoref{eq:shielded-coulomb} with $\alpha = 1$ and the screening
        parameter set to $a = 0.25$.
        In the simulation done by \citeauthor{zanghellini_2004} the parameters
        are chosen to be $\omega = 0.25$, $E_0 = 1$, and $\Omega = 8\omega$.
        The simulation is run on a one-dimensional grid where $x \in [-10, 10]$.
        We use a relatively small basis of $L = 20$ spin-orbitals, that is, the
        $L / 2 = 10$ first one-dimensional harmonic oscillator orbitals found by
        diagonalizing the one-body Hamiltonian.
        The time-evolution is run for $t = 4\tau$ where $\tau$ is a cycle of the
        laser given by
        \begin{align}
            \tau = \frac{2\pi}{\Omega}.
        \end{align}
        We use an integration time step of $\Delta t = \num{1e-2}$ and the
        Gauss-integrator with $s = 3$ and a fix-point convergence threshold of
        $\epsilon = \num{1e-6}$.

        We will reproduce part of the study done by
        \citeauthor{zanghellini_2004} using TDHF, TDCID, TDCISD, TDCCD, and
        TDCCSD as well as the their ground state counterparts.
        We do not include NOCCD nor OATDCCD, as part of the study is a
        comparison of the time-dependent overlap which we have not implemented
        for the latter method as dicussed in
        \autoref{subsec:autocorrelation-oatdccd}.
        In \autoref{fig:one-body-density-zanghellini} we plot the ground state
        particle densities using the general Hartree-Fock method and CISD, i.e.,
        full configuration interaction for two particles.
        In the lower plot in \autoref{fig:one-body-density-zanghellini} we
        demonstrate that CCSD reproduces the exact solution for the particle
        density as achieved from configuration interaction to a high precision
        by computing the absolute difference between each point on the grid,
        that is,
        \begin{align}
            \abs{\Delta \densityten(x, 0)}
            = \abs{\densityten_{CISD}(x, 0) - \densityten_{CCSD}(x, 0)}.
        \end{align}
        We've listed the ground state energies in \autoref{tab:gs-zanghellini}
        for the different solver methods.
        We compute the relative error by
        \begin{align}
            e_{rel} = \frac{\abs{E - E_{ref}}}{\abs{E_{ref}}},
        \end{align}
        where $E_{ref} = \SI{0.8247}{\hartree}$ as the exact value from
        \citeauthor{zanghellini_2004} \cite{zanghellini_2004}.
        Note that the relative error is computed for the rounded values.

        \begin{table}
            \centering
            \caption{In this table we list the ground state energies computed
            for the one-dimensional harmonic oscillator.
            The values are rounded to four decimal places.}
            \renewcommand{\arraystretch}{1.3}
            \begin{tabular}{@{}lll@{}}
                \toprule
                Method & Energy $[\si{\hartree}]$
                & Relative error \\
                \midrule
                HF     & $1.1798$ & $0.4306$ \\
                CID    & $1.0516$ & $0.2751$ \\
                CCD    & $1.0516$ & $0.2751$ \\
                CID-HF & $0.8384$ & $0.0166$ \\
                CCD-HF & $0.8384$ & $0.0166$ \\
                CISD   & $0.8253$ & $0.0007$ \\
                CCSD   & $0.8253$ & $0.0007$ \\
                \bottomrule
            \end{tabular}
            \label{tab:gs-zanghellini}
        \end{table}

        \begin{figure}
            \centering
            \begin{tikzpicture}
                \begin{groupplot}[
                        group style={
                            group size=1 by 2,
                            vertical sep=30pt,
                            xlabels at=edge bottom,
                        },
                        width=11cm,
                        height=6cm,
                        xlabel={$x$ $[\si{\bohr}]$},
                    ]
                    \nextgroupplot[
                            grid=major,
                            ylabel={$\densityten(x, 0)$},
                            restrict x to domain=-6:6,
                            enlarge x limits=false,
                            ymin=0,
                            ymax=0.4,
                            enlarge y limits=false,
                            title={Ground state particle density},
                        ]
                        \addplot+ [
                            mark=none,
                            thick,
                        ]
                        table
                        {results/benchmarks/zanghellini/dat/rho_tdcisd_real.dat};
                        \addlegendentry{TDCISD}

                        \addplot+ [
                            mark=none,
                            thick,
                            dashed,
                        ]
                        table
                        {results/benchmarks/zanghellini/dat/rho_tdhf_real.dat};
                        \addlegendentry{TDHF}
                    \nextgroupplot[
                            ymode=log,
                            grid=major,
                            ylabel={$\abs{\Delta\densityten(x, 0)}$},
                            restrict x to domain=-6:6,
                            enlarge x limits=false,
                        ]
                        \addplot+ [
                            mark=none,
                            thick,
                        ]
                        table
                        {results/benchmarks/zanghellini/dat/rho_diff_tdccsd_tdcisd.dat};
                \end{groupplot}
            \end{tikzpicture}
            \caption{In the top figure we have reproduced Figure 1. in the
            study done by \citeauthor{zanghellini_2004}
            \cite{zanghellini_2004}, that is, we have plotted the one-body
            particle density two one-dimensional quantum dots in an harmonic
            oscillator trap using full-configuration interaction and a
            general Hartree-Fock solver.
            In the lower figure we have plotted the absolute difference
            between the one-body particle densities from the
            full-configuration interaction and coupled cluster with singles
            and doubles methods.}
            \label{fig:one-body-density-zanghellini}
        \end{figure}

        In \autoref{fig:overlap-zanghellini} we show the
        time-dependent overlap for two one-dimensional quantum dots subject
        to the same dipole laser as in the study done by
        \citeauthor{zanghellini_2004} \cite{zanghellini_2004}.
        We also show how TDCCSD reproduces the exact TDCISD solution to a large
        degree by computing the absolute difference between the overlap from
        TDCISD and TDCCSD.
        By inspection we see that both the one-body particle density in
        \autoref{fig:one-body-density-zanghellini} and the time-dependent
        overlap in \autoref{fig:overlap-zanghellini} resembles the results from
        \citeauthor{zanghellini_2004} \cite{zanghellini_2004}, and we conclude
        that TDHF, TDCISD, and TDCCSD work as expected with the latter two
        methods serving as exact solutions within the finite single-particle
        space we explore.

        \begin{figure}
            \centering
            \begin{tikzpicture}
                \begin{groupplot}[
                        group style={
                            group size=1 by 2,
                            vertical sep=30pt,
                            xlabels at=edge bottom,
                        },
                        width=11cm,
                        height=6cm,
                        xlabel={$\Omega t / (2\pi)$},
                    ]
                    \nextgroupplot[
                            grid=major,
                            ylabel={$\abs{\braket*{\Psi(t)}{\Psi(0)}}^2$},
                            enlarge x limits=false,
                            title={Time-dependent overlap},
                        ]
                        \addplot+ [
                            mark=none,
                            thick,
                        ]
                        table
                        {results/benchmarks/zanghellini/dat/overlap_tdcisd_real.dat};
                        \addlegendentry{TDCISD}

                        \addplot+ [
                            mark=none,
                            thick,
                            dashed,
                        ]
                        table
                        {results/benchmarks/zanghellini/dat/overlap_tdhf_real.dat};
                        \addlegendentry{TDHF}
                    \nextgroupplot[
                            ymode=log,
                            grid=major,
                            ylabel={$\Delta\abs{\braket*{\Psi(t)}{\Psi(0)}}^2$},
                            restrict x to domain=-6:6,
                            enlarge x limits=false,
                        ]
                        \addplot+ [
                            mark=none,
                            thick,
                        ]
                        table
                        {results/benchmarks/zanghellini/dat/overlap_diff_tdccsd_tdcisd.dat};
                \end{groupplot}
            \end{tikzpicture}
            \caption{In the top figure we have plotted the time-dependent
            overlap between the initial ground state $\ket*{\Psi(0)}$ and the
            state $\ket*{\Psi(t)}$ at a later time.
            We compare the time-dependent general Hartree-Fock method and
            the time-dependent full configuration interaction method.
            The figure is a reproduction of Figure 2. in the study done by
            \citeauthor{zanghellini_2004} \cite{zanghellini_2004}.
            In the lower figure we show the absolute difference in the
            overlap between the time-dependent full configuration
            interaction method the time-dependent coupled cluster method
            with singles and doubles.}
            \label{fig:overlap-zanghellini}
        \end{figure}

        Before moving on to more exotic systems, we will look shortly at the
        performance of CCD and CID applied to the same one-dimensional harmonic
        oscillator system.
        The reason for doing this is to demonstrate the importance of the
        singles excitation operator.
        Looking at \autoref{fig:one-body-density-doubles-zanghellini} we can see
        how the doubles approximations CID and CCD with the harmonic oscillator
        basis are quite far from the full configuration-interaction benchmark.
        Performing a Hartree-Fock calulation and transforming to the new
        molecular orbital basis we see that CID and CCD are much better at
        reproducing the full configuration-interaction result.
        This demonstrates the effectiveness of Brillouin's theorem by moving to
        the optimal single-reference determinant found in Hartree-Fock.
        However, we see that this does not completely remove the need for a
        singles operator as the result for CID and CCD in the molecular orbital
        basis are still quite far from the exact results.

        \begin{figure}
            \centering
            \begin{tikzpicture}
                \begin{groupplot}[
                        group style={
                            group size=1 by 2,
                            vertical sep=30pt,
                            xlabels at=edge bottom,
                        },
                        width=11cm,
                        height=6cm,
                        xlabel={$x$ $[\si{\bohr}]$},
                    ]
                    \nextgroupplot[
                            grid=major,
                            ylabel={$\densityten(x, 0)$},
                            restrict x to domain=-6:6,
                            enlarge x limits=false,
                            ymin=0,
                            ymax=0.5,
                            enlarge y limits=false,
                            title={Ground state particle density},
                        ]

                        \addplot+ [
                            mark=none,
                            thick,
                        ]
                        table
                        {results/benchmarks/zanghellini/dat/rho_tdcid_real.dat};
                        \addlegendentry{TDCID}

                        \addplot+ [
                            mark=none,
                            thick,
                        ]
                        table
                        {results/benchmarks/zanghellini/dat/rho_tdcid_hf_real.dat};
                        \addlegendentry{TDCID-HF}

                        \addplot+ [
                            mark=none,
                            thick,
                            dashed,
                        ]
                        table
                        {results/benchmarks/zanghellini/dat/rho_tdcisd_real.dat};
                        \addlegendentry{TDCISD}

                    \nextgroupplot[
                            ymode=log,
                            grid=major,
                            ylabel={$\abs{\Delta\densityten(x, 0)}$},
                            restrict x to domain=-6:6,
                            enlarge x limits=false,
                        ]
                        \addplot+ [
                            mark=none,
                            thick,
                        ]
                        table
                        {results/benchmarks/zanghellini/dat/rho_diff_tdccd_tdcid.dat};
                        \addlegendentry{HO}

                        \addplot+ [
                            mark=none,
                            thick,
                        ]
                        table
                        {results/benchmarks/zanghellini/dat/rho_diff_tdccd_tdcid_hf.dat};
                        \addlegendentry{HF}
                \end{groupplot}
            \end{tikzpicture}
            \caption{In this figure we try to reproduce
            \autoref{fig:one-body-density-zanghellini} using plain CID and CID with
            Hartree-Fock basis.
            The lower figure shows the absolute difference between CCD and CID
            with harmonic oscillator (HO) basis and Hartree-Fock (HF) basis.}
            \label{fig:one-body-density-doubles-zanghellini}
        \end{figure}

        In \autoref{fig:overlap-doubles-zanghellini} we show how TDCID and TDCCD
        with and without a Hartree-Fock basis is unable to give a good
        reconstruction of the exact solution from full configuration
        interaction.
        The lack of singles excitations become glaringly apparent once we start
        the time-evolution of the system.
        This can be seen as an effect of the laser field being described by a
        singles operator.
        It is therefore likely that a singles-excitation operator best describes
        this interaction and the doubles approximation will have problems
        properly depicting the dynamics.
        In fact, comparing with \autoref{fig:overlap-zanghellini} we see that
        TDHF gives a qualitatively much better approximation to the exact
        results which demonstrates that the singles approximation is essential
        for the laser driven dynamics we explore.

        \begin{figure}
            \centering
            \begin{tikzpicture}
                \begin{groupplot}[
                        group style={
                            group size=1 by 2,
                            vertical sep=30pt,
                            xlabels at=edge bottom,
                        },
                        width=11cm,
                        height=6cm,
                        xlabel={$\Omega t / (2\pi)$},
                    ]
                    \nextgroupplot[
                            grid=major,
                            ylabel={$\abs{\braket*{\Psi(t)}{\Psi(0)}}^2$},
                            enlarge x limits=false,
                            title={Time-dependent overlap},
                        ]

                        \addplot+ [
                            mark=none,
                            thick,
                        ]
                        table
                        {results/benchmarks/zanghellini/dat/overlap_tdcid_real.dat};
                        \addlegendentry{TDCID}

                        \addplot+ [
                            mark=none,
                            thick,
                        ]
                        table
                        {results/benchmarks/zanghellini/dat/overlap_tdcid_hf_real.dat};
                        \addlegendentry{TDCID-HF}

                        \addplot+ [
                            mark=none,
                            thick,
                            dashed,
                        ]
                        table
                        {results/benchmarks/zanghellini/dat/overlap_tdcisd_real.dat};
                        \addlegendentry{TDCISD}

                    \nextgroupplot[
                            ymode=log,
                            grid=major,
                            ylabel={$\Delta\abs{\braket*{\Psi(t)}{\Psi(0)}}^2$},
                            restrict x to domain=-6:6,
                            enlarge x limits=false,
                        ]
                        \addplot+ [
                            mark=none,
                            thick,
                        ]
                        table
                        {results/benchmarks/zanghellini/dat/overlap_diff_tdccd_tdcid.dat};
                        \addlegendentry{HO}

                        \addplot+ [
                            mark=none,
                            thick,
                        ]
                        table
                        {results/benchmarks/zanghellini/dat/overlap_diff_tdccd_tdcid_hf.dat};
                        \addlegendentry{HF}
                \end{groupplot}
            \end{tikzpicture}
            \caption{Here we reproduce \autoref{fig:overlap-zanghellini} with
            the doubles approximations CID and CCD using both the original
            harmonic oscillator (HO) basis, and the Hartree-Fock (HF) basis.
            In the lower figure we demonstrate the absolute error between CID
            and CCD using both basis sets.}
            \label{fig:overlap-doubles-zanghellini}
        \end{figure}

        We therefore conclude that the pure doubles approximations are inferior
        at describing the dipole laser fields we explore and we will therefore
        not use these for any significant results in the remainder of this
        thesis.



    \section{Two-electron molecule}
        \label{sec:li}
        In a study on the optical response of molecules subject to an
        intense laser fields done by \citeauthor{li_2005} \cite{li_2005} a
        comparison of the time-dependent Hartree-Fock method and the
        time-dependent full configuration interaction\footnote{%
            Note that \citeauthor{li_2005} denotes the time-dependent
            full-configuration interaction method by: ``the time-dependent
            Schrödinger equation''.
        } is performed.
        This provides us with an ample opportunity to repeat the
        experiments in order for us to verify our methods.
        We will therefore reproduce parts of the study done by
        \citeauthor{li_2005} using TDHF, TDCISD, TDCCSD, and OATDCCD.

        We'll look at the smallest molecule \ch{H2}, that is, the
        Hydrogen-molecule with two electrons.
        This means that TDCISD, TDCCSD, and OATDCCD should produce exact
        results for the single-particle basis that is used.
        We express \ch{H2} in $14$ Gaussian type orbitals, that is, the
        6-311++G(d, p) basis set, with an equilibrium geometry of $R_e =
        \SI{0.7354}{\angstrom} \approx \SI{1.3897}{\bohr}$.
        We use PySCF \cite{pyscf} to create the basis set with the string
        \pyth{"6-311++Gss"}.
        We center the molecule around the origin with each atom located at
        $\pm R_e/2$ in the $z$-direction zero for the $x$- and
        $y$-direction.
        We then make the basis set doubly occupied by including spin.
        We perform a ground state calculation using the general Hartree-Fock
        solver and transform to this basis.
        We then do ground state computations using the time-independent version
        of the specific solvers berfore starting the time-evolution.
        The laser pulse used by \citeauthor{li_2005} is
        \begin{align}
            \vfg{E}(t)
            = \hat{\vfg{d}} \cdot \vfg{f}(t)\sin(\omega t),
            \label{eq:li-field}
        \end{align}
        where the envelope function is given by $\vfg{f}(t) =
        f(t)\vfg{\epsilon}$ with $\vfg{\epsilon}$ as the polarization vector.
        The envelope is described by
        \begin{align}
            f(t) = \begin{cases}
                (\omega t / 2\pi) E_m, & \omega t \in [0, 2\pi], \\
                E_m, & \omega t \in [2\pi, 4\pi], \\
                [3 - \omega t / (2\pi)] E_m, & \omega t \in [4\pi, 6\pi], \\
                0, & \omega t \neq [0, 6\pi],
            \end{cases}
            \label{eq:li-laser}
        \end{align}
        where we set $E_m = \SI{0.07}{\hartree} \approx
        \SI{1.72e14}{\watt/\cm^2}$ and $\omega =
        \SI{0.1}{\hartree/\hslash}$ which corresponds to a wavelength of
        $\lambda = \SI{456}{\nano\meter}$.
        We run the simulation for a total of $t_f = \SI{225}{\hslash/\hartree}$
        with the laser turned on from the start.
        We use the Gauss-integrator with $s = 3$ a fix-point convergence
        threshold of $\epsilon = \num{1e-6}$ and an integration time step of
        $\Delta t = \SI{1e-2}{\hslash/\hartree}$.
        %A plot of the laser field during the simulation is shown in
        %\autoref{fig:li-laser}.
        %\begin{figure}
        %    \centering
        %    \begin{tikzpicture}
        %        \begin{axis}[
        %            width=11cm,
        %            height=6cm,
        %            xlabel={$t$ $[\si{\hslash/\hartree}]$},
        %            ylabel={$E(t)$ $[\si{\hartree}]$},
        %            grid=major,
        %            enlarge x limits=false,
        %            enlarge y limits=false,
        %        ]
        %            \addplot+[
        %                mark=none,
        %                thick,
        %            ]
        %            table
        %            {results/benchmarks/li/dat/li_laser.dat};
        %        \end{axis}
        %    \end{tikzpicture}
        %    \caption{In this figure we've plotted the laser field from
        %    \autoref{eq:li-field} with the envelope described in
        %    \autoref{eq:li-laser} as done in the study by
        %    \citeauthor{li_2005} \cite{li_2005}.}
        %    \label{fig:li-laser}
        %\end{figure}
        In our programs the dipole moment is defined with a positive sign,
        which means that we need to introduce a negative sign in the
        envelope or the polarization vector to include the negative charge
        of the electrons.
        We have chosen to set the polarization vector along the negative
        $z$-direction.
        A plot of the instantaneous dipole moment of \ch{H2} is shown in
        \autoref{fig:dipole-moment-li}.
        A by-eye comparison of the top figure in
        \autoref{fig:dipole-moment-li} with figure 4 in \citetitle{li_2005}
        \cite{li_2005} shows that we are in perfect agreement with the
        results of \citeauthor{li_2005}.
        In the lower figure we show the absolute error in the induced dipole
        moment as calculated by the TDCCSD and OATDCCD methods compared with
        the exact full configuration interaction solution.
        As the error between the TDCISD, and TDCCSD and OATDCCD are smaller than
        the convergence threshold of the Gauss-integrator, we conclude that
        TDCCSD and OATDCCD reproduce the exact results to a satisfying degree.
        The TDHF-method is several orders of magnitude off from the
        coupled-cluster methods, but it is interesting to note in the top figure
        in \autoref{fig:dipole-moment-li} that the time-dependent Hartree-Fock
        method performs qualitatively well.
        There are small discrepancies visible to the eye, but at the intensity
        we use for our laser the mean-field approximation performs excellently.
        However, we expect that as the intensity increases, this method will
        prove inferior to the correlated methods.

        \begin{figure}
            \centering
            \begin{tikzpicture}
                \begin{groupplot}[
                        group style={
                            group size=1 by 2,
                            vertical sep=30pt,
                            xlabels at=edge bottom,
                        },
                        width=11cm,
                        height=6cm,
                        xlabel={$t$ $[\si{\hslash/\hartree}]$},
                    ]
                    \nextgroupplot[
                            grid=major,
                            ylabel={$\expval*{z(t)}$},
                            enlarge x limits=false,
                            enlarge y limits=false,
                            title={Dipole moment of \ch{H2}-molecule},
                        ]
                        \addplot+ [
                            mark=none,
                            thick,
                        ]
                        table
                        {results/benchmarks/li/dat/dipole_z_tdcisd_real.dat};
                        \addlegendentry{TDCISD}

                        \addplot+ [
                            mark=none,
                            thick,
                        ]
                        table
                        {results/benchmarks/li/dat/dipole_z_tdhf_real.dat};
                        \addlegendentry{TDHF}

                    \nextgroupplot[
                            ymode=log,
                            grid=major,
                            ylabel={$\abs{\Delta\expval*{z(t)}}$},
                            enlarge x limits=false,
                            enlarge y limits=false,
                            legend pos=south east,
                        ]
                        \addplot+ [
                            mark=none,
                            thick,
                        ]
                        table
                        {results/benchmarks/li/dat/dipole_z_diff_tdccsd_tdcisd.dat};
                        \addlegendentry{TDCCSD}

                        \addplot+ [
                            mark=none,
                            thick,
                        ]
                        table
                        {results/benchmarks/li/dat/dipole_z_diff_oatdccd_tdcisd.dat};
                        \addlegendentry{OATDCCD}

                        \addplot+ [
                            mark=none,
                            thick,
                        ]
                        table
                        {results/benchmarks/li/dat/dipole_z_diff_tdhf_tdcisd.dat};
                        \addlegendentry{TDHF}
                \end{groupplot}
            \end{tikzpicture}
            \caption{In the top figure we have plotted the instantaneous
            dipole by radiating a \ch{H2} molecule with a dipole laser.
            The lower figure shows the absolute error of the dipole using the
            TDHF, TDCCSD, and OATDCCD methods compared to the TDCISD solver.}
            \label{fig:dipole-moment-li}
        \end{figure}

    \section{Summing up the validation}
        We have now demonstrated that the bulk of our methods perform as
        expected and give satisfying results for small systems.
        Furthermore, we've demonstrated that the self-made one-dimensional
        system and the interface towards PySCF \cite{pyscf} are working.
        We've also demonstrated that we support several dipole laser fields and
        that we can compute various quantities from all methods.
        Recalling the words of E. W. Dijkstra we do note that we have not proven
        our methods to be free of bugs, but at least we have demonstrated that a
        wide range of our code works as expected.

\clearpage
